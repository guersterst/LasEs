\localauthor{Johannes Garstenauer}

Im Folgenden soll die Qualität der Testsuites bestimmt werden,
um ihre Aussagekraft hinsichtliche der Bestätigung der Qualität vom LaSes-System zu evaluieren.
Hierzu wird die Quelltextüberdeckung der \emph{Unittests} wie auch der \emph{Integrationstests} untersucht.
Schlußendlich werden die Ergebnisse interpretiert.

\subsection{Quelltextüberdeckung}\label{subsec:quelltextueberdeckung}
Zur Messung der Quelltextüberdeckung wurde die populäre \emph{Java Code Coverage Library \textbf{JaCoCo}} verwendet.
In der Analyse wird ein besonderes Augenmerk auf die Werte der \emph{Zeilen-} und \emph{Zweigüberdeckung} gelegt,
da diese am aussagekräftigsten bezüglich der Testqualität sind.

\subsubsection{Unit Tests}
Es existieren \emph{Whiteboxtests} zu den meisten Modulen, Klassen und Methoden des Projekts.
Es besteht jedoch kein Anspruch auf Vollständigkeit.
Die \emph{Unittests} entstanden größtenteils parallel zur Entwicklung.
Bestimmte Arten von Klassen wurden hierbei ausgespart:

\begin{itemize}
    \item \emph{Backing-Beans}: Es bestand die Erwartung, dass diese sinnvoller in den \emph{Blackboxtests}
    abgedeckt werden können
    \item \emph{Klassen in den internal-Paketen}: Hier war in der Regel ein großer \emph{Mockingaufwand} vonnöten,
    um erwartetes Verhalten testen zu können.
    Aufgrund des Wunsches, vonseiten der Entwickler, die begrenzte Entwicklungszeit
    wirkungsvoll einzusetzen und der oft überschaubaren \emph{JSF}-fremden Logik welche zu überprüfen war,
    wurden diese Klassen beim \emph{unittesting} oft ausgespart.
    Beispiele hierfür sind z.B. \emph{UncheckedExceptionHandler} oder \emph{MessageResourceBundleProducer}.
\end{itemize}
Die \emph{Quelltextüberdeckungswerte} sind vor diesem Hintergrund zu betrachten.

\newline

Die XXX Testmethoden der Unittests erreichen eine \emph{Zeilenüberdeckung} von \textbf{XX\%}
und eine \emph{Zweigüberdeckung} von \textbf{XX\%}.

Übersicht über die genauen Werte gegliedert nach Paketen.

\todo{Ausreißer}

\subsubsection{Integration Tests}
Die \emph{Blackboxtests} decken große Teile der vorgesehenen Funktionalität ab und beinhalten zusätzlich Tests
zu sicherheitskritischen Szenarien, wie \emph{Skript- und Sequelinjektionen}.
Es besteht wiederum kein Anspruch auf Vollständigkeit.
Erwartete Schwachstellen der \emph{Quelltextberdeckung} sind zu erwarten aufgrund von:
\begin{itemize}
    \item Den entgegengesetzten Prinzipien von Isolation und Integration im \emph{Software Testing}.
    Integrationstests weisen eine höhere Integration (der Features) auf Kosten der Isolation auf und
    erreichen aufgrunddessen auch eine geringere \emph{Quelltextüberdeckung}.
    \item Codeabschnitten, welche sich der defensiven Programmierung widmen, werden seltener überdeckt
\end{itemize}
Die \emph{Quelltextüberdeckungswerte} sind vor diesem Hintergrund zu betrachten.

\newline

Die XXX Testmethoden der Integrationtests erreichen eine \emph{Zeilenüberdeckung} von \textbf{XX\%}
und eine \emph{Zweigüberdeckung} von \textbf{XX\%}.

Übersicht über die genauen Werte gegliedert nach Paketen.

\todo{Ausreißer}

\subsubsection{Kombination der Unit- und Integrationstests}

Kombiniert wird eine
\emph{Zeilenüberdeckung} von \textbf{XX\%} und eine \emph{Zweigüberdeckung} von \textbf{XX\%} erreicht.

\subsubsection{Mutationstesting}

\subsubsection{Fazit}
Abschließend ist anzumerken, dass im \emph{Software Testing} generell kein Anspruch auf eine \emph{Quelltextüberdeckung}
von 100\% besteht.
Der Grund hierfür ist die Zeitintensivität von ausführlichem Testing (siehe Mockingprobleme in internals-Paketen).
Außerdem liegt es leider in der Natur der Sache, dass selbst eine 100\%ige Testsuite keine Garantie auf Buglosigkeit
der Anwendung garantiert, da auch der Testquellcode Bugs enthalten kann.
Grundsätzlich ist es ein Axiom in der Softwareentlickung,dass kein System ohne Bugs ist.
Unter allen diesen Gesichtspunkten lässt sich dem LaSes-System nichtsdestrotz attestieren, dass ...
Die Testsuite erfüllt die Anforderung (etwas, weniger, sehr) ...