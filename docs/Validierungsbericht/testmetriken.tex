\localauthor{Johannes Garstenauer}

Im Folgenden soll die Qualität der Testsuites bestimmt werden,
um ihre Aussagekraft hinsichtliche der Bestätigung der Qualität vom LaSes-System zu evaluieren.
Hierzu wird die Quelltextüberdeckung der \emph{Unittests} wie auch der \emph{Integrationstests} untersucht.
Schlußendlich werden die Ergebnisse interpretiert.

\subsection{Quelltextüberdeckung}\label{subsec:quelltextueberdeckung}
Zur Messung der Quelltextüberdeckung wurde die populäre \emph{Java Code Coverage Library \textbf{JaCoCo}} verwendet.
In der Analyse wird ein besonderes Augenmerk auf die Werte der \emph{Zeilen-} und \emph{Zweigüberdeckung} gelegt,
da diese am aussagekräftigsten bezüglich der Testqualität sind.

\subsubsection{Unit Tests}
Es existieren \emph{Whiteboxtests} zu den meisten Modulen, Klassen und Methoden des Projekts.
Es besteht jedoch kein Anspruch auf Vollständigkeit.
Die \emph{Unittests} entstanden größtenteils parallel zur Entwicklung.
Bestimmte Arten von Klassen wurden hierbei ausgespart:

\begin{itemize}
    \item \emph{Backing-Beans}: Es bestand die Erwartung, dass diese sinnvoller in den \emph{Blackboxtests}
    abgedeckt werden können
    \item \emph{Klassen in den internal-Paketen}: Hier war in der Regel ein großer \emph{Mockingaufwand} vonnöten,
    um erwartetes Verhalten testen zu können.
    Aufgrund des Wunsches, vonseiten der Entwickler, die begrenzte Entwicklungszeit
    wirkungsvoll einzusetzen und der oft überschaubaren \emph{JSF}-fremden Logik welche zu überprüfen war,
    wurden diese Klassen beim \emph{unittesting} oft ausgespart.
    Beispiele hierfür sind z.B. \emph{UncheckedExceptionHandler} oder \emph{MessageResourceBundleProducer}.
\end{itemize}
Die \emph{Quelltextüberdeckungswerte} sind vor diesem Hintergrund zu betrachten.

\newline

Die XXX Testmethoden der Unittests erreichen eine \emph{Zeilenüberdeckung} von \textbf{XX\%}
und eine \emph{Zweigüberdeckung} von \textbf{XX\%}.

\newline
Weiterhin ist anzumerken, im \emph{Software Testing} generell kein Anspruch auf eine \emph{Quelltextüberdeckung} von
100\% besteht.
Dies ist begründet Gesichtspunkten von effizienter Zeitnutzung etc.

Übersicht über die genauen Werte gegliedert nach Paketen.

\todo{Ausreißer}

\subsubsection{Integration Tests}
Die \emph{Blackboxtests} decken große Teile der vorgesehenen Funktionalität ab und beinhalten zusätzlich Tests
zu sicherheitskritischen Szenarien, wie \emph{Skript- und Sequelinjektionen}.
Die XXX Testmethoden der Unittests erreichen eine \emph{Zeilenüberdeckung} von \textbf{XX\%}
und eine \emph{Zweigüberdeckung} von \textbf{XX\%}.

\subsubsection{Kombination der Unit- und Integrationstests}