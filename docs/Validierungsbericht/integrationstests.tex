\localauthor{Stefanie Gürster, Johann Schicho}

Wir verwenden das \emph{Selenium Framework} um mehrere Nutzerabläufe im Webbrowser zu automatisieren.
Dabei übernimmt ein Programm den Browser und navigiert anhand der \emph{ID}
der Webelemente. Dabei wird dann gegen die angezeigten Meldungen getestet, um
den korrekten Ablauf sicherzustellen.

\subsection{Probleme zu Beginn}

\begin{itemize}
	\item Verwendung von \emph{IDs} in JSF-Komponenten. Bereits während der
	Implementierung wurden überall, nach Definition der Feinspezifikation, IDs in den \emph{Facelets} verwendet.

	Allerdings wurde übersehen, dass auch \emph{Forms} und \emph{Composite Components} immer eigene IDs benötigen, da ansonsten JSF diese dynamisch
	generiert. Die dynamisch generierten IDs können sich bei jedem \emph{Render Vorgang} wieder ändern und erlauben damit also leider keine Identifikation.\newline
	Es wurden an allen Stellen, an denen die IDs vergessen wurden, diese noch eingefügt.



\end{itemize}