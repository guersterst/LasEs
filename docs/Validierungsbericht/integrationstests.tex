\localauthor{Stefanie Gürster, Johann Schicho}

Wir verwenden das \emph{Selenium Framework} um mehrere Nutzerabläufe im Webbrowser zu automatisieren.
Dabei übernimmt ein Programm den Browser und navigiert anhand der \emph{ID}
der Webelemente. Dabei wird dann gegen die angezeigten Meldungen getestet, um
den korrekten Ablauf sicherzustellen.

\subsection{Probleme zu Beginn}

\begin{itemize}
	\item \textbf{Verwendung von \emph{IDs} in JSF-Komponenten.} Bereits während der
	Implementierung wurden überall, nach Definition der Feinspezifikation, IDs in den \emph{Facelets} verwendet.

	Allerdings wurde übersehen, dass auch \emph{Forms} und \emph{Composite Components} immer eigene IDs benötigen, da ansonsten JSF diese dynamisch
	generiert. Die dynamisch generierten IDs können sich bei jedem \emph{Render Vorgang} wieder ändern und erlauben damit also leider keine Identifikation.\newline
	Es wurden an allen Stellen, an denen die IDs vergessen wurden, diese noch eingefügt.

	\item \textbf{Testsuites mit JUnit5.} Testsuite erlauben eine Zusammenfassung von mehreren Tests in ein Gesamtpaket. Das ist eine wichtige
	Anforderung für unsere \emph{Integration Tests}, da diese von einander abhängig sind, also eine Reihenfolge festgelegt werden muss.

	JUnit5 besaß bis zu Release 5.8\footnote{\url{https://junit.org/junit5/docs/5.8.0/release-notes/}}
	am 12. September 2021 keine Möglichkeiten für Test Suites. Bisher waren immer Abhängigkeiten zu Junit4 nötig, um diese
	Funktionalität in JUnit5 zu verwenden. Unsere Testsuite benutzt die neue
	\emph{@Suite Annotation} von JUnit5. Anfangs gab es damit Startschwierigkeiten, da das Feature sehr neu ist und die offizielle
	Dokumentation spärlich ist.
\end{itemize}

\begin{table}
	\centering
	\begin{tabular}{m{1.2cm}|m{4.5cm}|m{4.5cm}|l}
		\toprule
		\textbf{Test} & \textbf{Änderungen} & \textbf{Bugfixes} & \textbf{Zuständigkeit} \\\midrule
		/T010/ & - & - & Johann Schicho \\\midrule
		/T020/ & - & - & Johann Schicho \\
	\end{tabular}
\end{table}

