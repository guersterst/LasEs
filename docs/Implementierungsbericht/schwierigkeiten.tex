\localauthor{Johann Schicho}

Wie bereits aus Kapitel 1 hervorgeht, gab es teilweise größere Abweichungen von den geplanten Zeiten. Diese werden hier nochmal aufgefasst und genauer beschrieben. Ziel ist es zu reflektieren und die Gründe für größere Verzögerungen zu erörtern. Zusätzlich wird auch auf Probleme in der Programmierung selbst eingegangen.

\subsection{Größere Abweichungen in der geschätzten Zeit}

\subsubsection{Arbeitspaket NewSubmission}

Dieses Arbeitspaket benötigte 13 Stunden mehr als ursprünglich geplant.
\emph{NewSubmission} war Teil des ersten Milestones. Dadurch war viel Code noch nicht vorhanden. Da wir alle Arbeitspakete vertikal implementiert haben, das heißt zu jedem Frontend-Code wurde der zugehörige Backend-Code von der gleichen Person implementiert, war es sehr viel Arbeit für einen Einzelnen. Zusätzlich benötigte dieses Arbeitspakete auch viele unterschiedliche \emph{Services} und damit verschob sich der Zeitaufwand immens.

\subsubsection{Arbeitspaket Pagination \& SortSearchColumn}

Diese Arbeitspakete benötigten während der Implementierungsphase keine weitere Zeit. Grund dafür war, das bereits Prototypen in den Wochen davor programmiert worden sind. Damit war die Arbeit bereits abgeschlossen und konnte direkt übernommen werden.

\subsubsection{Arbeitspaket Homepage}

Hier wurden acht Stunden mehr benötigt, als ursprünglich angenommen. Grund dafür war ähnlich wie bei \emph{NewSubmission}. Die Übersichtsseite sammelt Informationen aus vielen Ecken des Systems und stellt diese dar, dafür muss natürlich auch viele Code geschrieben werden. Den Aufwand, besonders während des ersten Milestones, haben wir hier unterschätzt.

\subsubsection{Arbeitspaket Submission ohne Gutachten}

Wie der Name schon impliziert, wurde hier bereits ein größeres Arbeitspaket aufgespalten. Die Seite einer Einreichung ist das Kernstück der Anwendung und umfasst damit viel Funktionalität. Der große Aufwand war uns bewusst, dennoch wurden zehn Stunden mehr benötigt als geplant. Grund hier war neben den bereits genannten Gründen natürlich auch Startschwierigkeiten mit  \emph{Jakarta Server Faces}. Der Projektumfang ist wesentlich angestiegen und somit auch die Anforderungen an den Programmierer, sich tiefer mit der Technologie auseinander zu setzen.

\subsubsection{Fazit}

Der Zeitaufwand für Arbeitspakete, bei denen bereits viel Zeit eingeschätzt war, hat sich häufig um viel mehr Zeit verlängert, als bei Arbeitspakten, bei welchen die Zeit von Beginn an geringer eingeschätzt war. Eine weitere Aufteilung der Arbeitspakete hätte das gegebenenfalls verhindern können, hätte aber auch gegenseitige Abhängigkeiten während der Implementierung erzeugt. Das hätte zu anderweitigen Verzögerungen geführt.


Abweichungen von zwei bis drei Stunden waren abzusehen und sind auch bei den meisten Arbeitspaketen aufgetreten, sowohl als Verlängerung als auch Verkürzung der tatsächlichen Arbeitszeit.

\subsection{Direkte Probleme bei der Programmierung}

\subsubsection{JDBC}

Die Verbindung zur Datenbank wurde nach Voraussetzung mit \emph{Java Database Connectivity} umgesetzt. Die Technologie war im geringen Maße bereits bekannt und konnte sehr schnell eingesetzt werden. Probleme entstanden durch benötigte Aufrufe von \emph{Close Methoden}. Diese werden benötigt um sowohl \emph{Connections} an den \emph{Connection Pool} zurückzugeben, als auch \emph{SQL Statements} zu schließen.

Anfangs führte das zu vielen \emph{Leaks} an Datenbankverbindungen, die nicht mehr ordnungsgemäß zurückgegeben wurden. Gelöst wurde das Problem durch besseres \emph{Logging} der Datenbankverbindungen und Verwendung von \emph{Try-with-Ressource} Codeblöcken. Das benötigte zusätzliche Zeit zu debuggen und implementieren und hat die Arbeitszeit bei vielen Arbeitspaketen gestreckt.

Ein weiteres Problem war das \emph{Error-Handling} von SQL Errors. \emph{PostgreSQL} und \emph{JDBC} geben hier nur numerische Fehlercodes zurück und keine direkte Information, ob es sich um eine \emph{Checked- oder Unchecked Exception} handelt. Hier wurde ein Prototyp von Sebastian Vogt entwickelt, allerdings zuerst nur vereinzelt von ihm umgesetzt. Es benötigte anschließend einen gemeinsamen Einsatz aller Teammitglieder alle Datenbankzugriffsmethoden mit der neuen Fehlerbehandlung auszustatten.

\subsubsection{Fehlerseiten Facelet}

\subsubsection{Testing}