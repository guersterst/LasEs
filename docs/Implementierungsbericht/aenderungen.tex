\localauthor{Stefanie Gürster}

\subsection{Klassen}

Damit ein Weiterleiten auf die nächste Seite einer \emph{Pagination} funktioniert, müssen alle zugehörigen Einträge gezählt werden. Dies übernehmen Methoden im Service sowie in den Repositories mit der Namenskonvention \emph{count...()}. Eine solche Methode wird zum Beispiel im PaperService verwendet, um alle Paper einer Submission zu zählen. Um Redundanzen im Folgenden zu vermeiden wird diese Methode im Weiteren nicht mehr angesprochen.

\todo{SubmissionService: addReviewer(ohne Submission)}

\subsubsection{business.internal}

\textbf{Lifetime} In der Methode \emph{startup()} wird nun zusätzlich ein FileDTO, welches \emph{logger properties} als \emph{ImputStream} enthält, als Parameter übergeben.

\textbf{PeriodicWorker} Um den PeriodicWorker zu implementieren wird ein Timer Objekt verwendet. Dieser Timer wird durch \emph{init()} gestartet und durch \emph{stop()} gestoppt.

\subsubsection{business.service}

\textbf{PaperService} Für die Methode \emph{getLatest()} wird nur ein Submission-DTO benötigt und kein User-DTO wie zuvor angenommen. \newline
Die Methode \emph{remove()} wurde entfernt, da sie nicht benötigt wird.

\textbf{ScienceFieldService} Die Funktion eine einzelnes Fachgebiet mit \emph{get()} zu laden wurde nicht benötigt.

\textbf{SubmissionService} In \emph{add()} werden zusätzlich ein Paper-DTO, sowie ein File-DTO mitübergeben. Im Gegensatz dazu wird eine Liste von Gutachter nicht mehr benötigt. \newline
Der Rückgabewert der Methode \emph{change()} wurde von \emph{void} zu \emph{boolean} verändert, um Rückmeldung der Änderung zu erhalten. \newline
Beispielsweise wird in der Toolbar eine Liste von Gutachtern der jeweiligen Einreichung geladen. Daher wurde die Methode \emph{getList(Submission submission)} neu hinzugefügt. \newline
Die Methode \emph{releaseReview()} wird bereits im \emph{ReviewService} durch \emph{change()} implementiert und ist somit hier überflüssig. \newline
Ebenso wird die Methode \emph{addCoAuthor()} nicht mehr verwendet, da Co-Autoren nur bei der Einreichung selbst hinzugefügt werden. Die Funktionalität liegt also schon in der Methode \emph{add()}.

\textbf{UserService} Um den Avatar eines Nutzers zu löschen, wird \emph{setAvatar()} verwendet. Daher ist \emph{deleteAvatar()} überflüssig. \newline
Die Methode \emph{getVerification()}, welche bestimmt ob eine E-Mail-Adresse schon verifiziert ist, wird nicht mehr benötigt, da die Verifikation bereits über \emph{isVerified()} aus dem User DTO abgefragt, oder über \emph{verify()} eine E-Mail-Adresse verifiziert werden kann.

\subsubsection{business.util}

\textbf{EmailUtil} Aufgrund von Benutzerfreundlichkeit wird
werden in einigen E-Mails auch passend generierte Links mit versendet. Dafür wurden folgende Methoden neu hinzugefügt: \emph{generateLinkForEmail, generateSubmissionURL, generateForumURL}.

\subsubsection{control.backing}

Diesem Paket wurden drei weitere Interfaces hinzugefügt: \emph{ScientificForumPaginationBacking, SubmissionPaginationBacking und UserPaginationBacking}. Der Grund dafür war die Erweiterung der \emph{Templates} um drei \emph{Composite Component}. Die einzelnen \emph{Paginations} werden an mehreren Orten verwendet. Um Codeduplikationen zu vermeiden und für die jeweilige Komponente, die benötigten Methoden zur Verfügung zu stellen, wurden die genannten Interfaces implementiert.

\textbf{AdministrationBacking} Um die Funktionalität zu erweitern, ein neues Logo hochladen zukönnen. wurde zusätzlich \emph{uploadNewLogo()} hinzugefügt. \newline
Damit das momentan ausgewählte Stylesheet geladen wird und dem Benutzer angezeigt werden kann, wurde \emph{getPathToStyle()} hinzugefügt.

\textbf{HomepageBacking} Der Name eines zugehörigen Forums zu einer Einreichung wird über \emph{getForumName()} geladen, damit dieser in der Pagination angezeigt werden kann. \newline
Um einzelne Tabs per \emph{CSS} dynamisch zu aktivieren, wird dies über \emph{get...CssClassSuffix()} gesetzt.

\localauthor{Johannes Garstenauer}

