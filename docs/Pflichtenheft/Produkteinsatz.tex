\localauthor{Thomas Kirz}

\subsection{Anwendungsbereiche \& Zielgruppen}

Die Applikation ermöglicht das Einreichen von Artikeln für Konferenzen und Journale.

Damit richtet sie sich an die einreichenden Wissenschaftler:innen, gutachtende \emph{peers} und Editor:innen, die zu den jeweiligen Konferenzen und Journalen gehören.
Die Wissenschaftler:innen und Gutachter:innen sollten dazu qualifiziert sein, in dem Bereich des Artikels wissenschaftliche Arbeiten schreiben zu können.
Die Editor:innen werden von den Konferenzen und Journalen gestellt und sind in der Lage, über die Annahme einer Einreichung zu entscheiden.

Die Anwendung wird auch von Administrator:innen genutzt, um die Software zu betreiben und zu konfigurieren.
Diese sollten daher Erfahrung mit der Installation, Verwaltung und Wartung von Web- und Datenbankapplikationen haben.
Sie haben auch die Aufgabe, Konferenzen und Journale einzurichten und müssen daher mit deren Repräsentanten in Kontakt stehen.

\subsection{Betriebsbedingungen}

LasEs ist als Webanwendung frei im World Wide Web verfügbar und kann daher von allen Nutzer:innen mit ihren eigenen Endgeräten mit gängigen Browsern weltweit bedient werden.

Die Software ist jederzeit zugänglich bis auf eine von dem/der Administrator:in festgelegte wöchentliche Stunde für Wartungsarbeiten.

Für den Betrieb des Systems sind ein Web- und ein Datenbankserver nötig.
Diese können getrennt sein oder zwei Dienste auf dem gleichen Server.
Dafür kann ein externes Rechenzentrum benutzt werden oder man betreibt einen eigenen Server in einer Umgebung mit adäquater Kühlungs- und Sicherheitsinfrastruktur.