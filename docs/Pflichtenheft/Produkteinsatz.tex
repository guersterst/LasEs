\localauthor{Thomas Kirz}

\subsection{Anwendungsbereiche \& Zielgruppen}

Die Applikation ermöglicht das Einreichen von Artikeln für Konferenzen und Journale.

Damit richtet sie sich an die einreichenden Wissenschaftler, gutachtende \emph{Peers} und Editoren, die zu den jeweiligen Konferenzen und Journalen gehören.
Die Wissenschaftler und Gutachter sollten dazu qualifiziert sein, in dem Bereich des Artikels wissenschaftliche Arbeiten schreiben zu können.
Die Editoren werden von den Konferenzen und Journalen initial gestellt und sind in der Lage, über die Annahme einer Einreichung zu entscheiden oder andere Editoren zu ernennen.

Die Anwendung wird außerdem von Administrator genutzt, um die Software zu betreiben und zu konfigurieren.
Diese sollten daher Erfahrung mit der Installation, Verwaltung und Wartung von Web- und Datenbankapplikationen haben.
Sie haben auch die Aufgabe, Konferenzen und Journale einzurichten und müssen daher mit deren Repräsentanten in Kontakt stehen.

\subsection{Betriebsbedingungen}

LasEs ist als Webanwendung frei im World Wide Web verfügbar und kann daher von allen Nutzern mit ihren eigenen Endgeräten mit gängigen Browsern weltweit bedient werden.

Die Software ist jederzeit zugänglich bis auf eine vom Administrator festgelegte wöchentliche Stunde für Wartungsarbeiten.

Für den Betrieb des Systems sind ein Web- und ein Datenbankserver nötig.
Diese können getrennt sein oder zwei Dienste auf dem gleichen Server tätigen.
Dafür kann ein externes Rechenzentrum benutzt werden oder man betreibt einen eigenen Server in einer Umgebung mit adäquater Kühlungs- und Sicherheitsinfrastruktur.