\localauthor{Sebastian Vogt, Thomas Kirz}

Die Notation \texttt{/TXXX/} erlaubt eine spätere Referenzierung der einzelnen Tests in diesem und weiteren
Dokumenten. \texttt{XXX} entspricht dabei immer einer dreistelligen Zahl.
\subsection{Setup}\label{setup}
Vor Ausführung jeglicher Tests sollten folgende Voraussetzungen erfüllt sein:
\begin{itemize}
	\item Das System ist vollständig eingerichtet.
	Insbesondere sind die Datenbankschemata erstellt und die globalen Einstellungen sind getroffen.
	\item Der Testmodus der Anwendung ist aktiviert.
	Dieser zeigt Links, die durch eine Aktion per E-Mail versendet werden, nach Durchführen dieser Aktion direkt in einer Hinweisbox auf der Website an.
	\item Es existiert eine Administratorin mit folgenden Nutzerdaten:
	\begin{itemize}
		\item \emph{E-Mail Adresse}: kirz@fim.uni-passau.de
		\item \emph{Vorname}: Johanna
		\item \emph{Nachname}: Mayer
		\item \emph{Passwort}: UniDorfen1870!
	\end{itemize}
	\item Es existiert ein Nutzer mit folgenden Nutzerdaten:
	\begin{itemize}
		\item \emph{E-Mail Adresse}: schicho@fim.uni-passau.de
		\item \emph{Vorname}: Franz
		\item \emph{Nachname}: Huber
		\item \emph{Passwort}: TSVDorfen2001!
	\end{itemize}
	\item Es existiert eine Nutzerin mit folgenden Nutzerdaten:
	\begin{itemize}
		\item \emph{E-Mail Adresse}: vogt@fim.uni-passau.de
		\item \emph{Vorname}: Petra
		\item \emph{Nachname}: Müller
		\item \emph{Passwort}: TSVDorfen2002!
	\end{itemize}
	\item Es existiert ein Nutzer mit folgenden Nutzerdaten:
	\begin{itemize}
		\item \emph{E-Mail Adresse}: guerster@fim.uni-passau.de
		\item \emph{Vorname}: Tuti
		\item \emph{Nachname}: Aslan
		\item \emph{Passwort}: SupaDöner1970!
	\end{itemize}
\end{itemize}
Die Tests werden in der angegebenen Reihenfolge ausgeführt.
Das heißt jeder Test kann die Zustandsänderungen, die durch vorherige Tests ausgelöst worden sind, als gegeben voraussetzen.
\subsection{Administratoren}
\begin{description}

	\XXitem{/T010/}{t010} \emph{Testet \hyperref[funkt:830]{/F830/}}.
	Die Administratorin meldet sich mit ihren Anmeldedaten im System an und wird zur Startseite weitergeleitet.
	Auf der Kopfzeile klickt sie auf ``Journals \& Conferences'' und wird zur Liste der wissenschaftlichen Foren weitergeleitet.
	Von dort aus navigiert sie zur Seite zur Erstellung eines neuen wissenschaftlichen Forums.
	Dort erstellt sie ein Forum mit folgenden Daten:
	\begin{itemize}
		\item \emph{Editoren}: Nutzer mit E-Mail-Adresse guerster@fim.uni-passau.de
		\item \emph{Name}: Chemie Tagung
		\item \emph{Deadline}: 30.12.2099
		\item \emph{Kurzbeschreibung}: Es geht um Chemie.
		\item \emph{URL}: https://ch.em.ie/
		\item \emph{Anleitung zur Begutachtung}: Begutachten Sie.
	\end{itemize}
	Die Konferenz wird erfolgreich gespeichert und sie wird auf die Seite des wissenschaftlichen Forums weitergeleitet.
	Sie meldet sich ab.

\end{description}

\subsection{Angemeldeter Nutzer I}
\begin{description}

	\XXitem{/T020/}{t020} \emph{Testet \hyperref[funkt:160]{/F160/}}.
	Die Nutzerin mit der E-Mail-Adresse vogt@fim.uni-passau.de meldet sich im System an.
	Anschließend gibt sie im Suchfeld in der Kopfzeile ``Chemie Tagung'' ein und schickt die Suchanfrage mit Enter ab.
	Nun wird sie auf die Seite mit den Suchergebnissen weitergeleitet und das Forum namens ``Chemie Tagung'' ist der einzige Eintrag in der angezeigten Liste.

	\XXitem{/T030/}{t030} \emph{Testet \hyperref[funkt:400]{/F400/}}.
	Sie klickt auf das angezeigte Forum und wird auf dessen Übersichtsseite weitergeleitet.
	Nun navigiert die Nutzerin per Mausklick auf die Seite für eine neue Einreichung.
	Dort ist das Feld mit dem wissenschaftlichen Forum bereits richtig ausgefüllt, und zwar mit ``Chemie Tagung''

	\XXitem{/T040/}{t040} \emph{Testet \hyperref[funkt:420]{/F420/}}.
	Anschließend lädt die Nutzerin folgende PDF-Datei hoch: \href{https://dl.acm.org/doi/pdf/10.1145/3321707.3321795}{https://dl.acm.org/doi/pdf/10.1145/3321707.3321795}.
	Das erfolgreiche Hochladen wird ihr durch Anzeige des Dateinamens bestätigt.

	\XXitem{/T045/}{t045} \emph{Testet \hyperref[funkt:450]{/F450/}}.
	Sie trägt in den Feldern des Formulars folgende Daten ein:
	\begin{itemize}
		\item \emph{Name der Einreichung}: P\neq NP
		\item \emph{Co-Autoren}: Ein Ko-Autor mit folgenden Daten:
		\begin{itemize}
			\item \emph{Vorname} Christian
			\item \emph{Nachname} Bachmaier
			\item \emph{E-Mail-Adresse} garstenaue
		\end{itemize}
		\item \emph{Editor}: guerster@fim.uni-passau.de
	\end{itemize}
	Sie klickt auf den Button ``Submit Paper''.
	Da die angegebene E-Mail-Adresse ``garstenaue'' nicht gültig ist, ist das Einreichen nicht erfolgreich.
	Sie bleibt auf der Einreichungsseite und wird mit einer Fehlermeldung über das Problem informiert.
	Die validen Daten bleiben jedoch erhalten.

	\XXitem{/T050/}{t050} \emph{Testet \hyperref[funkt:410]{/F410/}}.
	Sie bessert nun die E-Mail-Adresse aus: ``garstenaue@fim.uni-passau.de'' und klickt wieder auf ``Submit Paper''.
	Diesmal ist das Einreichen erfolgreich und sie wird auf die Übersichtsseite der Einreichung weitergeleitet.

	\XXitem{/T060/}{t060} \emph{Testet \hyperref[funkt:460]{/F460/}}.
	Frau Müller klickt auf den Profil-Button in der Kopfzeile und dann auf ``Log out''.
	Sie wird erfolgreich ausgeloggt und auf die Anmeldeseite weitergeleitet.
\end{description}

\subsection{Editor I}
\begin{description}

	\XXitem{/T080/}{t080} \emph{Testet \hyperref[funkt:680]{/F680/}}.
	Der Nutzer mit der E-Mail-Adresse guerster@fim.uni-passau.de meldet sich im System an.
	Von der Startseite aus ruft er die Einreichung ``P\neq NP'' auf und landet auf der Seite dieser Einreichung.
	Er gibt in das Formular zur Zuweisung von Gutachtern ``schicho@fim.uni-passau.de'' ein und schickt das Formular ab.
	Es wird eine Rückmeldung für das erfolgreiche Hinzufügen eines Gutachters angezeigt.
	Anschließend meldet er sich ab.
	Da der Testmodus aktiviert ist, sind Links zur Annahme und Ablehnung für den eingeladenen Gutachter bekannt.

\end{description}

\subsection{Gutachter}
\begin{description}

	\XXitem{/T085/}{t085} \emph{Testet \hyperref[funkt:690]{/F690/}}.
	schicho@fim.uni-passau.de hat eine E-Mail mit folgendem Inhalt erhalten:
	\begin{itemize}
		\item alle relevanten Informationen zur Einreichung
		\item ein Link zur Annahme der Begutachtungsanfrage
		\item ein Link zur Ablehnung der Begutachtungsanfrage.
	\end{itemize}
	Er nutzt den Link aus Test \hyperref[t080]{/T080/} zur Annahme der Begutachtungsanfrage und ist nun Gutachter.

	\XXitem{/T090/}{t090} \emph{Testet \hyperref[funkt:540]{/F540/}}.
	Der Nutzer mit der E-Mail-Adresse schicho@fim.uni-passau.de meldet sich im System an.
	Von der Startseite aus ruft er die Einreichung ``P\neq NP'' auf und landet auf der Seite dieser Einreichung.
	Er klickt auf ``Review''
	Er nutzt das angezeigte Formular, um eine einseitige PDF-Datei namens \emph{gutachten.pdf} als Gutachten hochzuladen und klickt auf ``Submit Review''.
	Das Gutachten wurde erfolgreich hochgeladen und ist jetzt auf der Seite der Einreichung sichtbar.
	Anschließend meldet er sich ab.

\end{description}

\subsection{Editor II}
\begin{description}

	\XXitem{/T100/}{t100} \emph{Testet \hyperref[funkt:685]{/F685/}}.
	Der Nutzer guerster@fim.uni-passau.de meldet sich wie oben beschrieben an und navigiert zur Seite der Einreichung ``P\neq NP''.
	Dort sieht er ein Gutachten von schicho@fim.uni-passau.de.
	Er betätigt die Schaltfläche zur Freigabe dieses Gutachtens.
	Eine Bestätigung über die Freigabe des Gutachtens wird angezeigt.
	Dann meldet er sich wieder ab.

\end{description}

\subsection{Anonyme Nutzer}

\begin{description}
	\XXitem{/T110/}{t2110} \emph{Testet \hyperref[funkt:010]{/F010/}}.
	Christian Bachmaier aus \hyperref[t050]{/T050/} ist noch nicht im System registriert.
	Er ruft LasEs auf und wird zur Anmeldeseite weitergeleitet.
	\XXitem{/T120/}{t120} \emph{Testet \hyperref[funkt:060]{/F060/}}.
	Er klickt auf den Link zur Registrierung und gibt seinen Namen, seine E-Mail-Adresse und das Passwort \texttt{einsZwei3!5678} an.
	Die Registrierung wird bestätigt.
	Da der Testmodus aktiviert ist, wird der Verifizierungslink, hier angezeigt.
	\XXitem{/T130/}{t130} \emph{Testet \hyperref[funkt:070]{/F070/}}.
	Er klickt auf den Verifizierungslink und wird auf die Verifizierungsseite weitergeleitet.
	Damit ist sein Profil erstellt.
	Er wird automatisch auf die Homepage weitergeleitet.
	\XXitem{/T140/}{t140} \emph{Testet \hyperref[funkt:260]{/F260/}, \hyperref[funkt:437]{/F437/}}.
	Da er als Ko-Autor in Test \hyperref[t050]{/T050/} eingetragen wurde, wird ihm die Einreichung auf der Homepage angezeigt.
	Der Nutzer meldet sich ab.
\end{description}

\subsection{Angemeldeter Nutzer II}

\begin{description}

	\XXitem{/T150/}{t150} \emph{Testet \hyperref[funkt:480]{/F480/}}.
	Nun meldet sich die Nutzerin vogt@fim.uni-passau.de an und navigiert zur Seite der Einreichung ``P\neq NP''.
	Dort ist das Gutachten von schicho@fim.uni-passau.de sichtbar.
	Nach Betätigung der Download Schaltfläche wird die PDF-Datei \emph{gutachten.pdf} vom Browser heruntergeladen.

	\XXitem{/T160/}{t160} \emph{Testet \hyperref[funkt:230]{/F230/}}.
	Auf der Navigationsleiste klickt die Nutzerin auf den Profil-Button.
	Die Profil-Seite wird angezeigt.
	Die Nutzerin betätigt nun die Schaltfläche ``Delete account''.
	Daraufhin wird eine Warnung angezeigt, dass dies sowohl alle ihre Einreichungen löscht, als auch die Editoren und Gutachter der eingereichten Paper per E-Mail benachrichtigt.
	Die Nutzerin akzeptiert diese Nachricht und wird auf die Anmeldeseite weitergeleitet.
	Die Administratorin meldet sich nun im System an, navigiert über die Navigationsleiste zur Liste der wissenschaftlichen Foren und von dort auf ``Chemie Tagung''.
	Dort stellt sie fest, dass keine Paper eingereicht sind.

\end{description}

\subsection{Fehlerhafte und illegale Zugriffe}

\begin{description}
	\XXitem{/T200/}{t200} \emph{Testet \hyperref[funkt:040]{/F040/}}.
	Der Nutzer guerster@fim.uni-passau.de meldet sich an und ruft mit seinem Browser \texttt{<URL zur LasEs-Startseite>/freebitcoin.html} auf.
	Er wird auf eine 404-Fehlerseite weitergeleitet.
	\XXitem{/T210/}{t210} \emph{Testet \hyperref[funkt:040]{/F040/}}.
	Der Nutzer guerster@fim.uni-passau.de meldet sich an und ruft mit seinem Browser \texttt{<URL zur LasEs-Startseite>/administration.xhtml} auf.
	Er wird auf eine 404-Fehlerseite weitergeleitet.
\end{description}

\subsection{Reset}
Der Programmzustand, der im Abschnitt \emph{\hyperref[setup]{Setup}} beschrieben ist, muss nach den Tests wiederhergestellt werden.
Dies ermöglicht eine weitere korrekte Durchführung der Tests.



