\localauthor{Sebastian Vogt}
\todo{Des nummoi erklären mit den TXXX Angelegenheiten}
\subsection{Test Setup}
Vor Ausführung jeglicher Tests sollten folgende Voraussetzungen erfüllt sein:
\begin{itemize}
	\item Das System ist vollständig eingerichtet.
	Insbesondere sind die Datenbankschemata erstellt.
	\item Es existiert eine Administratorin mit folgenden Nutzerdaten:
	\begin{itemize}
		\item \emph{E-Mail Adresse}: mayer@fim.uni-dorfen.de
		\item \emph{Vorname}: Johanna
		\item \emph{Nachname}: Mayer
		\item \emph{Passwort}: UniDorfen1870!
	\end{itemize}
	\item Es existiert ein Nutzer mit folgenden Nutzerdaten:
	\begin{itemize}
		\item \emph{E-Mail Adresse}: maier@fim.uni-dorfen.de
		\item \emph{Vorname}: Franz
		\item \emph{Nachname}: Maier
		\item \emph{Passwort}: TSVDorfen2001!
	\end{itemize}
	\item Es existiert eine Nutzerin mit folgenden Nutzerdaten:
	\begin{itemize}
		\item \emph{E-Mail Adresse}: müller@fim.uni-dorfen.de
		\item \emph{Vorname}: Petra
		\item \emph{Nachname}: Müller
		\item \emph{Passwort}: TSVDorfen2002!
	\end{itemize}
	\item Es existiert ein Nutzer mit folgenden Nutzerdaten:
	\begin{itemize}
		\item \emph{E-Mail Adresse}: tuti@fim.uni-dorfen.de
		\item \emph{Vorname}: Tuti
		\item \emph{Nachname}: Aslan
		\item \emph{Passwort}: SupaDöner1970!
	\end{itemize}
\end{itemize}

\subsection{Administrator:innen}
Zuerst testen wir die Funktionen der Administrator:innen.
Die Einstellungen, die von der Administratorin getroffen werden, können so im Folgenden als Voraussetzungen genutzt werden.
\begin{description}
	\XXitem{/T010/}{t010} Die Administratorin meldet sich mit ihren Anmeldedaten im System an und wird zur Startseite weitergeleitet.
	Über die Kopfzeile ruft sie nun die Liste der wissenschaftlichen Foren auf.
	Von dort aus navigiert sie zur Seite zur Erstellung eines neuen wissenschaftlichen Forums.
	Dort erstellt sie ein Forum mit folgenden Daten:
	\begin{itemize}
		\item \emph{Editor:innen}:
		\item \emph{Deadline}:
		\item \emph{Kurzbeschreibung}:
		\item \emph{URL}:
		\item \emph{Anleitung zur Begutachtung}:
	\end{itemize}
\end{description}