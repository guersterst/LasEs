\localauthor{Sebastian Vogt}

Die Notation \texttt{/TXXX/} erlaubt eine spätere Referenzierung der einzelnen Tests in diesem und weiteren
Dokumenten. \texttt{XXX} entspricht dabei immer einer dreistelligen Zahl.
\subsection{Setup}\label{setup}
Vor Ausführung jeglicher Tests sollten folgende Voraussetzungen erfüllt sein:
\begin{itemize}
	\item Das System ist vollständig eingerichtet.
	Insbesondere sind die Datenbankschemata erstellt und die globalen Einstellungen sind getroffen.
	\item Es existiert eine Administratorin mit folgenden Nutzerdaten:
	\begin{itemize}
		\item \emph{E-Mail Adresse}: kirz@fim.uni-passau.de
		\item \emph{Vorname}: Johanna
		\item \emph{Nachname}: Mayer
		\item \emph{Passwort}: UniDorfen1870!
	\end{itemize}
	\item Es existiert ein Nutzer mit folgenden Nutzerdaten:
	\begin{itemize}
		\item \emph{E-Mail Adresse}: schicho@fim.uni-passau.de
		\item \emph{Vorname}: Franz
		\item \emph{Nachname}: Huber
		\item \emph{Passwort}: TSVDorfen2001!
	\end{itemize}
	\item Es existiert eine Nutzerin mit folgenden Nutzerdaten:
	\begin{itemize}
		\item \emph{E-Mail Adresse}: vogt@fim.uni-passau.de
		\item \emph{Vorname}: Petra
		\item \emph{Nachname}: Müller
		\item \emph{Passwort}: TSVDorfen2002!
	\end{itemize}
	\item Es existiert ein Nutzer mit folgenden Nutzerdaten:
	\begin{itemize}
		\item \emph{E-Mail Adresse}: guerster@fim.uni-passau.de
		\item \emph{Vorname}: Tuti
		\item \emph{Nachname}: Aslan
		\item \emph{Passwort}: SupaDöner1970!
	\end{itemize}
\end{itemize}
Die Tests werden in der angegebenen Reihenfolge ausgeführt.
Das heißt jeder Test kann die Zustandsänderungen, die durch vorherige Tests ausgelöst worden sind, als gegeben voraussetzen.
\subsection{Administratoren}
Zuerst testen wir die Funktionen der Administratoren.
Die Einstellungen, die von der Administratorin getroffen werden, können so im Folgenden als Voraussetzungen genutzt werden.
\begin{description}

	\XXitem{/T010/}{t010} \emph{Testet \hyperref[funkt:830]{/F830/}}.
	Die Administratorin meldet sich mit ihren Anmeldedaten im System an und wird zur Startseite weitergeleitet.
	Über die Kopfzeile ruft sie nun die Liste der wissenschaftlichen Foren auf.
	Von dort aus navigiert sie zur Seite zur Erstellung eines neuen wissenschaftlichen Forums.
	Dort erstellt sie ein Forum mit folgenden Daten:
	\begin{itemize}
		\item \emph{Editoren}: Nutzer mit E-Mail-Adresse guerster@fim.uni-passau.de
		\item \emph{Name}: Chemie Tagung
		\item \emph{Deadline}: 30.12.2099
		\item \emph{Kurzbeschreibung}: Es geht um Chemie.
		\item \emph{URL}: ch.em.ie
		\item \emph{Anleitung zur Begutachtung}: Begutachten Sie.
	\end{itemize}
	Danach wird sie auf die Seite des wissenschaftlichen Forums weitergeleitet.

	\XXitem{/T015/}{t015} \emph{Testet \hyperref[funkt:830]{/F830/}}.
	Die Administratorin ruft über die Kopfzeile die Liste der wissenschaftlichen Foren auf.
	Von dort aus navigiert sie zur Seite zur Erstellung eines neuen wissenschaftlichen Forums.
	Dort erstellt sie ein Forum mit folgenden Daten:
	\begin{itemize}
		\item \emph{Editor:innen}: Nutzer mit E-Mail-Adresse guerster@fim.uni-passau.de
		\item \emph{Name}: Physik Tagung
		\item \emph{Deadline}: 30.12.2099
		\item \emph{Kurzbeschreibung}: Es geht um Physik.
		\item \emph{URL}: ph.ys.ik
		\item \emph{Anleitung zur Begutachtung}: Begutachten Sie.
	\end{itemize}
	Danach wird sie auf die Seite des wissenschaftlichen Forums weitergeleitet.
	Zum Schluss meldet sie sich mit der Logout-Schaltfläche in der Navigationsleiste vom System ab.

\end{description}

\subsection{Angemeldeter Nutzer I}
\begin{description}

	\XXitem{/T020/}{t020} \emph{Testet \hyperref[funkt:160]{/F160/}}.
	Die Nutzerin mit der E-Mail-Adresse vogt@fim.uni-passau.de meldet sich im System an.
	Anschließend gibt sie im Suchfeld in der Kopfzeile ``Chemie Tagung'' ein und schickt die Suchanfrage mit Enter ab.
	Nun wird sie auf die Seite mit den Suchergebnissen weitergeleitet und das Forum namens ``Chemie Tagung'' ist der einzige Eintrag in der angezeigten Liste.
	Nach einem Klick auf diesen Eintrag wird die Nutzerin auf die Seite des wissenschaftlichen Forums weitergeleitet.

	\XXitem{/T030/}{t030} \emph{Testet \hyperref[funkt:400]{/F400/}}.
	Nun navigiert die Nutzerin per Mausklick auf die Seite für eine neue Einreichung.
	dort ist das Feld mit dem wissenschaftlichen Forum bereits richtig ausgefüllt, und zwar mit ``Chemie Tagung''

	\XXitem{/T040/}{t040} \emph{Testet \hyperref[funkt:420]{/F420/}}.
	Anschließend lädt die Nutzerin folgende PDF-Datei hoch: \href{https://dl.acm.org/doi/pdf/10.1145/3321707.3321795}{https://dl.acm.org/doi/pdf/10.1145/3321707.3321795}.

	\XXitem{/T045/}{t045} \emph{Testet \hyperref[funkt:450]{/F450/}}.
	Sie trägt in den Feldern des Formulars folgende Daten ein:
	\begin{itemize}
		\item \emph{Name der Einreichung}: Wichtiges Papier
		\item \emph{Co-Autoren}: Ein Co-Autor mit folgenden Daten:
		\begin{itemize}
			\item \emph{Vorname} Valentin
			\item \emph{Nachname} Kasper
			\item \emph{E-Mail-Adresse} garstenaue
		\end{itemize}
		\item \emph{Editor}: guerster@fim.uni-dorfen.de
	\end{itemize}
	Da die angegebene E-Mail-Adresse ``garstenaue'' nicht gültig ist, ist die Registrierung nicht erfolgreich.
	Sie bleibt auf der Registrierungsseite und wird mit einer Fehlermeldung über das Problem informiert. Die validen Daten bleiben jedoch erhalten.

	\XXitem{/T050/}{t050} \emph{Testet \hyperref[funkt:410]{/F410/}}.
	Sie bessert nun die E-Mail-Adresse aus: ``garstenaue@fim.uni-passau.de''

	\XXitem{/T060/}{t060} \emph{Testet \hyperref[funkt:460]{/F460/}}.
	Nach erfolgreicher Absendung des Formulars wird die Nutzerin auf die Übersichtsseite der Einreichung weitergeleitet. Frau Müller ist jetzt fertig mit ihrer Arbeit und führt mit dem zugehörigen Link in der Kopfzeile den Logout durch.
	Sie befindet sich nun wieder auf den Anmeldeseite.
\end{description}

\subsection{Editor I}
\begin{description}

	\XXitem{/T080/}{t080} \emph{Testet \hyperref[funkt:680]{/F680/}}.
	Der Nutzer mit der E-Mail-Adresse guerster@fim.uni-passau.de meldet sich im System an.
	Von der Startseite aus ruft er die Einreichung ``Wichtiges Papier'' auf und landet auf der Seite dieser Einreichung.
	Er gibt in das Formular zur Zuweisung von Gutachtern ``schicho@fim.uni-passau.de'' ein und schickt das Formular ab.
	Es wird eine Rückmeldung für das erfolgreiche Hinzufügen eines Gutachters angezeigt.
	Anschließend meldet er sich ab.

\end{description}

\subsection{Gutachter}
\begin{description}

	\XXitem{/T085/}{t085} \emph{Testet \hyperref[funkt:690]{/F690/}}.
	schicho@fim.uni-passau.de hat eine E-Mail mit folgendem Inhalt erhalten:
	\begin{itemize}
		\item alle relevanten Informationen zur Einreichung
		\item ein Link zur Annahme der Begutachtungsanfrage
		\item ein Link zur Ablehnung der Begutachtungsanfrage.
	\end{itemize}
	Er nutzt den Link zur Annahme der Begutachtungsanfrage und ist nun Gutachter.

	\XXitem{/T090/}{t090} \emph{Testet \hyperref[funkt:540]{/F540/}}.
	Der Nutzer mit der E-Mail-Adresse schicho@fim.uni-passau.de meldet sich im System an.
	Von der Startseite aus ruft er die Einreichung ``Wichtiges Papier'' auf und landet auf der Seite dieser Einreichung.
	Er nutzt das angezeigte Formular um eine einseitige PDF-Datei namens \emph{gutachten.pdf} als Gutachten hochzuladen.
	Dann meldet er sich wieder ab mit der Logout Funktionalität.

\end{description}

\subsection{Editor II}
\begin{description}

	\XXitem{/T100/}{t100} \emph{Testet \hyperref[funkt:685]{/F685/}}.
	Der Nutzer guerster@fim.uni-passau.de meldet sich wie oben beschrieben an und navigiert wie oben beschrieben zur Seite der Einreichung ``Wichtiges Papier''.
	Dort sieht er ein Gutachten von schicho@fim.uni-passau.de.
	Er betätigt die Schaltfläche zur Freigabe dieses Gutachtens. Eine Bestätigung über die Freigabe des Gutachtens wird angezeigt.
	Dann meldet er sich wieder ab.

\end{description}

\subsection{Anonyme Nutzer}

\begin{description}
	\XXitem{/T200/}{t200} \emph{Testet \hyperref[funkt:010]{/F010/}}. Valentin Kasper aus \hyperref[t050]{/T050/} ist noch nicht im System registriert.
	Er ruft LasEs auf und wird zur Anmeldeseite weitergeleitet.
	\XXitem{/T210/}{t210} \emph{Testet \hyperref[funkt:060]{/F060/}}. Er klickt auf den Link zur Registrierung und gibt seinen Namen, seine E-Mail-Adresse und das Passwort \texttt{einsZwei3!5678} an.

	Die Registrierung wird bestätigt und er erhält eine Verifizierungs-E-Mail.
	\XXitem{/T220/}{t220} \emph{Testet \hyperref[funkt:070]{/F070/}}. Er klickt auf den Bestätigungslink in der E-Mail und wird auf die Verifizierungsseite weitergeleitet.
	Damit ist sein Profil erstellt.
	Er wird automatisch auf die Homepage weitergeleitet.
	\XXitem{/T230/}{t230} \emph{Testet \hyperref[funkt:260]{/F260/}}. Da er als Ko-Autor in Test \hyperref[t050]{/T050/} eingetragen wurde, wird ihm die Einreichung auf der Homepage angezeigt. Der Nutzer meldet sich ab.
\end{description}

\subsection{Angemeldeter Nutzer II}

\begin{description}

	\XXitem{/T110/}{t110} \emph{Testet \hyperref[funkt:480]{/F480/}}.
	Nun meldet sich die Nutzerin vogt@fim.uni-passau.de an und navigiert zur Seite der Einreichung ``Wichtiges Papier''.
	Dort ist das Gutachten von schicho@fim.uni-passau.de sichtbar.
	Nach Betätigung der Download Schaltfläche wird die PDF-Datei \emph{gutachten.pdf} vom Browser heruntergeladen.

	\XXitem{/T120/}{t120} \emph{Testet \hyperref[funkt:230]{/F230/}}.
	Auf der Navigationsleiste klickt die Nutzerin nun den Link, der zum Profil führt.
	Die Profil-Seite wird angezeigt.
	Die Nutzerin betätigt nun die Schaltfläche zur Löschung des Kontos.
	Daraufhin wird eine Warnung angezeigt, dass dies sowohl alle Einreichungen löscht, als auch die Editoren und Gutachter der eingereichten Paper per E-Mail benachrichtigt.
	Die Nutzerin akzeptiert diese Nachricht und wird auf die Anmeldeseite weitergeleitet.
	Der Nutzer schicho@fim.uni-passau.de hat nun eine E-Mail über die Löschung des Nutzers erhalten.
	Die Nutzerin vogt@fim.uni-passau.de meldet sich wieder ab.
	Die Administratorin meldet sich nun im System an, navigiert über die Navigationsleiste zur Liste der wissenschaftlichen Foren und von dort auf auf ``Chemie Tagung''. Dort stellt sie fest, dass keine Paper existieren.

\end{description}

\subsection{Reset}
Der Programmzustand, der im Abschnitt \emph{\hyperref[setup]{Setup}} beschrieben ist, muss nach den Tests wiederhergestellt werden.
Dies ermöglicht eine weitere korrekte Durchführung der Tests.



