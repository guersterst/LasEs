\localauthor{Sebastian Vogt}
\todo{Des nummoi erklären mit den TXXX Angelegenheiten}
\subsection{Test Setup}
Vor Ausführung jeglicher Tests sollten folgende Voraussetzungen erfüllt sein:
\begin{itemize}
	\item Das System ist vollständig eingerichtet.
	Insbesondere sind die Datenbankschemata erstellt und die globalen Einstellungen sind getroffen.
	\item Es existiert eine Administratorin mit folgenden Nutzerdaten:
	\begin{itemize}
		\item \emph{E-Mail Adresse}: kirz@fim.uni-passau.de
		\item \emph{Vorname}: Johanna
		\item \emph{Nachname}: Mayer
		\item \emph{Passwort}: UniDorfen1870!
	\end{itemize}
	\item Es existiert ein Nutzer mit folgenden Nutzerdaten:
	\begin{itemize}
		\item \emph{E-Mail Adresse}: schicho@fim.uni-passau.de
		\item \emph{Vorname}: Franz
		\item \emph{Nachname}: Huber
		\item \emph{Passwort}: TSVDorfen2001!
	\end{itemize}
	\item Es existiert eine Nutzerin mit folgenden Nutzerdaten:
	\begin{itemize}
		\item \emph{E-Mail Adresse}: vogt@fim.uni-passau.de
		\item \emph{Vorname}: Petra
		\item \emph{Nachname}: Müller
		\item \emph{Passwort}: TSVDorfen2002!
	\end{itemize}
	\item Es existiert ein Nutzer mit folgenden Nutzerdaten:
	\begin{itemize}
		\item \emph{E-Mail Adresse}: guerster@fim.uni-passau.de
		\item \emph{Vorname}: Tuti
		\item \emph{Nachname}: Aslan
		\item \emph{Passwort}: SupaDöner1970!
	\end{itemize}
\end{itemize}
Die Tests werden in der angegebenen Reihenfolge ausgeführt.
Das heißt jeder Test kann die Zustandsänderungen, die durch vorherige Tests ausgelöst worden sind, als gegeben voraussetzen.
\subsection{Administratoren}
Zuerst testen wir die Funktionen der Administratoren.
Die Einstellungen, die von der Administratorin getroffen werden, können so im Folgenden als Voraussetzungen genutzt werden.
\begin{description}

	\XXitem{/T010/}{t010} \emph{Testet \hyperref[funkt:830]{/F830/}}.
	Die Administratorin meldet sich mit ihren Anmeldedaten im System an und wird zur Startseite weitergeleitet.
	Über die Kopfzeile ruft sie nun die Liste der wissenschaftlichen Foren auf.
	Von dort aus navigiert sie zur Seite zur Erstellung eines neuen wissenschaftlichen Forums.
	Dort erstellt sie ein Forum mit folgenden Daten:
	\begin{itemize}
		\item \emph{Editor:innen}: Nutzer mit E-Mail-Adresse guerster@fim.uni-passau.de
		\item \emph{Name}: Chemie Tagung
		\item \emph{Deadline}: 30.12.2022
		\item \emph{Kurzbeschreibung}: Es geht um Chemie.
		\item \emph{URL}: ch.em.ie
		\item \emph{Anleitung zur Begutachtung}: Begutachten Sie.
	\end{itemize}
	Danach wird sie auf die Seite des wissenschaftlichen Forums weitergeleitet.

	\XXitem{/T015/}{t015} \emph{Testet \hyperref[funkt:830]{/F830/}}.
	Die Administratorin ruft über die Kopfzeile die Liste der wissenschaftlichen Foren auf.
	Von dort aus navigiert sie zur Seite zur Erstellung eines neuen wissenschaftlichen Forums.
	Dort erstellt sie ein Forum mit folgenden Daten:
	\begin{itemize}
		\item \emph{Editor:innen}: Nutzer mit E-Mail-Adresse guerster@fim.uni-passau.de
		\item \emph{Name}: Physik Tagung
		\item \emph{Deadline}: 30.12.2099
		\item \emph{Kurzbeschreibung}: Es geht um Physik.
		\item \emph{URL}: ph.ys.ik
		\item \emph{Anleitung zur Begutachtung}: Begutachten Sie.
	\end{itemize}
	Danach wird sie auf die Seite des wissenschaftlichen Forums weitergeleitet.

\end{description}

\subsection{Angemeldeter Nutzer}
\begin{description}

	\XXitem{/T020/}{t020} \emph{Testet \hyperref[funkt:160]{/F160/}}.
	Die Nutzerin mit der E-Mail-Adresse vogt@fim.uni-passau.de meldet sich im System an.
	Anschließend gibt sie im Suchfeld in der Kopfzeile ``Chemie Tagung'' ein und schickt die Suchanfrage mit Enter ab.
	Nun wird sie auf die Seite mit den Suchergebnissen weitergeleitet und das Forum namens ``Chemie Tagung'' ist der einzige Eintrag in der angezeigten Liste.
	Nach einem Klick auf diesen Eintrag wird die Nutzerin auf die Seite des wissenschaftlichen Forums weitergeleitet.

	\XXitem{/T030/}{t030} \emph{Testet \hyperref[funkt:400]{/F400/}}.
	Nun navigiert die Nutzerin per Mausklick auf die Seite für eine neue Einreichung.
	dort ist das Feld mit dem wissenschaftlichen Forum bereits richtig ausgefüllt, und zwar mit ``Chemie Tagung''

	\XXitem{/T040/}{t040} \emph{Testet \hyperref[funkt:420]{/F420/}}.
	Anschließend lädt die Nutzerin folgende PDF-Datei hoch: \href{https://dl.acm.org/doi/pdf/10.1145/3321707.3321795}{https://dl.acm.org/doi/pdf/10.1145/3321707.3321795}.

	\XXitem{/T045/}{t045} \emph{Testet \hyperref[funkt:450]{/F450/}}.
	Sie trägt in den Feldern des Formulars folgende Daten ein:
	\begin{itemize}
		\item \emph{Name der Einreichung}: Wichtiges Papier
		\item \emph{Co-Autoren}: Ein Co-Autor mit folgenden Daten:
		\begin{itemize}
			\item \emph{Vorname} Valentin
			\item \emph{Nachname} Kasper
			\item \emph{E-Mail-Adresse} garstenaue
		\end{itemize}
		\item \emph{Editor}: guerster@fim.uni-dorfen.de
	\end{itemize}
	Da die angegebene E-Mail-Adresse nicht gültig ist, ist die Registrierung nicht erfolgreich.
	Sie bleibt auf der Registrierungsseite und wird mit einer Fehlermeldung über das Problem informiert.

	\XXitem{/T050/}{t050} \emph{Testet \hyperref[funkt:410]{/F410/}}.
	Sie trägt wieder die gleichen Daten in das Formular ein wie in Test \hyperref[t050]{/T050/}, nur diesmal mit der E-Mail-Adresse \texttt{garstenaue@fim.uni-passau.de}

	\XXitem{/T060/}{t060} \emph{Testet \hyperref[funkt:460]{/F460/}}.
	Nach erfolgreicher Absendung des Formulars wird die Nutzerin auf die Übersichtsseite der Einreichung weitergeleitet.

	\XXitem{/T070/}{t070} \emph{Testet \hyperref[funkt:460]{/F460/}}.
	Frau Müller ist jetzt fertig mit ihrer Arbeit und führt mit dem zugehörigen Link in der Kopfzeile den Logout durch.
	Sie befindet sich nun wieder auf den Anmeldeseite.
\end{description}

\subsection{Editor}
\begin{description}

	\XXitem{/T080/}{t080} \emph{Testet \hyperref[funkt:680]{/F680/}}.
	Der Nutzer mit der E-Mail-Adresse guerster@fim.uni-passau.de meldet sich im System an.
	Von der Startseite aus ruft er die Einreichung ``Wichtiges Papier'' auf und landet auf der Seite dieser Einreichung.
	Er gibt in das Formular zur Zuweisung von Gutachtern ``schicho@fim.uni-passau.de'' ein und schickt das Formular ab.
	Anschließend meldet er sich ab.

	\todo{Evenutuell F690, aso die Versendung der Email noch testen...}

	\todo{F700 noch testen, aber nicht hier, sondern nach den Sachen vong Gutachter}

\end{description}

\subsection{Gutachter}
\begin{description}

	\XXitem{/T090/}{t090} \emph{Testet \hyperref[]{}}.
	Der Nutzer mit der E-Mailc-Adresse schicho@fim.uni-passau.de meldet sich im System an.
	Von der Startseite aus ruft er die Einreichung ``Wichtiges Papier'' auf und landet auf der Seite dieser Einreichung.

\end{description}

\subsection{Anonyme Nutzer}

\begin{description}
	\XXitem{/T200/}{t200} \emph{Testet \hyperref[funkt:010]{/F010/}}. Valentin Kasper aus \hyperref[t050]{/T050/} ist noch nicht im System registriert.
	Er ruft LasEs auf und wird zur Anmeldeseite weitergeleitet.
	\XXitem{/T210/}{t210} \emph{Testet \hyperref[funkt:060]{/F060/}}. Er klickt auf den Link zur Registrierung und gibt seinen Namen, seine E-Mail-Adresse und das Passwort \texttt{einsZwei3!5678} an.

	Die Registrierung wird bestätigt und er erhält eine Verifizierungs-E-Mail.
	\XXitem{/T220/}{t220} \emph{Testet \hyperref[funkt:070]{/F070/}}. Er klickt auf den Bestätigungslink in der E-Mail und wird auf die Verifizierungsseite weitergeleitet.
	Damit ist sein Profil erstellt.
	Er wird automatisch auf die Homepage weitergeleitet.
	\XXitem{/T230/}{t230} \emph{Testet \hyperref[funkt:260]{/F260/}}. Da er als Ko-Autor in Test \hyperref[t050]{/T050/} eingetragen wurde, wird ihm das Paper auf der Homepage angezeigt.
\end{description}

