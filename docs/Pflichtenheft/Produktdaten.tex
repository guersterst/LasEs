\localauthor{Sebastian Vogt}

Die Notation \texttt{/DXXX/} erlaubt eine spätere Referenzierung der einzelnen Produktdaten in diesem und weiteren
Dokumenten. \texttt{/DXXX/} steht dabei für ein Musskriterium, \texttt{/DWXXX/} für ein Wunschkriterium. \texttt{XXX} entspricht dabei immer einer dreistelligen Zahl.

\begin{description}

	\XXitem{/D010/}{d010} Für jeden \emph{Nutzer} sind folgende Informationen gespeichert:
	\begin{itemize}
		\item \hyperref[produktfunktionen]{Nutzerrolle}
		\item Titel
		\item Vorname
		\item Nachname
		\item E-Mail-Adresse
		\item Menge aller \hyperref[d025]{Einreichungen}
	\end{itemize}

	\XXitem{/DW015/}{d015} Für jeden Nutzer wird darüber hinaus gespeichert:
	\begin{itemize}
		\item Avatarbild
		\item Arbeitgeber
		\item seine \hyperref[d035]{Fachgebiete}
		\item Geburtsdatum
	\end{itemize}
	Siehe \hyperref[funkt:220]{/FW220/}, \hyperref[funkt:240]{/FW240/}.

	\XXitem{/D020/}{d020} Für jedes \emph{Paper} sind folgende Informationen abgespeichert:
	\begin{itemize}
		\item der Titel des Papers
		\item Titel, Vorname, Nachname und E-Mail-Adresse des Autors und jedes Ko-Autors
		\item die zugehörige PDF-Datei
		\item eine Versionsnummer
	\end{itemize}

	\XXitem{/D025/}{d025} Für jede \emph{Einreichung} wird gespeichert:
	\begin{itemize}
		\item das zugehörige \hyperref[d020]{Paper}
		\item der Zeitpunkt der Einreichung
		\item das zugehörige \hyperref[d030]{wissenschaftliche Forum}
		\item der \hyperref[funkt:editor]{Editor} der Einreichung
		\item der Status der Einreichung (\emph{schwarz}: Eingereicht, \emph{gelb}: Revision verlangt, \emph{rot}: Abgelehnt, \emph{grün}: Angenommen)
		\item Titel, Vorname, Nachname und E-Mail-Adresse jedes Gutachters
		\item die Abgabefrist für \hyperref[d040]{Gutachten}
		\item die abgegebenen \hyperref[d040]{Gutachten}
		\item eine Deadline für Gutachten
	\end{itemize}

	\XXitem{/DW021/}{d021} Für jede \emph{Einreichung} sind zusätzlich folgende Informationen abgespeichert:
	\begin{itemize}
		\item Frist für das Einreichen einer erneuten Revision
		\item alle Revisionen (in Form von \hyperref[d020]{Papers})
		\item die Information, ob diese Revisionen bereits für die Gutachter freigeschaltet sind
	\end{itemize}

	\XXitem{/DW022/}{d022} Eine \emph{Einreichung} speichert zusätzlich Titel, Vorname, Nachname und E-Mail-Adresse jedes Nutzers, der als Gutachter vorgeschlagen ist.
	Siehe \hyperref[funkt:430]{/FW430/}

	\XXitem{/D030/}{d030} Für jedes \emph{wissenschaftliche Forum} ist folgendes gespeichert:
	\begin{itemize}
		\item der Name
		\item Titel, Vorname, Nachname und E-Mail-Adresse jedes Editoren
		\item die Deadline für \hyperref[d025]{Einreichungen}
		\item eine Kurzbeschreibung
		\item eine URL zur Website des Forums
		\item eine Anleitung zur Begutachtung
		\item das \hyperref[d035]{Fachgebiet} des wissenschaftlichen Forums
	\end{itemize}

	\XXitem{/D035/}{d035} Es wird eine Liste mit allen im System zulässigen \emph{Fachgebieten} gespeichert.

	\XXitem{/D040/}{d040} Für jedes \emph{Gutachten} wird gespeichert:
	\begin{itemize}
		\item der Inhalt des Gutachtens als PDF
		\item ein Kommentar als Freitext
		\item Titel, Vorname, Nachname und E-Mail-Adresse des verfassenden \hyperref[funkt:Gutachter]{Gutachters}
		\item die zugehörige \hyperref[d025]{Einreichung}
		\item die Information, ob das Gutachten zur Ansicht für den Einreichenden freigeschaltet ist
		\item die Ja/Nein Empfehlung des Gutachters, ob die Einreichung angenommen werden soll
	\end{itemize}

	\XXitem{/D050/}{d050} Systemweit ist gespeichert:
	\begin{itemize}
		\item das Logo der betreibenden Einrichtung
		\item der Name der betreibenden Einrichtung
		\item das Farbschema der Kopf- und Fußzeile
		\item das Impressum
	\end{itemize}

	\XXitem{/DW051/}{d051} Die gespeicherten Informationen aus \hyperref[d050]{/D050/} sind für jedes \hyperref[d030]{wissenschaftliche Forum} separat gespeichert.
\end{description}