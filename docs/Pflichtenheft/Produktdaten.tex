\localauthor{Sebastian Vogt}

Die Notation \texttt{/DXXX/} erlaubt eine spätere Referenzierung der einzelnen Produktdaten in diesem und weiteren
Dokumenten. \texttt{/DXXX/} steht dabei für ein Musskriterium, \texttt{/DWXXX/} für ein Wunschkriterium. \texttt{XXX} entspricht dabei immer einer dreistelligen Zahl.

\begin{description}
	\XXitem{/D010/}{d010} Für jeden \emph{Nutzer} sind folgende Informationen gespeichert: \hyperref[funkt:nutzer]{Nutzerrolle}, Titel, Vorname, Nachname, E-Mail Adresse, sowie die Menge aller \hyperref[d025]{Einreichungen}.

	\XXitem{/D020/}{d020} Für jedes \emph{Manuskript} sind folgende Informationen abgespeichert: Der Titel, die Namen und E-Mail Adressen der Co-Autoren und die zugehörige PDF Datei.

	\XXitem{/D025}{d025} Für jede \emph{Einreichung} wird das zugehörige \hyperref[d020]{Manuskript}, der Zeitpunkt der Einreichung, das zugehörige \hyperref[d030]{wissenschaftliche Forum}, der \hyperref[funkt:editor]{Editor} der Einreichung, der Status der Einreichung (\emph{schwarz}: Eingereicht, \emph{gelb}: Revision verlangt, \emph{rot}: Abgelehnt, \emph{grün}: Angenommen), die Gutachter, die Abgabefrist für \hyperref[d040]{Gutachten} und die abgegebenen \hyperref[d040]{Gutachten} gespeichert.

	\XXitem{/DW021}{d021} Für jede \emph{Einreichung} sind zusätzlich folgende Informationen abgespeichert: Frist für das Einreichen einer erneuten Revision, alle Revisionen (in Form von \hyperref[d020]{Manuskripten}) und die Information, ob diese Revisionen bereits für die Gutachter freigeschaltet sind.

	\XXitem{/DW022}{d022} Ein \emph{Einreichung} speichert zusätzlich, welche Nutzer als Gutacher:innen vorgeschlagen sind. Diese Werden mit Vorname, Nachname und E-Mail Adresse gespeichert, da sie nicht als Nutzer in der Datenbank existieren müssen.

	\XXitem{/D030/}{d030} Für jedes \emph{wissenschaftliche Forum} sind die zugehörigen Editor:innen, die Deadline für \hyperref[d025]{Einreichungen}, eine Kurzbeschreibung, Eine URL zur Website des Forums und eine Anleitung zur Begutachtung gespeichert.

	\XXitem{/D040/}{d040} Für jedes \emph{Gutachten} wird der Inhalt des Gutachtens als PDF gespeichert, ein Kommentar als Freitext und zusätzlich noch der zugehörige \hyperref[funkt:Gutachter]{Gutachter} und die zugehörige \hyperref[d025]{Einreichung}.

	\XXitem{/D050/}{d050} Systemweit ist das Logo und der Name der betreibenden Einrichtung sowie das Look and Feel der Applikation gespeichert.

	\XXitem{/DW051}{d051} Die gespeicherten Informationen aus \hyperref[d050]{/D050/} sind für jedes \hyperref[d030]{wissenschaftliche Forum} separat gespeichert.
\end{description}