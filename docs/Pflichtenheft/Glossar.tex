\localauthor{Stefanie Gürster}

\begin{description}
	\XXitem{Administrator}{glo:admin} Ein Administrator ist der Betreiber und Verwalter von LasEs. Dieser kann neue \hyperref[glo:wissForum]{Wissenschaftliche Foren} erstellen und Nutzer entfernen oder hinzufügen. Ein Administrator besitzt allumfassende Rechte.

	\XXitem{Anonymer Nutzer}{glo:anon} Anonyme Nutzende sind nicht eingeloggte oder registrierte Websitenbesucher. Sie haben keinen Zugriff auf systeminterne Daten und können nur die Anmeldungs- oder Registrierungsseite sehen.

	\XXitem{Build Tool}{glo:buildtool} \href{https://maven.apache.org/what-is-maven.html}{Apache Maven} ist ein Build System für Java Anwendungen. Es erlaubt die einfache Einbindung von weiteren Softwarebibliotheken und übernimmt den Bau eines \hyperref[glo:war]{\texttt{war}} Archivs.

	\XXitem{Client}{glo:client} Rechner eines Webseitenbenutzers.

	\XXitem{CPU}{glo:cpu} \emph{Central Processing Unit} Zentrale Recheneinheit des Prozessors der Rechenbefehle ausführt.

	\XXitem{Editor}{glo:editor} Ein Editor ist einem wissenschaftlichen Forum zugewiesen und verwaltet Einreichungen.

	\XXitem{GitLab}{glo:gitlab} \href{https://fimgit.fim.uni-passau.de/users/sign_in}{GitLab} ist ein zentraler Speicherplatz für alle Entwickler. Darüber kann die Zusammenfügung einzelner Codestücke verwaltet werden.

	\XXitem{Gutachter}{glo:gutachter} Gutachter:innen können anonym die Paper anderer Wissenschaftler \emph{peer-reviewen} und Änderungen verlangen oder es als gut befinden.

	\XXitem{HTTPS}{glo:https} Protokoll zur verschlüsselten Datenübertragung über das Internet.

	\XXitem{IBM RSA}{glo:rsa} Der \emph{IBM Rational Software Architect} erlaubt \emph{Round-Trip Engineering}. Damit können gleichzeitig zur Programmierung auch die aus dem Code hervorgehenden Diagramme erstellt werden.

	\XXitem{Inkscape}{glo:inkscape} Inkscape ist ein Vektorgrafikbearbeitungsprogramm. Vektorgrafiken haben den Vorteil bei nahem \emph{heranzoomen} nicht unscharf zu werden.

	\XXitem{In-Memory Datenbank}{glo:ramdb} Zur vereinfachten Entwicklung wird während der Entwicklungsphase nicht eine echte Datenbank mit hoher Latenzzeit verwendet, sondern eine lokale Arbeitsspeicherdatenbank.

	\XXitem{IDE}{glo:ide} \emph{Integrated Development Environment} Programm, in der die Webanwendung programmiert wird und bei der Entwicklungsarbeit unterstützt.

	\XXitem{JDK}{glo:jdk} \href{https://www.oracle.com/java/technologies/downloads/}{\emph{Java Development Kit}} Komplette Softwarebibliothek der Java Programmiersprache. Enthält die Grundbausteine der Anwendung.

	\XXitem{Journal}{glo:journal} Zu einem Journal kann ein Wissenschaftler ein Manuskript in Form eines PDFs abgeben. Ein Journal hat keine Deadline zur Abgabe.

	\XXitem{JSF}{glo:jsf} \href{https://jakarta.ee/specifications/faces/}{\emph{Jakarta Server Faces}} ist das Grundgerüst von LasEs. Es erlaubt die Erstellung von Webanwendungen in der Programmiersprache Java.

	\XXitem{Konferenz}{glo:konf} Zu einer Konferenz kann ein Wissenschaftler ein Manuskript in Form eines PDFs abgeben. Eine Konferenz hat eine Deadline zur Abgabe.

	\XXitem{\LaTeX}{glo:latex} \href{https://www.latex-project.org/}{Latex} ist das Textsatzsystem zum Verfassen der Dokumente. Es ermöglicht die parallele Bearbeitung von Textdokumenten.

	\XXitem{RAM}{glo:ram} \emph{Random Access Memory} Arbeitsspeicher eines Computers. Hier sind Daten gespeichert die ein \hyperref[glo:cpu]{CPU} während der Befehlsabarbeitung benötigt.

	\XXitem{Registrierter Nutzer}{glo:regnutzer} Ein Nutzer, welcher ein Nutzerkonto erstellt hat und dieses per E-Mail verifiziert hat. Der authentifizierte Nutzer kann Papers einreichen.

	\XXitem{Pagination}{glo:pagination} Eine lange Liste mit mehr als 25 Einträgen wird auf mehrere Seiten aufgeteilt.

	\XXitem{Server}{glo:server} Rechner, auf welcher die Webanwendung ausgeführt wird und die Datenbank gespeichert ist.

	\XXitem{Submission}{glo:submission} Eine \emph{Einreichung} ist ein Manuskripts, welches auf den Datenbankserver durch den veröffentlichenden Wissenschaftler hochgeladen wird. Anschließend können Gutachter dieses Manuskript reviewen.

	\XXitem{war}{glo:war} \emph{Web Application Resource} oder \emph{Web Archive} Archivdateiformat. Bündelt die Anwendung in eine einzige Datei, die damit leicht installierbar ist.

	\XXitem{Mailto-Link}{glo:mailto} Hyperlinks über die Benutzer eine E-Mail an eine vorgegebene E-Mail-Adresse senden können, ohne diese zuvor in einem E-Mail-Programm eingeben zu müssen. Diese Mail kann eine vorgefertigte Nachricht enthalten.

	\XXitem{Wissenschaftliches Forum}{glo:wissForum} Überbegriff für \hyperref[glo:journal]{Journale} und \hyperref[glo:konf]{Konferenzen}.

	\XXitem{yEd}{glo:yed} \href{https://www.yworks.com/products/yed}{yEd} ist ein Bearbeitungsprogramm zum Erstellen von Graphen und Diagrammen.
\end{description}