\localauthor{Johann Schicho}

Durch die Verwendung von Java ist die serverseitige Ausführung von LasEs
grundsätzlich plattformunabhängig. Auch sind Java EE Application Server für verschiedene Plattformen verfügbar.
Die Anwenderseite setzt nur einen modernen Webbrowser voraus.

\subsection{Hardware}

\begin{itemize}
	\item \textbf{Client:} Computer (PC oder Laptop) mit Internetanschluss, um darauf einen modernen Webbrowser zu verwenden. Die Mindestauflösung des Bildschirms muss $1280 \times 720$ Pixel betragen.

	\item \textbf{Server:} Rechner mit Internetanschluss, um darauf Webserver und Datenbankserver laufen zu lassen. Datenbankserver und Webserver können auch auf zwei unterschiedlichen Rechnern ausgeführt werden.

	\phantomsection
	\label{dbspezi}
	Referenzsystem für den Datenbankserver ist der FIM Rechner \texttt{bueno}.\\
	\texttt{\textbf{bueno}} führt PostgreSQL 12.x aus.

	\phantomsection
	\label{spezi}

	Referenzsystem für den Webserver ist der FIM CIP Pool Rechner \texttt{ds9}.\\
	\texttt{\textbf{ds9}} hat folgende Systemspezifikationen:

	\begin{itemize}
		\XXitem{CPU:}{spez:cpu} Intel Core i7-4790 @ 3.60GHz x 8

		\XXitem{RAM:}{spez:ram} 16 GiB

		\XXitem{Festplattenkapazität:}{spez:rom} 256 GiB

		\XXitem{Systemarchitektur:}{spez:arch} 64-bit

		\XXitem{Betriebssystem}{spez:os} Debian GNU/Linux 11 (bullseye)
	\end{itemize}


\end{itemize}

\subsection{Software}

\begin{itemize}

	\item \textbf{Client:} Betriebssystem (Windows, MacOS, Linux, etc.) und ein installierter Webbrowser.

	\begin{itemize}
		\item Google Chrome $95.0$
		\item Mozilla Firefox $93.0$
		\item Microsoft Edge $95.0$
	\end{itemize}

	Da der HTML 5 Standard eingehalten wird, sollte auch mit allen anderen Browsern, die diesen Standard unterstützen, LasEs problemlos verwendet werden können.

	\item \textbf{Datenbankserver:} Betriebssystem (Windows, Linux, etc.) mit folgenden weiteren Voraussetzungen:

	\begin{itemize}
		\item PostgreSQL 12.x SQL Datenbank Server
	\end{itemize}

	\item \textbf{Anwendungsserver:} Betriebssystem (Windows, Linux, etc.) mit folgenden weiteren Voraussetzungen:

	\begin{itemize}
		\item JDK $16.0.2$ Installation
		\item Apache Tomcat $10.0.10$ Applicationserver
	\end{itemize}

\end{itemize}

	Datenbankserver und Anwendungsserver können bei ausreichend Rechnenleistung auf dem selben Rechner ausgeführt werden. Die Performance kann dadurch eingeschränkt sein. Dazu sind dann beide Server-Softwarevoraussetzungen auf einem System zu installieren.\\
	Getestet wird mit einem Setup, in dem Datenbankserver und Anwendungsserver getrennt sind.

\subsection{Orgware}

\begin{itemize}
	\item Installation der Softwarevoraussetzungen

	\item Konfiguration der Anwendung (Erstmaliges Starten der Anwendung, Verbindung mit PostgreSQL, Erstellung des Datenbank Schemata)

	\item Internetanschluss mit einer Bandbreite von 1 GBit/s für den Webserver, der über das öffentliche \emph{freie} Internet zugänglich ist und genauso schneller Internetanschluss oder lokale Netzwerkverbindung zu dem Datenbankserver.

	\item Verschlüsselte Kommunikation über HTTPS. Verwendung einer statischen IP-Adresse und eines vertrauenswürdigen TLS Zertifikats einer Zertifizierungsstelle.

	\item SMTP E-Mail-Server mit E-Mail-Konto zur Versendung automatisierter Benachrichtigung.

	\item Internetverbindung der Clients mit einer Bandbreite von mindestens 1 MBit/s zum Aufrufen der Webanwendung.

	\item E-Mail Konto mit E-Mail-Client auf Rechner des Benutzers um E-Mails zu anderen Benutzern versenden und empfangen zu können.
\end{itemize}

