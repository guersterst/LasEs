\localauthor{Johann Schicho}

Durch die Verwendung von Java ist die serverseitige Ausführung von LasEs
grundsätzlich plattformunabhängig.
Die Verwendung von Anwenderseite setzt nur einen modernen Webbrowser voraus.

\subsection{Hardware}

\begin{itemize}
	\item \textbf{Client:} Computer (PC oder Laptop) mit Internetanschluss, um darauf einen Webbrowser zu verwenden.

	\item \textbf{Server:} Rechner mit Internetanschluss, um darauf Webserver und Datenbankserver laufen zu lassen. Datenbankserver und Webserver können auch auf zwei unterschiedlichen Rechnern ausgeführt werden.

	\phantomsection
	\label{dbspezi}
	Referenzsystem für den Datenbankserver ist der FIM Rechner \texttt{bueno}.

	\texttt{\textbf{bueno}} führt PostgreSQL 12.x aus.

	\phantomsection
	\label{spezi}

	Referenzsystem für den Webserver ist der FIM CIP Pool Rechner \texttt{ds9}.

	\texttt{\textbf{ds9}} hat folgende Systemspezifikationen:

	\begin{itemize}
		\XXitem{CPU:}{spez:cpu} Intel Core i7-4790 @ 3.60GHz x 8

		\XXitem{RAM:}{spez:ram} 16 GiB

		\XXitem{Festplattenkapazität:}{spez:rom} 256 GiB

		\XXitem{Systemarchitektur:}{spez:arch} 64-bit

		\XXitem{Betriebssystem}{spez:os} Debian GNU/Linux 11 (bullseye)
	\end{itemize}


\end{itemize}

\subsection{Software}

\begin{itemize}

	\item \textbf{Client:} Betriebssystem (Windows, MacOS, Linux, etc.) und ein installierter Webbrowser.

	\begin{itemize}
		\item Google Chrome
		\item Mozilla Firefox
		\item Microsoft Edge
	\end{itemize}

	\item \textbf{Datenbankserver:} Betriebssystem (Windows, Linux, etc.) mit folgenden weiteren Voraussetzungen:

	\begin{itemize}
		\item PostgreSQL 12.x SQL Datenbank Server
	\end{itemize}

	\item \textbf{Anwendungsserver:} Betriebssystem (Windows, Linux, etc.) mit folgenden weiteren Voraussetzungen:

	\begin{itemize}
		\item JDK 16 Installation
		\item JSF Referenzimplementation Mojarra $3.0.1$ (mitgeliefert in der Anwendung)
		\item CDI Framework Weld $4.0.2$ (mitgeliefert in der Anwendung)
		\item JDBC (mitgeliefert in der Anwendung)
		\item Apache Tomcat $10.0.x$ HTTP Webserver
	\end{itemize}

\end{itemize}

	Datenbankserver und Anwendungsserver können auf dem selben Rechner ausgeführt werden. Dazu sind dann beide Server-Softwarevoraussetzungen auf einem System zu installieren.

\subsection{Orgware}

\begin{itemize}
	\item Installation der Softwarevoraussetzungen

	\item Konfiguration der Anwendung (Erstmaliges Starten der Anwendung, Verbindung mit PostgreSQL, Erstellung des Datenbank Schemata)

	\item Internetanschluss für den Webserver, der über das öffentliche \textit{freie} Internet zugänglich ist und Internetanschluss oder lokale Netzwerkverbindung zu dem Datenbankserver.

	\item Verschlüsselte Kommunikation über HTTPS. Verwendung einer statischen IP-Adresse und und eines vertrauenswürdigem TLS Zertifikats.

	\item Email-Server mit Email-Konto zur Versendung automatisierter Benachrichtigungsemails.

	\item Email-Client auf Rechner des Benutzers um Emails zu anderen Benutzern versenden zu können.
\end{itemize}

