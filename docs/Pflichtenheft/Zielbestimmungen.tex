\localauthor{Thomas Kirz}

\subsection{Musskriterien}
Ziel des Projektes ist ein Programm, dass von mehreren Nutzenden mit verschiedenen Rollen und Rechten benutzt werden kann.

An oberster Stelle in der Hierarchie stehen die Administrator:innen, welche für das Betreiben der Webanwendung zuständig sind.
Sie können das System gemäß den Anforderungen und Wünsche des Betreibers konfigurieren, Konferenzen und Journale einrichten und Benutzende verwalten.
Dafür ernennt er auch Editor:innen, die die Einreichungen für ihre Konferenz oder Journal übersehen.
Sie laden Gutachter:innen für das review einer Einreichung ein und einscheiden nach Fertigstellung des Gutachtens auf dessen Basis über das weitere Vorgehen.
Ein Artikel kann entweder direkt angenommen werden, abgelehnt (mit Begründung) oder die Editor:innen verlangen eine Revision des Artikels, die erneut begutachtet werden kann.

Einreichen kann jede:r Wissenschaftler:in nach Registrierung und Anmeldung am System.
Sie können mit Hilfe einer Liste oder Suche einen geeigneten Kongress oder ein Journal finden und ihren Artikel als PDF-Datei hochladen und mit Metainformationen wie Daten der Koautoren versehen.
Dabei sucht man sich aus, welche:r Editor:in für die Einreichung verantwortlich sein soll.
Über eine Entscheidung der Editor:innen werden sie per E-Mail benachrichtigt.

\subsection{Wunschkriterien}

Über die nötigen Funktionen hinaus gibt es noch folgende wünschenswerte Kriterien.

Es wäre möglich, dass bei der Einreichung eine:n Editor:in nicht verbindlich ausgesucht, sondern nur vorgeschlagen wird;
der/die Editor:in kann dann selbst entscheiden, wer ein Gutachten durchführen soll.