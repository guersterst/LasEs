\localauthor{Thomas Kirz}

\subsection{Musskriterien}
Ziel des Projektes ist eine Webapplikation, dass von mehreren Nutzenden mit verschiedenen Rollen und Rechten benutzt werden kann.

An oberster Stelle in der Hierarchie stehen die Administrator:innen, welche für das Betreiben der Anwendung zuständig sind.
Sie können das System gemäß den Anforderungen und Wünsche des Betreibers konfigurieren, Konferenzen und Journale einrichten und Benutzende verwalten.
Dafür ernennt er auch Editor:innen, die die Einreichungen für ihre Konferenz oder Journal übersehen.
Sie laden Gutachter:innen für das review einer Einreichung ein und einscheiden nach Fertigstellung des Gutachtens auf dessen Basis über angenommen oder (mit Begründung )abgelehnt werden soll.

Einreichen kann jede:r Wissenschaftler:in nach Registrierung und Anmeldung am System und der Angabe, wer Editor:in der Arbeit sein soll.
Sie können mit Hilfe einer Liste oder Suche einen geeigneten Kongress oder ein Journal finden und ihren Artikel als PDF-Datei hochladen und mit Metainformationen wie Daten der Koautoren versehen.
Dabei sucht man sich aus, welche:r Editor:in für die Einreichung verantwortlich sein soll.
Über eine Entscheidung der Editor:innen werden sie per E-Mail benachrichtigt.

Eine Erweiterung des Systems um weitere Funktionen wie z.B.\ das Publizieren von Artikeln soll einfach möglich sein.

\subsection{Wunschkriterien}

Über die nötigen Funktionen hinaus gibt es noch folgende wünschenswerte Kriterien.

Es wäre möglich, dass bei der Einreichung ein:e Editor:in nicht verbindlich ausgesucht, sondern nur vorgeschlagen wird;
die Editor:innnen können dann selbst unter sich ausmachen, wer welche Einreichung betreut.
Außerdem könnte es die Funktion geben, Gutachter:innen bei der Einreichung unverbindlich vorzuschlagen.

Weitere Möglichkeiten zur Personalisierung sind möglich, zum einen die Anzeige von eigener Logos und Farbschemata für Konferenzen und Journale oder auch Avatarbilder für die Nutzerprofile.

Neben direkter Annahme oder Ablehnung einer Einreichung könnte noch das Verlangen einer Revision möglich sein.
Die Einreichung müssten also mit gewünschten Änderungen erneut eingesendet werden und erneut begutachtet und evtl.\ akzeptiert werden.

Schließlich gibt es noch die Möglichkeit, die Webseite zweisprachig auf Englisch und Deutsch anzubieten.

\subsection{Abgrenzungskriterien}

Die Veröffentlichung und Lizenzierung von Artikeln ist keine Funktion der Software, die Annahme oder Ablehnung ist der letzte Schritt einer Einreichung für LasEs. Auch Zahlungsabwicklungen werden nicht unterstützt.