\localauthor{Thomas Kirz}

\subsection{Musskriterien}
Ziel des Projekts ist eine Webapplikation mit Benutzeroberfläche in englischer Sprache, die von mehreren Nutzenden mit verschiedenen Rollen und Rechten benutzt werden kann.
Alle Rollen, außer die anonymen Nutzenden, werden als authentifizierte Nutzende bezeichnet und können Papers einreichen.
Falls sie die Rolle eines Administrators, eines Gutachters oder eines Editors bekleiden, stehen ihnen weitere Funktionen zur Verfügung.

Eine Erweiterung des Systems um Funktionen, die über die folgenden Kriterien hinausgehen, wie z.B.\ das Publizieren von Artikeln, ist einfach möglich.

\subsubsection{Administratoren}\label{mkrit:admin}
An oberster Stelle in der Hierarchie stehen die Administratoren mit allumfassenden Rechten, welche als Betreiber der Anwendung fungieren.
Sie konfigurieren das System gemäß den Anforderungen und Wünschen des Betreibers und verwalten Benutzende.
Außerdem richten sie Konferenzen und Journale ein, bei denen wissenschaftliche Artikel veröffentlicht werden.
Dafür ernennen sie jeweils Editoren, die die Einreichungen für ihre Konferenzen und Journale verwalten.

\subsubsection{Editoren}\label{mkrit:editor}
Editoren haben die Möglichkeit, weitere Editoren für ein \hyperref[glo:wissForum]{wissenschaftliches Forum} zu ernennen.
Sie laden Gutachter für das Review von Einreichungen ein und entscheiden nach Fertigstellung der Gutachten auf deren Basis, ob sie angenommen oder (mit Begründung) abgelehnt werden sollen.

\subsubsection{Gutachter}\label{mkrit:gutachter}
Gutachter können Einladungen zur Begutachtung einer wissenschaftlichen Arbeit annehmen oder ablehnen.
Nehmen sie diese an, so müssen sie sich spätestens zu diesem Zeitpunkt registrieren.
Danach laden sie das Dokument herunter und erstellen ihr Gutachten in einem eigenen PDF-Dokument.
Unter Einhaltung einer vom Editor festgelegten Deadline laden sie dieses hoch.
Außerdem stehen ihnen Tools wie ein Kommentarfeld oder die Möglichkeit, eine Empfehlung abzugeben, zur Verfügung.

\subsubsection{Angemeldete Nutzer}\label{mkrit:angemeldet}
Einreichen kann jeder Wissenschaftler nach Registrierung und Anmeldung am System.
Sie finden mithilfe einer Liste oder Suche eine geeignete Konferenz oder ein Journal.
Dort laden sie ihre Artikel als PDF-Datei hoch und versehen sie mit Metainformationen wie Daten der Ko-Autoren.
Dabei suchen sie aus, welcher Editor für die Einreichung verantwortlich sein soll.
Über eine Entscheidung des Editors werden sie per E-Mail benachrichtigt.

\subsubsection{Anonyme Nutzer}\label{mkrit:anon}
Anonyme Nutzer können sich initial nur registrieren und dann erst weitere Funktionen benutzen.


\subsection{Wunschkriterien}

Über diese Funktionen hinaus gibt es noch folgende wünschenswerte Kriterien:

Eine Möglichkeit ist, Gutachter bei der Einreichung unverbindlich vorschlagen zu können.

Weitere Einstellungen zur Personalisierung sind möglich:
Zum einen die Anzeige eigener Logos und Farbschemata für Konferenzen und Journale oder auch Avatarbilder für die Nutzerprofile.

Neben direkter Annahme oder Ablehnung einer Einreichung ist das Anfordern einer Revision optional.
Die Einreichung müsste also mit gewünschten Änderungen erneut eingereicht werden, erneut begutachtet und eventuell akzeptiert werden.

Für Gutachter kann es die Möglichkeit geben, eigene Gutachten zurückzuziehen und löschen zu lassen.

Schließlich gibt es noch die Möglichkeit, die Webseite zweisprachig auf Englisch und Deutsch anzubieten.

\subsection{Abgrenzungskriterien}

Artikel müssen als PDF-Datei eingereicht werden. Das Hochladen von bspw.\ Word- oder \hyperref[glo:latex]{\LaTeX}-Dateien ist nicht möglich.
Außerdem können diese Dokumente nicht direkt auf der Webseite angezeigt oder bearbeitet werden;
die Nutzer müssen die Dateien herunterladen und dafür eigene Tools benutzen.

Die Veröffentlichung und Lizenzierung von Artikeln ist keine Funktion von LasEs,
die Annahme oder Ablehnung ist der letzte Schritt einer Einreichung.
Auch Zahlungsabwicklungen werden nicht unterstützt.

Die Webseite ist für Laptops und Desktopgeräte optimiert, es wird keine für mobile Endgeräte nutzerfreundliche Oberfläche gewährleistet.
