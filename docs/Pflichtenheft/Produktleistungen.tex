\localauthor{Johann Schicho}

Die Notation \texttt{/LXXX/} erlaubt eine spätere Referenzierung der einzelnen Produktleistungen in diesem und weiteren
Dokumenten. \texttt{XXX} entspricht dabei immer einer dreistelligen Zahl.

\subsection{Skalierbarkeit}

\begin{description}

	\XXitem{/L010/}{leist:10} Die Anwendung wird bis zu 100000 verschiedene Nutzer und 2000 verschiedene Konferenzen und Journale verwalten können.

	\XXitem{/L020/}{leist:20} Die Anwendung wird bis zu 50 gleichzeitig angemeldete und auf der Webanwendung agierende Nutzer verarbeiten können. Dabei wird die Reaktionszeit jeglicher Nutzerinteraktion mit der Webanwendung nicht mehr als 3 Sekunden dauern. Ausnahme davon ist die allgemeine Suchfunktion, die nicht mehr als 5 Sekunden dauert, um ein Ergebnis anzuzeigen.

\end{description}

\subsection{Usability}

\begin{description}
	\XXitem{/L030/}{leist:30} Die Benutzeroberfläche ist intuitiv zu bedienen sein und zeigt pro Seite nur gezielt Informationen, die der Nutzer sehen will, und ist nicht überladen.

	\XXitem{/L040/}{leist:40} Lange Listen werden paginiert. Das heißt sie sind über mehrere Seiten aufgeteilt. Ein solcher Umbruch findet nach je 25 Listeneinträgen statt. Durch Anklicken eines Listeneintrags wird dann auf die zugehörige Übersichtsseite weitergeleitet.

	\XXitem{/L042/}{leist:042} Listen werden filterbar sein. Eine Spalte kann also auf einen bestimmten Wert gefiltert werden.

	\XXitem{/L045/}{leist:045} Alle Listen sind nach ihren Spalten auf- oder absteigend sortierbar.

	\XXitem{/L050/}{leist:050} Über die Kopfzeile, die auf jeder nach dem Login vorzufindenden Seite der Webanwendung
	vorhanden ist, steht eine Suchfunktion zur Verfügung.

	\XXitem{/L055/}{leist:055} Über \emph{Tooltips} kann zu wesentlichen Funktionen der Anwendung eine Kurzhilfe eingeblendet werden. Damit können etwaige Fragen der Nutzer geklärt werden.

	\XXitem{/L060/}{leist:60} Bei fehlerhaften Eingaben wird der Nutzer informiert und kann diese ohne erneutes Eingeben korrigieren, da sie stehen bleiben. Eine Ausnahme ist die Eingabe eines neues Passworts.

	\XXitem{/L062/}{leist:062} Nach erfolgreichen Datenänderungen wird der Nutzer darüber in kurzen Hinweistexten informiert.

	\XXitem{/L065/}{leist:65} Die Verwendung von Cookies ist nicht zwingend erforderlich.
\end{description}

\subsection{Datensicherheit}

\begin{description}
	\XXitem{/L070/}{leist:070} Alle in der Anwendung erfassten Daten werden in der PostgreSQL Datenbank persistent abgelegt.

	\XXitem{/L080/}{leist:080} Die Konsistenz der Daten über Änderungen hinweg wird durch Transaktionen sichergestellt.

	\XXitem{/L090/}{leist:090} Sollten durch Löschen bestimmter Datensätze davon abhängige weitere Datensätze gelöscht werden, wird der Nutzer vorher deutlich gewarnt.
\end{description}

\subsection{Datenschutz}

\begin{description}
	\XXitem{/L100/}{leist:100} Es wird sichergestellt, dass keine sensiblen Daten für unberechtigte Dritte zugänglich sind.

	\XXitem{/L110/}{leist:110} Alle personenbezogenen Daten werden, wie Nutzerdaten und Passwörter werden über eine verschlüsselte HTTPS Verbindung übertragen.

	\XXitem{/L120/}{leist:120} Der \hyperref[funkt:080]{Login} in das System ist über die Emailadresse und Passwort möglich.

	\XXitem{/L130/}{leist:130} Das Passwort muss 8-100 Zeichen lang sein und mindestens Groß- und Kleinbuchstaben, Zahlen und Sonderzeichen enthalten.

	\XXitem{/L140/}{leist:140} Die Registrierung muss durch eine \hyperref[funkt:060]{Bestätigungsemail}, die an die vom Nutzer angegebene Email-Adresse gesendet wird, authentifiziert werden.

	\XXitem{/L142/}{leist:142} Bei Änderung der hinterlegten Email-Adresse muss diese ebenfalls bestätigt werden. Dazu wird eine Bestätigungsemail an die neue Email-Adresse gesendet.
\end{description}

\subsection{Logging}

\begin{description}
	\XXitem{/L145/}{leist:145} Das System verfügt über einen detaillierten Logger, der alle Fehler und Events während der Laufzeit des Systems sammelt. Hierbei werden die Fehler in 3 verschiedene Fehlerklassen eingeteilt: \emph{Error}, \emph{Warning} und \emph{Debug}.
\end{description}

\subsection{Internationalisierbarkeit}

\begin{description}
	\XXitem{/L150/}{leist:150} Die Texte auf der Website werden in UTF-8 kodiert
	und ausgeliefert, um \textit{Internationalization (i18n)} in verschiedenen Sprachen zu ermöglichen.

	\XXitem{/L155/}{leist:155} Die Standard Sprache der Webanwendung ist Englisch.

	\XXitem{/WL160/}{leist:160} Die \hyperref[funkt:020]{mehrsprachige Implementierung} umfasst Englisch und Deutsch.
 \end{description}

\subsection{Installation}

\begin{description}
	\XXitem{/L170/}{leist:170} Es gibt eine kurze Installationsanleitung\footnote{Siehe mitgelieferte INSTALL.txt} für den Systemadministrator. Diese leitet weiter auf Installationsbeschreibungen der einzelnen Softwarevoraussetzungen.

	\XXitem{/L180/}{leist:180} Die Erstinbetriebnahme soll die Möglichkeit bieten, die benötigten Datenbankschemata automatisch zu erstellen.
\end{description}
