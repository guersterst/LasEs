\localauthor{Johann Schicho}

Die Notation \texttt{/LXXX/} erlaubt eine spätere Referenzierung der einzelnen Produktleistungen in diesem und weiteren
Dokumenten. \texttt{XXX} entspricht dabei immer einer dreistelligen Zahl.

\subsection{Skalierbarkeit}

\begin{description}

	\XXitem{/L010/}{leist:10} Die Anwendung soll bis zu 1000 verschiedene Nutzer und 100 verschiedene Konferenzen und Journale verwalten können.

	\XXitem{/L020/}{leist:20} Die Anwendung soll bis zu 50 gleichzeitig angemeldete und auf der Webanwendnung agierende Nutzer verarbeiten können.

\end{description}

\subsection{Usability}

\begin{description}
	\XXitem{/L030/}{leist:30} Die Benutzeroberfläche soll intuitiv zu bedienen sein und pro Seite nur gezielt Informationen anzeigen, die der Nutzer sehen will, und nicht überladen sein.

	\XXitem{/L040/}{leist:40} Lange Listen sollen paginiert werden. Das heißt sie sind über mehrere Seiten aufgeteilt. Ein solcher Umbruch findet nach je 25 Listeneinträgen statt.

	\XXitem{/L050/}{leist:50} Über die Kopfzeile, die auf jeder nach dem Login vorzufindenden Seite der Webanwendung
	vorhanden ist, soll eine Suchfunktion zur Verfügung stehen.

	\XXitem{/L055/}{leist:55} Über die Fußzeile soll zu jeder Seite eine Kurzhilfe angeboten werden. Durch einen Link wird ein neuer Tab geöffnet, der dann einen kurzen Hilfetext anzeigt.

	\XXitem{/L060/}{leist:60} Die Benutzer sollen über Fehleingaben benachrichtigt werden und diese anschließend ohne erneute Eingabe aller Daten korrigieren können.

	\XXitem{/L065/}{leist:65} Die Verwendung von Cookies ist nicht zwingend erforderlich.
\end{description}

\subsection{Datensicherheit}

\begin{description}
	\XXitem{/L070/}{leist:070} Alle in der Anwendung erfassten Daten werden in der PostgreSQL Datenbank persistent abgelegt.

	\XXitem{/L080/}{leist:080} Die Konsistenz der Daten über Änderungen hinweg wird durch Transaktionen sichergestellt.

	\XXitem{/L090/}{leist:090} Sollten durch Löschen bestimmter Datensätze davon abhängige weitere Datensätze gelöscht werden, wird der Nutzer vorher deutlich gewarnt.
\end{description}

\subsection{Datenschutz}

\begin{description}
	\XXitem{/L100/}{leist:100} Es wird sichergestellt, dass keine sensiblen Daten für unberechtigte Dritte zugänglich sind.

	\XXitem{/L110/}{leist:110} Alle personenbezogenen Daten werden, wie Nutzerdaten und Passwörter werden über eine verschlüsselte HTTPS Verbindung übertragen.

	\XXitem{/L120/}{leist:120} Der \hyperref[funkt:080]{Login} in das System ist über die Emailadresse und Passwort möglich.

	\XXitem{/L130/}{leist:130} Das Passwort muss 8-100 Zeichen lang sein und mindestens Groß- und Kleinbuchstaben, Zahlen und Sonderzeichen enthalten.

	\XXitem{/L140/}{leist:140} Die Registrierung muss durch eine \hyperref[funkt:060]{Bestätigungsemail}, die an die vom Nutzer angegebene Email-Adresse gesendet wird, authentifiziert werden.
\end{description}

\subsection{Logging}

\begin{description}
	\XXitem{/L145/}{leist:145} Das System verfügt über einen detaillierten Logger, der alle Fehler während der Laufzeit des Systems sammelt. Damit soll Debugging und Fehlersuche erleichtert werden.
\end{description}

\subsection{Internationalisierbarkeit}

\begin{description}
	\XXitem{/L150/}{leist:150} Die Texte auf der Website werden in UTF-8 kodiert
	und ausgeliefert, um \textit{Internationalization (i18n)} in verschiedenen Sprachen zu ermöglichen.

	\XXitem{/L155/}{leist:155} Die Standard Sprache der Webanwendung ist Englisch.

	\XXitem{/WL160/}{leist:160} Die \hyperref[funkt:020]{mehrsprachige Implementierung} umfasst Englisch und Deutsch.
 \end{description}

\subsection{Installation}

\begin{description}
	\XXitem{/L170/}{leist:170} Es gibt eine kurze Installationsanleitung für den Systemadministrator. Diese leitet weiter auf Installationsbeschreibungen der einzelnen Softwarevoraussetzungen.

	\XXitem{/L180/}{lesit:180} Die Erstinbetriebnahme soll die Möglichkeit bieten, die benötigtem Datenbankschemata automatisch zu erstellen.
\end{description}
