\localauthor{Sebastian Vogt}
\subsection{Programmierung}
\begin{itemize}
	\item \textbf{Betriebssystem}: Die Entwickler verwenden Windows 10, Windows 11 und Ubuntu 20.04
	\item \textbf{IDE \todo{Glossar}}: JetBrains IntelliJ
	\item \textbf{Build Tool \todo{Glossar}}: Apache Maven
\end{itemize}
\subsection{Referenzumgebung}
\begin{itemize}
	\item \textbf{Webserver \todo{Glossar}}: Tomcat 10.0.10
	\item \textbf{Betriebssystem}: Debian 11
	\item \textbf{Deployment \todo{Glossar}}: Das Projekt wird durch das \textbf{WAR Plugin in Maven} als \textbf{WAR Datei \todo{Glossar}} exportiert. Diese Datei wird dann von Tomcat zum Zugriff zur Verfügung gestellt.
\end{itemize}
\subsection{Versionskontrolle}
\begin{itemize}
	\item \textbf{Git} Version 2.25.1
	\item \textbf{Zusammenarbeit} im Team wird über den \textbf{GitLab \todo{Glossar}} Server der Fakultät für Informatik und Mathematik der Universität Passau gehandhabt.
\end{itemize}
\subsection{Dokumente}
\begin{itemize}
	\item \textbf{LaTex Compiler \todo{Glossar}}: LuaHBTeX, Version 1.13.2
	\item \textbf{LaTex Distribution \todo{Glossar}}: TeX Live 2021
\end{itemize}
\subsection{Diagramme}
\begin{itemize}
	\item \textbf{Klassendiagramm \todo{Glossar}}: IBM Rational Software Architect auf Debian 11
	\todo{Die Programme, die wir für Diagramme verwenden noch ergänzen.}
\end{itemize}