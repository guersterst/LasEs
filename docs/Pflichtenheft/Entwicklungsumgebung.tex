\localauthor{Sebastian Vogt}
\subsection{Programmierung}
\begin{itemize}
	\item \emph{Entwicklerrechner}: Die Entwickler verwenden folgende Systeme für die Entwicklung:
		\begin{itemize}
			\item Lenovo IdeaPad C340-14IML, Intel(R) Core(TM) i5-10210U CPU @ 1.60GHz 2.11GHz, 16GB RAM, Windows 11
			\item Lenovo IdeaPad Flex 5 14IIL05, Intel(R) Core(TM) i5-1035G1 CPU @ 1.00GHz 1.19GHz, 8GB RAM, Windows 10
			\item Acer Swift SF314-55, Intel(R) Core(TM) i5-8265U CPU @ 1.60GHz 1.80GHz, 8GB RAM, Windows 10
			\item Acer Aspire A515-54G, Intel(R) Core(TM) i5-8265U CPU @ 1.60GHz 1.80GHz, 8GB RAM, Ubuntu 20.04.3 LTS
			\item Lenovo ThinkPad E490, Intel(R) Core(TM) i5-8265U CPU @ 1.60GHz 1.80GHz, 8GB RAM, Ubuntu 20.04.3 LTS
		\end{itemize}
	\item \emph{IDE}: JetBrains IntelliJ 2021.2
	\item \emph{JDK}: Adopt-OpenJDK 16.0.2
	\item \emph{Application Server}: Tomcat 10.0.10
	\item \emph{Build Tool}: Apache Maven 3.6.3
	\item \emph{Testing Frameworks}: JUnit Jupiter 5.8.1, Selenium 3.141.59, Mockito 4.0.0
	\item \emph{In-Memory Datenbank}: H2 Database Engine 1.4.200
	\item \emph{Webbrowser}: Mozilla Firefox 93.0
	\item \emph{Mail Client}: Mozilla Thunderbird 91.2.0
\end{itemize}
Die Referenzumgebung für den Applikationsserver wird \hyperref[spezi]{hier} beschrieben.
Als Datenbankserver wird in der Entwicklung bereits der Referenzserver verwendet. Dieser wird \hyperref[dbspezi]{hier} beschrieben.
\subsection{Versionskontrolle}
\begin{itemize}
	\item \emph{Git} Version 2.25.1
	\item \emph{Zusammenarbeit} im Team wird über den \emph{GitLab} Server der Fakultät für Informatik und Mathematik der Universität Passau gehandhabt.
\end{itemize}
\subsection{Dokumente}
\begin{itemize}
	\item \emph{Textsatz}: \LaTeX
	\item \emph{\LaTeX\ Compiler}: LuaHBTeX, Version 1.13.2
	\item \emph{\LaTeX\ Distribution}: TeX Live 2021
	\item \emph{\LaTeX\ Editor}: TeXstudio 4.0.0
	\item \emph{PDF Reader}: Adobe Acrobat Reader DC 2021.007.20099, Evince 3.36.10
\end{itemize}
\subsection{Diagramme}
\begin{itemize}
	\item \emph{Klassendiagramm}: IBM Rational Software Architect 9.7 auf Debian 11
	\item \emph{Sequenzdiagramm}: PlantUML mit IntelliJ Plugin "PlantUML integration" 5.6.1
	\item \emph{Vektorgrafik Software}: Inkscape 1.1.1, Affinity 1.10.1.1142
	\item \emph{Graph Editor}: yEd 3.21.1
	\item \emph{Kollaboratives Design-Tool}: Figma Linux 0.9.2
\end{itemize}
\subsection{Orgware}
\begin{itemize}
	\item \emph{Internetanbindung} mit mindestens 1 Mbit/s Bandbreite
	\item \emph{E-Mail Dienst} der FIM mit einer ``@fim.uni-passau.de'' Adresse für jeden Entwickler und der Adresse ``sep21g02@fim.uni-passau.de'' für das gesamte Team.
\end{itemize}
\subsection{Sonstige Software}
\begin{itemize}
	\item \emph{Kommunikation}: Whatsapp 2.2140.5, Discord 1.0.9003
	\item \emph{Datei Sharing}: LRZ Sync and Share
	\item \emph{Projektmanagement}: ProjectLibre 1.9.3
\end{itemize}