%todo items permutieren um gleichheit zu verschleiern
%todo querverweise zu zb produktleistungen und zu produktdaten
%todo navigationen zwischen seiten -> siehe flussdiagramm -> welche seite ist von wo aus erreichbar.
%todo namenschaos -> einheitliche Namen für Seiten
\localauthor{Johannes Garstenauer}
Die Funktionalität vom LasEs-System soll nach den Benutzerrollen
%TODO verlinken
\textit{anonymer Nutzer}, \textit{angemeldeter Nutzer}, \textit{Editor}, und
\textit{Administrator} untergliedert werden. Da ein Nutzer mehrere Rollen einnehmen
kann bestehen Inklusionsbeziehungen die im Folgenden näher erklärt werden.
%todo inklusionsbeziehungen

\subsection{Anonymer Nutzer}
Anonyme Nutzer sind unauthentifzierte Nutzer, deren Zugriffsrechte sich
auf die Registrierung und Anmeldung im System beschränken.

\subsubsection{Allgemein}
\begin{description}
    \XXitem{/F010/}{funkt:010} Beim Aufruf einer Seite durch einen nichtangemeldeten Benutzer
    wird dieser auf die Anmeldeseite weitergeleitet und zur
    Anmeldung aufgefordert. Ausgenommen davon ist die Hilfeseite zur Anmeldung und die
    Registrierungsseite.
    %in dieser allgemeinheit? alle seiten?
    \XXitem{/F020/}{funkt:020} Die Standardsprache des Systems ist abhängig von der im Browser
    eingestellten Sprache. Es werden Deutsch und Englisch angeboten.
    Sonst ist die Standardsprache Englisch. Die Sprache der Anwendung kann über die
    Fußzeile geändert werden.
    \XXitem{/F030/}{funkt:030} Auf den wichtigsten Seiten lassen sich von der Fußzeile aus
    Hilfetexte zu den angebotenen Funktionalitäten und der jeweiligen Rolle
    des Nutzers anzeigen. Hierzu öffnet sich ein neuer Tab.
    \XXitem{/F040}{funkt:040} Beim erstmaligen Aufruf der Anwendungen wird der Nutzer gebeten
    seine Cookieauswahl zu treffen. Selbst wenn er ablehnt soll die Anwendung
    für ihn funktionstüchtig bleiben.
    %Fehlerseite falls Ressource nicht ex. -> auch in angemeld. Nutzer darauf verweisen.
\end{description}

\subsubsection{Registrierung}
\begin{description}
    \XXitem{/F050/} Ein anonymer Nutzer kann von der Anmeldeseite aus mittels eines
    Buttons auf die Registrierungsseite navigieren.
    \XXitem{/F060/} Über ein Registrierungsformular wird der Nutzer zur Eingabe seiner
    Daten aufgefordert. Verlangt wird die Eingabe eines Benutzernamens und Passworts
    %doppelt? -> sicherheitsanforderungen. benutzername schon verwendet -> jsf validierung.
    %email regex validierung.
    %optionale daten
    sowie von Vor- und Nachname. Außerdem ist ein Arbeitsgeber, ein oder mehrere Spezialgebiete
    und das Geburtsdatum
    anzugeben. Optional ist das Einfügen eines Avatarbildes. Letztlich muss die Emailadresse
    angegeben werden. Durch das Absenden des Formulars wird der E-Mail Verifizierungsprozess
    gestartet.
    %link email-ver
    \XXitem{/F070/} Nach der Registrierung wird eine automatisierte E-Mail
    an die angegebene Mailadresse gesendet. Der Inhalt der Nachricht Inhalt einen Hinweis auf
    die versuchte Registrierung sowie einen Verifizierungslink, welcher auf eine Seite führt, welche
    die erfolgreiche Registrierung bestätigt. Der Account ist nun registriert. Nach einem Augenblick wird der Nutzer auf die Startseite
    weitergeleitet.
    \XXitem{/F070/} Von der Registrierungsseite kann man auf die Anmeldeseite gelangen.
\end{description}

\subsubsection{Login}
\begin{description}
    \XXitem{/F080/}{funkt:080} Mittels eines Anmeldeformulars erfolgt eine Anmeldung durch korrekte Zugangsdaten.
    Diese umfassen den Benutzernamen und das Passwort. Ein anonymer Nutzer wird so zum angemeldeten Benutzer.
    \XXitem{/F090/} Von der Anmeldeseite ist es möglich zur Registrierungsseite zu gelangen.
    % Passwort vergessen.
\end{description}

\subsection{Angemeldeter Nutzer}
Angemeldete Nutzer haben Zugriff auf die Funktionen
\hyperref[funkt:010]{/F010/}, \hyperref[funkt:020]{/F020/}, \hyperref[funkt:030]{/F030/}.
Nach der Anmeldung \hyperref[funkt:080]{/F080/},
stehen außerdem folgende weitere Funktionen zur Verfügung.

\subsubsection{Allgemein}
\begin{description}
    \XXitem{/F100} Bei Zugriff auf die Anmelde- oder Registrationsseite
    wird auf die Startseite weitergeleitet.
    \XXitem{/F110} Über einen Button in der Kopfzeile kann ein Logout durchgeführt werden.
    Der nun anonyme Nutzer wird auf die Anmeldeseite weitergeleitet.
    \XXitem{/F120/} Die Fußzeile ermöglicht eine Navigation auf das Impressum.
    %%url def?
\end{description}

\subsubsection{globale suche}
%% publ alle
% nutzer admin, editor

\subsubsection{Profil}
\begin{description}
    \XXitem{} Über die Kopfzeile kann sich ein Benutzer zu seinem Profil navigieren.
    \XXitem{} Auf der Profilseite kann der Nutzer alle dynamischen Daten über dieses Profil einsehen,
    Eine explizite Ausnahme hiervon ist das gehashte Passwort. %link /D/
    \XXitem{/F130/} Auf der eigenen Profilseite kann der Nutzer jedes dieser Daten %link /D/
    welche über ihn gespeichert sind permanent verändern.
    \XXitem{/F140/} Bei Änderung der Mailadresse wird der E-Mailverifikationsprozess erneut
    begonnen. %link
    \XXitem{} Der Nutzer kann auf der eigenen Profilseite ein Avatarbild mit einer maximalen
    Größe von 4MB hochladen oder sein altes Avatarbild entfernen oder austauschen. %link /D/
    \XXitem{} Auf der eigenen Profilseite kann der Nutzer sein Profil und alle damit verbundenen persistenten
    Daten löschen. Auch seine Einreichungen werden gelöscht und Gutachter sowie Editoren dieser
    Einreichung per Mail informiert. Bevor die Löschung vollzogen wird, wird dem Nutzer
    eine Warnung über diese Konsequenzen angezeigt.
\end{description}

\subsubsection{Startseite}
\begin{description}
    \XXitem{} Die Startseite ist zu jeder Zeit über die Kopfzeile erreichbar.
    \item
    \XXitem{/F130/} Ein Nutzer bekommt auf der Startseite alle Namen von Konferenzen
    \& Journale in einer Listensicht angezeigt in denen er aktive Einreichungen hat.
    Die Namen dieser aktiven Einreichungen werden unter den Namen der Konferenzen \& Journale angezeigt.
    \XXitem{/F140/} Die Einreichungen lassen sich nach Namen und Datum
    der Einreichung sortieren. Die Konferenzen lassen sich nach ihrem Namen sortieren.
    %suchfunktion?
    \XXitem{} Durch den Klick auf einen Eintrag der Liste gelangt man auf die jeweilige Übersichtsseite
    der Einreichung oder auf die Seite des jew. wissenschaftlichen Forums.
%info über benachrichtigungen? -> Benachrichtigungssystem? nur mail?
\end{description}

\subsubsection{Liste der Journals \& Konferenzen}
\begin{description}
    %% ergebnisse globale suche
    \XXitem{/F130/} Über eine Suchleiste kann nach bestimmten Journals oder Konferenzen mittels
    ihres Namens gesucht werden.
    \XXitem{/F140/} Die Ergebnisse lassen sich sortieren nach %produktdaten journals & kongresse, eig nur Name oder?
    \XXitem{/F150/} Durch Klick auf den Namen eines Eintrags wird man auf die Seite des
    jeweiligen Journals oder der jeweiligen Konferenz gelenkt.
\end{description}

\subsubsection{Publikationseite}
\begin{description}
    \XXitem{/F130/} Auf der Seite eines wissensch. Forums werden die zugehörigen Informationen
    angezeigt. %link /D/
    \XXitem{/F140/} Dem Nutzer werden seine eigenen Einreichungen in Form einer Liste mit Namen angezeigt.
    \XXitem{/F160/} Durch Klick auf den Namen einer Einreichung gelange ich auf die Übersichtsseite
    dieser Einreichung.
    \XXitem{/F140/} Die Einreichungen lassen sich nach Namen und Datum
    der Einreichung sortieren und nach Namen der Einreichung durchsuchen. %durchsuchen meinung?
    \XXitem{} Der Nutzer kann auf die Seite zur Erstellung einer Einreichung navigieren. Hierbei ist
    das Feld, welches die Konferenz oder das Journal bestimmt bei dem eingereicht wird, bereits mit
    der Konferenz bzw. dem Journal befüllt von dessen Übersichtsseite aus die Navigation auf diese
    Seite ausgeführt wurde.
\end{description}


\subsubsection{Erstellen einer Einreichung}
\begin{description}
    \XXitem{} Der Nutzer kann eine Einreichung im System erstellen. Hierzu gibt er in einem
    Formular die nötigen Informationen wie Name der Einreichung, Namen und E-Mail Adressen der Ko-Autoren,
    sowie den gewünschten Editor angeben.
    \XXitem{} Der Nutzer lädt seine Einreichung in Form einer PDF hoch. Die Abgabe darf eine Dateigröße
    von 20MB nicht überschreiten und muss zwingenderweise im PDF-Format erfolgen.
    \XXitem{} Durch Absenden des Formulars wird der Editor des Journals bzw. der Konferenz
    informiert. Das Datum der Einreichung wird auf das Datum zum Zeitpunkt der Einreichung fest-
    gelegt.
    \XXitem{} Die Einreichung ist erfolgreich, wenn alle Felder ausgefüllt sind und eine PDF
    hochgeladen wurde. Andernfalls wird der Nutzer über das fehlschlagen informiert.
    \XXitem{} Nach der erfolgreichen Einreichungen wird der Nutzer auf die Übersichtsseite der
    Einreichung weitergeleitet.
    %zugriff: navbar, startseite, publikationsseite
\end{description}

\subsubsection{Einreichungsseite}
\begin{description}
    \XXitem{/F130/} Dem Nutzer werden Informationen zu seiner Einreichung angezeigt.
    Hierzu gehören der Status der Einreichung, das Datum der Einreichung, das zugehörige
    Journal- bzw Konferenz, Namen und E-Mail Adressen der Ko-Autoren, sowie ein Download zur Einreichung.
    Außerdem werden die Gutachten in einer
    Liste dargestellt zusammen mit ihrem Erstellungsdatum, Gutachterempfehlung und Download. %link /D/
    \XXitem{} Die Gutachten lassen sich nach Namen und Datum
    des Gutachten sortieren und nach Namen des Gutachten durchsuchen. %suche optional, meinung?
%nav
\end{description}

\subsection{Editor}
\subsubsection{asdasd} ads

\subsubsection{Startseite}
%siehe nutzer, konferenz + papers to review

\subsubsection{Pages}

\subsection{Gutachter}

\subsubsection{Startseite}
%siehe nutzer, konferenz + papers to review

\subsubsection{Pages}

\subsection{Administrator}
%edit user profiles

\subsubsection{Pages}
