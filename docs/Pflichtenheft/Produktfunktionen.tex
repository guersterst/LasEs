%todo items permutieren um gleichheit zu verschleiern
%todo querverweise zu zb produktleistungen und zu produktdaten
%todo navigationen zwischen seiten -> siehe flussdiagramm -> welche seite ist von wo aus erreichbar.
%todo namenschaos -> einheitliche Namen für Seiten
%todo optionale als FW (funktionswunsch) -> so konzipieren, dass >= 50% sicher implementiert werden.
\localauthor{Johannes Garstenauer}

Die Funktionalität vom LasEs-System soll nach den Benutzerrollen
%TODO verlinken
\emph{anonymer Nutzer}, \emph{angemeldeter Nutzer}, \emph{Editor}, und
\emph{Administrator} untergliedert werden. Da ein Nutzer mehrere Rollen einnehmen
kann bestehen Inklusionsbeziehungen, die im Folgenden näher erklärt werden.
%todo inklusionsbeziehungen

\subsection{Anonymer Nutzer}
Anonyme Nutzer sind unauthentifzierte Nutzer, deren Zugriffsrechte sich
auf die Registrierung und Anmeldung im System beschränken.

\subsubsection{Allgemein}
\begin{description}
    \XXitem{/F010/}{funkt:010} Beim Aufruf einer Seite durch einen nichtangemeldeten Benutzer
    wird dieser auf die Anmeldeseite weitergeleitet und zur
    Anmeldung aufgefordert. Ausgenommen davon ist die Hilfeseite zur Anmeldung und die
    Registrierungsseite.
    %in dieser allgemeinheit? alle seiten?+
    %sprachen optional fw
    \XXitem{/F020/}{funkt:020} Die Standardsprache des Systems ist abhängig von der im Browser
    eingestellten Sprache. Es werden Deutsch und Englisch angeboten.
    Sonst ist die Standardsprache Englisch. Die Sprache der Anwendung kann über die
    Fußzeile geändert werden.
    \XXitem{/F025/}{funkt:025} Im Header Befindet sich eine Miniaturansicht des Logos \todo{Link auf Produktdaten?}. Das Clicken dieses Logos führt auf die Startseite. Anonyme Nutzer werden dabei auf die Login Seite weitergeleitet.
    \XXitem{/F030/}{funkt:030} Auf den wichtigsten Seiten lassen sich von der Fußzeile aus
    Hilfetexte zu den angebotenen Funktionalitäten und der jeweiligen Rolle
    des Nutzers anzeigen. Hierzu öffnet sich ein neuer Tab.
    \XXitem{/F040}{funkt:040} Beim erstmaligen Aufruf der Anwendungen wird der Nutzer gebeten
    seine Cookieauswahl zu treffen. Selbst wenn er ablehnt soll die Anwendung
    für ihn funktionstüchtig bleiben.
    %Fehlerseite falls Ressource nicht ex. -> auch in angemeld. Nutzer darauf verweisen.
\end{description}

\subsubsection{Registrierung}
\begin{description}
    \XXitem{/F050/}{funkt:050} Ein anonymer Nutzer kann von der Anmeldeseite aus mittels eines
    Buttons auf die Registrierungsseite navigieren.
    \XXitem{/F060/}{funkt:060} Über ein Registrierungsformular wird der Nutzer zur Eingabe seiner
    Daten aufgefordert. Verlangt wird die Eingabe eines Benutzernamens und Passworts
    %doppelt? -> sicherheitsanforderungen. benutzername schon verwendet -> jsf validierung.
    %email regex validierung.
    %optionale daten
    sowie von Vor- und Nachname. Optional ist das Einfügen eines Avatarbildes. 
    \todo{Avatarbild in den Profileisntellungen und nicht beim Registrieren}
    Letztlich muss die Emailadresse
    angegeben werden. Durch das Absenden des Formulars wird der E-Mail Verifizierungsprozess
    gestartet.
    \XXitem{/FW065/}{funkt:065} Ein Nutzer kann in seinen Profileinstellungen seinen Arbeitgeber, sein Spezialgebiet und sein Geburtsdatum angeben.
    %link email-ver
    \XXitem{/F070/}{funkt:070} Nach der Registrierung wird eine automatisierte E-Mail
    an die angegebene Mailadresse gesendet. Der Inhalt der Nachricht besteht aus einem Hinweis auf
    die versuchte Registrierung sowie einem Verifizierungslink. Dieser führ auf eine Seite, die
    die erfolgreiche Registrierung bestätigt. Der Account ist nun registriert und der Nutzer wird auf die Startseite
    weitergeleitet.
\end{description}

\subsubsection{Login}
\begin{description}
    \XXitem{/F080/}{funkt:080} Mittels eines Anmeldeformulars erfolgt eine Anmeldung durch korrekte Zugangsdaten.
    Diese umfassen den Benutzernamen und das Passwort. Ein anonymer Nutzer wird so zum angemeldeten Benutzer.
    % Passwort vergessen.
\end{description}

\subsection{Angemeldeter Nutzer}
Angemeldete Nutzer haben Zugriff auf die Funktionen
\hyperref[funkt:010]{/F010/}, \hyperref[funkt:020]{/F020/}, \hyperref[funkt:030]{/F030/}.
Nach der Anmeldung \hyperref[funkt:080]{/F080/},
stehen außerdem folgende weitere Funktionen zur Verfügung.

\subsubsection{Allgemein}
\begin{description}
    \XXitem{/F100}{funkt:100} Bei Zugriff auf die Anmelde- oder Registrationsseite
    wird auf die Startseite weitergeleitet.
    \XXitem{/F110}{funkt:110} Über einen Button in der Kopfzeile kann ein Logout durchgeführt werden.
    Der nun anonyme Nutzer wird auf die Anmeldeseite weitergeleitet.
    \XXitem{/F120/}{funkt:120} Die Fußzeile ermöglicht eine Navigation auf das Impressum.
    %%url def?
\end{description}

\subsubsection{globale suche}
%% publ alle
% nutzer admin, editor
%papers, admin, nutzer eigene
%editoren & gutachter zugeordnete papers

\subsubsection{Profil}
\begin{description}
    \XXitem{}{} Über die Kopfzeile kann sich ein Benutzer zu seinem Profil navigieren.
    \XXitem{}{} Auf der Profilseite kann der Nutzer alle dynamischen Daten über dieses Profil einsehen,
    Eine explizite Ausnahme hiervon ist das gehashte Passwort. %link /D/
    \XXitem{/F130/}{funkt:130} Auf der eigenen Profilseite kann der Nutzer jedes dieser Daten %link /D/
    welche über ihn gespeichert sind permanent verändern.
    \XXitem{/F140/}{funkt:140} Bei Änderung der Mailadresse wird der E-Mailverifikationsprozess erneut
    begonnen. %link
    \XXitem{}{} Der Nutzer kann auf der eigenen Profilseite ein Avatarbild mit einer maximalen
    Größe von 4MB hochladen oder sein altes Avatarbild entfernen oder austauschen. %link /D/
    \XXitem{}{} Auf der eigenen Profilseite kann der Nutzer sein Profil und alle damit verbundenen persistenten
    Daten löschen. Auch seine Einreichungen werden gelöscht und Gutachter sowie Editoren dieser
    Einreichung per Mail informiert. Bevor die Löschung vollzogen wird, wird dem Nutzer
    eine Warnung über diese Konsequenzen angezeigt.
\end{description}

\subsubsection{Startseite}
\begin{description}
    \XXitem{}{} Die Startseite ist zu jeder Zeit über die Kopfzeile erreichbar.
    \XXitem{/F130/}{funkt:130} Ein Nutzer bekommt auf der Startseite alle Namen von Konferenzen
    \& Journale in einer Listensicht angezeigt in denen er aktive Einreichungen hat.
    Die Namen dieser aktiven Einreichungen werden unter den Namen der Konferenzen \& Journale angezeigt.
    \XXitem{/F140/}{funkt:140} Die Einreichungen lassen sich nach Namen und Datum
    der Einreichung sortieren. Die Konferenzen lassen sich nach ihrem Namen sortieren.
    %suchfunktion?
    \XXitem{}{} Durch den Klick auf einen Eintrag der Liste gelangt man auf die jeweilige Übersichtsseite
    der Einreichung oder auf die Seite des jew. wissenschaftlichen Forums.
%info über benachrichtigungen? -> Benachrichtigungssystem? nur mail?
\end{description}

\subsubsection{Liste der Journals \& Konferenzen}
\begin{description}
    %% ergebnisse globale suche
    \XXitem{/F130/}{funkt:130} Über eine Suchleiste kann nach bestimmten Journals oder Konferenzen mittels
    ihres Namens gesucht werden.
    \XXitem{/F140/}{funkt:140} Die Ergebnisse lassen sich sortieren nach %produktdaten journals & kongresse, eig nur Name oder?
    \XXitem{/F150/}{funkt:150} Durch Klick auf den Namen eines Eintrags wird man auf die Seite des
    jeweiligen Journals oder der jeweiligen Konferenz gelenkt.
\end{description}

\subsubsection{Publikationseite}
\begin{description}
    \XXitem{/F130/}{funkt:130} Auf der Seite eines wissensch. Forums werden die zugehörigen Informationen
    angezeigt. %link /D/
    \XXitem{/F140/}{funkt:140} Dem Nutzer werden seine eigenen Einreichungen in Form einer Liste mit Namen angezeigt.
    \XXitem{/F160/}{funkt:160} Durch Klick auf den Namen einer Einreichung gelangt der Nutzer auf die Übersichtsseite
    dieser Einreichung.
    \XXitem{/F170/}{funkt:170} Die Einträge lassen sich nach Namen und Datum
    der Einreichung sortieren und nach Namen der Einreichung durchsuchen. %durchsuchen meinung?
    \XXitem{}{} Der Nutzer kann auf die Seite zur Erstellung einer Einreichung navigieren. Hierbei ist
    das Feld, welches die Konferenz oder das Journal bestimmt bei dem eingereicht wird, bereits mit
    der Konferenz bzw. dem Journal befüllt von dessen Übersichtsseite aus die Navigation auf diese
    Seite ausgeführt wurde.
\end{description}

\subsubsection{Erstellen einer Einreichung}
\begin{description}
    \XXitem{}{} Der Nutzer kann eine Einreichung im System erstellen. Hierzu gibt er in einem
    Formular die nötigen Informationen wie Name der Einreichung, Namen und E-Mail Adressen der Ko-Autoren,
    sowie den gewünschten Editor angeben.
    %informieren des editors
    \XXitem{}{} Der Nutzer lädt seine Einreichung in Form einer PDF hoch. Die Abgabe darf eine Dateigröße
    von 20MB nicht überschreiten und muss zwingenderweise im PDF-Format erfolgen.
    \XXitem{}{} Durch Absenden des Formulars wird der Editor des Journals bzw. der Konferenz
    informiert. Das Datum der Einreichung wird auf das Datum zum Zeitpunkt der Einreichung fest-
    gelegt.
    \XXitem{}{} Die Einreichung ist erfolgreich, wenn alle Felder ausgefüllt sind und eine PDF
    hochgeladen wurde. Andernfalls wird der Nutzer über das fehlschlagen informiert.
    \XXitem{}{} Nach der erfolgreichen Einreichungen wird der Nutzer auf die Übersichtsseite der
    Einreichung weitergeleitet.
    %zugriff: navbar, startseite, publikationsseite
    %FW Vorschlag von Gutachtern
\end{description}

\subsubsection{Einreichungsseite}
\begin{description}
    \XXitem{/F130/}{funkt:130} Dem Nutzer werden Informationen zu seiner Einreichung angezeigt.
    Hierzu gehören der Status der Einreichung, das Datum der Einreichung, das zugehörige
    Journal- bzw Konferenz, Namen und E-Mail Adressen der Ko-Autoren, sowie ein Download zur Einreichung.
    \XXitem{}{} Außerdem werden die Gutachten in einer
    Liste dargestellt zusammen mit ihrem Erstellungsdatum, Gutachterempfehlung und Download. %link /D/
    \XXitem{}{} Die Gutachten lassen sich nach Namen und Datum
    des Gutachten sortieren und nach Namen des Gutachten durchsuchen. %suche optional, meinung?
%nav
\end{description}

\subsection{Gutachter}
Gutachter haben diesselben Funktion wie gewöhnliche angemeldete Nutzer. Zusätzlich hierzu kommen
die Folgenden:
%ohne /F/ einreichungsseite alle gutachten sehen

\subsubsection{Startseite}
\begin{description}
    \XXitem{}{} Dem Gutachter werden auf seiner personalisierten Startseite zusätzlich zu den eigenen
    Einreichungen und zugehörigen Konferenzen \& Journalen diejenigen angezeigt für die er als Gutachter
    zugeordnet ist.
    Für diese Einträge gelten dieselben Funktionalitäten wie für die eigenen Einreichungen.
\end{description}

\subsubsection{Publikationsseite}
\begin{description}
    \XXitem{}{} Dem Gutachter werden zusätzlich zu den eigenen Einreichungen diejenigen Einreichungen angezeigt,
    welchen er als Gutachter zugeordnet ist. Für diese Einträge gelten dieselben Funktionalitäten wie für die
    eigenen Einreichungen. %link
\end{description}

\subsubsection{Einreichungsseite}
\begin{description}
    \XXitem{}{} Der Gutachter sieht auf der Übersichtseite einer Einreichung welcher als Gutachter
    zugeordnet ist, diejenigen Gutachten welche er selbst erstellt hat in einer
    Liste mit ihrem Erstellungsdatum, Gutachterempfehlung und Download.
    \XXitem{}{} Der Gutachter hat zusätzlich die Möglichkeit zu einer Einreichung der er als Gutachter zugeordnet ist
    ein Gutachten mittels eines Formulars einzureichen. Hierfür ist eine PDF hoc
    \XXitem{FW}{} Auf der Einreichungsseite von Einreichungen denen der Gutachter zugeordnet ist,
    kann er in der Liste eigene eingereichte Gutachten zurückziehen. Hieraufhin werden sie aus
    der Datenbank entfernt und nicht mehr angezeigt.
\end{description}

\subsection{Editor}
Editoren haben Zugriff auf alle Funktionen welche angemeldeten Nutzern zur Verfügung stehen.
Außerdem hat ein Editor die Folgenden zusätzlichen Funktionalitäten:
%keine Editierung von eigenen Einreichungen.

\subsubsection{Startseite}
\begin{description}
    \XXitem{}{} Dem Editor werden auf seiner personalisierten Startseite zusätzlich zu den eigenen
    Einreichungen und zugehörigen Konferenzen \& Journalen diejenigen Konferenzen \& Journale in einer
    Liste angezeigt welchen er als Editor zugeordnet ist.
    Für diese Einträge gelten dieselben Funktionalitäten wie für die eigenen Einreichungen. %link
\end{description}

\subsubsection{Publikationsseite}
\begin{description}
    \XXitem{}{} Einem Editor werden auf der Publikationsseite einer Konferenz bzw. eines Journals für
    welches er als Editor fungiert alle aktiven Einreichungen in einer Liste dargestellt.
    Für diese Einträge gelten dieselben Funktionalitäten wie für die eigenen Einreichungen. %link
    %%editoren ernennen/aberkennen
    %%nutzern editorenn auf dieser seite als ansprechpartner anzeigen
\end{description}

\subsubsection{Einreichungsseite}
\begin{description}
    \XXitem{}{} Ein Editor kann auf der Seiter einer Einreichung, welcher er als Editor zugeordnet ist,
    in einem Formular Gutachter zuweisen. Hierzu gibt er deren E-Mail Adressen an.
    %FW Nutzersuche (registrierte Nutzerliste durchsuchbar)
    \XXitem{}{} Wird eine E-Mail Adresse als Gutachter angegeben, so wird nach Absenden des Formulars
    eine automatisierte Mail an diese Adresse versendet. Sie enthält eine Nachricht mit den relevanten
    Informationen zu Einreichung, Konferenz bzw. Journal, sowie
    \begin{itemize}
        \item Einen Link zur \emph{Annahme der Begutachtungsanfrage} welcher, sobald geklickt,
        auf die Loginseite verweist aud der eine Nachricht des Dankes anzeigt und zur Anmeldung bzw.
        Registrierung auffordert.
        \item Einen Mailto-Link zur \emph{Ablehnung der Begutachtungsanfrage} welcher, sobald
        geklickt, einen Mailentwurf öffnet mit vorausgefülltem Empfänger (zugehöriger Editor)
        und einem Infotext in welchen ein Ablehnungsgrundes eingefügt werden kann.
        Abgesehen hiervon wird zusätzlich eine automatisierte Mail an den Editor versendet in der er
        kurz über die Ablehnung informiert wird, falls die obige E-Mail nie versendet wird.
    \end{itemize}
\end{description}

\subsection{Administrator}
%edit user profiles

