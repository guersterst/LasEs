%todo Erstellen eines Nutzerprofils als admin? -> FW
%todo längere einträge in listen aufsplitten
%todo suchen jeweils mit dropdown
%todo ablehnung -> aus Forum entfernt -> auf startseite des nutzers zu sehen
%todo nutzer einreichung zurücknehmen
\localauthor{Johannes Garstenauer}

Die Funktionalität vom LasEs-System soll nach den Benutzerrollen
%link
\textit{anonymer Nutzer}, \textit{angemeldeter Nutzer}, \textit{Editor}, und
\textit{Administrator} untergliedert werden. Da ein Nutzer mehrere Rollen einnehmen
kann bestehen Inklusionsbeziehungen die im Folgenden näher erklärt werden.

\subsection{Anonymer Nutzer}
Anonyme Nutzer sind unauthentifzierte Nutzer, deren Zugriffsrechte sich
auf die Registrierung und Anmeldung im System beschränken.

\subsubsection{Allgemein}
\begin{description}
    \XXitem{/F010/}{funkt:010} Beim Aufruf einer Seite durch einen nichtangemeldeten Benutzer
    wird dieser auf die Anmeldeseite weitergeleitet und zur
    Anmeldung aufgefordert. Ausgenommen davon ist die Hilfeseite zur Anmeldung und die
    Registrierungsseite.
    \XXitem{/FW020/}{funkt:020} Die Standardsprache des Systems ist abhängig von der im Browser
    eingestellten Sprache. Es werden Deutsch und Englisch angeboten.
    Sonst ist die Standardsprache Englisch. Die Sprache der Anwendung kann über die
    Fußzeile geändert werden.
    \XXitem{/F030/}{funkt:030} Auf den wichtigsten Seiten lassen sich von der Fußzeile aus
    Hilfetexte zu den angebotenen Funktionalitäten und der jeweiligen Rolle
    des Nutzers anzeigen. Hierzu öffnet sich ein neuer Tab.
    \XXitem{/F040/}{funkt:040} Ist eine Ressource über eine URL nicht erreichbar (z.B. weil sie nicht existiert/
     die URL fehlerhaft ist) wird eine Fehlerseite angezeigt.%in ang. nutzer darauf verweisen
\end{description}

\subsubsection{Registration}
\begin{description}
    \XXitem{/F050/}{funkt:050} Ein anonymer Nutzer kann von der Anmeldeseite aus mittels eines
    Buttons auf die Registrierungsseite navigieren.
    \XXitem{/F060/}{funkt:060} Über ein Registrierungsformular wird der Nutzer zur Eingabe seiner
    Daten aufgefordert. Verlangt wird die Eingabe eines Passworts,
    sowie von Vor- und Nachname. Optional ist das Einfügen eines Avatarbildes. Letztlich muss die Emailadresse
    angegeben werden, welche einzigartig im System sein muss. Durch das Absenden des Formulars wird der E-Mail Verifizierungsprozess
    gestartet.
    %link email-ver
    \XXitem{/F070/}{funkt:070} Nach der Registrierung wird eine automatisierte E-Mail
    an die angegebene Mailadresse gesendet. Der Inhalt der Nachricht Inhalt einen Hinweis auf
    die versuchte Registrierung sowie einen Verifizierungslink, welcher auf eine Seite führt, welche
    die erfolgreiche Registrierung bestätigt. Der Account ist nun registriert. Nach einem Augenblick wird der Nutzer
    auf die Startseite weitergeleitet.
\end{description}

\subsubsection{Anmeldung}
\begin{description}
    \XXitem{/F080/}{funkt:080} Mittels eines Anmeldeformulars erfolgt eine Anmeldung durch korrekte Zugangsdaten.
    Diese umfassen die Mailadresse und das Passwort. Ein anonymer Nutzer wird so zum angemeldeten Benutzer.
    Nach erfolgreicher Anmeldung erfolgt eine Weiterleitung auf die Startseite.
    \XXitem{/F090/}{funkt:090} Von der Anmeldeseite ist es möglich zur Registrierungsseite zu gelangen.
    \XXitem{/F100/}{funkt:100} Über einen Klick auf das Logo der Anwendung gelangt der Nutzer jederzeit
    auf die Seite zur Anmeldung.
\end{description}

\subsection{Angemeldeter Nutzer}
Angemeldete Nutzer haben Zugriff auf die Funktionen
\hyperref[funkt:010]{/F110/}, \hyperref[funkt:020]{/F020/}, \hyperref[funkt:030]{/F030/}.
Nach der Anmeldung \hyperref[funkt:080]{/F080/},
stehen außerdem folgende weitere Funktionen zur Verfügung.

\subsubsection{Allgemein}
\begin{description}
    \XXitem{/F120/}{funkt:120} Bei Zugriff auf die Anmelde- oder Registrationsseite
    wird auf die Startseite weitergeleitet.
    \XXitem{/F130}{funkt:130} Über einen Button in der Kopfzeile kann ein Logout durchgeführt werden.
    Der nun anonyme Nutzer wird auf die Anmeldeseite weitergeleitet.
    \XXitem{/F140/}{funkt:140} Die Fußzeile ermöglicht eine Navigation auf das Impressum.

    \XXitem{/F150/}{funkt:150} Über einen Klick auf das Logo der Anmeldung gelangt ein angemeldeter Nutzer auf die
    Startseite.
\end{description}

\subsubsection{Suche}
\begin{description}
    \XXitem{/F160/}{funkt:160} Über die globale Suche kann ein Nutzer jederzeit seine eigenen Papers und nach wissenschaftlichen Foren,
    suchen. Nach Absenden der Suche wird eine Resultatliste angezeigt. Für Papers werden Name, Datum und Status, für
    wissenschaftliche Foren nur der Name abgebildet. %link /D/
    Sie sind nach Namen sortiert.
    \XXitem{/F170/}{funkt:170} Alle Einträge können anhand der angezeigten Informationen sortiert werden.
    Mit einem Klick auf einen Eintrag wird der Nutzer auf die jeweilige Ansichtseite
    der Ressource navigiert.
    \XXitem{/F180/}{funkt:180} Während der Eeingabe in das Suchfeld werden bis zu 10 mögliche Suchergebnisse in
    einem Dropdown Menü angezeigt.
\end{description}

\subsubsection{Profil}
\begin{description}
    \XXitem{/F190/}{funkt:190} Über die Kopfzeile kann sich ein Benutzer zu seinem Profil navigieren.
    \XXitem{/F200/}{funkt:200} Auf der Profilseite kann der Nutzer alle dynamischen Daten über dieses Profil einsehen,
    Eine Ausnahme hiervon ist das gehashte Passwort. %link /D/
    \XXitem{/F205/}{funkt:205} Auf der eigenen Profilseite kann der Nutzer jedes Datum %link /D/
    welches über ihn gespeichert ist persistent verändern.
    \XXitem{/F210/} Bei Änderung der Mailadresse wird der E-Mailverifikationsprozess erneut
    begonnen. Die Mailadresse muss im System einzigartig sein. %link
    \XXitem{/F220/}{funkt:220} Der Nutzer kann auf der eigenen Profilseite ein Avatarbild mit einer maximalen
    Größe von 4MB hochladen oder sein altes Avatarbild entfernen oder austauschen. %link /D/
    \XXitem{/F230/}{funkt:230} Auf der eigenen Profilseite kann der Nutzer sein Profil und alle damit verbundenen persistenten
    Daten löschen. Auch seine Einreichungen werden gelöscht und Gutachter sowie Editoren dieser
    Einreichung per Mail informiert. Bevor die Löschung vollzogen wird, wird dem Nutzer
    eine Warnung über diese Konsequenzen angezeigt.
    \XXitem{/FW240/}{funkt:240} Der Nutzer kann außerdem seinen Arbeitsgeber, ein oder mehrere Spezialgebiete
    und sein Geburtsdatum angeben.
\end{description}

\subsubsection{Startseite}
\begin{description}
    \XXitem{/F250/}{funkt:250} Die Startseite ist zu jeder Zeit über die Kopfzeile erreichbar.
    \XXitem{/F260/}{funkt:260} Ein Nutzer bekommt auf der Startseite alle Namen von wissenschaftlichen Foren
    in einer Listensicht angezeigt in denen er aktive Einreichungen hat.
    Die Namen, das Datum und der Status dieser aktiven Einreichungen werden unter den Namen der wissenschaftlichen
    Foren angezeigt.
    \XXitem{/F270/}{funkt:270} Die Einreichungen lassen sich nach Namen und Datum und Status
    der Einreichung sortieren. Die wisenschaftlichen Foren lassen sich nach ihrem Namen sortieren.
    \XXitem{/FW280/}{funkt:280} Die Listen lassen sich nach den Namen der Einträge durchsuchen.
    \XXitem{/F290/}{funkt:290} Durch den Klick auf den Namen eines Eintrags der Liste gelangt man auf die jeweilige Übersichtsseite
    der Einreichung oder auf die Seite des jew. wissenschaftlichen Forums.
\end{description}

\subsubsection{Liste der wissenschaftlichen Foren}
\begin{description}
    \XXitem{/F300/}{funkt:300} In einer Liste werden die Namen von wissenschaftlichen Foren angezeigt.
    \XXitem{/F310/}{funkt:310} Über eine Suchleiste kann nach bestimmten wissenschaftlichen Foren mittels
    ihres Namens gesucht werden.
    \XXitem{/F320/}{funkt:320} Die Einträge lassen sich ahand ihres Namens alphabetisch sortieren.
    \XXitem{/FW330/}{funkt:330} Die Einträge lassen sich anhand des Namens durchsuchen.
    \XXitem{/F340/}{funkt:340} Durch einen Klick auf den Namen eines Eintrags wird man auf die Seite des
    jeweiligen wissenschaftlichen Forums navigiert.
\end{description}

\subsubsection{Wissenschaftliches Forum}
\begin{description}
    \XXitem{/F350/}{funkt:350} Auf der Seite eines wissensch. Forums werden die zugehörigen wesentlichen Daten
    angezeigt. %link /D/
    \XXitem{/F360/}{funkt:360} Dem Nutzer werden seine eigenen Einreichungen in Form einer Liste mit Namen, Datum und Status
    angezeigt.
    \XXitem{/F370/}{funkt:370} Durch einen Klick auf den Namen einer Einreichung gelangt der Nutzer auf die Übersichtsseite
    dieser Einreichung.
    \XXitem{/F380/}{funkt:380} Die Einträge lassen sich nach Namen, Datum und Status
    der Einreichung sortieren.
    \XXitem{/FW390/}{funkt:390} Die Einträge lassen sich anhand ihres Namens alphabetisch sortieren.
    \XXitem{/F400/}{funkt:400} Der Nutzer kann auf die Seite zur Erstellung einer Einreichung navigieren. Hierbei ist
    das Feld, welches die wissenschaftliche Forum bestimmt wird bei dem eingereicht wird, bereits mit
    dem wissenschaftlichen Forum befüllt von dessen Übersichtsseite aus die Navigation auf diese
    Seite ausgeführt wurde.
\end{description}

\subsubsection{Einreichungserstellung}
\begin{description}
    \XXitem{/F410/}{funkt:410} Der Nutzer kann eine Einreichung im System erstellen. Hierzu gibt er in einem
    Formular die nötigen Informationen wie Namen der Einreichung, Namen und E-Mail Adressen der Ko-Autoren,
    sowie den gewünschten Editor angeben. Der Editor wird nach erfolgreicher Erstellung hierüber durch eine
    automatisierte Mail informiert.
    \XXitem{/F420/}{funkt:420} Der Nutzer lädt seine Einreichung in Form einer PDF hoch. Die Abgabe darf eine Dateigröße
    von 20MB nicht überschreiten und muss zwingenderweise im PDF-Format erfolgen.
    \XXitem{/FW430/}{funkt:430} Der Nutzer kann bei Einreichung gewünschte Gutachter vorschlagen.
    \XXitem{/F440/}{funkt:440} Durch Absenden des Formulars wird der Editor des Journals bzw. der Konferenz
    informiert. Das Datum der Einreichung wird auf das Datum zum Zeitpunkt der Einreichung fest-
    gelegt.
    \XXitem{/F450/}{funkt:450} Die Einreichung ist erfolgreich, wenn alle Felder ausgefüllt sind und eine PDF
    hochgeladen wurde. Andernfalls wird der Nutzer über das fehlschlagen informiert.
    \XXitem{/F460/}{funkt:460} Nach der erfolgreichen Einreichungen wird der Nutzer auf die Übersichtsseite der
    Einreichung weitergeleitet.
\end{description}

\subsubsection{Einreichung}
\begin{description}
    \XXitem{/F470/}{funkt:470} Dem Nutzer werden Informationen zu seiner Einreichung angezeigt.
    Hierzu gehören der Status der Einreichung, das Datum der Einreichung, das zugehörige
    Journal- bzw Konferenz, Namen und E-Mail Adressen der Ko-Autoren, sowie ein Download zur Einreichung.
    \XXitem{/F480/}{funkt:480} Außerdem werden die Gutachten in einer
    Liste dargestellt zusammen mit ihrem Erstellungsdatum, Gutachterempfehlung und Download. %link /D/
    \XXitem{} Die Gutachten lassen sich nach Namen und Datum
    des Gutachten sortieren
    \XXitem{/FW490/}{funkt:490}... und nach Namen des Gutachten durchsuchen.
%nav
\end{description}

\subsection{Gutachter}
Gutachter haben diesselben Funktion wie gewöhnliche angemeldete Nutzer. Zusätzlich hierzu kommen
die Folgenden:
%ohne /F/ einreichungsseite alle gutachten sehen

\subsubsection{Startseite}
\begin{description}
    \XXitem{/F500/}{funkt:500} Dem Gutachter werden auf seiner personalisierten Startseite zusätzlich zu den eigenen
    Einreichungen und zugehörigen wissenschaftlichen Foren diejenigen angezeigt für die er als Gutachter
    zugeordnet ist. Diese sind als solche gekennzeichnet.
    Für sie gelten dieselben Funktionalitäten wie für eigene Einreichungen. %link
\end{description}

\subsubsection{Suche}
\begin{description}
    \XXitem{/F510/}{funkt:510} Ein Gutachter kann ebenfalls Einreichungen finden, welcher er als Gutachter
    zugeordnet ist. Diese sind als solche gekennzeichnet.
\end{description}

\subsubsection{Wissenschaftliches Forum}
\begin{description}
    \XXitem{/F520/}{funkt:520} Dem Gutachter werden zusätzlich zu den eigenen Einreichungen diejenigen Einreichungen angezeigt,
    welchen er als Gutachter zugeordnet ist. Für diese Einträge gelten dieselben Funktionalitäten wie für die
    eigenen Einreichungen. %link
\end{description}

\subsubsection{Einreichung}
\begin{description}
    \XXitem{/F530/}{funkt:530} Der Gutachter sieht auf der Übersichtseite einer Einreichung welcher als Gutachter
    zugeordnet ist, diejenigen Gutachten welche er selbst erstellt hat in einer
    Liste mit ihrem Erstellungsdatum, Gutachterempfehlung und Download.
    \XXitem{/F540/}{funkt:540} Der Gutachter hat zusätzlich die Möglichkeit zu einer Einreichung der er als Gutachter zugeordnet ist
    ein Gutachten mittels eines Formulars einzureichen. Hierfür ist eine PDF hoc
    \XXitem{/FW550/}{funkt:550} Auf der Einreichungsseite von Einreichungen denen der Gutachter zugeordnet ist,
    kann er in der Liste eigene eingereichte Gutachten zurückziehen. Hieraufhin werden sie aus
    der Datenbank entfernt und nicht mehr angezeigt.
    \XXitem{/F560/}{funkt:560} Der Einreicher und Editor werden über neue oder entfernte Gutachten mit einer automatisierten
    Mail informiert.
    %einreicher & editor über gutachten informieren
\end{description}

\subsection{Editor}
Editoren haben Zugriff auf alle Funktionen welche angemeldeten Nutzern zur Verfügung stehen.
Außerdem hat ein Editor die Folgenden zusätzlichen Funktionalitäten:
%keine Editierung von eigenen Einreichungen.
%Hat Profilsichtrechte von Nutzer.

\subsubsection{Startseite}
\begin{description}
    \XXitem{/F570/}{funkt:570} Dem Editor werden auf seiner personalisierten Startseite zusätzlich zu den eigenen
    Einreichungen und zugehörigen wissenschaftlichen Foren diejenigen in einer
    Liste angezeigt welchen er als Editor zugeordnet ist. Sie sind als solche gekennzeichnet.
    Für diese Einträge gelten dieselben Funktionalitäten wie  %link
\end{description}

\subsubsection{Suche}
\begin{description}
    \XXitem{/F580/}{funkt:580} Ein Editor kann ebenfalls Einreichungen finden, welchen er als Editor zugeordnet ist
    Diese sind als solche gekennzeichnet.
    \XXitem{/F590/}{funkt:590} Ein Editor kann ebenfalls Einträge zu allen Nutzern finden.
\end{description}

\subsubsection{Benutzer}
\begin{description}
    \XXitem{/F600/}{funkt:600} Ein Editor kann auf die Nutzerliste über die Kopfzeile zugreifen.
    \XXitem{/F610/}{funkt:610} Hier werden ihm alle Nutzer mit Namen und E-Mail übersichtlich in einer Liste angezeigt.
    \XXitem{/F620/}{funkt:620} Diese Liste kann alphabetisch nach Namen und Mailaddresse sortiert werden.
    \XXitem{/FW630/} Die Liste kann anhand von Namen und Mailadresse durchsucht werden.
    \XXitem{/F640/}{funkt:640} Mit einem Klick auf einen Eintrag wird der Editor auf das zugehörige Profil navigiert.
\end{description}

\subsubsection{Wissenschaftliches Forum}
\begin{description}
    \XXitem{/F650/}{funkt:650} Einem Editor werden auf der Publikationsseite einer Konferenz bzw. eines Journals für
    welches er als Editor fungiert alle aktiven Einreichungen in einer Liste dargestellt.
    Solche bei denen er als Editor eingesetzt wird werden als solche gekennzeichnet.
    Für diese Einträge gelten dieselben Funktionalitäten wie für die eigenen Einreichungen. %link
    \XXitem{/F660/}{funkt:660} Ein Editor kann andere Editoren ernennen indem er sie mittels ihrer Mailadresse identifiziert.
    Diese müssen bereits als Nutzer regitriert sein.
    \XXitem{/F670/}{funkt:670} Ein Editor kann anderen Editoren den Status als Editor aberkennen.
\end{description}

\subsubsection{Einreichung}
\begin{description}
    \XXitem{/F680/}{funkt:680} Ein Editor kann auf der Seiter einer Einreichung, welcher er als Editor zugeordnet ist,
    in einem Formular Gutachter zuweisen. Hierzu gibt er deren E-Mail Adressen an.
    %FW Nutzersuche (registrierte Nutzerliste durchsuchbar)
    \XXitem{/F690/}{funkt:690} Wird eine E-Mail Adresse als Gutachter angegeben, so wird nach Absenden des Formulars
    eine automatisierte Mail an diese Adresse versendet. Sie enthält eine Nachricht mit den relevanten
    Informationen zu Einreichung, wissenschaftlichem Forum, sowie
    \begin{itemize}
        \item ... einen Link zur \textbf{Annahme der Begutachtungsanfrage} welcher, sobald geklickt,
        auf die Loginseite verweist aud der eine Nachricht des Dankes anzeigt und zur Anmeldung bzw.
        Registrierung auffordert.
        \item ... einen Mailto-Link zur \textbf{Ablehnung der Begutachtungsanfrage} welcher, sobald
        geklickt, einen Mailentwurf öffnet mit vorausgefülltem Empfänger (zugehöriger Editor)
        und einem Infotext in welchen ein Ablehnungsgrundes eingefügt werden kann.
        Abgesehen hiervon wird zusätzlich eine automatisierte Mail an den Editor versendet in der er
        kurz über die Ablehnung informiert wird, falls die obige E-Mail nie versendet wird.
    \end{itemize}
    \XXitem{/F700/}{funkt:700} Ein Editor kann über Einreichungen eine Annahmeentscheidung treffen. Über diese werden
    beteiligte Gutachter, der Einreicher und beteiligte Ko-Autoren per automatisierter Mail benachrichtigt.
\end{description}

\subsection{Administrator}
Der Administrator besitzt zu Verwaltungszwecken alle Rechte, welche auch Nutzern, Editoren
und Gutachtern zustehen. %Genau genug definiert?
Einem Administrator werden in Listen grundsätzlich alle aktiven Einträge angezeigt.
%UserList editor linken
%Das ausgliedern in /F/s? oder einfach verlinken der /F/s? -> oder von fall zu fall je nachdem
%/F/ hier aussagekräftig ist.
% Einreichungsseite selbe Recht wie Gutachter oder Editor zu verwaltungsszwecken.
\subsubsection{Suche}
\begin{description}
    \XXitem{/F710/}{funkt:710} Ein Administrator kann in der globalen Suche ebenfalls alle Nutzer finden.
    \XXitem{/F720/}{funkt:720} Ein Administrator kann alle vorhandenen Einreichungen finden.
\end{description}

\subsection{Wissenschaftliches Forum}
\begin{description}
    \XXitem{/F730/}{funkt:730} Einem Administrator werden auf der Publikationsseite einer Konferenz bzw. eines Journals
    alle aktiven Einreichungen in einer Liste dargestellt.
    Für diese Einträge gelten dieselben Funktionalitäten wie für eigene Einreichungen. %link
\end{description}

\subsubsection{Profil}
\begin{description}
    \XXitem{/F740/}{funkt:740} Ein Administrator kann einen anderen Nutzer auf dessen Profilseite zum Administrator ernennen.
    \XXitem{/F750/}{funkt:750} Ein Administrator kann einem anderen Administrator auf dessen Profilseite seine
    Administratorrolle aberkennen.
    \XXitem{/FW760/}{funkt:760} Vor dem An- oder Aberkennen von Administratorrechten ist eine gültige
    Passworteingabe erforderlich.
    \XXitem{/F770/}{funkt:770} Der Administrator besitzt auf allen Profilseiten dieselben Rechte zur Änderung
    der persistierten Daten wie ein Nutzer auf seiner eigenen Profilseite. %link
\end{description}

\subsection{Liste der Wisssenschaftlichen Foren}
\begin{description}
    \XXitem{/F780/}{funkt:780} Auf dieser Seite kann der Admin zur Seite navigieren auf welcher er ein neues
    wissenschaftliches Forum anlegen kann. %link
\end{description}

\subsubsection{Erstellung wissenschaftlicher Foren}
\begin{description}
    \XXitem{/F790/}{funkt:790} Auf der Seite zum Erstellen eines wissenschaftlichen Forums kann der Administrator dessen
    wesentliche Daten nach /D/ %link
    festlegen. Bei der Anngabe von Editoren wird überprüft, dass diese bereits als Nutzer im System registriert sind.
    Der Name des Forums muss ebenfalls einzigartig sein.
    Nach erfolgreicher Erstellung wird der Administrator auf die Seite dieses wissenschaftlichen Forums
    navigiert.
\end{description}

\subsection{Wissenschaftliches Forum}
\begin{description}
    \XXitem{/F800/}{funkt:800} Auf der Seite eines wissenschaftlichen Forums kann der Administrator alle wesentlichen Daten
    verändern. Insbesondere erfolgt hierbei die Ernennung von Editoren, welche bereits im System
    registriert sein  müssen. Der Name des weissenschaftlichen Forums muss einzigartig sein.
    \XXitem{/FW810/}{funkt:810} Der Administrator kann den Look des wissenschaftlichen Forums durch die folgenden Daten
    verändern. %link
    \XXitem{/F820/}{funkt:820} Auf der Seite eines wissenschaftlichen Forums kann der Administrator diese aus dem System zu löschen.
    Hierbei wird er dazu aufgefordert seine Entscheidung ein zweites Mal zu bestätigen.
    Daraufhin werden alle zugehörigen Daten /D/ und Einreichungen aus dem System entfernt. %link /D/
    %link
\end{description}

\subsection{Konfiguration}
\begin{description}
    \XXitem{/F830/}{funkt:830} Ein Administrator kann auf der Konfigurationsseite den vom Betreiber gewünschten
    'Look and Feel' des Systems festlegen. Hierzu bestimmt der die Daten wie in /D/ definiert. %link
\end{description}