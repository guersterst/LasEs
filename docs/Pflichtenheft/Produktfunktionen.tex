\localauthor{Johannes Garstenauer}

Die Funktionalität vom LasEs-System wird nach den Benutzerrollen
\textit{anonymer Nutzer}, \textit{angemeldeter Nutzer}, \textit{Gutachter}, \textit{Editor}, und
\textit{Administrator} untergliedert.

Es gilt darüber hinaus, dass alle Funktionen eines einfachen \hyperref[mkrit:angemeldet]{angemeldeten Nutzers} auch den höherrangigen Benutzern, wie
\hyperref[mkrit:gutachter]{Gutachtern}, \hyperref[mkrit:editor]{Editoren} und \hyperref[mkrit:admin]{Administratoren} zur Verfügung stehen.
Ein Nutzer kann mehrere Rollen annehmen. Diese hierarchische Ordnung wird in Bemerkungen zu den jeweiligen Rollen im Folgenden
übersichtlich erklärt.
Die Funktionen werden nach dem Schema \texttt{/FXXX/} bzw. \texttt{/FWXXX/} für Wunschfunktionen benannt, wobei \texttt{XXX} eine dreistellige Ganzzahl ist.

\subsection{Anonymer Nutzer}\label{funkt:nutzer}
Anonyme Nutzer sind nicht authentifizierte Nutzer, deren Zugriffsrechte sich
auf die Registrierung, Anmeldung und Verifizierung im System beschränken.

\subsubsection{Allgemein}
\begin{description}
    \XXitem{/F010/}{funkt:010} Beim Aufruf einer der Anwendung zugeordneten URL durch einen nicht-angemeldeten Benutzer
    wird dieser auf die \hyperref[an:log]{Anmelde- und Willkommensseite} weitergeleitet. Ausgenommen davon ist die \hyperref[an:reg]{Registrierungsseite}.
    \XXitem{/FW020/}{funkt:020} Die Standardsprache des Systems ist abhängig von der im Web-Browser
    eingestellten bevorzugten Sprache. Es werden Deutsch und Englisch angeboten, standardmäßig ist Englisch ausgwählt.
    Die Sprache der Anwendung kann zusätzlich über die Fußzeile geändert werden. (\hyperref[leist:160]{/L160/})
    \XXitem{/F030/}{funkt:030} Auf jeder Seite lassen sich
    Hilfetexte \hyperref[leist:055]{/L055/} als Tooltips zu den angebotenen Funktionalitäten und der jeweiligen Rolle
    des Nutzers, sowie das Impressum anzeigen.
    \XXitem{/F040/}{funkt:040} Ist eine Ressource über eine URL nicht erreichbar (z.B. weil sie nicht existiert,
     die URL fehlerhaft ist oder Zugriffsrechte fehlen) wird eine Fehlerseite angezeigt.
    \XXitem{/F045/}{funkt:045} E-Mail Benachrichtigungen werden grundsätzlich immer zweisprachig,
    auf Deutsch und Englisch, versendet.
\end{description}

\subsubsection{Registration}\label{an:reg}
\begin{description}
    \XXitem{/F050/}{funkt:050} Ein anonymer Nutzer kann von der \hyperref[an:log]{Anmelde- und Willkommensseite} aus mittels eines
    Buttons auf die Registrierungsseite navigieren.
    \XXitem{/F060/}{funkt:060} Über ein Registrierungsformular wird der Nutzer zur Eingabe seiner
    Daten aufgefordert. Verlangt wird die Eingabe eines Passworts (\hyperref[leist:130]{/L130/}),
    sowie von Vor- und Nachname. Letztlich muss die E-Mail-Adresse
    sowie von Vor- und Nachname. Die Eingabe eines Titels ist optional. Letztlich muss die E-Mailadresse
    angegeben werden, welche einzigartig im System sein muss. Durch das Absenden des Formulars wird der E-Mail Verifizierungsprozess
    \hyperref[funkt:070]{/F070/} gestartet.
    \XXitem{/FW061/}{funkt:061} Optional ist das Einfügen eines Avatarbildes, siehe \hyperref[d015]{/DW015/},
    bei der Registration.
    \XXitem{/FW062/}{funkt:062} Weiterhin kann der Nutzer optional weitere Daten wie in \hyperref[funkt:240]{/FW240/} beschrieben angeben.
    \XXitem{/F070/}{funkt:070} Nach der Registrierung wird eine automatisierte E-Mail
    an die angegebene Mailadresse gesendet. Die Nachricht beinhaltet einen Hinweis auf
    die versuchte Registrierung, sowie einen Verifizierungslink, der eine Stunde gültig ist und auf die Verifzierungsseite führt.
    Damit ist die Registrierung erfolgreich abgeschlossen. Nach einem Augenblick wird der Nutzer von hier
    auf die \hyperref[nut:start]{Startseite} weitergeleitet. (\hyperref[leist:140]{/L140/})
\end{description}

\subsubsection{Anmeldung}\label{an:log}
\begin{description}
    \XXitem{/F080/}{funkt:080} Mittels eines Anmeldeformulars erfolgt eine Anmeldung durch korrekte Zugangsdaten.
    Diese umfassen die Mailadresse und das Passwort. (\hyperref[leist:120]{/L120/})
    Ein anonymer Nutzer wird so zum angemeldeten Benutzer.
    \XXitem{/F085/}{funkt:085} Nach erfolgreicher Anmeldung erfolgt eine Weiterleitung
    auf die \hyperref[nut:start]{Startseite}.
\end{description}

\subsection{Angemeldeter Nutzer}
Angemeldete Nutzer haben Zugriff auf die Funktionen anonymer Nutzer.
Es stehen außerdem folgende weitere Funktionen zur Verfügung.

\subsubsection{Allgemein}
\begin{description}
    \XXitem{/F130}{funkt:130} Über einen Button in der Kopfzeile kann ein Logout durchgeführt werden.
    Der nun anonyme Nutzer wird auf die \hyperref[an:log]{Anmelde- und Willkommensseite} weitergeleitet.
    \XXitem{/F150/}{funkt:150} Über einen Klick auf das Logo der Anmeldung gelangt ein angemeldeter Nutzer auf die
    \hyperref[nut:start]{Startseite}.
\end{description}

\subsubsection{Suche}
Umgesetzt nach \hyperref[leist:050]{/L050/}
\begin{description}
    \XXitem{/F160/}{funkt:160} Über die globale Suche kann ein Nutzer jederzeit seine eigenen Einreichungen per Namen und Ko-Autoren
    und nach \hyperref[glo:wissForum]{wissenschaftlichen Foren} per Namen oder Kategorie
    suchen. Nach Absenden der Suche werden Resultatlisten nach \hyperref[leist:40]{/L040/} angezeigt.
    Für Einreichungen werden Name, Zeitpunkt der Einreichung und Status, für wissenschaftliche Foren nur der Name und
    Kategorien abgebildet
    \XXitem{/FW180/}{funkt:180} Während der Eingabe in das Suchfeld werden bis zu 10 mögliche Suchergebnisse in
    einem Dropdown Menü angezeigt.
\end{description}

\subsubsection{Profil} \label{nut:profil}
\begin{description}
    \XXitem{/F190/}{funkt:190} Über die Kopfzeile kann sich ein Benutzer zu seinem Profil navigieren.
    \XXitem{/F200/}{funkt:200} Auf der Profilseite kann der Nutzer alle dynamischen Daten \hyperref[d010]{/D010/} über dieses Profil einsehen.
    Eine Ausnahme hiervon ist das gehashte Passwort.
    \XXitem{/F205/}{funkt:205} Auf der eigenen Profilseite kann der Nutzer jedes Datum \hyperref[d010]{/D010/},
    welches über ihn gespeichert ist, persistent verändern.
    \XXitem{/F210/}{funkt:210} Bei Änderung der Mailadresse wird der E-Mailverifikationsprozess \hyperref[funkt:070]{/F070/} erneut
    begonnen. Die Mailadresse muss im System einzigartig sein.
    \XXitem{/FW220/}{funkt:220} Der Nutzer kann auf der eigenen Profilseite ein Avatarbild \hyperref[d015]{/DW015/} mit einer maximalen
    Größe von 4MB (dieser Wert ist nicht konfigurierbar) hochladen oder sein altes Avatarbild entfernen oder austauschen.
    \XXitem{/F230/}{funkt:230} Auf der eigenen Profilseite kann der Nutzer sein Profil und alle damit verbundenen persistenten
    Daten löschen. Auch seine Einreichungen werden gelöscht und Gutachter, sowie Editoren dieser
    Einreichung per Mail automatisiert informiert, wenn der Status der Einreichung noch nicht abgeschlossen ist.
    Bevor die Löschung vollzogen wird, wird dem Nutzer eine Warnung über diese Konsequenzen angezeigt.
    \XXitem{/FW240/}{funkt:240} Der Nutzer kann außerdem seinen Arbeitgeber, ein oder mehrere Spezialgebiete
    und sein Geburtsdatum angeben und verändern, siehe \hyperref[d015]{/DW015}.
\end{description}

\subsubsection{Startseite} \label{nut:start}
\begin{description}
    \XXitem{/F250/}{funkt:250} Die Startseite ist zu jeder Zeit über das Logo in der Kopfzeile erreichbar.
    \XXitem{/F260/}{funkt:260} Ein Nutzer bekommt auf der Startseite alle Namen von \hyperref[glo:wissForum]{wissenschaftlichen Foren},
    für die er Einreichungen hat (d.h. Ersteller oder Ko-Autor), in einer Listensicht nach \hyperref[leist:40]{/L040/} angezeigt.
    Die Namen, das Datum und der Status dieser aktiven Einreichungen werden unter den Namen der wissenschaftlichen
    Foren angezeigt.
\end{description}

\subsubsection{Liste der wissenschaftlichen Foren}
\begin{description}
    \XXitem{/F300/}{funkt:300} In einer Liste werden die Namen und Kategorien von wissenschaftlichen Foren nach \hyperref[leist:40]{/L040/}
    angezeigt.
\end{description}

\subsubsection{Wissenschaftliches Forum} \label{nut:wissfor}
\begin{description}
    \XXitem{/F350/}{funkt:350} Auf der Seite eines wissenschaftlichen Forums werden die
    zugehörigen wesentlichen Daten \hyperref[d030]{/D030},
    Name, Kurzbeschreibung, Editoren, potentielle Deadlines und Website angezeigt.
    \XXitem{/F360/}{funkt:360} Dem Nutzer werden seine eigenen \hyperref[nut:ein]{Einreichungen} in Form einer Liste  nach \hyperref[leist:40]{/L040/ und Folgenden}
    mit Namen, Datum und Status angezeigt.
    \XXitem{/F400/}{funkt:400} Der Nutzer kann auf die \hyperref[nut:eein]{Seite zur Erstellung einer Einreichung} navigieren. Hierbei ist
    das Eingabefeld, welches das wissenschaftliche Forum bestimmt bei dem eingereicht wird, bereits mit
    dem wissenschaftlichen Forum befüllt, von dessen \hyperref[nut:wissFor]{Übersichtsseite} aus die Navigation auf diese
    Seite ausgeführt wurde.
\end{description}

\subsubsection{Einreichungserstellung}\label{nut:eein}
\begin{description}
    \XXitem{/F410/}{funkt:410} Der Nutzer kann eine Einreichung im System erstellen. Hierzu gibt er in einem
    Formular die nötigen Informationen wie Name der Einreichung, sowie den gewünschten Editor durch ein Dropdown-Menü
    (siehe \hyperref[funkt:665]{/F665/})
    an.
    Der Editor wird nach erfolgreicher Einreichung hierüber durch eine automatisierte Mail informiert.
    \XXitem{/F420/}{funkt:420} Der Nutzer lädt seine Einreichung in Form eines PDF hoch. Die Abgabe darf eine Dateigröße
    von 20MB nicht überschreiten (dieser Wert ist nicht konfigurierbar) und muss im PDF-Format erfolgen.
    \XXitem{/FW430/}{funkt:430} Der Nutzer kann bei Erstellung der Einreichung maximal 10 gewünschte Gutachter per
    Eingabefeld für E-Mailadressen angeben.
    \XXitem{/F435/}{funkt:435} Der Nutzer kann bei Erstellung der Einreichung maximal 10 Ko-Autoren per
    Eingabefeld für E-Mailadressen und zusätzlich mit Name, Vorname und Titel angeben.
    \XXitem{/F437/}{funkt:437} Ko-Autoren werden mittels einer automatisierten E-Mail informiert. Wenn sie bereits
    registriert sind gelangen sie über einen Link in der E-Mail auf die \hyperref[nut:ein]{Seite der Einreichung} bzw.auf die \hyperref[an:log]{Seite des Logins},
    falls keine aktive angemeldete Session besteht.
    Falls unregistriert, gelangt der anonyme Nutzer über den Link zur \hyperref[an:reg]{Registrierungsseite}.
    Nach Accounterstellung erscheint die Einreichung, für welche er als Ko-Autor fungiert, auf der \hyperref[nut:start]{Startseite}.
    \XXitem{/F440/}{funkt:440} Durch Absenden des Formulars wird der Editor des wissenschaftlichen Forums
    informiert. Das Datum der Einreichung wird auf das Datum zum Zeitpunkt der Einreichung festgelegt.
    \XXitem{/F450/}{funkt:450} Die Einreichung ist erfolgreich, wenn alle Felder ausgefüllt sind und eine PDF
    hochgeladen wurde. Andernfalls wird der Nutzer über das Fehlschlagen informiert.
    \XXitem{/F460/}{funkt:460} Nach der erfolgreichen Einreichungen wird der Nutzer auf die Übersichtsseite der
    \hyperref[nut:ein]{Einreichung} weitergeleitet.
\end{description}

\subsubsection{Einreichung}\label{nut:ein}
\begin{description}
    \XXitem{/F470/}{funkt:470} Dem Nutzer werden die Informationen \hyperref[d025]{/D025/} zu seiner Einreichung auf einer eigenen Seite
    angezeigt. Hierzu gehören der Status der Einreichung, das Datum der Einreichung, das zugehörige
    wissenschaftliche Forum, Namen und E-Mail Adressen der Ko-Autoren, ggf. eine Deadline für Gutachten und Frist für Revisionen,
    sowie ein Downloadlink des Papers.
    \XXitem{/F480/}{funkt:480} Außerdem werden die freigeschalteten Gutachten in einer
    Liste nach \hyperref[leist:40]{/L040/} dargestellt, zusammen mit Versionsnummer (identifiziert zugehörige Revision oder Paper)
    Erstellungsdatum, Gutachterempfehlung, Gutachterkommentar und Download siehe \hyperref[d040]{/D040/}.
    \XXitem{/F495/}{funkt:495} Der Nutzer kann die Einreichung zurückziehen. Es werden die zugehörigen Daten
    \hyperref[d025]{/D025/} gelöscht und die Editoren per automatisierter Mail informiert.
    Der Nutzer wird vorher auf die Konsequenzen hingewiesen. Dies ist selbst dann möglich, wenn die Einreichung schon akzeptiert wurde.
    \XXitem{/FW496/}{funkt:496} Der Nutzer kann, wenn verlangt (gelber Status: \hyperref[d025]{/D025/}, Funktion: \hyperref[funkt:702]{/FW702}),
    eine Revision des Paper \hyperref[d020]{/D020/} innerhalb der Frist hochladen.
    Editoren werden hierüber per automatisierter Mail informiert.
\end{description}

\subsection{Gutachter}\label{funkt:Gutachter}
Gutachter haben die selben Funktion wie gewöhnliche angemeldete Nutzer. Zusätzlich hierzu kommen die Folgenden:

\subsubsection{Startseite}
\begin{description}
    \XXitem{/F500/}{funkt:500} Dem Gutachter werden auf seiner personalisierten Startseite zusätzlich zu den eigenen
    Einreichungen und zugehörigen wissenschaftlichen Foren, diejenigen Einreichungen angezeigt, für die er als Gutachter
    zugeordnet ist.
    Für sie gelten \hyperref[nut:start]{dieselben Funktionalitäten} wie für die Darstellung eigener Einreichungen.
\end{description}

\subsubsection{Suche}
\begin{description}
    \XXitem{/F510/}{funkt:510} Ein Gutachter kann bei der Suche ebenfalls Einreichungen finden, welche er begutachtet.
\end{description}

\subsubsection{Wissenschaftliches Forum}
\begin{description}
    \XXitem{/F520/}{funkt:520} Dem Gutachter werden zusätzlich zu den eigenen Einreichungen diejenigen Einreichungen angezeigt
    \hyperref[funkt:360]{/F360/}, welche er begutachtet.
\end{description}

\subsubsection{Einreichung} \label{gut:ein}
\begin{description}
    \XXitem{/F530/}{funkt:530} Der Gutachter sieht auf der Übersichtsseite einer Einreichung, welcher als Gutachter
    zugeordnet ist, diejenigen Gutachten welche er selbst erstellt hat in einer
    Liste mit ihrer Versionsnummer, Erstellungsdatum, Gutachterempfehlung, Gutachterkommentar und Download,
    siehe \hyperref[d040]{/D040/}. Explizit nicht zu sehen sind fremde Gutachten.
    \XXitem{/F540/}{funkt:540} Der Gutachter hat zusätzlich die Möglichkeit zu einer Einreichung, welcher er als Gutachter
    zugeordnet ist ein Gutachten einzureichen. Ein Gutachten kann jeweils nur für das aktuellste Paper oder die aktuellste Revision
    eingereicht werden, außerdem beschränkt sich die Anzahl auf maximal ein Gutachten pro Paper/Revision der Einreichung.
    Dies erfolgt mittels eines Formulars definiert durch \hyperref[d040]{/D040/}.
    Sie werden durch Nummern der jeweiligen Revision bzw. Einreichung zugeordnet.
    Hierfür ist eine PDF hochzuladen. Dies ist möglich solange keine Deadline zur Gutachtenseinreichung überschritten wurde.
    \XXitem{/FW550/}{funkt:550} Auf der Einreichungsseite von Einreichungen denen der Gutachter zugeordnet ist,
    kann er in der Liste eigene eingereichte Gutachten zurückziehen.
    Daraufhin werden sie aus der Datenbank entfernt und nicht mehr angezeigt.
    \XXitem{/F560/}{funkt:560} Der Einreicher und Editor werden über neue oder entfernte Gutachten mittels einer automatisierten
    Mail informiert.
\end{description}

\subsection{Editor}\label{funkt:editor}
Editoren haben Zugriff auf alle Funktionen welche angemeldeten Nutzern zur Verfügung stehen.
Editoren haben alle Funktionen von Gutachtern auf Einreichungen, welche sie verwalten.
Außerdem hat ein Editor die Folgenden zusätzlichen Funktionalitäten:

\subsubsection{Startseite}
\begin{description}
    \XXitem{/F570/}{funkt:570} Dem Editor werden auf seiner personalisierten Startseite in getrennten Listen nach \hyperref[leist:040]{/L040/}
    zusätzlich zu den eigenen
    Einreichungen und zugehörigen wissenschaftlichen Foren diejenigen angezeigt, welche er verwaltet.
    Für sie gelten \hyperref[nut:start]{dieselben Funktionalitäten} wie für die Darstellung eigener Einreichungen.
\end{description}

\subsubsection{Suche} \label{ed:suche}
\begin{description}
    \XXitem{/F580/}{funkt:580} Ein Editor kann ebenfalls Einreichungen finden, welche er verwaltet.
    Diese sind als solche gekennzeichnet.
    \XXitem{/F590/}{funkt:590} Ein Editor kann ebenfalls Einträge zu allen Nutzern per Namen, E-Mail und
    evtl. (\hyperref[funkt:240]{/FW240/}) nach Arbeitgeber und Spezialgebieten finden.
    Die zugehörige Resultatliste enthält alle dem Nutzer zugehörigen Daten \hyperref[d010]{/D010/} und evtl. \hyperref[d015]{/DW015/}
    und ist nach \hyperref[leist:40]{/L040/ und Folgenden} umgesetzt.
\end{description}

\subsubsection{Benutzer} \label{ed:benutzer}
\begin{description}
    \XXitem{/F600/}{funkt:600} Ein Editor kann auf die Nutzerliste über die Kopfzeile zugreifen.
    \XXitem{/F610/}{funkt:610} Die Nutzerliste enthält alle dem Nutzer zugehörigen Daten \hyperref[d010]{/D010/} und evtl. \hyperref[d015]{/DW015/}
    und ist nach \hyperref[leist:040]{/L040/ und Folgenden} umgesetzt.
    \XXitem{/F640/}{funkt:640} Mit einem Klick auf einen Eintrag wird der Editor auf das zugehörige Profil navigiert.
    Auf welchem er die Sichtrechte \hyperref[funkt:200]{/F200/} besitzt.
\end{description}

\subsubsection{Wissenschaftliches Forum} \label{ed:wissFor}
\begin{description}
    \XXitem{/F650/}{funkt:650} Einem Editor werden auf der Seite eines wissenschaftlichen Forums, für
    welches er als Editor fungiert, alle Einreichungen in einer Liste dargestellt.
    Solche bei denen er als Editor eingesetzt wird werden als solche gekennzeichnet.
    Für diese Einträge gelten dieselben Funktionalitäten wie für die
    eigenen \hyperref[nut:ein]{Einreichungen}.
    \XXitem{/F660/}{funkt:660} Ein Editor kann, in einem wissenschaftlichen Forum, welchem er als Editor zugeordnet ist,
    andere Editoren ernennen, indem er sie aus einem Dropdown-Menü (\hyperref[funkt:665]{/F665/}) auswählt.
    \XXitem{/F665/}{funkt:665} Das Dropdown-Menü ist nach Vor-/Namen und E-Mailadressen
    (optional auch nach Spezialgebieten und Arbeitgebern \hyperref[funkt:240]{/FW240/}) durchsuchbar.
    \XXitem{/F670/}{funkt:670} Ein Editor kann, in einem wissenschaftlichen Forum, welcher er als Editor zugeordnet ist,
    anderen Editoren den Status als Editor aberkennen, solange ein weiterer Editor vorhanden ist.
    \XXitem{/F671/}{funkt:671} Ein Editor kann in einem wissenschaftlichen Forum, welchem er als Editor zugeordnet ist,
    außerdem alle weiteren zugehörigen Daten \hyperref[d030]{/D030/} editieren.
\end{description}

\subsubsection{Einreichung} \label{ed:ein}
\begin{description}
    \XXitem{/F680/}{funkt:680} Ein Editor kann auf der Seite einer Einreichung, welcher er als Editor zugeordnet ist,
    in einem Formular Gutachter zuweisen. Die Angabe von Gutachtern erfolgt über ein Dropdown-Menü
    wie in \hyperref[funkt:665]{/665/} beschrieben oder über ein Eingabefeld für E-Mailadressen.
    Zweiteres ist vorgesehen für (noch) unregistrierte Personen.
    \XXitem{/F682/}{funkt:682} Dem Editor werden auf Einreichungen, welche er editiert,
    zu Gutachten zusätzlich der Name des Gutachters angezeigt.
    \XXitem{/F685/}{funkt:685} Ein Editor kann auf Einreichungen, welche er editiert, noch nicht freigeschaltete Gutachten einsehen und freischalten.
    \XXitem{/F690/}{funkt:690} Wird eine E-Mail Adresse oder ein registrierter Nutzer als Gutachter angegeben,
    so wird nach Absenden des Formulars eine vorgefüllte \hyperref[glo:mailto]{Mailto-Nachricht} an diesen Nutzer bzw. diese Mailadresse angeboten.
    Sie enthält eine Nachricht mit den relevanten Informationen zu Einreichung, wissenschaftlichem Forum, sowie
    \begin{itemize}
        \item \ldots einen Link zur \textbf{Annahme der Begutachtungsanfrage} welcher, sobald geklickt,
        auf die \hyperref[an:log]{Loginseite} verweist auf der eine Nachricht des Dankes anzeigt und zur Anmeldung bzw.
        Registrierung auffordert, wenn der Gutachter noch unregistriert ist. Sonst leitet der Link auf die
        \hyperref[nut:ein]{Seite der Einreichung} weiter.
        \item \ldots einen Mailto-Link zur \textbf{Ablehnung der Begutachtungsanfrage} welcher, sobald
        geklickt, einen \hyperref[glo:mailto]{Mailentwurf} öffnet mit vorausgefülltem Empfänger (zugehöriger Editor)
        und einem Infotext in welchem ein Ablehnungsgrund eingefügt werden kann.
    \end{itemize}
    \XXitem{/F695/}{funkt:695} Ein Editor kann einer Einreichung, welcher er als Editor zugewiesen ist, einen anderen Editor
    über ein Dropdown-Menü (\hyperref[funkt:665]{/F665/}) zuweisen. Er selbst ist nun kein Editor dieser Einreichung mehr.
    Der neue Editor wird per automatisierter E-Mail hierüber informiert.
    \XXitem{/F700/}{funkt:700} Ein Editor kann über Einreichungen, welcher er als Editor zugeordnet ist,
    eine Annahmeentscheidung ("Rot", "Grün": \hyperref[d025]{/D025/}) treffen.
    Der Editor kann Einreicher und beteiligte Ko-Autoren per \hyperref[glo:mailto]{Mailto-Nachricht} mit vorgefertigtem Inhalt
    hierüber informieren.
    \XXitem{/FW702/}{funkt:702} Ein Editor kann für Einreichungen, welche er verwaltet,
    eine Revision ("Gelb": \hyperref[d025]{/D025/}) verlangen. In diesem Fall gibt er eine Frist für die Revision an.
    Der Editor benachrichtigt den Einreicher hierüber per \hyperref[glo:mailto]{Mailto-Nachricht} mit vorgefertigtem Inhalt.
    \XXitem{/FW703}{funkt:703} Ein Editor kann Revisionen für Gutachter freischalten, sodass diese für sie sichtbar werden.
    Hierüber werden die beteiligten Gutachter per \hyperref[glo:mailto]{Mailto-Nachricht} informiert.
    \XXitem{/F705/}{funkt:705} Ein Editor kann einer Einreichung, welcher er als Editor zugeordnet ist, eine Deadline für
    Gutachten zuweisen oder entfernen.
    \XXitem{/F706/}{funkt:706} Ein Editor kann eine Einreichung, welcher er zugeordnet ist, wie in
    \hyperref[funkt:495]{/F495/} beschrieben, entfernen.
    Statt Editoren werden hierüber beteiligt Einreicher, Ko-Autoren und Gutachter informiert.
\end{description}

\subsection{Administrator}
Der Administrator besitzt zu Verwaltungszwecken umfassende Funktionalitäten.
Er kann ebenfalls Editorrollen und Gutachterrollen annehmen.
Einem Administrator werden in Listen grundsätzlich keine Einträge vorenthalten.
Er besitzt alle Funktionalitäten die einem angemeldeten Benutzer zur Verfügung stehen, auf allen Einreichungen.
Er besitzt alle Funktionalitäten des Editors auf allen \hyperref[ed:wissFor]{wissenschaftlichen Foren} und \hyperref[ed:ein]{Einreichungen},
alle Funktionen eines Gutachters auf allen \hyperref[gut:ein]{Einreichungen}
und zusätzlich die Folgenden.

\subsubsection{Suche}
\begin{description}
    \XXitem{/F710/}{funkt:710} Ein Administrator kann in der globalen Suche ebenfalls alle Nutzer finden.
    \XXitem{/F720/}{funkt:720} Ein Administrator kann alle vorhandenen Einreichungen finden.
\end{description}

\subsubsection{Wissenschaftliches Forum}
\begin{description}
    \XXitem{/F730/}{funkt:730} Einem Administrator werden auf der Seite eines wissenschaftlichen Forums
    alle Einreichungen in einer Liste, nach \hyperref[leist:40]{/L040/ und Folgenden}, dargestellt.
    \XXitem{/FW750/}{funkt:750} Der Administrator kann den Look \hyperref[d051]{/D051/}
    des wissenschaftlichen Forums verändern.
    \XXitem{/F760/}{funkt:760} Auf der Seite eines wissenschaftlichen Forums kann der Administrator diese aus dem System löschen.
    Hierbei wird er dazu aufgefordert seine Entscheidung ein zweites Mal zu bestätigen.
    Daraufhin werden alle zugehörigen Daten \hyperref[d030]{/D030/}, \hyperref[d051]{/D051/}
    und Einreichungen aus dem System entfernt.
\end{description}

\subsubsection{Profil}
\begin{description}
    \XXitem{/F770/}{funkt:770} Ein Administrator kann einen anderen Nutzer auf dessen Profilseite zum Administrator ernennen.
    \XXitem{/F780/}{funkt:780} Ein Administrator kann einem anderen Administrator auf dessen Profilseite seine
    Administratorrolle aberkennen, solange ein weiterer Admin vorhanden ist.
    \XXitem{/FW790/}{funkt:790} Vor dem An- oder Aberkennen von Administratorrechten ist eine gültige
    Passworteingabe erforderlich.
    \XXitem{/F800/}{funkt:800} Der Administrator besitzt auf allen Profilseiten dieselben Rechte zur Änderung
    der persistierten Daten wie ein Nutzer auf seiner eigenen Profilseite
    \hyperref[nut:profil]{(Link-Profil)}. Es ist keine E-Mailverifikation notwendig.
\end{description}

\subsubsection{Liste der Wissenschaftlichen Foren}
\begin{description}
    \XXitem{/F810/}{funkt:810} Auf dieser Seite kann der Administrator zur Seite navigieren auf welcher er ein neues
    wissenschaftliches Forum anlegen kann (\hyperref[funkt:820]{/F820/}).
\end{description}

\subsubsection{Erstellung wissenschaftlicher Foren}
\begin{description}
    \XXitem{/F820/}{funkt:820} Auf der Seite zum Erstellen eines wissenschaftlichen Forums kann der Administrator dessen
    wesentliche Daten \hyperref[d030]{/D030/} und \hyperref[d051]{/D051/}
    festlegen.
    \begin{itemize}
        \item Deadline, Kurzbeschreibung, URL, Anleitung zur Begutachtung sind hierbei optional.
        \item  Die Kategorie (Algorithmik, Graphentheorie, Analysis, \ldots siehe \hyperref[d035]{/D035/}) kann aus existierenden ausgewählt
        oder neu erstellt und der Liste hinzugefügt werden.
        \item Bei der Angabe von Editoren wird sichergestellt, dass diese bereits als Nutzer im System registriert sind.
        Es muss mindestens ein Editor über ein Dropdown-Menü, wie in \hyperref[funkt:665]{/665/} beschrieben, eingetragen werden,
        welcher per automatisierter E-Mail darüber informiert wird.
        \item Der Name des Forums muss einzigartig sein.
    \end{itemize}
    Nach erfolgreicher Erstellung wird der Administrator auf die Seite dieses wissenschaftlichen Forums
    navigiert.
\end{description}

\subsubsection{Konfiguration}
\begin{description}
    \XXitem{/F830/}{funkt:830} Ein Administrator kann auf der Konfigurationsseite den vom Betreiber gewünschten
    'Look and Feel' des Systems festlegen. Hierzu bestimmt der die Daten wie in \hyperref[d050]{/D050/} definiert.
\end{description}

\subsubsection{Nutzeranlegung}
\begin{description}
    \XXitem{/F840/}{funkt:840} Ein Administrator kann von der \hyperref[ed:benutzer]{Seite welche die Nutzerliste} darstellt auf die Seite zur
    Erstellung neuer Nutzer gelangen.
    \XXitem{/F850/}{funkt:850} Ein Administrator kann auf der Seite zur Nutzeranlegung
    einen neuen Nutzer im System wie bei der Registration \hyperref[funkt:060]{/F060/} und \hyperref[funkt:070]{/F70/}
    anlegen. Hierbei ist keine E-Mail-Verifikation \hyperref[funkt:070]{/F070/} nötig. Außerdem kann festgelegt werden ob dieser Nutzer ein Administrator ist.
    \XXitem{/F860}{funkt:860} Nach erfolgeicher Erstellung eines Nutzers wird der Administrator auf dessen \hyperref[ad:profil]{Profilseite} weitergeleitet.
\end{description}