JSF lässt sich auf verschiedenste Art \& Weisen konfigurieren, wie beispielsweise
mithilfe der \emph{faces-config.xml}.
Die Lases-Anwendung bietet nebenher weitere selbstdefinierte Konfigurationsmöglichkeiten an.
Die Konfigurationsdateien befinden sich im \emph{webapp/WEB-INF/classes} Verzeichnis.
Die Konfigurationsdateien werden intern in \emph{Properties}-Pakete umgewandelt,
wobei jedes \emph{Properties}-Paket eine semantische Aufgabe übernimmt.
So dient beispielsweise die \emph{mail.properties} der Konfiguration des Mailing.
Wenn ein Paket bestimmte \emph{Properties} auslagert,
so landen diese in der Anwendungskonfiguration \emph{config.properties}.
%todo letzten teil kürzen zB letzten satz zu kofig der anw auslagern

\subsection{Konfiguration der Anwendung}
Die Anwendung kann auf einige Weisen personalisiert konfiguriert werden.
Hierzu dient die \emph{config.properties}-Datei,  %todo hyperref
welche zentrale Parameter der Anwendung definiert.

\subsubsection{Konfiguration des Mailings}
Damit die LasEs-Anwendung Mails versenden und Mailto-Links generieren kann
müssen einige Konfigurationsentscheidungen getroffen werden.
Hierzu dient die \emph{mail.properties}-Datei.

\subsubsection{Konfiguration der Datenbank}
Damit die LasEs-Anwendung einen persistenten Datenbankzugriff besitzen kann,
muss dieser konfiguriert werden. Nennenswert sind ebenfalls die
Konfigurationsentscheidungen bezüglich der SSL-Verschlüsselung.
Hierzu dient die \emph{database.properties}-Datei.

\subsubsection{Konfiguration des Logging}
Damit die LasEs-Anwendung aussagekräftige \emph{Log}-Dateien ausgeben kann,
wird der \emph{Logger} konfiguriert.



