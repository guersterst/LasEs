JSF lässt sich auf verschiedenste Art \& Weisen konfigurieren, wie beispielsweise
mithilfe der \emph{faces-config.xml}.
Die Lases-Anwendung bietet nebenher weitere selbstdefinierte Konfigurationsmöglichkeiten an.
Die Konfigurationsdateien befinden sich im \emph{webapp/WEB-INF/classes} Verzeichnis.
\newline\newline
Die Konfigurationsdateien werden intern in \emph{Properties}-Pakete umgewandelt,
wobei jedes \emph{Properties}-Paket eine semantische Aufgabe übernimmt.
So dient beispielsweise die \emph{mail.properties} der Konfiguration des Mailing.
Wenn ein Paket bestimmte \emph{Properties} auslagert,
so landen diese in der Anwendungskonfiguration \emph{config.properties}.
%todo letzten teil kürzen zB letzten satz zu kofig der anw auslagern

\subsection{Konfiguration der Anwendung}
Die Anwendung kann auf einige Weisen personalisiert konfiguriert werden.
Hierzu dient die \emph{config.properties}-Datei,
welche zentrale Parameter der Anwendung definiert.

\begin{lstlisting}[language=XML, caption = Die Anwendungskonfiguration \emph{config.properties}]
# The configuration for the LasEs application.
# The convention to follow is: <key> = <value>.

#------------------- GENERAL ------------------------#

# The maximum filesize for a pdf upload.
MAX_PDF_FILE_SIZE_MB = 20

# The maximum filesize for a image upload.
MAX_AVATAR_FILE_SIZE_MB = 5

# The maximum length of paginated lists.
MAX_PAGINATION_LIST_LENGTH = 25

#------------------- MAIL -------------------------------#

MAIL_ADDRESS_FROM = system@lases.de

#------------------- DATABASE ---------------------------#

# The URL with which the database can be accessed.
DB_URL = jdbc:postgresql://bueno.fim.uni-passau.de/sep21g02

# The database driver.
DB_DRIVER = org.postgresql.Driver

# The minimum amount of database connections provided
# by the ConnectionPool.
MIN_DB_CONNECTIONS = 10

# The maximum amount of database connections provided
# by the ConnectionPool.
MAX_DB_CONNECTIONS = 50
\end{lstlisting}

\subsubsection{Konfiguration des Mailings}
Damit die LasEs-Anwendung Mails versenden und Mailto-Links generieren kann
müssen einige Konfigurationsentscheidungen getroffen werden.
Hierzu dient die \emph{mail.properties}-Datei.

\begin{lstlisting}[language=XML, caption = Die Mailingkonfiguration \emph{mail.properties},
    breaklines=true]
# The configuration for the mailing API.
# The convention to follow is: <key> = <value>.

# For documentation see:
# https://jakarta.ee/specifications/mail/1.6/apidocs/index.html?com/ sun/mail/smtp/package-summary.html


#------------------- SMTP ---------------------------#

mail.smtp.host = mail.fim.uni-passau.de

mail.smtp.port = 587

mail.smtp.auth = true

mail.smtp.starttls.enable = true
\end{lstlisting}

\subsubsection{Konfiguration der Datenbank}
Damit die LasEs-Anwendung einen persistenten Datenbankzugriff besitzen kann,
muss dieser konfiguriert werden. Nennenswert sind ebenfalls die
Konfigurationsentscheidungen bezüglich der SSL-Verschlüsselung.
Hierzu dient die \emph{database.properties}-Datei.

\begin{lstlisting}[language=XML, caption = Die Datenbankkonfiguration \emph{database.properties}]
# The configuration for the database.
# The convention to follow is: <key> = <value>.
# For documentation see:
# https://jdbc.postgresql.org/documentation/head/connect.html

#------------------- ACCESS ------------------------#

# The username for database access.
user = sep21g02

# The password for database access.
password = ieQu2aeShoon

#------------------- SSL ---------------------------#

# Is SSL protection being used.
ssl = true

# The SSL factory used for SSL protection.
sslfactory = org.postgresql.ssl.DefaultJavaSSLFactory
\end{lstlisting}


\subsubsection{Konfiguration des Logging}
Damit die LasEs-Anwendung aussagekräftige \emph{Log}-Dateien ausgeben kann,
wird der \emph{Logger} konfiguriert.

\begin{lstlisting}[language=XML, caption = Die Loggingkonfiguration \emph{logger.properties}]
# The configuration for the logger.
# For documentation see:
# https://docs.oracle.com/javase/7/docs/api/java/util/logging/FileHandler.html

#------------------- LOGGING ---------------------------#

# The handler for logging.
handlers= java.util.logging.FileHandler

# The level of logs that are to be logged. With 'Info' all logs
# with higher priority will also be logged.
.level= INFO

# The directory where the logs will be placed.
# We need help deciding where this should be.
java.util.logging.FileHandler.pattern = %h/lases%u.log
\end{lstlisting}

\subsubsection{Sprachkonfiguration}
Die Dateien, welche die dargestellten \emph{Strings} enthalten, befinden sich im
\emph{src/main/ressources}-Verzeichnis.
Die initiale Standardsprache (Defaultwert Englisch) wird in der \emph{faces-config.xml} im
Verzeichnis \emph{webapp/WEB-INF} festgelegt.

\begin{itemize}
    \item \emph{Resource Bundle 'label'} enthält solche \emph{Strings}, welche als \emph{Labels} für die Benutzeroberfläche verwendet
    werden.
    \item \emph{Resource Bundle 'help'} enthält solche \emph{Strings}, welche zum Anzeigen der \emph{Tooltips} verwendet
    werden.
    \item \emph{Resource Bundle 'message'} enthält solche \emph{Strings}, welche zum Anzeigen von Hinweisen und Warnungen
    an den Nutzer verwendet werden.
\end{itemize}

Ein \emph{Resource Bundle} enthält jeweils drei \emph{.properties}-Dateien.
Jeweils eine für:
\begin{itemize}
    \item \emph{Strings} in deutscher Sprache.
    \item \emph{Strings} in englischer Sprache.
    \item sprachneutrale \emph{Strings}.
\end{itemize}

Diese \emph{Strings} werde nach der Konvention:
\textbf{<key> = <value>}
abgelegt, wobei der \emph{<key>} in \emph{camelCase} geschrieben ist.

Hier ein Beispiel mit einigen ausgewählten \emph{key-value}-Paaren aus
dem \emph{Resource Bundle 'labels'}.
\begin{lstlisting}[language=XML, caption = Beispielhafter Ausschnitt aus \emph{label.properties}]
    # Language-neutral translations of labels displayed in the UI.
    # The convention to follow is: <key> = <value>.

    appName = LasEs
\end{lstlisting}
\begin{lstlisting}[language=XML, caption = Beispielhafter Ausschnitt aus \emph{label\_de.properties}]
    # German language translations of labels displayed in the UI.
    # The convention to follow is: <key> = <value>.

    login = Anmelden
    register = Registrieren
    logout = Abmelden
    save = speichern
    acceptSubmission = Annehmen
    rejectSubmission = Ablehnen
    prevPage = Nächste Seite
    lastPage = Letzte Seite
\end{lstlisting}

\begin{lstlisting}[language=XML, caption = Beispielhafter Ausschnitt aus \emph{label\_en.properties}]
    # English language translations of labels displayed in the UI.
    # The convention to follow is: <key> = <value>.

    login = Login
    register = Register
    logout = Logout
    saveEditor = Save
    acceptSubmission = Accept
    rejectSubmission = Reject
    prevPage = Previous Page
    lastPage = Last Page
\end{lstlisting}