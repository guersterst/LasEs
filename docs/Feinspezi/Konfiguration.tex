JSF lässt sich auf verschiedenste Art \& Weisen konfigurieren, wie beispielsweise
mithilfe der \emph{faces-config.xml}.
Die Lases-Anwendung bietet nebenher weitere selbstdefinierte Konfigurationsmöglichkeiten an.
Die Konfigurationsdateien befinden sich im \emph{webapp/WEB-INF/classes} Verzeichnis.
Die Konfigurationsdateien werden intern in \emph{Properties}-Pakete umgewandelt,
wobei jedes \emph{Properties}-Paket eine semantische Aufgabe übernimmt.
So dient beispielsweise die \emph{mail.properties} der Konfiguration des Mailing.
Wenn ein Paket bestimmte \emph{Properties} auslagert,
so landen diese in der Anwendungskonfiguration \emph{config.properties}.
%todo letzten teil kürzen zB letzten satz zu kofig der anw auslagern

\subsection{Konfiguration der Anwendung}
Die Anwendung kann auf einige Weisen personalisiert konfiguriert werden.
Hierzu dient die \emph{config.properties}-Datei,  %todo hyperref
welche zentrale Parameter der Anwendung definiert.

\subsubsection{Konfiguration des Mailings}
Damit die LasEs-Anwendung Mails versenden und Mailto-Links generieren kann
müssen einige Konfigurationsentscheidungen getroffen werden.
Hierzu dient die \emph{mail.properties}-Datei.

\subsubsection{Konfiguration der Datenbank}
Damit die LasEs-Anwendung einen persistenten Datenbankzugriff besitzen kann,
muss dieser konfiguriert werden. Nennenswert sind ebenfalls die
Konfigurationsentscheidungen bezüglich der SSL-Verschlüsselung.
Hierzu dient die \emph{database.properties}-Datei.

\subsubsection{Konfiguration des Logging}
Damit die LasEs-Anwendung aussagekräftige \emph{Log}-Dateien ausgeben kann,
wird der \emph{Logger} konfiguriert.

\subsubsection{Sprachkonfiguration}
Die Dateien, welche die dargestellten \emph{Strings} enthalten, befinden sich im
\emph{src/main/ressources}-Verzeichnis.

\begin{itemize}
    \item \emph{Resource Bundle 'labels'} enthält solche \emph{Strings}, welche als \emph{Labels} für die Benutzeroberfläche verwendet
    werden.
    \item \emph{Resource Bundle 'help'} enthält solche \emph{Strings}, welche zum Anzeigen der \emph{Tooltips} verwendet
    werden.
    \item \emph{Resource Bundle 'messages'} enthält solche \emph{Strings}, welche zum Anzeigen von Hinweisen und Warnungen
    an den Nutzer verwendet werden.
    \item \emph{Resource Bundle 'texts'} enthält solche \emph{Strings}, welche zum Anzeigen längerer Texte, wie beispielsweise die
    Begrüßung auf der Anmeldeseite, verwendet werden.
\end{itemize}

Ein \emph{Resource Bundle} enthält jeweils drei \emph{.properties}-Dateien.
Jeweils eine für:
\begin{itemize}
    \item \emph{Strings} in deutscher Sprache.
    \item \emph{Strings} in englischer Sprache.
    \item sprachneutrale \emph{Strings}.
\end{itemize}

Diese \emph{Strings} werde nach der Konvention:
\textbf{<key> = <value>} %todo latex let look like code
abgelegt, wobei der \emph{<key>} in \emph{camelCase} geschrieben ist.
-sprach konfiguration -> faces.xml %todo