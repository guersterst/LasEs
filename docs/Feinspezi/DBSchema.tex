\localauthor{Johann Schicho}

\subsection{Konvertierung des ER-Diagramms}

Das ER-Diagramm aus dem Entwurf wird hier in \emph{Data Definition Language} übersetzt. Dabei werden die Details des Übertragungsprozesses dargestellt und erklärt.

\subsubsection{SQL Standard}

\emph{SQL} ist vielfach standardisiert und alle Datenbank Management System weichen teilweise von diesen Standards ab und haben dadurch ihren eigenen \emph{SQL Dialekt}. Im Folgenden wird der PostgreSQL Dialekt verwendet, da PostgreSQL, wie im Pflichtenheft definiert, als SQL Server verwendet wird.

Abweichungen vom ISO SQL Standard sind Enumerationen, \texttt{TIMESTAMP} Zeitpunkte, \texttt{BYTEA} als binärer Datentyp und einzelne Unterschiede in der Syntax.

\subsubsection{Normalform}

Das Datenbankschema befindet sich in \emph{Boyce-Codd-Normalform} (BCNF).

Durch die Verwendung von Surrogatschlüsseln in den Entities sind die vorkommenden funktionalen Abhängigkeiten offensichtlich. Für all diese gilt, dass die Surrogatschlüssel Superschlüssel der Relation sind. Die \emph{UNIQUE}s bilden ebenfalls funktionale Abhängigkeiten und sind auch Superschlüssel der jeweiligen Relation.

\subsubsection{Namenskonventionen}

Die Tabellennamen sind immer im englischen Singular. Also beispielsweise \texttt{submission} anstatt \texttt{submissions}.
Alle Bezeichner in SQL werden in \texttt{snake\_case} geschrieben. Damit ist eine eindeutige optische Trennung zu Java's \texttt{camelCase} möglich. Die Werte von Enumerationen werden in \texttt{UPPER\_CASE} geschrieben.

\subsubsection{Relationship-Typen}

1 : n Relationships benötigen keine extra Tabellen in der Datenbank, da diese durch Referenz des Primärschlüssels der 1-Relation als Fremdschlüssel in der n-Relation abgespeichert werden können.

n : m Relationships benötigen dahingegen eine eigene Relation in der die Schlüsselpaare abgespeichert werden.

Abhängigkeiten, wie zum Beispiel Einreichungen, die zu einem Nutzer gehören, werden durch \texttt{ON DELETE CASCADE} modelliert. Wird der referenzierte Beziehungspartner (\emph{User}) gelöscht, so werden auch kaskadierend alle abhängigen weiteren Daten (\emph{Submissions}) automatisch gelöscht.

\subsubsection{Vererbung}

Durch \emph{IS-A} werden in ER-Diagrammen verschiedene Arten eines Typs definiert. Diese können dann in der \emph{DDL} unterschiedlich übersetzt werden.

Anstatt einer Enumeration wird in unserem Fall die Rechtezuordnung der registrierten Nutzer durch ein \emph{boolean} für Administratorrechte und durch Relationen für Gutachter und Editor modelliert, da ein Nutzer mehrere Aufgaben in LasEs wahrnehmen kann.

Die Unterscheidung zwischen Journalen und Konferenzen findet nur durch eine gesetzte oder \emph{NULL} Deadline statt, da sich beide funktional nicht weiter unterscheiden.

\subsection{Entities}

Hier werden die einzelnen Entities in der \emph{Data Definition Language} beschrieben. Durch die Verwendung von \emph{SERIAL} werden die Surrogatschlüssel der Relationen automatisch von PostgreSQL erstellt und die Primärschlüsseleigenschaft sichergestellt.\\

\begin{lstlisting}[caption={DDL von Nutzern}]
CREATE TABLE "user" (
	id SERIAL PRIMARY KEY NOT NULL,
	email_address VARCHAR(70) UNIQUE NOT NULL,
	is_administrator BOOLEAN NOT NULL,
	firstname VARCHAR(50) NOT NULL,
	lastname VARCHAR(50) NOT NULL,
	title VARCHAR(20),
	employer VARCHAR(100),
	birthdate DATE,
	avatar_thumbnail BYTEA,
	is_registered BOOLEAN NOT NULL,
	-- 128 Bit, Base64 encoded
	password_hash VARCHAR(24),
	-- 16 Byte Salt, Base64 encoded
	password_salt VARCHAR(24),

	CONSTRAINT valid_name CHECK (length(firstname) >= 1
	AND length(lastname) >= 1),
	CONSTRAINT valid_birthdate CHECK (birthdate < CURRENT_DATE),
	CONSTRAINT email_address_pattern CHECK (email_address LIKE '_%@_%._%')
);
\end{lstlisting}

\begin{lstlisting}[caption={DDL von Verifizierung}]
CREATE TABLE verification (
	id INTEGER NOT NULL,
	validation_random VARCHAR,
	is_verified BOOLEAN,
	timestamp_validation_started TIMESTAMP,
	unvalidated_email_address VARCHAR,

	PRIMARY KEY (id),
	FOREIGN KEY (id) REFERENCES "user"(id)
);
\end{lstlisting}

\begin{lstlisting}[caption={DDL von wissenschaftlichen Foren}]
CREATE TABLE scientific_forum(
	id SERIAL PRIMARY KEY NOT NULL,
	name VARCHAR(200) UNIQUE NOT NULL,
	description VARCHAR(500),
	url VARCHAR(150),
	review_manual VARCHAR(500),
	timestamp_deadline TIMESTAMP
);
\end{lstlisting}

\begin{lstlisting}[caption={DDL von Fachgebieten}]
CREATE TABLE science_field(
	name VARCHAR(30) PRIMARY KEY NOT NULL
);
\end{lstlisting}
\begin{lstlisting}[caption={DDL von Einreichungen}]
CREATE TYPE submission_state AS ENUM (
	'SUBMITTED',
	'REVISION_REQUIRED',
	'ACCEPTED',
	'REJECTED'
);

CREATE TABLE submission (
	id SERIAL PRIMARY KEY NOT NULL,
	title VARCHAR(200) NOT NULL,
	state submission_state NOT NULL,
	timestamp_submission TIMESTAMP,
	requires_revision BOOLEAN NOT NULL DEFAULT FALSE,
	timestamp_deadline_revision TIMESTAMP,

	author_id INTEGER REFERENCES "user"(id) ON DELETE CASCADE,
	editor_id INTEGER REFERENCES "user"(id) ON DELETE SET NULL,
	forum_id INTEGER REFERENCES scientific_forum(id) ON DELETE CASCADE
);
\end{lstlisting}

\begin{lstlisting}[caption={DDL von Papers}]
CREATE TABLE paper (
	version INTEGER NOT NULL,
	submission_id SERIAL NOT NULL,
	timestamp_upload TIMESTAMP NOT NULL DEFAULT NOW(),
	is_visible BOOLEAN NOT NULL,
	pdf_file BYTEA NOT NULL,

	PRIMARY KEY (submission_id, version),
	FOREIGN KEY (submission_id) REFERENCES submission(id) ON DELETE CASCADE
);
\end{lstlisting}

\begin{lstlisting}[caption={DDL von Gutachten}]
CREATE TABLE review (
	reviewer_id INTEGER REFERENCES "user"(id) ON DELETE CASCADE,
	version INTEGER NOT NULL,
	submission_id INTEGER NOT NULL,

	timestamp_upload TIMESTAMP NOT NULL DEFAULT NOW(),
	is_visible BOOLEAN NOT NULL,
	is_recommended BOOLEAN NOT NULL,
	comment VARCHAR,
	pdf_file BYTEA,

	PRIMARY KEY (submission_id, version, reviewer_id),
	FOREIGN KEY (submission_id, version) REFERENCES
	paper(submission_id, version) ON DELETE CASCADE
);
\end{lstlisting}

\begin{lstlisting}[caption={DDL von den Systemeinstellungen}]
CREATE TABLE system(
	id INTEGER PRIMARY KEY,
	company_name VARCHAR(100),
	welcome_heading VARCHAR(200),
	welcome_description VARCHAR(500),
	imprint VARCHAR,
	logo_image BYTEA
);
\end{lstlisting}

\subsection{Indizes}

Um die E-Mail-Adressen unabhängig von der Groß- oder Kleinschreibung in der Datenbank tatsächlich \emph{UNIQUE} zu halten benötigt es in PostgreSQL einen eigenen Index, da das \emph{UNIQUE} Schlüsselwort selbst \emph{case-sensitive} ist.\\

\begin{lstlisting}[caption={DDL des UNIQUE Index auf E-Mail-Adressen}]
CREATE UNIQUE INDEX email_unique_case_insensitive ON "user"
	(LOWER(email_address));
\end{lstlisting}

\subsection{Relationships}

n : m Relationships benötigen eigenen Relationen.\\

\begin{lstlisting}[caption={DDL der Ko-Autoren}]
CREATE TABLE co_authored(
	user_id INTEGER NOT NULL,
	submission_id INTEGER NOT NULL,

	PRIMARY KEY (user_id, submission_id),
	FOREIGN KEY (user_id) REFERENCES "user"(id) ON DELETE CASCADE,
	FOREIGN KEY (submission_id) REFERENCES submission(id) ON DELETE CASCADE
);
\end{lstlisting}

\begin{lstlisting}[caption={DDL der Interessensgebiete}]
CREATE TABLE interests (
	user_id INTEGER NOT NULL,
	science_field_name VARCHAR NOT NULL,

	PRIMARY KEY (user_id, science_field_name),
	FOREIGN KEY (user_id) REFERENCES "user"(id) ON DELETE CASCADE,
	FOREIGN KEY (science_field_name) REFERENCES science_field(name)
	ON DELETE CASCADE
);
\end{lstlisting}

\begin{lstlisting}[caption={DDL der verwaltenden Editoren}]
CREATE TABLE member_of(
	editor_id INTEGER NOT NULL,
	scientific_forum_id INTEGER NOT NULL,

	PRIMARY KEY (editor_id, scientific_forum_id),
	FOREIGN KEY (editor_id) REFERENCES "user"(id) ON DELETE CASCADE,
	FOREIGN KEY (scientific_forum_id) REFERENCES scientific_forum(id)
	ON DELETE CASCADE
);
\end{lstlisting}

\begin{lstlisting}[caption={DDL der begutachtenden Gutachter}]
CREATE TYPE review_task_state AS ENUM (
	'NO_DECISION',
	'ACCEPTED',
	'REJECTED'
);

CREATE TABLE reviewed_by(
	reviewer_id INTEGER NOT NULL,
	submission_id INTEGER NOT NULL,
	has_accepted review_task_state,
	timestamp_deadline TIMESTAMP,

	PRIMARY KEY (reviewer_id, submission_id),
	FOREIGN KEY (reviewer_id) REFERENCES "user"(id) ON DELETE CASCADE,
	FOREIGN KEY (submission_id) REFERENCES submission(id) ON DELETE CASCADE
);
\end{lstlisting}

\begin{lstlisting}[caption={DDL der Themengebiete eines Forums}]
CREATE TABLE topics(
	science_field_name VARCHAR NOT NULL,
	scientific_forum_id INTEGER NOT NULL,

	PRIMARY KEY (science_field_name, scientific_forum_id),
	FOREIGN KEY (science_field_name) REFERENCES science_field(name)
	ON DELETE CASCADE,
	FOREIGN KEY (scientific_forum_id) REFERENCES scientific_forum(id)
	ON DELETE CASCADE
);
\end{lstlisting}