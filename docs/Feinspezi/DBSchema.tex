\localauthor{Johann Schicho}

\subsection{Konvertierung des ER-Diagramms}

Das ER-Diagramm aus dem Entwurf wird hier in \emph{Data Definition Language} übersetzt. Dabei werden die Details des Übertragungsprozesses dargestellt und erklärt.

\subsubsection{SQL Standard}

\emph{SQL} ist vielfach standardisiert und jedes Datenbank Management System weicht teilweise von diesen Standards ab und haben dadurch ihren eigenen \emph{SQL Dialekt}. Im Folgenden wird der PostgreSQL Dialekt verwendet, da PostgreSQL, wie im Pflichtenheft definiert, als SQL Server verwendet wird.

Abweichungen vom ISO SQL Standard sind Enumerationen, \texttt{TIMESTAMP} Zeitpunkte, \texttt{BYTEA} als binärer Datentyp und einzelne Unterschiede im Syntax.

\subsubsection{Namenskonventionen}

Die Tabellennamen sind immer im Singular. Also beispielsweise \texttt{submission} anstatt \texttt{submissions}.
Alle Bezeichner in SQL werden in \texttt{snake\_case} geschrieben. Damit ist eine eindeutige optische Trennung zu Java's \texttt{camelCase} möglich. Die Werte von Enumerationen werden in \texttt{UPPER\_CASE} geschrieben.

\subsubsection{Relationship-Typen}

\begin{lstlisting}[language=SQL, caption={DDL von Einreichungen},captionpos=below,frame=tb,numbers=left, tabsize=4]
CREATE TYPE submission_state AS ENUM (
	'SUBMITTED',
	'REVISION_REQUIRED',
	'ACCEPTED',
	'REJECTED'
);

CREATE TABLE submission (
	id SERIAL PRIMARY KEY,
	title VARCHAR NOT NULL,
	state submission_state NOT NULL,
	timestamp_submission TIMESTAMP,
	requires_revision BOOLEAN NOT NULL DEFAULT FALSE,
	timestamp_deadline_revision TIMESTAMP,

	author_id SERIAL REFERENCES user(id) NOT NULL,
	editor_id SERIAL REFERENCES user(id) NOT NULL,
	forum_id SERIAL REFERENCES scientificForum(id) NOT NULL
);
\end{lstlisting}