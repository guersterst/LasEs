\localauthor{Johann Schicho}

Zur Entwicklung und Verwendung von LasEs werden einige Softwarebibliotheken benötigt. Um keine weiteren Gebühren geltend machen zu müssen, werden nur Bibliotheken mit freien Lizenzen verwendet.

\begin{table}[H]
	\centering
	\begin{tabularx}{\columnwidth}{|l|X|l|}
		\hline
		Name & Beschreibung & Lizenz \\
		\hline\hline
		Mojarra v3.0.1 & Jakarta Server Faces Referenzimplementierung der Eclipse Foundatation & EPL v2.0 \\
		\hline
		JBoss Weld v4.0.2.Final & Context and Dependency Injection Referenz Implementierung & Apache 2.0 \\
		\hline
		Primefaces v10.0.0 & UI Komponenten Bibliothek & MIT \\
		\hline
		Bootstrap v5.1.3 & CSS Framework & MIT \\
		\hline
		PostreSQL JDBC v42.3.1 & Datenbank Treiber für PostgreSQL & BSD 2-Clause \\
		\hline
		Jakarta Mail API v2.0.1 & Automatisierte E-Mail Versendung & EPL v2.0 \\
		\hline
		JUnit v5.8.1 & Unit Tests für Java Anwendungen & EPL v2.0 \\
		\hline
		Mockito v4.0.0 & Erzeugung von Mockups zum testen & MIT \\
		\hline
		Selenium v4.0.0 & Automatisierte Browser Tests & Apache 2.0 \\
		\hline



	\end{tabularx}
\end{table}