%% Macros
\newcommand{\ftable}[1]{\begin{sidewaystable}
\begin{tabular}[H]{ m{2cm} m{3cm} m{6cm} m{2.5cm} m{2cm} m{2cm} }
    \toprule
    \textbf{ID} & \textbf{Typ} & \textbf{Beschreibung} & \textbf{Binding} & \textbf{Constraints} & \textbf{Validator \newline Converter} \\
    \midrule
    #1
    \bottomrule
\end{tabular}
\end{sidewaystable}
}

\newcommand{\fentry}[6]{#1 & #2 & #3 & #4 & #5 & #6\\}

\localauthor{Stefanie Gürster}

\subsection{Namenskonvention}

Zur Vereinheitlichung der Namensgebung der Komponenten in den Facelets verfahren wir nach dem \emph{kebab-case} Prinzip. Um den genauen Typ in die ID mit aufzunehmen, wird das entsprechende Suffix angehängt.

\todo{Suffix:}
 graphicImage = gIm link =lnk inputText =itext commandButton = cbtn

\subsection{Templates}

\begin{samepage}
	\todo{Serachfield needs view param?}
\textbf{navigation.xhtml} \phantomsection \label{flt:navbar} ist die Kopfzeile der Webanwendung. Diese bietet die Suchfunktion, Links zu verschiedenen Listen und dem Profil an.
\nopagebreak

\todo{Absprechen mit Sebastian und BB nicht vollstäandig}
\ftable{
	\fentry{logo-gimg}{<h:graphicImage>}{Logo der Applikation.}{value="#{}"}{/}{/}
	
	\fentry{direct-to-home-lnk}{<h:link>}{Weiterleitung zur Homepage.}{outcome="/authenticated/homepage" \newline title="#{tooltip.toHome}"}{/}{/}
	
	\fentry{search-frm}{<h:form>}{Formular zur Suche.}{/}{/}{/}
	
	\fentry{search-Field-itxt}{<h:inputText>}{Suchleiste}{value="#{navigationBacking.resultListParameters.globalSearchWord}"}{/}{/}
	
	\fentry{search-cbtn}{<h:commandButton>}{Suche ausführen.}{action="#{navigationBacking.search}"}{/}{/}
	
	\fentry{user-list-lnk}{<h:link>}{Link zur Übersichtsseite aller Nutzer}{outcome="/editor/userList}{render=""}
	
	\fentry{forumListLink}{link}{Link zur Übersichtsseite alle Journale und Konferenzen}{Alle}
	
	\fentry{logoutButton}{commandButton}{Loggt den Nutzer aus dem System aus und leitet zur Loginseite weiter}{Alle}
	
	\fentry{profileLink}{link}{Link zur Profilübersicht}{N}


\end{samepage}

\subsection{Seiten}

