%% Macros
\newcommand{\ftable}[1]{\begin{sidewaystable}
\begin{tabular}[H]{ m{2cm} m{3cm} m{6cm} m{2.5cm} m{2cm} m{2cm} }
    \toprule
    \textbf{ID} & \textbf{Typ} & \textbf{Beschreibung} & \textbf{Binding} & \textbf{Constraints} & \textbf{Validator \newline Converter} \\
    \midrule
    #1
    \bottomrule
\end{tabular}
\end{sidewaystable}
}

\newcommand{\fentry}[6]{#1 & #2 & #3 & #4 & #5 & #6\\}

\localauthor{Stefanie Gürster}

\subsection{Namenskonvention}

Zur Vereinheitlichung der Namensgebung der Komponenten in den Facelets verfahren wir nach dem \emph{kebab-case} Prinzip. Um den genauen Typ in die ID mit aufzunehmen, wird das entsprechende Suffix angehängt.

\todo{Suffix: graphicImage = gIm}

\subsection{Templates}

\begin{samepage}
\textbf{navigation.xhtml} \phantomsection \label{flt:navbar} ist die Kopfzeile der Webanwendung. Diese bietet die Suchfunktion, Links zu verschiedenen Listen und dem Profil an.
\nopagebreak

\ftable{
	\fentry{logo-gIm}{<h:graphicImage>}{Logo der Applikation.}{value="#{Alle}}
	
	\fentry{directToHome}{link}{Weiterleitung zur Homepage.}{Alle}
	
	\fentry{searchField}{inputText}{Suchleiste}{Alle}
	
	\fentry{search}{commandButton}{Suche ausführen.}{Alle}
	
	\fentry{userListLink}{link}{Link zur Übersichtsseite aller Nutzer}{E}
	
	\fentry{forumListLink}{link}{Link zur Übersichtsseite alle Journale und Konferenzen}{Alle}
	
	\fentry{logoutButton}{commandButton}{Loggt den Nutzer aus dem System aus und leitet zur Loginseite weiter}{Alle}
	
	\fentry{profileLink}{link}{Link zur Profilübersicht}{N}


\end{samepage}

\subsection{Seiten}

