%% Macros
\newcommand{\ftable}[3]{\begin{longtable}{m{1.5cm} m{1.5cm} m{3cm} m{7cm} m{4cm} m{4cm}}
	\caption{#2}\label{flt:#1} \\
	\toprule
	\textbf{ID} & \textbf{Typ} & \textbf{Beschreibung} & \textbf{Binding} & \textbf{Constraints} & \textbf{Validator/Converter}
	\\
	\midrule
	\endfirsthead

	\caption{Forsetzung}\\
    \toprule
    \textbf{ID} & \textbf{Typ} & \textbf{Beschreibung} & \textbf{Binding} & \textbf{Constraints} & \textbf{Validator/Converter}
	\\
	\midrule
	\endhead

	\midrule
	\multicolumn{6}{r}{{Fortsetzung auf der nächsten Seite}}
	\endfoot

	\bottomrule
	\endlastfoot

    #3
\end{longtable}
}

\newcommand{\fentry}[6]{\footnotesize#1 &\footnotesize#2 &\footnotesize#3 &\footnotesize#4 &\footnotesize#5 &\footnotesize#6\\}

\localauthor{Stefanie Gürster}

\subsection{Namenskonvention}

Zur Vereinheitlichung der Namensgebung der Komponenten in den Facelets verfahren wir nach dem \emph{kebab-case} Prinzip. Um den genauen Typ in die ID mit aufzunehmen, wird das entsprechende Suffix angehängt.


\begin{table}[H]
	\centering
	\begin{tabular}{|c|c|}
		\hline
		\textbf{Typ} & \textbf{Suffix} \\
		\hline \hline
		graphicImage & gimg \\ \hline
		link & lnk \\ \hline
		inputText & itxt \\ \hline
		outputText & otxt \\ \hline
		commandButton & cbtn \\ \hline
		messages & msgs \\ \hline
		dataTable & dtbl \\ \hline
		selectOneMenu & slctom \\ \hline
		panelGrid & pgrd \\ \hline
		commandLink & clnk \\ \hline
		inputSecret & iscrt \\ \hline
		inputFile & ifile \\ \hline
		column & clmn \\ \hline
		selectBooleanCheckbox & slctbc \\ \hline
	\end{tabular}
\end{table}

\subsection{Templates} $~$

%\begin{samepage}

%\nopagebreak

\begin{landscape}
\captionsetup{width=\linewidth}

\ftable{navigation}{\textbf{navigation.xhtml} ist die Kopfzeile der Webanwendung. Diese bietet die Suchfunktion, Links zu verschiedenen Listen und dem Profil an.}{
	\fentry{logo-gimg}{<h:graphicImage>}{Logo der Applikation.}{value="\#\{navigationBacking.systemSettings.logoImage\}"\}"}{/}{/}

	\fentry{direct-to-home-lnk}{<h:link>}{Weiterleitung zur Homepage.}{outcome="/facelets/authenticated/homepage" \newline title="\#\{help.toHome\}"}{/}{/}

	\fentry{search-frm}{<h:form>}{Formular zur Suche.}{/}{/}{/}

	\fentry{search-Field-itxt}{<h:inputText>}{Suchleiste}{value="\#\{navigationBacking.resultListParameters.globalSearchWord\}" \newline title="\#\{help.search\}"}{/}{/}

	\fentry{search-cbtn}{<h:commandButton>}{Suche ausführen.}{action="\#\{navigationBacking.search\}"}{/}{/}

	\fentry{user-list-lnk}{<h:link>}{Link zur Übersichtsseite aller Nutzer}{value="\#\{label.userList\}" \newline outcome="/editor/userList \newline title="\#\{help.userList\}"}{rendered="\#\{navigationBacking.sessionInformation.user.editor\}"}{/}

	\fentry{forum-list-lnk}{<h:link>}{Link zur Übersichtsseite alle Journale und Konferenzen}{value="\#\{label.forumList\}" \newline outcome="/facelets/authenticated/scientificForumList" \newline title="\#\{help.forumList\}"}{/}{/}

	\fentry{logout-cbtn}{<h:commandButton>}{Loggt den Nutzer aus dem System aus und leitet zur Loginseite weiter.}{value="\#\{label.logout\}" \newline action="\#\{navigationBacking.logout\}" \newline title="\#\{help.logout\}"}{/}{/}

	\fentry{profile-lnk}{<h:link>}{Link zur Profilübersicht}{value="\#\{label.nav.profile\}" \newline outcome="/facelets/authenticated/profile"}{/}{/}
}

%\end{samepage}


\begin{samepage}

	\ftable{main}{\textbf{main.xhtml} ist das Template, welches den Inhalt der Seiten zwischen Kopf- und Fußzeile einbettet.}{
		\fentry{/}{<ui:include>}{Kopfzeile}{src="navigation.xhtml"}{/}{/}

		\fentry{global-msgs}{<h:mes-sages>}{Anzeigen globaler Nachrichten.}{/}{/}{/}

		\fentry{/}{<ui:insert>}{Seiteninhalt}{name="main-content"}{/}{/}

		\fentry{/}{<ui:include>}{Fußzeile}{src="footer.xhtml"}{/}{/}
	}
\end{samepage}

\begin{samepage}

	\ftable{footer}{\textbf{footer.xhtml} ist die Fußzeile der Webanwendung und für alle sichtbar.}{
		\fentry{imprint-lnk}{<h:link>}{Link zur Seite des Impressums}{value="\#\{label.imprint\}" \newline outcome="/facelets/anonymous/imprint" \newline title="\#\{help.imprint\}"}{/}{/}

		\fentry{language-otxt}{<h:outputText>}{Anzeige der eingestellten Sprache.}{value="\#\{footerBacking.sessionInformation.locale.getLanguage()\}"}{/}{/}

	}
\end{samepage}

\begin{samepage}

	\ftable{toolbar}{\textbf{toolbar.xhtml} ist die Seitenleiste für Editoren und Administratoren auf der Einreichungsübersicht. Hier wird die Einreichung verwaltet.}{
		\fentry{add-reviewer-iTxt}{<h:inputText>}{Eingabefeld zur Angabe einer E-Mail-Adresse}{value="\#\{toolbarBacking.reviewerInput.email\}" \newline title="\#{help.reviewerEmail}}{/}{emailAddressExistsValidator \newline emailAddressLayoutValidator \newline required="true" \newline requiredMessage=\newline"\#\{message.addReviewer\}"}

		\fentry{add-deadline-reviewer-iTxt}{<h:inputText>}{Eingabefeld zur Angabe einer Deadline.}{value="\#\{toolbarBacking.reviewedByInput.timestampDeadline\}" \newline title="\#\{help.reviewerDeadline\}"}{/}{converterDateTime \newline patter="dd/MM/yyyy"}

		\fentry{add-reviewer-cbtn}{<h:commandButton>}{Knopf zum Hinzufügen des Gutachters}{value="\#\{label.addReviewer\}" \newline action="\#\{toolbarBacking.addReviewer\}" \newline title="\#\{help.addReviewer\}"}{/}{/}

		\fentry{reviewer-dtbl}{<h:dataTable>}{Liste der Gutachter}{value="\#\{toolbarbacking.reviewer\}" \newline var="review"}{/}{/}

		\fentry{remove-reviewer-cbtn}{<h:commandButton>}{Entferne einen bestimmten Gutachters.}{action="\#\{toolbarBacking.removeReviewer\}" \newline title="\#\{help.removeReviewer\}"}{/}{/}

		\fentry{select-editor-slctom}{<h:selectOneMenu>}{Auswahländerung des verwaltenden Editors}{value="\#\{toolbarBacking.editorInput.firstName + toolbarBacking.editorInput.lastName\}" \newline title="\#\{help.selectEditor\}"}{/}{/}

		\fentry{/}{<f:selectItems>}{Liste der möglichen Editoren.}{value="\#\{toolbar.editors\}" \newline var="entry" \newline itemLabel="\#\{entry.firstName + entry.lastName\}" \newline itemValue="\#\{entry\}"}{}{}

		\fentry{save-editor-cbtn}{<h:commandButton>}{Speichern der Änderung des Editors.}{value="\#\{label.saveEditor\}" \newline action=\newline"\#\{toolbarBacking.chooseNewManagingEditor\}" \newline title="\#\{help.saveEditor\}"}{/}{/}

		\fentry{show-editor-otxt}{<h:outputText>}{Anzeige des aktuellen Editors.}{value="\#\{toolbarBacking.currentEditor\}"}{/}{/}

		\fentry{revision-deadline-iTxt}{<h:inputText>}{Eingabe zur Angabe einer Deadline.}{value="\#\{toolbarBacking.submission.deadlineRevision\}" \newline title="\#\{help.revisionDeadline\}"}{/}{converterDateTime \newline patter="dd/MM/yyyy"}

		\fentry{require-revision-cbtn}{<h:commandButton>}{Fordert den Einreicher auf, seine Einreichung zu überarbeiten}{value="\#\{label.requireRevision\}" \newline action="\#\{toolbarBacking.requireRevision\}" \newline title="\#\{help.requireRevision\}"}{/}{/}

		\fentry{accept-submission-cbtn}{<h:commandButton>}{Akzeptiere die Einreichung}{value="\#\{label.acceptSubmission\}" \newline action="\#\{toolbarBacking.acceptSubmission\}" \newline title="\#\{help.accesptSubmission\}"}{/}{/}

		\fentry{reject-submission-cbtn}{<h:commandButton>}{Lehne die Einreichung ab}{value="\#\{label.rejectSubmission\}" \newline action="\#\{toolbarBacking.rejectSubmission\}" \newline title="\#\{help.rejectSubmission\}"}{/}{/}

	}
\end{samepage}

\begin{samepage}

	\ftable{pagination}{\textbf{pagination.xhtml} ist ein Composite Component zur Paginierung von Listen.}{
		\fentry{/}{<composite:attribute>}{Paginator Attribut.}{name="paginator" \newline type="de.lases.control.internal.Pagination"}{required="true"}{/}

		\fentry{is-empty-otxt}{<h:outputText>}{Hinweis wenn die Tabelle leer ist.}{value="\#\{label.emptyList\}"}{rendered="\#\{cc.attrs.paginator.entries.isEmpty()\}"}{/}

		\fentry{pagination-dtbl}{<h:dataTable>}{Hier werden die Spalten eingefügt. Durch Klicken auf die Spaltennamen werden die Einträge sortiert. Zusätzlich kann durch ein Textfeld die Liste gefiltert werden.}{value="\#\{cc.attrs.paginator.entries\}"\newline var="var"}{rendered="\#\{!cc.attrs.paginator.entries.isEmpty()\}"}{/}

		\fentry{/}{<composite:insertChildren>}{Einfügen benötigter Spalten.}{/}{/}{/}

		\fentry{apply-cbtn}{<h:commandButton>}{Mit diesem Knopf können gewählte Filter angewandt werden.}{value="\#\{label.applyFilter\}" \newline action="\#\{cc.attrs.paginator.loadData\}" \newline title="\#\{help.applyFilter\}"}{/}{/}

		\fentry{navigation-pgrd}{<h:panelGrid>}{Hier befinden sich die Knöpfe zur Navigation zwischen den Seiten der Paginierungen.}{columns="5"}{rendered="\#\{!cc.attrs.paginator.entries.isEmpty()\}"}{/}

		\fentry{first-page-cbtn}{<h:commandButton>}{Zurück zur ersten Seite.}{value="\#\{label.firstPage\}" \newline action="\#\{cc.attrs.paginator.firstPage\}" \newline title="\#\{help.firstPage\}"}{/}{/}

		\fentry{prev-page-cbtn}{<h:commandButton>}{Eine Seite zurück.}{value="\#\{label.prevPage\}" \newline action="\#\{cc.attrs.paginator.priviousPage\}" \newline title="\#\{help.prevPage\}"}{/}{/}

		\fentry{current-page-otxt}{<h:outputText>}{Nummer der aktuellen Seite.}{value="\#\{cc.attrs.paginator.currentPage\}" \newline title="\#\{help.currentPage\}"}{/}{/}

		\fentry{next-page-cbtn}{<h:commandButton>}{Eine Seite weiter.}{value="\#\{label.nextPage\}" \newline action="\#\{cc.attrs.paginator.nextPage\}" \newline title="\#\{help.nextPage\}"}{/}{/}

		\fentry{last-page-cbtn}{<h:commandButton>}{Zur letzten Seite.}{value="\#\{label.lastPage\}" \newline action="\#\{cc.attrs.paginator.lastPage\}" \newline title="\#\{help.lastPage\}"}{/}{/}

	}
\end{samepage}

\begin{samepage}

	\ftable{sortfiltercolumn}{\textbf{sortfiltercolumn.xhtml} ist ein Composite Component, das eine sortierbare und filterbare Spalte einer Liste modelliert.}{
		\fentry{/}{<composite:attribute>}{Implementation der abstrakten Klasse.}{name="paginator" \newline type="de.lases.control.internal.Pagination"}{required="true"}{/}

		\fentry{/}{<composite:attribute>}{Anzeigename im Spaltenkopf.}{name="columnIdent" \newline type="java.lang.String"}{required="true"}{/}

		\fentry{/}{<composite:attribute>}{String Identifier der Spalte.}{name="columnName" \newline type="java.lang.String"}{required="true"}{/}

		\fentry{/}{<composite:attribute>}{Spaltennummer, null indiziert.}{name="coulumnNo" \newline type="java.lang.String"}{required="true"}{/}

		\fentry{column-name-clnk}{<h:commandLink>}{Durch Klicken auf den Spaltennamen wird nach dieser Spalte aufsteigend, bzw. absteigend sortiert.}{value="\#\{cc.attrs.columnNamei18n\}" \newline action="\#\{cc.attrs.paginator.sortBy[cc.attrs.columnNo]\}"}{/}{/}

		\fentry{column-filter-itxt}{<h:inputText>}{Hier kann ein Wert eingegeben werden, nach dem die Spalte dann gefiltert wird.}{value="\#\{cc.attrs.paginator.searchWords[cc.attrs.columnNo]\}"}{/}{/}
	}

\end{samepage}

\subsection{Seiten}$~$

\subsubsection{Anonymer Nutzer}$~$

\begin{samepage}

	\ftable{welcome}{\textbf{welcome.xhtml} Auf der Login- bzw. Welcome-Seite wird das System vorgestellt.
		Zusätzlich gibt es ein Login-Formular zur Anmeldung im System.
		Für nicht registrierte Nutzer wird man über einen gegebenen Link zu \emph{registration.xhtml} weitergeleitet.}{

		\fentry{welcome-heading-otxt}{<h:outputText>}{Überschrift der Welcomepage.}{value="\#\{welcomeBacking.systemSettings.headlineWelcomePage\}"}{/}{/}

		\fentry{welcome-otxt}{<h:outputText>}{Bewerben der Anwendung mithilfe einer Kurzbeschreibung.}{value="\#\{welcomeBacking.systemSettings.messageWelcomePage\}"}{A/}{/}

		\fentry{email-itxt}{<h:inputText>}{Textfeld für E-Mail Eingabe.}{value="\#\{welcomeBacking.loginInput.emailAddress\}" \newline title="\#\{help.loginEmail\}"}{/}{emailAddressExistsValidator, emailAddressLayoutValidator, required="true" \newline requiredMessage=\newline"\#\{message.email\}"}

		\fentry{password-iscrt}{<h:inputSecret>}{Textfeld für Passwort Eingabe.}{value="\#\{welcomeBacking.loginInput.passwordNotHashed\}" title="\#\{help.loginPassword\}"}{/}{passwordValidator \newline required="true" \newline requiredMessage=\newline"\#\{message.password\}"}

		\fentry{login-cbtn}{<h:commandButton>}{Ausführen des Login-Prozesses.}{value="\#\{label.login\}" \newline action="\#\{welcomeBacking.login\}" \newline title="\#\{help.login\}"}{/}{/}

		\fentry{register-cbtn}{<h:commandButton>}{Weiterleitung zur Registrierung.}{value="\#\{label.register\}" \newline action="\#\{welcomeBacking.goToRegister\}" \newline title="\#\{help.register\}"}{/}{/}

	}
\end{samepage}

\begin{samepage}

	\ftable{registration}{\textbf{registration.xhtml} Seite zur Registrierung anonymer Nutzer.}{
		\fentry{title-itxt}{<h:inputText>}{Angabe eines Titels.}{value="\#\{registrationBacking.newUser.title\}" \newline title="\#\{help.title\}"}{/}{/}

		\fentry{first-name-itxt}{<h:inputText>}{Angabe des Vornamens.}{value="\#\{registrationBacking.newUser.firstName\}" \newline title="\#\{help.firstName\}"}{/}{required="true" \newline requiredMessage=\newline"\#\{message.firstName\}"}

		\fentry{last-name-itxt}{<h:inputText>}{Angabe des Nachnamen.}{value="\#\{registrationBacking.newUser.lastName\}" \newline title="\#\{help.lastName\}"}{/}{required="true" \newline requiredMessage=\newline"\#\{message.lastName\}"}

		\fentry{password-iscrt}{<h:inputSecret>}{Angabe eines Passwortes.}{value="\#\{registrationBacking.newUser.passwordNotHashed\}" \newline title="\#\{help.password\}"}{/}{passwordValidator \newline required="true" \newline requiredMessage=\newline"\#\{message.password\}"}

		\fentry{email-itxt}{<h:inputText>}{Angabe einer validen E-Mail.}{value="\#\{registrationBacking.newUser.emailAddress\}" \newline title="\#\{help.email\}"}{ /}{emailAddressExistsValidator \newline emailAddressLayoutValidator \newline required=true" \newline requiredMessage=\newline"\#\{message.email\}"}

		\fentry{register-cbtn}{<h:commandButton>}{Ausführen des Registrations-Prozesses.}{value="\#\{label.register\}"\newline action="\#\{registrationBacking.register\}"}{/}{/}

		\fentry{welcomepage-lnk}{<h:link>}{Weiterleitung zur Login Seite.}{value="\#\{label.welcomeLink\}" \newline outcome="/facelets/anonymous/welcome" \newline title="\#\{help.welcomepage\}"}{/}{/}
	}
\end{samepage}

\begin{samepage}

	\ftable{verification}{\textbf{verification.xhtml} Wird bei erfolgreicher Verifikation der E-Mail angezeigt.}{
		\fentry{go-to-home-cbtn}{<h:commandButton>}{Button zur Weiterleitung auf die Homepage bei erfolgreicher Verifizierung. Ansonsten wird auf \emph{welcome.xhtml} weiter geleitet.}{/}{/}

	}
\end{samepage}


\begin{samepage}

	\ftable{errorPage}{\textbf{errorPage.xhtml} Auf diese Seite wird navigiert, wenn auf eine nicht existierende URL oder auf eine URL zugegriffen wird, auf die man keine Zugriffsrechte hat.}{

		\fentry{error-message-otxt}{<h:outputText>}{Anzeige einer Fehlermeldung.}{value="\#\{errorPageBacking.errorMessage.message\}"}{/}{/}

		\fentry{stack-trace-otxt}{<h:outputText>}{Stacktrace für den "Development Mode".}{value="\#\{errorPageBacking.errorMessage.stackTrace\}"}{/}{rendered="\#\{errorPageBacking.isDevelopmentMode()\}"}

	}
\end{samepage}

\begin{samepage}

	\ftable{imprint}{\textbf{imprint.xhtml} Das Impressum gibt die vom Administrator angegebenen Kontaktdaten des Betreibers wieder.}{
		\fentry{logo-gimg}{<h:graphicImage>}{Logo des Betreibers.}{value="\#\{imprintBacking.systemSettings.logoImage\}"}{/}{/}

		\fentry{imprint-heading-otxt}{<h:outputText>}{Überschrift der Ansicht}{value="\#{imprintBacking.systemSetting.companyName}"}{/}{/}

		\fentry{imprint-otxt}{<h:outputText>}{Impressum des Betreibers.}{value="\#{imprintBacking.systemSetting.imprint}"}{/}{/}
	}
\end{samepage}

\subsubsection{Angemeldeter Nutzer}$~$


\begin{samepage}

	\ftable{homepage}{\textbf{homepage.xhtml} Startseite die den Überblick über alle Einreichungen, aufgeteilt in Reiter, beinhaltet.}{
		\fentry{your-submission-clnk}{<h:commandLink>}{Reiter zum Anzeigen der eigenen Einreichungen.}{value="\#\{label.yourSubmission\}" \newline action="\#\{homepageBacking\newline.showOwnSubmissionsTab\}" \newline title="\#\{help.yourSubmissions\}"}{/}{/}

		\fentry{your-reviews-clnk}{<h:commandLink>}{Reiter zum Anzeigen der zu begutachtenden Einreichungen}{value="\#\{label.yourReviews\}" \newline action="\#\{homepageBacking\newline.showSubmissionToReviewTab\}" \newline title="\#\{help.yourReviews\}"}{rendered="\#\{homepageBacking.sessionInformation.user.reviewer\}"}{/}

		\fentry{editorial-overview-clnk}{<h:commandLink>}{Reiter zum Anzeigen aller Einreichungen, welche man editiert.}{value="\#\{label.editorialOverview\}" \newline action="\#\{homepageBacking\newline.showSubmissionToEditTab\}" \newline title="\#\{help.editorialOverview\}"}{rendered="\#\{homepageBacking.sessionInformation.editor\}"}{/}
	}
\end{samepage}



\begin{samepage}

	\ftable{paginations}{Die Tabellen \emph{submissionPagination}, \emph{reviewPagination} und \emph{editorPagination} sind nach den gleichen Punkten gegliedert und daher nur einmal ausführlich aufgegliedert: Titel des Papers, Datum, Deadline und Status der Einreichung und Name des zugehörigen Forums.}{
		\fentry{submission-pg}{<pg:pagination>}{Liste aller eignen Einreichungen.}{paginator="\#\{homepageBacking.submissionPagination\}"}{/}{/}

		\fentry{title-ssc}{<ssc:sortsearchcolumn>}{Spalte für den Titel des Papers.}{paginator="\#\{homepageBacking.submissionPagination\}"\newline columnNamei18n="\#\{label.paperTitle\}"}{/}{/}

		\fentry{title-lnk}{<h:link>}{Titel des Einreichung.}{value="\#\{var.title\}" \newline outcome="/facelets/authenticated/submission" \newline includeViewParams="true"}{/}{/}

		\fentry{/}{<f:param>}{Parameter für id der Einreichung.}{value="\#\{var.id\}" name="id"}{/}{/}

		\fentry{date-ssc}{<ssc:sortsearchcolumn>}{Spalte für das Abgabedatum des Papers.}{paginator="\#\{homepageBacking.submissionPagination\}"\newline columnNamei18n="\#\{label.date\}"}{/}{/}

		\fentry{date-scltom}{<h:selectOneMenu>}{Auswahl der Filtermöglichkeit eines Datums.}{value="\#\{homepageBacking.dateFilterSelectSub\}" \newline title="\#\{help.dateSelect\}"}{/}{/}

		\fentry{/}{<f:selectItems>}{Mögliche Filterung des Datums.}{value="\#\{homepageBacking.dateSelects\}" \newline var="entry" \newline itemLabel="\#\{entry.toString()\}" \newline itemValue="\#\{entry\}"}{/}{/}

		\fentry{date-otxt}{<h:outputText>}{Abgabedatum des Papers.}{value="\#\{var.submissionTime\}"}{/}{convertDateTime \newline pattern=""yyyy-MM-dd"}

		\fentry{deadline-ssc}{<ssc:sortsearchcolumn>}{Spalte für die Deadline der Revision.}{paginator="\#\{homepageBacking.submissionPagination\}"\newline columnNamei18n="\#\{label.deadline\}"}{/}{/}

		\fentry{deadline-otxt}{<h:outputText>}{Deadline des Papers.}{value="\#\{var.deadlineRevision\}"}{/}{convertDateTime \newline pattern=""yyyy-MM-dd"}

		\fentry{state-ssc}{<ssc:sortsearchcolumn>}{Spalte für den Status der Einreichung.}{paginator="\#\{homepageBacking.submissionPagination\}"\newline columnNamei18n="\#\{label.stateSub\}"}{/}{/}

		\fentry{state-scltom}{<h:selectOneMenu>}{Auswahl der Filtermöglichkeit eines Status.}{value="\#\{homepageBacking.stateFilterSelectSub\}" \newline title="\#\{help.stateSubSelect\}"}{/}{/}

		\fentry{/}{<f:selectItems>}{Mögliche Filterung des Status.}{value="\#\{homepageBacking.submissionStates\}" \newline var="entry" \newline itemLabel="\#\{entry.toString()\}" \newline itemValue="\#\{entry\}"}{/}{/}

		\fentry{state-otxt}{<h:outputText>}{Status der Einreichung.}{value="\#\{var.state\}"}{/}{/}

		\fentry{forum-ssc}{<ssc:sortsearchcolumn>}{Spalte für zugehöriges Forum.}{paginator="\#\{homepageBacking.submissionPagination\}"\newline columnNamei18n="\#\{label.forumSub\}"}{/}{/}

		\fentry{forum-lnk}{<h:link>}{Zugehöriges Forum der Einreichung.}{value="\#\{homepageBacking.scientificForumForSubmission\}" outcome="/facelets/authenticated/scientificForum" \newline includeViewParams="true"}{/}{/}

		\fentry{/}{<f:param>}{Parameter für id des Forums.}{value="\#\{var.scientificForumId\}" name="id"}{/}{/}
	}
\end{samepage}

\begin{samepage}

	\ftable{submission}{\textbf{sumbission.xhtml} ist die Übersichtsseite einer Abgabe.
		Hier werden alle mit der Einreichung verbundenen Aktivitäten abgebildet.}{
		\fentry{/}{<ui:include>}{\hyperref[flt:toolbar]{Seitenleiste mit Tools.}}{rendered="\#\{submissionBacking.sessionInformation.user.editor\}"}{/}

		\fentry{/}{<f:viewParam>}{Id der Einreichung.}{value="\#\{submissionBacking.submission.id\}" \newline name="id"}{/}{/}

		\fentry{/}{<f:viewAction>}{Laden der entsprechenden Einreichung.}{action="\#\{submissionBacking.onLoad\}"}{}{}

		\fentry{upload-revision-ifile}{<h:inputFile>}{Upload einer Revision.}{value="\#\{submissionBacking.uploadedRevisionPDF\}" }{rendered="\#\{submissionBacking.isViewerSubmitter\}"}{required="true" \newline requiredMessage=\newline"\#\{message.inputFile\}"}

		\fentry{upload-clnk}{<h:commandLink>}{Ausführen des Uploads.}{value="\#\{label.uploadFile\}" \newline action="\#\{submissionBacking.uploadPDF\}" \newline includeViewParams="true" title=\newline"\#\{help.uploadRevision\}"}{rendered="\#\{submissionBacking.isViewerSubmitter\}"}{/}

		\fentry{new-review-cbtn}{<h:commandButton>}{Weiterleitung zur Einreichung eines neuen Gutachtens.}{value="\#\{label.uploadReview\}" \newline action="\#\{submissionBacking.uploadReview\}" \newline title="\#\{help.uploadRevision\}"}{rendered="\#\{submissionBacking.sessionInformation.user.reviewer\}"}{/}

		\fentry{deadline-review-otxt}{<h:outputText>}{Deadline der Einreichung.}{value="\#\{submissionBacking.reviewedBy.timestampDeadline\}"}{rendered="\#\{submissionBacking.sessioninformation.user.reviewer\}"}{/}

		\fentry{accept-reviewing-cbtn}{<h:commandButton>}{Annahme der Rolle eines Gutachters für diese Einreichung.}{value="\#\{label.acceptReviewing\}" \newline action="\#\{submissionBacking.acceptReviewing\}" \newline title="\#\{help.acceptReviewing\}"}{rendered="\#\{submissionBacking.sessionInformation.user.reviewer\}"}{/}

		\fentry{decline-reviewing-cbtn}{<h:command>}{Ablehnen der Rolle eines Gutachters für diese Einreichung.}{value="\#\{label.declineReviewing\}" \newline action="\#\{submissionBacking.declineReviewing\}" \newline title="\#\{help.declineReviewing\}"}{rendered="\#\{submissionBacking.sessionInformation.user.reviewer\}"}{/}

		\fentry{state-slctom}{<h:selectOneMenu>}{Status der Submission.}{value="\#\{submissionBacking.submission.state\}"}{readonly="\#\{not submissionBacking.sessionInformation.user.editor\}"}{/}

		\fentry{/}{<f:selectItems>}{Auswahl des Status.}{value="\#\{submissionBacking.submissionState\}"\newline var="entry" \newline itemLabel="\#\{entry.toString()\}" \newline itemValue="\#\{entry\}"}{/}{/}

		\fentry{state-otxt}{<h:outputText>}{Status der Einreichung.}{value="\#\{submissionBacking.submission.state\}"}{/}{/}

		\fentry{forum-lnk}{<h:link>}{Name des Forums und Weiterleitung zur Forumsübersicht.}{value="\#\{submissionBacking.scientificForum.name\}" \newline outcome="/facelets/authenticated/scientificForum" \newline includeViewParams="true" \newline title="\#\{help.directToForum\}"}{disabled="\#\{not submissionBacking.sessionInformation.user.editor\}"}{/}

		\fentry{/}{<f:param>}{Parameter für id des Forums.}{value="\#\{submissionBacking.scientificForum.scientificForumId\}" name="id"}{/}{/}

		\fentry{author-lnk}{<h:link>}{Name des Autors und Weiterleitung zum Profil.}{value="\#\{submissionBacking.author.firstName + submissionBacking.author.lastName\}" \newline outcome="/facelets/authenticated/profile" \newline includeViewParams="true" \newline title="\#\{help.directProfile\}"}{disabled="\#\{not submissionBacking.sessionInformation.user.editor\}"}{/}

		\fentry{/}{<f:param>}{Parameter für id des Profil.}{value="\#\{submissionBacking.author.id\}" name="id"}{/}{/}

		\fentry{email-lnk}{<h:link>}{E-Mail-Adressen der Co-Autoren und Mailto-Link.}{value="\#\{submissionBacking.author.emailAddress\}"\newline  outcome="\#\{'mailto:'.concat(submissionBacking\newline.author.emailAddress)\}"}{disabled="\#\{not submissionBacking.sessionInformation.user.editor\}"}

		\fentry{deadline-revision-otxt}{<h:outputText>}{Deadline der Revision}{value="\#\{submissionBacking.submission.deadlineRevision\}"}{/}{/}
	}
\end{samepage}

\begin{samepage}

	\ftable{paperPagination}{\emph{paperPagination} enthält die Spalten: Version, Datum, Sichtbarkeit(E) und Download.Die Elemente der Liste, der unterschiedlichen Versionen der Einreichung, sieht wie folgt aus:}{
		\fentry{paper-pg}{<pg:pagination>}{Liste aller eignen Einreichungen.}{paginator="\#\{submissionBacking.paperPagination\}"}{/}{/}

		\fentry{version-ssc}{<ssc:sortsearchcolumn>}{Spalte für die Version des Papers.}{paginator="\#\{submissionBacking.paperPagination\}"\newline columnNamei18n="\#\{label.paperVersion\}"}{/}{/}

		\fentry{version-otxt}{<h:outputText>}{Version des Papers.}{value="\#\{var.submissionId\}"}{/}{/}

		\fentry{date-ssc}{<ssc:sortsearchcolumn>}{Spalte für das Abgabedatum des Papers.}{paginator="\#\{submissionBacking.paperPagination\}"\newline columnNamei18n="\#\{label.date\}"}{/}{/}

		\fentry{date-scltom}{<h:selectOneMenu>}{Auswahl der Filtermöglichkeit eines Datums.}{value="\#\{submissionBacking.dateFilterSelectPaper\}" \newline title="\#\{help.dateSelect\}"}{/}{/}

		\fentry{/}{<f:selectItems>}{Mögliche Filterung des Datums.}{value="\#\{submissionBacking.dateSelect\}" \newline var="entry" \newline itemLabel="\#\{entry.toString()\}" \newline itemValue="\#\{entry\}"}{/}{/}

		\fentry{date-otxt}{<h:outputText>}{Abgabedatum des Papers.}{value="\#\{var.uploadTime\}"}{/}{convertDateTime \newline pattern=""yyyy-MM-dd"}

		\fentry{release-revision-ssc}{<ssc:sortsearchcolumn>}{Spalte für den Freigabestatus.}{paginator="\#\{submissionBacking.paperPagination\}"\newline columnNamei18n="\#\{label.releaseRevision\}"}{rendered="\#\{submissionBacking.sessionInformation.user.editor || submissionBacking.sessionInformation.user.reviewer \}"}{/}

		\fentry{release-revision-scltom}{<h:selectOneMenu>}{Auswahl der Filtermöglichkeit Freischalten der Revision}{value="\#\{submissionBacking.visibleFilterInputPaper\}" title="\#\{help.releaseRevisionSelect\}"}{rendered="\#\{submissionBacking.sessionInformation.user.editor || submissionBacking.sessionInformation.user.reviewer \}"}{/}

		\fentry{/}{<f:selectItem>}{Revision freigeschaltet.}{itemValue="true" \newline itemLabel="\#\{label.releaseRevisionTrue\}"}{/}{/}

		\fentry{/}{<f:selectItem>}{Revision nicht freigeschaltet.}{itemValue="false" \newline itemLabel="\#\{label.releaswRevisionFalse\}"}{/}{/}

		\fentry{release-review-otxt}{<h:outputText>}{Status der Freischaltung.}{value="\#\{label.releasedReview\}"}{rendered="\#\{var.visible \&\& submissionBacking.sessionInformation.user.reviewer\}"}{/}

		\fentry{release-revision-cbtn}{<h:commandButton>}{Freischalten einer Revision.}{value="\#\{label.releaseRevision\}" action="\#\{submissionBacking.releaseRevision(var)\}" title="\#\{help.releaseRevision\}"}{rendered="\#\{submissionBacking.sessionInfromation.user.editor\}"}{/}

		\fentry{download-file-clmn}{<h:column>}{Spalten für den Download der Datei.}{/}{/}{/}
	}
\end{samepage}

\begin{samepage}

	\ftable{reviewPagination}{\emph{reviewPagination} enthält die Spalten: Gutachter, Datum, Status, Empfehlung, Kommentar und Download.
		Ist ein Gutachten freigeschaltet, so erhält auch ein Nutzer Leserechte.
		Die Elemente der Liste von Gutachten sind folgende:}{
		\fentry{review-pg}{<pg:pagination>}{Liste aller oder der eignen Gutachten.}{paginator="\#\{submissionBacking.reviewPagination\}"}{/}{/}

		\fentry{reviewer-ssc}{<ssc:sortsearchcolumn>}{Spalte für den Namen des Gutachters.}{paginator="\#\{submissionBacking.reviewPagination\}"\newline columnNamei18n="\#\{label.reviewerName\}"}{rendered="\#\{submissionBacking.sessionInformation.user.editor \&\& submissionBacking.sessionInformation.user.reviewer\}"}{/}

		\fentry{reviewer-lnk}{<h:link>}{Namen des Gutachters.}{value="\#\{submissionBacking.getReviewerForReview(var)\}" \newline outcome="/facelets/authenticated/profile" \newline includeViewParams="true"}{rendered="\#\{submissionBacking.sessionInformation.user.editor \&\& submissionBacking.sessionInformation.user.reviewer\}"}{/}{/}

		\fentry{/}{<f:param>}{Parameter für id des Profil.}{value="\#\{submissionBacking.getReviewerForReview(var).id\}" name="id"}{/}{/}

		\fentry{date-ssc}{<ssc:sortsearchcolumn>}{Spalte für das Abgabedatum des Gutachtens.}{paginator="\#\{submissionBacking.reviewPagination\}"\newline columnNamei18n="\#\{label.date\}"}{/}{/}

		\fentry{date-scltom}{<h:selectOneMenu>}{Auswahl der Filtermöglichkeit eines Datums.}{value="\#\{submissionBacking.dateFilterSelectReview\}" title="\#\{help.dateSelect\}"}{/}{/}

		\fentry{/}{<f:selectItems>}{Mögliche Filterung des Datums.}{value="\#\{dateSelect\}" \newline var="entry" itemValue="\#\{entry\}" itemLabel=\newline"\#\{entry.toString()\}"}{/}{/}

		\fentry{date-otxt}{<h:outputText>}{Abgabedatum des Gutachtens.}{value="\#\{var.timestampUpdloaded\}"}{/}{convertDateTime \newline pattern=""yyyy-MM-dd"}

		\fentry{release-review-ssc}{<ssc:sortsearchcolumn>}{Spalte für den Freigabestatus.}{paginator="\#\{submissionBacking.reviewPagination\}"\newline columnNamei18n="\#\{label.releaseReview\}"}{rendered="\#\{submissionBacking.sessionInfromation.user.editor\}"}{/}

		\fentry{release-review-scltom}{<h:selectOneMenu>}{Auswahl der Filtermöglichkeit Freischalten der Gutachten}{value="\#\{submissionBacking.visibleFilterInputPaper\}" title="\#\{help.releaseReviewSelect\}"}{rendered="\#\{submissionBacking.sessionInformation.user.editor || submissionBacking.sessionInformation.user.reviewer \}"}{/}

		\fentry{/}{<f:selectItem>}{Gutachten freigeschaltet.}{itemValue="true" \newline itemLabel="\#\{label.releaseReviewTrue\}"}{/}{/}

		\fentry{/}{<f:selectItem>}{Gutachten nicht freigeschaltet.}{itemValue="false" \newline itemLabel="\#\{label.releaswReviewFalse\}"}{/}{/}

		\fentry{release-review-otxt}{<h:outputText>}{Status der Freischaltung.}{value="\#\{label.releasedReview\}"}{rendered="\#\{var.visible\}"}{/}

		\fentry{release-review-cbtn}{<h:commandButton>}{Freischalten eines Gutachtens.}{value="\#\{label.releaseReview\}" action="\#\{submissionBacking.releaseReview(var)\}" title="\#\{help.releaseReview\}"}{rendered="\#\{submissionBacking.sessionInfromation.user.editor\}"}{/}

		\fentry{recommen-dation-review-ssc}{<ssc:sortsearchcolumn>}{Spalte für die Empfehlung.}{paginator="\#\{submissionBacking.reviewPagination\}"\newline columnNamei18n="\#\{label.recommendation\}"}{rendered="\#\{submissionBacking.sessionInformation.user.editor || submissionBacking.sessionInformation.user.reviewer}{/}

		\fentry{recommen-dation-scltom}{<h:selectOneMenu>}{Auswahl der Filtermöglichkeit Empfehlung.}{value="\#\{submissionBacking.recommendationFilterInputReview\}" \newline title="\#\{help.recommendationSelect\}"}{rendered="\#\{submissionBacking.sessionInformation.user.editor || submissionBacking.sessionInformation.user.reviewer}{/}

		\fentry{/}{<f:selectItem>}{Positive Empfehlung.}{itemValue="true" \newline itemLabel="\#\{label.recommendationTrue\}"}{/}{/}

		\fentry{/}{<f:selectItem>}{Negative Empfehlung.}{itemValue="false" \newline itemLabel="\#\{label.recommendationFalse\}"}{/}{/}

		\fentry{recommen-dation-slctbc}{<h:selectBooleanCheckbox>}{Empfehlung eines Gutachters.}{value="\#\{var.acceptPaper\}" readonly="true"}{rendered="\#\{submissionBacking.sessionInformation.user.editor || submissionBacking.sessionInformation.user.reviewer}{/}

		\fentry{comment-ssc}{<ssc:sortsearchcolumn>}{Spalte für den Kommentar des Gutachters.}{paginator="\#\{submissionBacking.reviewPagination\}"\newline columnNamei18n="\#\{label.comment\}"}{/}{/}

		\fentry{comment-otxt}{<h:outputText>}{Kommentar des Gutachters.}{value="\#\{var.comment\}"}{/}{/}

		\fentry{download-file-clmn}{<h:column>}{Spalten für den Download der Datei.}{/}{/}{/}
	}
\end{samepage}

\begin{samepage}

	\ftable{paper-download}{Die Spalte zum Download der einzelnen Paper ist wie folgt definiert und ist beiden Paginations vorhanden:}{

		\fentry{download-header}{<f:facet>}{Spaltenkopf für den Download.}{name="header"}{/}{/}

		\fentry{pdf-submission-cbtn}{<h:commandButton>}{Download der Einreichung}{value="\#\{label.downloadFile\}" action="\#\{submissionBacking.downloadPaper(var)\}" title="\#\{help.downloadFile\}"}{/}{/}
	}

\end{samepage}



%% Ab hier sind Bastis hinzugefügte Änderungen


\begin{samepage}

	\ftable{newSubmission}{\textbf{newSubmission.xhtml} Hier können neue Paper eingereicht werden.}{
		\fentry{submission-title-itxt}{<h:inputText>}{Titel des abzugebenden Papers.}{value="\#\{newSubmissionBacking.newSubmission.title\}"}{/}{required="true" \newline requiredMessage=\newline "\#\{message.titleSub\}"}

		\fentry{forum-name-itxt}{<h:inputText>}{Name des Forums, bei welchem abgegeben wird.}{value="\#\{newSubmissionBacking.newSubmission.forumInput.name\}"}{/}{required="true" \newline requiredMessage=\newline "\#\{message.forum\}"}

		\fentry{editor-slctom}{<h:selectOneMenu>}{Angabe eines Editors}{value="\#\{newSubmissionBacking.selectedEditor\}"}{/}{/}

		\fentry{editor-lst}{<f:selectItems>}{Liste des Auswahlmöglichkeiten}{value="\#\{newSubmissionBacking.editors\}"}{/}{/}

		\fentry{pfd-upload-ifile}{<h:inputText>}{Angabe der Abgabedatei.}{value="\#\{newSubmissionBacking.uploadedPDF\}"}{/}{required="true" \newline requiredMessage=\newline "\#\{message.inputFile\}"}

		\fentry{titel-itxt}{<h:inputText>}{Angabe eines Titels eines Co-Autors.}{value="\#\{newSubmissionBacking.coAuthorInput.title\}"}{/}{/}

		\fentry{co-author-firstname-itxt}{<h:inputText>}{Angabe des Vornamen eines Co-Autors.}{value="\#\{newSubmissionBacking.coAuthorInput.firstName\}"}{/}{/}

		\fentry{co-author-lastname-itxt}{<h:inputText>}{Angabe des Namen eines Co-Autors.}{value="\#\{newSubmissionBacking.coAuthorInput.lastName\}"}{/}{/}

		\fentry{co-author-email-itxt}{<h:inputText>}{Angabe der E-Mail eines Co-Autors.}{value="\#\{newSubmissionBacking.coAuthorInput.emailAddress\}"}{/}{/}

		\fentry{submit-co-author-cbtn}{<h:commandButton>}{Speichern des Co-Autors.}{value="\#\{label.save\}"\newline action="\#\{newSubmissionBacking.submitCoAuthor\}"}{/}{/}

		\fentry{co-author-list}{<h:dataTable>}{Anzeige aller Co-Autoren in einer Liste.}{value="\#\{newSubmissionBacking.coAuthors\}"}{/}{/}

		\fentry{delete-co-author-cbtn}{<h:commandButton>}{Löschen des zugehörigen Co-Authors.}{value="\#\{label.delete\}"\newline action="\#\{newSubmissionBacking.deleteCoAuthor(var)\}"}{/}{/}

		\fentry{submit-cbtn}{<h:commandButton>}{Ausführen des Einreichungsprozess.}{value="\#\{label.submit\}"\newline
		action="\#\{newSubmissionBacking.submit\}"}{/}{/}
	}
\end{samepage}

\begin{samepage}

	\begin{table}[ph!]
		\caption{\textbf{searchResult.xhtml}}
		\begin{tabular}{|l|}

			\hline
			Wird nach den eigenen Einreichungen, nach denen, die man begutachtet oder nach Einreichungen in eigener editorialer Verantwortung gesucht, \\ so erscheinen die Listen im gleichen Schema wie auch auf der Homepage.
			Ergibt die Suche eine Liste von User, so wird diese wie in der Userliste \\ für den Administrator und dem Editor angezeigt.
			Ist das Ziel der Suche, wissenschaftliche Foren zu finden, so erhält man bei passender Eingabe \\ eine Liste im selben Format wie diejenige auf der Seite der Forumsliste. \\\hline
		\end{tabular}
	\end{table}

\end{samepage}


\begin{samepage}

	\ftable{scientificForumList}{\textbf{scientificForumList.xhtml} Hier erhalten die Nutzer eine Übersicht über alle wissenschaftliche Foren.}{
		\fentry{date-select-slctom}{<h:selectOneMenu>}{Auswahl der Filtermöglichkeit eines Datums.}{value="\#\{scientificForumListBacking.deadlineFilterSelect\}"}{/}{/}

		\fentry{deadline-lst}{<f:selectItems>}{Liste des Auswahlmöglichkeiten}{value="\#\{scientificForumListBacking.dateSelects\}"}{/}{/}

		\fentry{forums-pagination-pg}{<pg:pagination>}{Liste aller Foren im System.}{paginator="\#\{scientificForumListBacking.scientificForumPagination\}"}{/}{/}

		\fentry{name-ssc}{<ssc:sortsearchcolumn>}{Spalte für den Namen des Forums.}{paginator="\#\{scientificForumListBacking.scientificForumPagination\}"\newline columnNamei18n="\#\{label.name\}"}{/}{/}

		\fentry{name-link}{<h:link>}{Name des Forums.}{value="\#\{var.name\}"\newline includeViewParams="true"}{/}{/}

		\fentry{/}{<f:param>}{URL Parameter Forum ID}{name="\#\{'id'\}"\newline value="\#\{var.id\}"}{/}{/}

		\fentry{deadline-ssc}{<ssc:sortsearchcolumn>}{Spalte für die Deadline des Forums.}{paginator="\#\{scientificForumListBacking.scientificForumPagination\}"\newline columnNamei18n="\#\{label.deadline\}"}{/}{/}

		\fentry{deadline}{outputText}{Abgabefrist des jeweiligen Forums.}{value="\#\{var.deadline\}"}{/}{/}
	}
\end{samepage}

\begin{samepage}

	%%\todo{wie hier url von bild wissen?}

\ftable{profile}{\textbf{profile.xhtml} Die Profilseite eines angemeldeten Nutzers ist nur vom Nutzer selbst und vom Administrator editierbar.
	Ein Editor besitzt nur Leserechte. Alle zuvor gespeicherten Angaben werden in den jeweiligen Feldern angezeigt.}{

	\fentry{avatar-gimg}{<h:graphicImage>}{Aktueller Avatar.}{value="\#\{'https://lases.de/image/'.concat(profileBacking.user.id)\}"}{/}{/}

	\fentry{new-avatar-ifile}{<h:inputFile>}{Hochladen eines neuen Avatars.}{value="\#\{profileBacking.uploadedAvatar\}"}{rendered="\#\{profileBacking.hasViewerEditRights\}"}{/}

	\fentry{submit-avatar-cbtn}{<h:commandButton>}{Lade neuen Avatar hoch.}{value="\#\{profileBacking.uploadAvatar\}"}{rendered="\#\{profileBacking.hasViewerEditRights\}"}{/}

	\fentry{delete-avatar-cbtn}{<h:commandButton>}{Löschen des Avatars.}{value="\#\{profileBacking.deleteAvatar\}"}{rendered="\#\{profileBacking.hasViewerEditRights\}"}{/}

	\fentry{role-otxt}{<h:outputText>}{Angabe der Rolle eines Nutzers}{?}{/}{/}

	\fentry{is-admin-cbx}{<h:selectBooleanCheckbox>}{Zuteilung der Rolle des Administrators.}{value="\#\{profileBacking.user.admin\}"}{rendered="\#\{profileBacking.sessionInformation.admin\}"}{/}

	\fentry{title-itxt}{<h:inputText>}{Titel des Nutzenden.}{value="\#\{profileBacking.user.title\}"}{readonly="\#\{!profileBacking.hasViewerEditRights\}"}{/}

	\fentry{firstname-itxt}{<h:inputText>}{Vorname des Nutzenden.}{value="\#\{profileBacking.user.firstName\}"}{readonly="\#\{!profileBacking.hasViewerEditRights\}"}{required="true" \newline requiredMessage=\newline "\#\{message.firstName\}"}

	\fentry{lastname-itxt}{<h:inputText>}{Nachname des Nutzenden.}{value="\#\{profileBacking.user.lastName\}"}{readonly="\#\{!profileBacking.hasViewerEditRights\}"}{required="true" \newline requiredMessage=\newline "\#\{message.lastName\}"}

	\fentry{password-iscrt}{<h:inputSecret>}{Leeres Eingabefeld für ein neues Passwort.}{value="\#\{profileBacking.user.passwordNotHashed\}"}{rendered="\#\{!profileBacking.hasViewerEditRights\}"}{passwordValidator \newline required="true" \newline requiredMessage=\newline "\#\{message.password\}"}

	\fentry{email-itxt}{<h:inputText>}{E-Mail des Nutzenden.}{value="\#\{profileBacking.user.emailAddress\}"}{readonly="\#\{!profileBacking.hasViewerEditRights\}"}{emailAddressExistsValidator \newline emailAddressExistsValidator \newline required="true" \newline requiredMessage=\newline "\#\{message.email\}"}

	\fentry{/}{<f:attribute>}{Parameter zur E-Mail Validierung.}{name="id" \newline value="\#\{profilBacking.user.id\}"}{/}{/}

	\fentry{submission-number-otxt}{<h:outputText>}{Anzahl der Einreichungen des Nutzers}{?}{/}{/}

	\fentry{employer-itxt}{<h:inputText>}{Angabe des Arbeitgebers}{value="\#\{profileBacking.user.employer\}"}{readonly="\#\{!profileBacking.hasViewerEditRights\}"}{/}

	\fentry{science-field-lst}{<h:dataTable>}{Spezialgebiete des Nutzers}{value="\#\{profileBacking.usersScienceFields\}"}{/}{/}

	\fentry{add-science-field-slctolb}{<p:selectOneListbox>}{Auswahl der Spezialgebiete des Nutzers}{value="\#\{profileBacking.selectedScienceField\}"}{rendered="\#\{!profileBacking.hasViewerEditRights\}"}{/}

	\fentry{field-lst}{<f:selectItems>}{Liste des Auswahlmöglichkeiten}{value="\#\{profileBacking.scienceFields\}"}{/}{/}

	\fentry{delete-field-cbtn}{<h:commandButton>}{Löschen des ausgewählten Fachgebietes.}{value="\#\{label.delete\}"\newline action="\#\{profileBacking.deleteScienceField\}"}{rendered="\#\{!profileBacking.hasViewerEditRights\}"}{/}

	\fentry{add-field-cbtn}{<h:commandButton>}{Füge Fachgebiet zur Liste hinzu.}{value="\#\{label.add\}"\newline action="\#\{profileBacking.addScienceField\}"}{rendered="\#\{!profileBacking.hasViewerEditRights\}"}{/}

	\fentry{date-of-birth-itxt}{<h:inputText>}{Angabe des Geburtstages}{value="\#\{profileBacking.user.dateOfBirth\}"}{readonly="\#\{!profileBacking.hasViewerEditRights\}"}{/}

	\fentry{save-cbtn}{<h:commandButton>}{Daten werden gespeichert.}{value="\#\{label.save\}"\newline action="\#\{profileBacking.submitChanges\}"}{rendered="\#\{!profileBacking.hasViewerEditRights\}"}{/}

	\fentry{delete-cbtn}{<h:commandButton>}{Nutzer und alle damit verbundenen Daten werden gelöscht.}{value="\#\{label.deleteUser\}"\newline action="\#\{profileBacking.deleteProfile\}"}{rendered="\#\{!profileBacking.hasViewerEditRights\}"}{/}

}
\end{samepage}



\ftable{admin-confirmation-popup}{Sollte ein Administrator einem anderen Nutzer die Rolle des Administrators zuweisen oder entziehen, so erscheint beim Auslösen des Save-Buttons ein Popup-Dialog.
	In diesem Fenster wird der Administrator angewiesen, die Änderung mit seinem Passwort zu bestätigen.}{

	\fentry{password-admin-iscrt}{<h:inputSecret>}{Angabe eines Passworts.}{value="\#\{profileBacking.adminPasswordInPopup.passwordNotHashed\}"}{/}{passwordValidator \newline required="true" \newline requiredMessage=\newline "\#\{message.password\}"}

	\fentry{admin-note-otxt}{<h:outputText>}{Hinweis zur Veränderung der Adminrolle.}{value="\#\{label.adminNote\}"}{/}{/}

	\fentry{abort-cbtn}{<h:commandButton>}{Abbruch der Änderung.}{value="\#\{label.save\}"\newline action="\#\{profileBacking.saveInPopup\}"}{/}{/}

	\fentry{save-cbtn}{<h:commandButton>}{Speichern aller Änderungen.}{value="\#\{label.abort\}"\newline action="\#\{profileBacking.abortInPopup\}"}{/}{/}

}

\ftable{scientificForum}{\textbf{scientificForum.xhtml} Die Ansicht eines Forums dient zur Ausgabe von Informationen über die jeweilige Konferenz oder das jeweilige Journal.
	Das Anzeigen und Filtern der Tabellen \emph{editorialPagination}, \emph{reviewPagination} und \emph{submissionPagination} wird genauso gehandhabt, wie auf \emph{homepage.xhtml}.
	Ausnahme dabei ist die Spalte Forum, welche hier nicht angezeigt wird.}{

	\fentry{/}{<f:viewParam>}{URL Parameter holen}{name="\#\{'id'\}"\newline value="\#\{scientificForumBacking.scientificForum.id\}"}{/}{required="true"}

	\fentry{/}{<f:viewAction>}{Action Methode ausführen.}{action="\#\{scientificForumBacking.onLoad\}"}{/}{/}

	\fentry{forum-name-itxt}{inputText}{Name des Forums.}{value="\#\{scientificForumBacking.scientificForum.name\}"}{readonly="\#\{!scientificForumBacking.isViewerEditor\}"}{required="true" \newline requiredMessage=\newline "\#\{message.forumName\}"}

	\fentry{editor-lst}{<h:dataTable>}{Liste der verantwortliche Editoren.}{value="\#\{scientificForumBacking.editors\}"\newline title="\#\{help.managingEditors\}"}{/}{/}

	\fentry{new-editor-itxt}{<h:inputText>}{Angabe der E-Mail eines Editors.}{value="\#\{scientificForumBacking.newEditorInput.emailAddress\}"}{rendered="\#\{scientificForumBacking.sessionInformation.user.editor\}"}{emailAddressExistsValidator \newline emailAddressLayoutValidator, required="true"\newline requiredMessage=\newline "\#\{message.email\}" }

	\fentry{add-editor-cbtn}{<h:commandButton>}{Füge Editor zur Liste hinzu.}{value="\#\{label.add\}"\newline action="\#\{scientificForumBacking.addEditor\}"}{rendered="\#\{scientificForumBacking.sessionInformation.user.editor\}"}{/}

	\fentry{remove-editor-cbtn}{<h:commandButton>}{Lösche ausgewählten Editor.}{value="\#\{label.remove\}"\newline action="\#\{scientificForumBacking.removeEditor\}"}{rendered="\#\{scientificForumBacking.sessionInformation.user.editor\}"}{/}

	\fentry{deadline-itxt}{<h:inputText>}{Deadline des Forums.}{value="\#\{scientificForumBacking.scientificForum.deadline\}"}{readonly="\#\{!scientificForumBacking.isViewerEditor\}"}{required="true" \newline requiredMessage=\newline "\#\{message.deadline\}"}

	\fentry{description-itxt}{inputTextArea}{Kurzbeschreibung des Forums.}{value="\#\{scientificForumBacking.scientificForum.description\}"}{readonly="\#\{!scientificForumBacking.isViewerEditor\}"}{required="true" \newline requiredMessage=\newline "\#\{message.description\}"}

	\fentry{url-lnk}{<h:outputLink>}{Link zur Konferenz oder zum Journal.}{value="\#\{scientificForumBacking.scientificForum.url\}"}{/}{/}

	\fentry{change-url-itxt}{inputText}{Angabe einer neuen URL.}{value="\#\{scientificForumBacking.scientificForum.url\}"}{rendered="\#\{!scientificForumBacking.isViewerEditor\}"}{/}

	\fentry{review-instructions-itxt}{<h:inputTextArea>}{Anleitung für eine Begutachtung.}{value="\#\{scientificForumBacking.scientificForum.reviewManual\}"}{readonly="\#\{!scientificForumBacking.isViewerEditor\}"}{/}

	\fentry{science-field-dtbl}{<h:dataTable>}{Fachgebiet des Forums.}{value="\#\{scientificForumBacking.selectedScienceFields\}"}{/}{/}

	\fentry{new-science-field}{<p:selectOneListbox>}{Hinzufügen eines neuen Fachgebietes.}{value="\#\{scientificForumBacking.selectedScienceFieldInput\}"}{rendered="\#\{!scientificForumBacking.isViewerEditor\}"}{/}

	\fentry{field-lst}{<f:selectItems>}{Liste des Auswahlmöglichkeiten}{value="\#\{scientificForumBacking.scienceFields\}"}{/}{/}

	\fentry{add-science-field-cbtn}{<h:commandButton>}{Speichern des neuen Fachgebietes.}{value="\#\{label.add\}"\newline action="\#\{scientificForumBacking.\newline
		addScienceField\}"}{rendered="\#\{!scientificForumBacking.isViewerEditor\}"}{/}

	\fentry{remove-science-field-cbtn}{commandButton}{Löschen eines Fachgebietes.}{value="\#\{label.remove\}"\newline action="\#\{scientificForumBacking.\newline
		removeScienceField\}"}{rendered="\#\{!scientificForumBacking.isViewerEditor\}"}{/}

	\fentry{save-cbtn}{<h:commandButton>}{Speichere Veränderungen.}{value="\#\{label.save\}"\newline action="\#\{scientificForumBacking.submitChanges\}"}{rendered="\#\{!scientificForumBacking.isViewerEditor\}"}{/}

	\fentry{submission-pg}{<pg:pagination>}{Paginierte Liste aller Einreichungen, die der Nutzer eingereicht hat.}{paginator="\#\{scientificForumBacking.submissionPagination\}"}{/}{/}

	\fentry{title-ssc}{<ssc:sortsearchcolumn>}{Spalte für den Titel des Papers.}{paginator="\#\{scientificForumBacking.submissionPagination\}"\newline columnNamei18n="\#\{label.submissionTitle\}"}{/}{/}

	\fentry{title-otxt}{<h:outputText>}{Titel der Einreichung.}{value="\#\{var.title\}"}{/}{/}

	\fentry{date-ssc}{<ssc:sortsearchcolumn>}{Spalte für das Abgabedatum der Einreichung.}{paginator="\#\{scientificForumBacking.submissionPagination\}"\newline columnNamei18n="\#\{label.date\}"}{/}{/}

	\fentry{deadline-scltom}{<h:selectOneMenu>}{Auswahl der Filtermöglichkeit eines Datums.}{value="\#\{scientificForumBacking.dateFilterSelectSub\}" title="\#\{help.deadlineSelect\}"}{/}{/}

	\fentry{deadline-lst}{<f:selectItems>}{Liste des Auswahlmöglichkeiten}{value="\#\{scientificForumBacking.dateSelects\}"}{/}{/}

	\fentry{date-otxt}{<h:outputText>}{Abgabedatum der Einreichung.}{value="\#\{var.submissionTime\}"}{/}{convertDateTime \newline pattern=""yyyy-MM-dd"}

	\fentry{deadline-ssc}{<ssc:sortsearchcolumn>}{Spalte für die Deadline der Revision.}{paginator="\#\{scientificForumBacking.submissionPagination\}"\newline columnNamei18n="\#\{label.deadline\}"}{/}{/}

	\fentry{deadline-otxt}{<h:outputText>}{Deadline für eine Revision.}{value="\#\{var.deadlineRevision\}"}{/}{convertDateTime \newline pattern=""yyyy-MM-dd"}

	\fentry{state-ssc}{<ssc:sortsearchcolumn>}{Spalte für den Status der Einreichung.}{paginator="\#\{scientificForumBacking.submissionPagination\}"\newline columnNamei18n="\#\{label.stateSub\}"}{/}{/}

	\fentry{state-scltom}{<h:selectOneMenu>}{Auswahl der Filtermöglichkeit eines Status.}{value="\#\{scientificForumBacking.stateFilterSelectSub\}" title="\#\{help.stateSubSelect\}"}{/}{/}

	\fentry{state-lst}{<f:selectItems>}{Liste des Auswahlmöglichkeiten}{value="\#\{scientificForumBacking.submissionStates\}"}{/}{/}

	\fentry{state-otxt}{<h:outputText>}{Status der Einreichung.}{value="\#\{var.state\}"}{/}{/}

	\midrule
	\multicolumn{6}{c}{Weitere Paginierungen mit den gleichen Spalten wie in der vorherigen Tabelle.} \\ %das hline verursacht keine probleme, wird sind in einem tabular
	\midrule

	\fentry{editorial-pg}{<pg:pagination>}{Paginierte Liste aller Einreichungen, die der Editor verwaltet.}{paginator="\#\{scientificForumBacking.editedPagination\}"}{rendered="\#\{!scientificForumBacking.isViewerEditor\}"}{/}

	\fentry{review-pg}{<pg:pagination>}{Paginierte Liste aller Einreichungen, die der Gutachter bearbeitet.}{paginator="\#\{scientificForumBacking.reviewedPagination\}"}{rendered="\#\{userListBacking.sessionInformation.user.reviewer\}"}{/}
}

\subsubsection{Gutachter}$~$

\localauthor{Johann Schicho ff.}

\begin{samepage}

	\ftable{newReview}{\textbf{newReview.xhtml} Möchte ein Gutachter eines Papers seine Beurteilung abgeben, so ist dies auf dieser Seite möglich.}{

		\fentry{version-number-otxt}{<h:outputText>}{Versionsnummer des eingereichten Papers, zu welchem der Gutachter ein Gutachten abgeben kann.}{}{/}{/}

		\fentry{recommend\newline ation-cbx}{<h:selectBooleanCheckbox>}{Empfehlung des Gutachters ein Paper zu akzeptieren.}{value="\#\{newReviewBacking.review.acceptPaper\}"}{/}{/}

		\fentry{comment-itxt}{<h:inputTextarea>}{Kommentar eines Gutachters}{value="\#\{newReviewBacking.review.comment\}"}{/}{/}

		\fentry{review-pdf-ifile}{<h:inputFile>}{Einzureichendes Gutachten.}{value="\#\{newReviewBacking.uploadedPDF\}"}{/}{/}

		\fentry{submit-review-btn}{<h:commandButton>}{Gutachten einreichen.}{value="\#\{newReviewBacking.addReview\}"}{/}{/}
	}
\end{samepage}

\subsubsection{Editor}$~$

\begin{samepage}


	\ftable{userList}{\textbf{userList.xhtml} Editoren und Administratoren erhalten hier einen Überblick über alle Nutzer. Die Seiten sind paginiert.}{
		\fentry{user-pg}{<pg:pagination>}{Paginierte Liste aller Nutzer}{paginator="\#\{userListBacking.userPagination\}"}{rendered="\#\{userListBacking.sessionInformation.user.isEditor\}"}{/}

		\fentry{rol-ssc}{<ssc:sortsearchcolumn>}{Spalte für die Nutzerrolle}{paginator="\#\{userListBacking.userPagination\}"\newline columnNamei18n="\#\{label.role\}"}{rendered="\#\{userListBacking.sessionInformation.user.isEditor\}"}{/}

		\fentry{role-otxt}{<h:outputText>}{Rolle des Nutzers}{value="\#\{var.role\}"}{rendered="\#\{userListBacking.sessionInformation.user.isEditor\}"}{/}

		\fentry{name-ssc}{<ssc:sortsearchcolumn>}{Spalte für den Namen des Nutzers}{paginator="\#\{userListBacking.userPagination\}"\newline columnNamei18n="\#\{label.name\}"}{rendered="\#\{userListBacking.sessionInformation.user.isEditor\}"}{/}

		\fentry{name-link}{<h:link>}{Name des Nutzers. Link mit URL-Parameter.}{value="\#\{var.firstName.concat( ).concat(var.lastName)\}"\newline outcome="/facelets/authenticated/profile"\newline includeViewParams="true"}{rendered="\#\{userListBacking.sessionInformation.user.isEditor\}"}{/}

		\fentry{/}{<f:param>}{Nutzer ID als URL Parameter.}{name="\#\{'id'\}"\newline value="\#\{var.id\}"}{/}{/}

		\fentry{email-ssc}{<ssc:sortsearchcolumn>}{Spalte für die E-Mail des Nutzers}{paginator="\#\{userListBacking.userPagination\}"\newline columnNamei18n="\#\{label.email\}"}{rendered="\#\{userListBacking.sessionInformation.user.isEditor\}"}{/}

		\fentry{email-link}{<h:link>}{Mailto-Link mit E-Mail-Adresse des Nutzers.}{value="\#\{var.emailAddress\}"\newline  outcome="\#\{'mailto:'.concat(var.emailAddress)\}"}{rendered="\#\{userListBacking.sessionInformation.user.isEditor\}"}{/}

		\fentry{employer-ssc}{<ssc:sortsearchcolumn>}{Spalte für den Arbeitsgeber des Nutzers}{paginator="\#\{userListBacking.userPagination\}"\newline columnNamei18n="\#\{label.employer\}"}{rendered="\#\{userListBacking.sessionInformation.user.isEditor\}"}{/}

		\fentry{employer-otxt}{<h:outputText>}{Arbeitgeber des Nutzers.}{value="\#\{var.employer\}"}{rendered="\#\{userListBacking.sessionInformation.user.isEditor\}"}{/}
	}
\end{samepage}

\subsubsection{Admin}$~$

\begin{samepage}

	\ftable{initialConfig}{\textbf{initialConfig.xhtml} Auf dieser Seite landet man beim ersten Systemstart. Sie ist nur für Administratoren zugänglich.
		Hier kann der Administrator die Datenbankschemata erstellen lassen.}{
		\fentry{db-connection-state-otxt}{<h:outputText>}{Information über die Verbindung mit der Datenbank}{value="\#\{initialConfigBacking.getDatasourceConnectionState\}"}{/}{/}

		\fentry{create-db-btn}{<h:commandButton>}{Erstellt die Datenbankschemata}{value="\#\{label.createDB\}"\newline action="\#\{initialConfigBacking.createDatasource\}"}{/}{/}
	}
\end{samepage}

\begin{samepage}

	\ftable{newUser}{\textbf{newUser.xhtml} Der Administrator kann hier einen neuen Nutzer anlegen.}{
		\fentry{password-iscrt}{<h:inputSecret>}{Angabe des Passworts.}{value="\#\{newUserBacking.newUser.passwordNotHashed\}"}{render="\#\{newUserBacking.sessionInformation.user.isAdmin\}"}{passwordValidator \newline required="true" \newline requiredMessage=\newline "\#\{message.password\}"}

		\fentry{firstname-itxt}{<h:inputText>}{Angabe des Vornamen.}{value="\#\{newUserBacking.newUser.firstName\}"}{render="\#\{newUserBacking.sessionInformation.user.isAdmin\}"}{required="true" \newline requiredMessage=\newline "\#\{message.firstName\}"}

		\fentry{lastname-itxt}{<h:inputText>}{Angabe des Nachnamen.}{value="\#\{newUserBacking.newUser.lastName\}"}{render="\#\{newUserBacking.sessionInformation.user.isAdmin\}"}{required="true" \newline requiredMessage=\newline "\#\{message.lastName\}"}

		\fentry{titel-itxt}{<h:inputText>}{Angabe eines Titels.}{value="\#\{newUserBacking.newUser.title\}"}{render="\#\{newUserBacking.sessionInformation.user.isAdmin\}"}{/}

		\fentry{e-mail-itxt}{<h:inputText>}{Angabe einer E-Mail-Adresse.}{value="\#\{newUserBacking.newUser.emailAddress\}"}{render="\#\{newUserBacking.sessionInformation.user.isAdmin\}"}{emailAddressExistsValidator \newline emailAddressLayoutValidator \newline required="true" \newline requiredMessage=\newline "\#\{message.email\}"}

		\fentry{is-admin-lbl}{<p:outputLabel>}{Label für die Administrator Checkbox}{value="\#\{label.setAdmin\}"}{render="\#\{newUserBacking.sessionInformation.user.isAdmin\}"}{/}

		\fentry{is-admin-cbx}{<h:selectBooleanCheckbox>}{Zuordnung einer Administratorenrolle.}{value="\#\{newUserBacking.admin\}"}{render="\#\{newUserBacking.sessionInformation.user.isAdmin\}"}{/}

		\fentry{abort-btn}{<h:commandButton>}{Abbruch des Vorgangs: Nutzererstellen.}{value="\#\{newUserBacking.abort\}"}{render="\#\{newUserBacking.sessionInformation.user.isAdmin\}"}{/}

		\fentry{save-btn}{<h:commandButton>}{Speichern des neuen Nutzers.}{value="\#\{newUserBacking.saveUser\}"}{render="\#\{newUserBacking.sessionInformation.user.isAdmin\}"}{/}

	}
\end{samepage}

\begin{samepage}

	\ftable{administration}{\textbf{administration.xhtml} Hier kann ein Administrator Konfigurationen vornehmen.}{
		\fentry{select-theme-slctom}{<h:selectOneMenu>}{Auswahl eines Farbthemas.}{value="\#\{administrationBacking.systemSettings.style\}"}{render="\#\{administrationBacking.sessionInformation.user.isAdmin\}"}{/}

		\fentry{styles-lst}{<f:selectItems>}{Liste des Auswahlmöglichkeiten}{value="\#\{administrationBacking.styles\}"}{/}{/}

		\fentry{logo-img}{<h:graphicImage>}{Logo des Systems.}{value="\#\{administrationBacking.systemSettings.logoImage\}"}{render="\#\{administrationBacking.sessionInformation.user.isAdmin\}"}{/}

		\fentry{change-logo-ifile}{<h:inputFile>}{Hochladen eines neuen Logos.}{title="\#\{help.uploadNewLogo\}"\newline value="\#\{administrationBacking.uploadedImage\}"}{render="\#\{administrationBacking.sessionInformation.user.isAdmin\}"}{/}

		\fentry{welcome-heading-itxt}{<h:inputText>}{Angabe einer Überschrift auf der \emph{welcomepage}}{value="\#\{administrationBacking.systemSettings.headlineWelcomePage\}"}{render="\#\{administrationBacking.sessionInformation.user.isAdmin\}"}{required="true" \newline requiredMessage=\newline "\#\{message.welcomeHeading\}"}

		\fentry{welcome-text-itxt}{<h:inputTextArea>}{Angabe eines Textes auf der \emph{welcomepage}}{value="\#\{administrationBacking.systemSettings.shortMessageWelcomePage\}"}{render="\#\{administrationBacking.sessionInformation.user.isAdmin\}"}{required="true" \newline requiredMessage=\newline "\#\{message.welcomeText\}"}

		\fentry{institution-itxt}{<h:inputText>}{Angabe des Namens der Einrichtung.}{value="\#\{administrationBacking.systemSettings.companyName\}"}{render="\#\{administrationBacking.sessionInformation.user.isAdmin\}"}{required="true" \newline requiredMessage=\newline "\#\{message.institution\}"}

		\fentry{imprint-itxt}{<h:inputTextArea>}{Angabe des Impressums der Einrichtung.}{value="\#\{administrationBacking.systemSettings.imprint\}"}{render="\#\{administrationBacking.sessionInformation.user.isAdmin\}"}{required="true" \newline requiredMessage=\newline "\#\{message.imprint\}"}

		\fentry{abort-btn}{<h:commandButton>}{Abbruch des Änderungsvorgangs.}{value="\#\{label.abort\}"\newline action="\#\{administrationBacking.abort\}"}{render="\#\{administrationBacking.sessionInformation.user.isAdmin\}"}{/}

		\fentry{save-btn}{<h:commandButton>}{Speichern der Änderungen.}{value="\#\{label.save\}"\newline action="\#\{administrationBacking.save\}"}{render="\#\{administrationBacking.sessionInformation.user.isAdmin\}"}{/}

		\fentry{new-user-link}{<h:link>}{Weiterleitung zu \emph{newUser.xhtml}.}{value="\#\{label.gotoNewUser\}" \newline outcome="/facelets/authenticated/newUser"}{render="\#\{administrationBacking.sessionInformation.user.isAdmin\}"}{/}

		\fentry{new-scientific-forum-link}{<h:link>}{Weiterleitung zu \emph{newScientificForum.xhtml}.}{value="\#\{label.gotoNewScientificForum\}" \newline outcome="/facelets/authenticated/newScientificForum"}{render="\#\{administrationBacking.sessionInformation.user.isAdmin\}"}{/}
	}
\end{samepage}

\begin{samepage}

	\ftable{newScientificForum}{\textbf{newScientificForum.xhtml} Auf der Seite zum Erstellen eines wissenschaftlichen Forums kann ein Administrator dessen wesentlichen Daten festlegen.
	}{

	\fentry{forum-name-itxt}{<h:inputText>}{Angabe des Name des Forums.}{value="\#{newScientificForumBacking.newScientificForum.name}"}{render="\#{newScientificForumBacking.sessionInformation.user.isAdmin}"}{required="true" \newline requiredMessage=\newline "\#\{message.forumName\}"}

	\fentry{editors-list}{<h:dataTable>}{Liste der Editoren}{value="\#{newScientificForumBacking.editors}"}{render="\#{toolbarBacking.sessionInformation.user.isAdmin}"}{/}

	\fentry{email-editor-itxt}{<h:inputText>}{Angabe der E-Mail-Adresse eines Editor.}{value="\#{newScientificForumBacking.newEditorInput.emailAddress}"}{render="\#{newScientificForumBacking.sessionInformation.user.isAdmin}"}{emailAddressValidator \newline emailAddressLayoutValidator \newline required="true" \newline requiredMessage=\newline "\#\{message.email\}"}

	\fentry{add-editor-btn}{<h:commandButton>}{Hinzufügen eines Editors in eine Liste.}{value="\#\{label.add\}"\newline  action="\#\{newScientificForumBacking.addEditor\}"\newline title="\#\{help.addEditor\}"}{render="\#\{newScientificForumBacking.sessionInformation.user.isAdmin\}"}{/}

	\fentry{delete-editor-btn}{<h:commandButton>}{Löschen des zugehörigen Editors von der Liste.}{value="\#\{label.remove\}"\newline  action="\#\{newScientificForumBacking.removeEditor(var)\}"\newline title="\#\{help.removeEditor\}"}{render="\#\{newScientificForumBacking.sessionInformation.user.isAdmin\}"}{/}

	\fentry{deadline-itxt}{<h:inputText>}{Hinzufügen einer Deadline des Forums.}{value="\#\{newScientificForumBacking.newScientificForum.deadline\}"}{render="\#{newScientificForumBacking.sessionInformation.user.isAdmin}"}{converterDateTime \newline patter="dd/MM/yyyy"}

	\fentry{description-itxt}{<h:inputTextArea>}{Angabe einer Kurzbeschreibung.}{value="\#\{newScientificForumBacking.newScientificForum.description\}"}{render="\#{newScientificForumBacking.sessionInformation.user.isAdmin}"}{required="true" \newline requiredMessage=\newline "\#\{message.description\}"}

	\fentry{url-itxt}{<h:inputText>}{Link zur Konferenz oder zum Journal.}{value="\#\{newScientificForumBacking.newScientificForum.url\}"}{render="\#{newScientificForumBacking.sessionInformation.user.isAdmin}"}{/}

	\fentry{review-instructions-itxt}{<h:inputTextArea>}{Angabe einer Anleitung für eine Begutachtung.}{value="\#\{newScientificForumBacking.newScientificForum.reviewManual\}"}{render="\#{newScientificForumBacking.sessionInformation.user.isAdmin}"}{required="true" \newline requiredMessage=\newline "\#\{message.instruction\}"}

	\fentry{science-field-list}{<h:outputText>}{Liste der ausgewählten Fachgebiete.}{value="\#\{newScientificForumBacking.selectedScienceFields\}"}{render="\#{newScientificForumBacking.sessionInformation.user.isAdmin}"}{/}

	\fentry{science-field-list}{<p:selectOneListbox>}{Auswahl von Fachgebieten.}{value="\#\{newScientificForumBacking.scienceFields\}"}{render="\#{newScientificForumBacking.sessionInformation.user.isAdmin}"}{/}

	\fentry{new-science-field-itxt}{<h:inputText>}{Hinzufügen neuer Fachgebiete.}{value="\#\{newScientificForumBacking.newScienceFieldInput.name\}"}{render="\#{newScientificForumBacking.sessionInformation.user.isAdmin}"}{required="true" \newline requiredMessage=\newline "\#\{message.newScienceField\}"}

	\fentry{add-science-field-btn}{<h:commandButton>}{Ausführen der Hinzufüge-Aktion.}{value="\#\{label.add\}"\newline action="\#\{newScientificForumBacking.addScienceField\}"}{render="\#{newScientificForumBacking.sessionInformation.user.isAdmin}"}{/}

	\fentry{save-btn}{<h:commandButton>}{Speichern des neuen Forums.}{value="\#\{label.save\}"\newline action="\#\{newScientificForumBacking.create\}"}{render="\#{newScientificForumBacking.sessionInformation.user.isAdmin}"}{/}

	\fentry{abort-btn}{<h:commandButton>}{Abbruch des Erstellungsprozesses.}{value="\#\{label.abort\}"\newline action="\#\{newScientificForumBacking.abort\}"}{render="\#{newScientificForumBacking.sessionInformation.user.isAdmin}"}{/}
}
\end{samepage}

\end{landscape}
