%% Macros
\newcommand{\ftable}[1]{\begin{sidewaystable}
\begin{tabular}[H]{ |m{2cm}| m{3cm}| m{4cm}| m{6cm} |m{4cm}| m{2cm}| }
    \toprule
    \textbf{ID} & \textbf{Typ} & \textbf{Beschreibung} & \textbf{Binding} & \textbf{Constraints} & \textbf{Validator \newline Converter} \\
    \midrule
    #1
    %\bottomrule
\end{tabular}
\end{sidewaystable}
}

\newcommand{\fentry}[6]{#1 & #2 & #3 & #4 & #5 & #6\\\hline}

\localauthor{Stefanie Gürster}

\subsection{Namenskonvention}

Zur Vereinheitlichung der Namensgebung der Komponenten in den Facelets verfahren wir nach dem \emph{kebab-case} Prinzip. Um den genauen Typ in die ID mit aufzunehmen, wird das entsprechende Suffix angehängt.

\todo{Suffix:}
 graphicImage = gIm link =lnk inputText =itext commandButton = cbtn messages = msgs dataTable = dtbl selectOneMenu = slctom

\subsection{Templates}

\begin{samepage}
	\todo{Serachfield needs view param?}
	\todo{Sebastians Rollenprüfung}
\textbf{navigation.xhtml} \phantomsection \label{flt:navbar} ist die Kopfzeile der Webanwendung. Diese bietet die Suchfunktion, Links zu verschiedenen Listen und dem Profil an.
\nopagebreak

\todo{Absprechen mit Sebastian und BB nicht vollstäandig}
\todo{outcome von links ist immer noch im ordner facelets. mit Johannes abklären und ausprobieren.}
\ftable{
	\fentry{logo-gimg}{<h:graphicImage>}{Logo der Applikation.}{value="\#{}"}{/}{/}
	
	\fentry{direct-to-home-lnk}{<h:link>}{Weiterleitung zur Homepage.}{outcome="/authenticated/homepage" \newline title="\#{help.toHome}"}{/}{/}
	
	\fentry{search-frm}{<h:form>}{Formular zur Suche.}{/}{/}{/}
	
	\fentry{search-Field-itxt}{<h:inputText>}{Suchleiste}{value="\#{navigationBacking.resultListParameters.globalSearchWord}" \newline title="\#{help.search}"}{/}{/}
	
	\fentry{search-cbtn}{<h:commandButton>}{Suche ausführen.}{action="\#{navigationBacking.search}"}{/}{/}
	
	\fentry{user-list-lnk}{<h:link>}{Link zur Übersichtsseite aller Nutzer}{value="\#{label.userList}" \newline outcome="/editor/userList \newline title="\#{help.userList}"}{render="\#{navigationBacking.sessionInformation.user.editor}"}{/}
	
	\fentry{forum-list-lnk}{<h:link>}{Link zur Übersichtsseite alle Journale und Konferenzen}{value="\#{label.forumList}" \newline outcome="/authenticated/scientificForumList" \newline title="\#{help.forumList}"}{/}{/}
	
	\fentry{logout-cbtn}{<h:commandButton>}{Loggt den Nutzer aus dem System aus und leitet zur Loginseite weiter.}{value="\#{label.logout}" \newline action="\#{navigationBacking.logout}" \newline title="\#{help.logout}"}{/}{/}
	
	\fentry{profile-lnk}{<h:link>}{Link zur Profilübersicht}{value="\#{label.nav.profile}" \newline outcome="/authenticated/profile"}{/}{/}
	}
	
\end{samepage}

\todo{wo css einbidnen?}

\begin{samepage}
	\textbf{main.xhtml} ist das Template, welches den Inhalt der Seiten zwischen Kopf- und Fußzeile einbettet.
	
	\ftable{
		\fentry{/}{<ui:include>}{\hyperref[flt:navbar]{Kopfzeile}}{src="navigation.xhtml"}{/}{/}
		
		\fentry{global-msgs}{<h:messages>}{Anzeigen globaler Nachrichten.}{/}{/}{/}		
		
		\fentry{/}{<ui:insert>}{\hyperref[flt:pages]{Seiteninhalt}}{name="main-content"}{/}{/}
		
		\fentry{/}{<ui:include>}{\hyperref[flt:footer]{Fußzeile}}{src="footer.xhtml"}{/}{/}
	}
\end{samepage}

\todo{language bearbeiten}
\begin{samepage}
	\textbf{footer.xhtml} \phantomsection \label{flt:footer} ist die Fußzeile der Webanwendung und für alle sichtbar.
	\nopagebreak
	
	\ftable{
		\fentry{imprint-lnk}{<h:link>}{Link zur Seite des Impressums}{value="\#{label.footer.imprint}" \newline outcome="/anonymous/imprint" \newline too}{/}{/}
		
		\fentry{language-otxt}{<h:outputText>}{Anzeige der eingestellten Sprache.}{value="\#{}"}{/}{/}
		
	}
\end{samepage}

\todo{Toolbarbacking fehlt}
\todo{Validierung?}
\todo{Schätze wir brauchen doch min und mx length was ist wenn man zu lange usernames angibt?}
\todo{Datum internationalisieren?}
\todo{var als binding oder constraint?}
\todo{evtl kein rendern sonder abhängiges include in submission.xhtml?}
\todo{selectItem?}
\begin{samepage}
	\textbf{toolbar.xhtml} \phantomsection \label{flt:toolbar} ist die Seitenleiste für Editoren und Administratoren auf der Einreichungsübersicht. Hier wird die Einreichung verwaltet.
	\nopagebreak
	
	\ftable{
		\fentry{add-reviewer-iTxt}{<h:inputText>}{Eingabefeld zur Angabe einer E-Mail-Adresse}{value="\#{toolbarBacking.reviewerInput.email}" \newline title="\#{help.reviewerEmail}}{render="\#{toolbarBacking.sessionInformation.user.editor}"}{emailAddressExistsValidator \newline required="true" \newline pattern="([\^.@]+)(\.[\^.@]+)*@([\^.@]+\.)+([\^.@]+)"}
		
		\fentry{add-deadline-reviewer-iTxt}{<h:inputText>}{Eingabefeld zur Angabe einer Deadline.}{value="\#{toolbarBacking.reviewedByInput.timestampDeadline}" \newline title="\#{help.reviewerDeadline}"}{render="\#{toolbarBacking.sessionInformation.user.editor}"}{converterDateTime \newline patter="dd/MM/yyyy"}
		
		\fentry{add-reviewer-cbtn}{<h:commandButton>}{Knopf zum Hinzufügen des Gutachters}{value="\#{label.addReviewer}" \newline action="\#{}" \newline title="\#{help.addReviewer}"}{render="\#{toolbarBacking.sessionInformation.user.editor}"}{/}
		
		\fentry{reviewer-dtbl}{<h:dataTable>}{Liste der Gutachter}{value="\#{toolbarbacking.reviewer}" \newline var="review"}{render="\#{toolbarBacking.sessionInformation.user.editor}"}{/}
		
		\fentry{remove-reviewer-cbtn}{<h:commandButton>}{Entferne einen bestimmten Gutachters.}{action="\#{toolbarBacking.removeReviewer}" \newline title="\#{help.removeReviewer}"}{render="\#{toolbarBacking.sessionInformation.user.editor}"}{/}
		
		\fentry{select-editor-slctom}{<h:selectOneMenu>}{Auswahländerung des verwaltenden Editors}{value="\#{}" \newline title="\#{help.selectEditor}"}{render="\#{toolbarBacking.sessionInformation.user.editor}"}{/}
		
		\fentry{save-editor-cbtn}{<h:commandButton>}{Speichern der Änderung des Editors.}{value="\#{label.saveEditor}" \newline action="\#{toolbarBacking.chooseNewManagingEditor}" \newline title="\#{help.saveEditor}"}{render="\#{toolbarBacking.sessionInformation.user.editor}"E}{/}
		
		\fentry{show-editor-otxt}{<h:outputText>}{Anzeige des aktuellen Editors.}{value="\#{toolbarBacking.currentEditor}"}{render="\#{toolbarBacking.sessionInformation.user.editor}"}{/}
		
		\fentry{revision-deadline-iTxt}{<h:inputText>}{Eingabe zur Angabe einer Deadline.}{value="\#{toolbarBacking.submission.deadlineRevision}" \newline title="\#{help.revisionDeadline}"}{render="\#{toolbarBacking.sessionInformation.user.editor}"}{converterDateTime \newline patter="dd/MM/yyyy"}
		
		\fentry{require-revision-cbtn}{<h:commandButton>}{Fordert den Einreicher auf, seine Einreichung zu überarbeiten}{value="\#{label.requireRevision}" \newline action="\#{toolbarBacking.requireRevision}" \newline title="\#{help.requireRevision}"}{render="\#{toolbarBacking.sessionInformation.user.editor}"}{/}
		
		\fentry{accept-submission-cbtn}{<h:commandButton>}{Akzeptiere die Einreichung}{value="\#{label.acceptSubmission}" \newline action="\#{toolbarBacking.acceptSubmission}" \newline title="\#{help.accesptSubmission}"}{render="\#{toolbarBacking.sessionInformation.user.editor}"}{/}
		
		\fentry{reject-submission-cbtn}{<h:commandButton>}{Lehne die Einreichung ab}{value="\#{label.rejectSubmission}" \newline action="\#{toolbarBacking.rejectSubmission}" \newline title="\#{help.rejectSubmission}"}{render="\#{toolbarBacking.sessionInformation.user.editor}"}{/}
		
	}
\end{samepage}

\subsection{Seiten}

