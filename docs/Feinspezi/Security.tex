\localauthor{Thomas Kirz}

Im Folgenden wird erklärt, wie verschiedene Sicherheitsangriffe verhindert und Sicherheitsrisiken ausgeschlossen werden.

\subsection{Cross Site Scripting (XSS)}
Da JSF components für die HTML-Ausgabe verwendet werden, bereinigt JSF alle Nutzereingaben.
Zum Beispiel werden also \code{'<'} und \code{'>'} durch \code{'\&lt;'} und \code{'\&gt;'} ersetzt und Nutzereingaben können keine HTML-Tags in der Ausgabe erstellen.
Dadurch wird XSS bzw.\ HTML-Injection unmöglich.
Auch JavaScript kann so nicht injected werden, da sich JavaScript-Code in einem HTML-Tag (\code{<script>}) befinden müsste.

\subsection{SQL-Injection}
Für SQL-Code werden \code{PreparedStatements} benutzt, damit die SQL-Statements ohne von Nutzereingaben abhängige Parameter vorkompiliert werden.
Die Parameter werden dann an die Datenbank gesendet und von ihr getrennt vom Code als Daten betrachtet.
Die Ausführung des Statements kann also nicht durch die Nutzereingaben modifiziert werden und eine SQL-Injection wird verhindert

