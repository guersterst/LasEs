\localauthor{Thomas Kirz}

Im Folgenden wird erklärt, wie verschiedene Sicherheitsangriffe verhindert und Sicherheitsrisiken ausgeschlossen werden.

\subsection{Cross Site Scripting (XSS)}\label{subsec:xss}
Da JSF components für die HTML-Ausgabe verwendet werden, bereinigt JSF alle Nutzereingaben.
Zum Beispiel werden also \code{'<'} und \code{'>'} durch \code{'\&lt;'} und \code{'\&gt;'} ersetzt und Nutzereingaben können keine HTML-Tags in der Ausgabe erstellen.
Dadurch wird XSS bzw.\ HTML-Injection unmöglich.
Auch JavaScript kann so nicht injected werden, da sich JavaScript-Code in einem HTML-Tag (\code{<script>}) befinden müsste.

\subsection{SQL-Injection}\label{subsec:sql-injection}
Für SQL-Code werden \code{PreparedStatements} benutzt, damit die SQL-Statements ohne von Nutzereingaben abhängige Parameter vorkompiliert werden.
Die Parameter werden dann an die Datenbank gesendet und von ihr getrennt vom Code als Daten betrachtet.
Die Ausführung des Statements kann also nicht durch die Nutzereingaben modifiziert werden und eine SQL-Injection wird verhindert

\subsection{Session Hijacking \& Fixation}\label{subsec:session-hijacking}
In der \code{web.xml}-Datei ist ein Timeout für Sessions konfigurierbar.
Dieser wirkt Session Hijacking sowie unerwünschten physischen Zugriffen auf den Browser des Clients entgegen.
JSF setzt für das Session-Cookie standardmäßig das \code{http-only}-Flag, damit kann darauf nicht mit JavaScript zugegriffen werden und Hijacking wird erschwert.

Nach jeder Anmeldung wird die Session verworfen und neu generiert mit
{\small
\begin{lstlisting}
FacesContext.getCurrentInstance().getExternalContext()
    .getRequest().changeSessionId()
\end{lstlisting}
}
Nach einer Authentifizierung bleibt also keine alte Session bestehen und \emph{Session-Fixation}-Attacken werden verhindert.

\subsection{Passwörter}\label{subsec:passwords}
Passwörter werden mit der modernen und sicheren PBKDF2-Funktion gehasht.
Dabei wird für jedes Passwort ein eigener zufällig generierter Salt benutzt, welcher mit dem Hash in der Datenbank gespeichert wird.
Eingegebene Passwörter werden so früh wie möglich gehasht, damit die Klartextpasswörter eine minimale Lebenszeit auf dem Server haben.
Für das Generieren eins Salts und das Hashing bietet die Klasse \code{Hashing} aus \code{de.java.business.util}
Methoden an.
Sie verwendet die robuste \code{javax.crypto}-Bibliothek.

Um das Erraten von Passwörtern durch Brute-Force-Attacken o.ä.\ zu erschweren, müssen Passwörter 8--100 Zeichen lang sein und Groß- und Kleinbuchstaben, Zahlen und Sonderzeichen enthalten.

\subsection{Information Leakage}\label{subsec:information-leakage}
Alle Informationen über das System, die nach außen gelangen, können Angreifern helfen.
Dazu gehören Details über die technische Umgebung und verwendete Software des Systems, interne Fehlermeldungen und Implementierungsdetails.
Folgende Maßnahmen werden ergriffen, um Leakage solcher Informationen zu verhindern.
Fehlermeldungen zeigen im Produktionsmodus keine Stacktraces an, sondern eigene Fehlermeldungen, die nur Informationen enthalten, die für den Endnutzer relevant sind.
Wir verwenden eigene Fehlerseiten bei Exceptions und ersetzen die 404 und 500 Fehlerseiten von JSF durch unsere eigenen wie folgt in der web.xml-Datei:
    {\small
\begin{lstlisting}
<error-page>
    <error-code>404</error-code>
    <location>/WEB-INF/errorpages/error404.xhtml</location>
</error-page>
\end{lstlisting}
}
Dadurch ist nicht ersichtlich, dass wir JSF/Mojarra verwenden.
Das wird auch durch die Verwendung von .xhtml-Dateiendungen für unsere Facelets und Ersetzen des Bezeichners \code{JSESSONID} für Session-Cookies und URL-Rewriting in der web.xml-Datei verborgen:
    {\small
\begin{lstlisting}
<session-config>
    <cookie-config>
        <name>sesssionid</name>
    </cookie-config>
</session-config>
\end{lstlisting}
}

\subsection{Access Control}\label{subsec:access-control}
Um unerlaubten Zugriff auf Seiten, Informationen und Aktionen zu verhindern, werden Zugriffe auf nicht-öffentliche Ressourcen (das ist alles außer Start-, Registrierungs- und Verifikationsseite) standardmäßig abgelehnt.
Die Klasse \code{TrespassListener} entscheidet dann, ob der Nutzer die nötigen Rechte hat und akzeptiert eine Anfrage nur dann, sonst wird auf eine 404-Fehler-Seite weitergeleitet.
Dazu wird über die angefragte URL das zugehörige Facelet bestimmt und geprüft, ob darauf mit der Rolle des Nutzers zugegriffen werden darf.
Auf die Rolle kann der \code{TrespassListener} über die SessionInformation zugreifen.
Der Zugriff hängt darüber hinaus auch von der Beziehung des Nutzers zur angefragten Ressource ab (ein normaler Nutzer kann z.B.\ nur sein \emph{eigenes} Profil oder eine \emph{eigene} Einreichung sehen).
Deshalb prüft für die relevanten Seiten das jeweilige Backing Bean diese Beziehung und leitet ggf.\ auf eine Fehlerseite weiter.
Das verhindert \emph{Insercure Direct Object Reference}, da IDs von Submissions, Scientific Forums und Profilen über URL-Parameter preisgegeben werden.

Durch den Eintrag
\begin{lstlisting}
    <servlet>
        <servlet-name>facesServlet</servlet-name>
        <servlet-class>jakarta.faces.webapp.FacesServlet</servlet-class>
        <load-on-startup>1</load-on-startup>
    </servlet>
    <servlet-mapping>
        <servlet-name>facesServlet</servlet-name>
        <url-pattern>/faces/*</url-pattern>
    </servlet-mapping>
\end{lstlisting}
in der web.xml-Datei werden Aufrufe auf alle unsere Facelet-Dateien dem Faces Servlet weitergeleitet.
Damit wird aufgrund der Entscheidung des \code{TrespassListener}s entweder das Facelet richtig als HTML-Datei gerendert, oder zur 404-Seite weitergeleitet. Ein Herunterladen der eigentlichen auf dem Server gespeicherten Facelet-Datei ist aber nicht möglich.