\localauthor{Johannes Garstenauer}

\subsection{Schichten}\label{arch:schichten}

\subsection{Pakete}\label{arch:pakete}

\subsubsection{edu.uni\_passau.lases.control}
Dieses Paket enthält alle Klassen der Kontrollschicht.
\newline\newline
\textbf{\emph{edu.uni\_passau.lases.control.conversion}}
enthält alle Klassen die zur Konvertierung von Nutzereingaben
in Facelets verwendet werden.
\newline\newline
\textbf{\emph{edu.uni\_passau.lases.control.validiation}}
enthält alle Klassen die zur Validierung von Nutzereingaben
in Facelets verwendet werden.
\newline\newline
\textbf{\emph{edu.uni\_passau.lases.control.backing}}\label{arch:backing}
enthält alle Managed Beans welche zur Darstellung der Facelets aus der
%todo link
View Verwendung finden. Zu jedem Facelet [facelet] existiert genau ein
Backing-Bean, welches [facelet]Backing genannt wird.
%todo
%\subsubsection*{edu.uni_passau.lases.control.servlet}

\subsubsection{edu.uni\_passau.lases.business}
Dieses Paket enthält alle Klassen der Logikschicht.
\newline\newline
\textbf{\emph{edu.uni\_passau.lases.business.service}}
enthält alle Klassen die den Hauptteil der Anwendungslogik.
Das sind also alle Dienste welche den
\hyperref[arch:backing]{Backing Beans} zur Verfügung
gestellt werden und auf die auf die %todo link repos
Repositories zugreifen.
\newline\newline
\textbf{\emph{edu.uni\_passau.lases.business.util}}
enthält die Hilfsklassen, welche Dienste zur Verfügung stellen,
die von der Anwendungslogik selbst getrennt werden können.
\newline\newline
\textbf{\emph{edu.uni\_passau.lases.business.exception}}
enthält die Exceptions, die ein Fehlverhalten im System repräsentieren.
\newline\newline
\textbf{\emph{edu.uni\_passau.lases.business.internal}
enthält die Klassen, welche zur Verwaltung intern verwendet werden.
Hierzugehört beispielsweise die Sessionverwaltung, sowie der Systemstart und -stop.

\subsubsection{edu.uni\_passau.lasses.persistence}
Dieses Paket enthält alle Schichten der Persistenzschicht.
\newline\newline
\textbf{\emph{edu.uni\_passau.lases.persistence.repository}}
enthält alle Klassen die den Zugriff auf, in der Datenbank gelagerte,
    Daten ermöglichen
    \newline\newline
    \textbf{\emph{edu.uni\_passau.lases.persistence.util}
    enthält die Hilfsklassen zur Datenbankverwaltung wie einen
    Connection-Pool oder Mailverwaltung.
    \newline\newline
    \textbf{\emph{edu.uni\_passau.lases.persistence.exception}
    enthält die Exceptions welche ein Fehlverhalten im Umgang mit der
    Datenbank repräsentieren.
    \subsection{Frameworks}\label{arch:frameworks}

    \subsection{Patterns}\label{arch:patterns}