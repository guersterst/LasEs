\localauthor{Thomas Kirz}

\subsection{Logging}\label{subsec:logging}
Es gibt verschiedene Loglevel, zu denen bei relevanten Ereignissen ein Eintrag in eine Logdatei hinzugefügt wird: \texttt{ERROR}, \texttt{WARNING} und \texttt{INFO}

\subsection{Sicherheit}\label{subsec:sicherheit}
Relevante Sicherheitsrisiken nach OWASP Top Ten \todo{link}

\subsubsection*{Fehlerhafte Zugangskontrolle}
aber wir ham gut abegtrennte rollen und nen guten trespass listener und so

\subsubsection*{Kryptographie-Fehler}
da machen wir wenig selbst und benutzen robuste Java-Bibliotheken

\subsubsection*{Injection \& XSS}
wir sanitizen alles brav

\subsubsection*{Verwendung veralteter Komponenten}
Wir benutzen moderne dependencies und Technologien.
Da Java-Code so rückwärtskompatibel ist, ist der Upgrade auf neue Versionen einfach.

\subsection{Wartungsthread für Säuberung unso}\label{subsec:wartungsthread}
\todo{mach ma ned oder}

\subsection{Systemstart}\label{subsec:systemstart}
Logging wird initialisiert, Konfigurationsdatei gelesen

\subsection{Systemstopp}\label{subsec:systemstopp}
Bei Systemstopp (Applikationsserver via shutdown port herunterfahren) wird der Connection Pool gestoppt und dabei alle Datenbankverbindungen geschlossen