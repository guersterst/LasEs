%% Macros
\newcommand{\ftable}[1]{\begin{longtable}[H]{|m{2cm}|m{3cm}|m{6cm}|m{2.5cm}|}
                            \hline
                            \textbf{ID} & \textbf{Typ} & \textbf{Beschreibung} & \textbf{Sichtbarkeit} \\
                            \hline
                            \hline
                            #1
\end{longtable}
}

\newcommand{\fentry}[4]{#1 & #2 & #3 & #4 \\
\hline}


\localauthor{Stefanie Gürster, Johann Schicho}

Im Folgendem Abschnitt werden Abkürzungen für die verschiedenen Rollen eingeführt:
\textbf{A} steht für Administrator, \textbf{N} für einen angemeldeten Nutzer, \textbf{E} für einen Editor und \textbf{G} steht für einen Gutachter.
Ist eine Funktion für alle Benutzerrollen vorgesehen, so werden diese unter dem Begriff \textbf{Alle} zusammengefasst.

\todo{OutputText oder graphicimages für Bezeichnung hinzufügen}

\todo{Suchleiste hinzufügen + Datefilter}
\todo{Deadline in Toolbar setzen}

\subsection{Templates}

\localauthor{Johann Schicho}

% https://stackoverflow.com/questions/4003473/make-an-unbreakable-block-in-tex
% Mit samepage bleiben überschrift und tabelle auf der glechen Seite

\begin{samepage}
    \textbf{navigation.xhtml} ist die Kopfzeile der Webanwendung. Diese bietet die Suchfunktion an und Links zu verschiedenen Listen und dem Profil.
    \nopagebreak

    \ftable{
        \fentry{searchfield}{inputText}{Suchleiste}{N }

        \fentry{user-list-link}{link}{Link zur Übersichtsseite aller Nutzer}{A, E }

        \fentry{forum-list-link}{link}{Link zur Übersichtsseite alle Journale und Konferenzen}{N }

        \fentry{logout-button}{commandButton}{Loggt den Nutzer aus dem System aus und leitet zur Loginseite weiter}{N }

        \fentry{profile-link}{link}{Link zur Profilübersicht}{N }

    }
\end{samepage}

\begin{samepage}
    \textbf{main.xhtml} ist das Template, welches die Seiten zwischen Kopf- und Fußzeile einbettet.

    \ftable{
        \fentry{navigation-bar}{include}{Kopfzeile}{Alle}

        \fentry{main-content}{insert}{Seiteninhalt}{Alle }

        \fentry{footer-bar}{include}{Fußzeile}{Alle }

        \fentry{tool-bar}{include}{Seitenleiste}{A, E }

    }
\end{samepage}

\begin{samepage}
    \textbf{footer.xhtml} ist die Fußzeile der Webanwendung und für alle sichtbar.
    \nopagebreak

    \ftable{
        \fentry{imprint-link}{link}{Link zur Seite des Impressums}{Alle}

    }
\end{samepage}

\begin{samepage}
    \textbf{toolbar.xhtml} ist die Seitenleiste für Editoren und Administratoren auf der Einreichungsübersichtsseite. Hier wird die Einreichung verwaltet.
    \nopagebreak

    \ftable{
        \fentry{add-reviewer-field}{inputText}{Eingabefeld zur Angabe einer E-Mailadresse}{A, E}

        \fentry{add-reviewer-btn}{commandButton}{Knopf zum hinzufügen des Gutachters}{A, E}

        \fentry{reviewer-table}{dataTable}{Liste der Gutachter}{A,E}

        \fentry{remove-reviewer}{commandButton}{Entferne einen bestimmten Gutacher}{A, E}

        \fentry{select-editor}{selectOneMenu}{Auswahländerung des verwaltenden Editors}{A, E}

        \fentry{save-editor}{commandButton}{Speichern des geänderten Editors}{A, E}

        \fentry{require-revision-btn}{commandButton}{Fordert den Einreicher auf, seine Einreichung zu überarbeiten}{A, E}

        \fentry{accept-submission-btn}{commandButton}{Akzeptiere die Einreichung}{A, E}

        \fentry{reject-submission-btn}{commandButton}{Lehne die Einreichung ab}{A, E}

    }
\end{samepage}

\localauthor{Stefanie Gürster}

\begin{samepage}
\textbf{listPagination.xhtml} Composite Component zur paginierung von Listen.
\emph{listPagination.xhtml} wird in jeder Tabelle oder Liste eingebunden.
\todo{Beschreibung von ListPagination verändern.}
\nopagebreak

\ftable{
        \fentry{back}{commandButton}{Laden der vorherigen Seite der Liste.}{Alle}
        \fentry{next}{commandButton}{Laden der nächsten Seite der Liste.}{Alle}
        \fentry{page}{selectOneMenue}{Laden einer bestimmten Seite}{Alle}
}
\end{samepage}

\subsection{Seiten}

\localauthor{Stefanie Gürster}

\begin{samepage}
    \textbf{initialConfig.xhtml} Auf dieser Seite landet man beim ersten Systemstart. Sie ist nur für Administratoren zugänglich. Hier kann der Administrator die Datenbankschemata erstellen lassen.
    \nopagebreak

    \ftable{
        \fentry{db-connection-state}{outputText}{Information über die Verbindung mit der Datenbank}{A}

        \fentry{create-db-btn}{commandButton}{Erstellt die Datenbankschemata}{A}
    }
\end{samepage}

\begin{samepage}
    \textbf{welcome.xhtml} Auf der Login- bzw. Welcome-Seite wird \emph{LasEs} vorgestellt.
    Zusätzlich gibt es ein Login-Formular zur Anmeldung im System.
    Für nicht registrierte Nutzer wird man über einen gegebenen Link zu \emph{registration.xhtml} weitergeleitet.
    \nopagebreak

    \ftable{

        \fentry{welcomeHeading}{outputText}{Überschrift der Welcomepage.}{Alle}

        \fentry{welcomeText}{outputText}{Bewerben der Applikation mithilfe einer Kurzbeschreibung.}{Alle }

        \fentry{username}{inputText}{Textfeld für Username Eingabe.}{Alle }

        \fentry{password}{inputSecret}{Textfeld für Passwort Eingabe.}{Alle }

        \fentry{login}{commandButton}{Ausführen des Login-Prozesses.}{Alle }

        \fentry{register}{link}{Weiterleitung zur Registrierung.}{Alle }

    }
\end{samepage}

\begin{samepage}
    \textbf{registration.xhtml} Seite zur Registrierung anonymer Nutzer.
    \nopagebreak

    \ftable{

        \fentry{title}{inputText}{Angabe eines Titels.}{Alle }

        \fentry{firstName}{inputText}{Angabe des Vornamens.}{Alle }

        \fentry{name}{inputText}{Angabe des Nachnamen.}{Alle }

        \fentry{password}{inputSecret}{Angabe eines Passwortes.}{Alle }

        \fentry{email}{inputText}{Angabe einer validen E-Mail.}{Alle }

        \fentry{register}{commandButton}{Ausführen des Registrations-Prozesses.}{Alle }

        \fentry{welcomepage}{link}{Weiterleitung zur Login Seite.}{Alle}
    }
\end{samepage}

\begin{samepage}
    \textbf{verification.xhtml} Wird bei erfolgreicher Verifikation der E-Mail angezeigt.
    \nopagebreak

    \ftable{

        \fentry{successText}{outputText}{Text zur erfolgreichen Verifizierung der E-Mail}{Alle}

        \fentry{countdown}{timer}{Timer bis zur Weiterleitung auf die Homepage.}{Alle }

    }
\end{samepage}

\textbf{homepage.xhtml}\label{flt:homepage} Startseite die den Überblick über alle Einreichungen beinhaltet.

\begin{samepage}
    \ftable{
        \fentry{yourSubmission}{commandLink}{Reiter zum Anzeigen der eigenen Einreichungen.}{Alle}
        \fentry{yourReviews}{commandLink}{Reiter zum Anzeigen der zu begutachtenden Einreichungen}{G}
        \fentry{editorialOverview}{commandLink}{Reiter zum Anzeigen aller Einreichungen, welche man editiert.}{E}
        \fentry{filterLabel}{outputLabel}{Label für Filter}{Alle}
        \fentry{stateLabel}{outputLabel}{Label für Filter vom Status der Einreichung}{Alle}
        \fentry{stateSelect}{selectOneMenu}{Filtern des Status von Einreichungen}{Alle}
        \fentry{dateLabel}{outputLabel}{Label für Filter von Datum.}{Alle}
        \fentry{dateUp}{commandButton}{Einreichungen werden aufsteigend angezeigt.}{Alle}
        \fentry{dateDown}{commandButton}{Einreichungen werden absteigend angezeigt.}{Alle}
        \fentry{apply}{commandButton}{Wende Filtereinstellungen an.}{Alle}
        \fentry{submissionTable}{dataTable}{Liste aller eignen Einreichungen.}{Alle}
        \fentry{reviewTable}{dataTable}{Liste aller Gutachten.}{G}
        \fentry{editorTable}{dataTable}{Liste aller Einreichungen für einen Editor.}{E}
        \fentry{pagination}{listPagination}{Jede Tabelle enthält eine Paginierung.}{Alle}
    }
    \todo{kann nach Forum gefiltert werden wenn ja wie?}
\end{samepage}
\begin{samepage}
    Die einzelnen Tabellen \emph{submissionTable}, \emph{reviewTable} und \emph{editorTable} sind nach den gleichen Punkten gegliedert:

    \ftable{
        \fentry{title}{Link}{Titel des Papers und Weiterleitung zur Einreichung}{Alle}
        \fentry{date}{outputText}{Datum der Einreichung.}{Alle}
        \fentry{state}{outputText}{Status der Einreichung.}{Alle}
        \fentry{forum}{Link}{Name des zugehörigen Forums und Weiterleitung zum Forum.}{Alle}
    }
\end{samepage}

\begin{samepage}

    \textbf{submission.xhtml} ist die Übersichtsseite einer Abgabe.
    Hier werden alle mit der Einreichung verbundenen Aktivitäten abgebildet.
    Für den \textbf{Editor} und den \textbf{Administrator} wird zusätzlich eine \emph{toolbar.xhtml} eingebunden.
    \nopagebreak

        \ftable{
            \fentry{title}{outputText}{Titel des Papers.}{Alle}
            \fentry{forumLabel}{outputLabel}{Label für ein Forum.}{Alle}
            \fentry{forum}{Link}{Name des Forums und Weiterleitung zur Forumsübersicht.}{Alle}
            \fentry{authorLabel}{outputLabel}{Label für Autor.}{Alle}
            \fentry{author}{Link}{Name des Autors und Weiterleitung zum Profil.}{Alle}
            \fentry{coAuthorLabel}{outputLabel}{Namen der Co-Autoren.}{Alle}
            \fentry{emailLabel}{outputLabel}{Label für E-Mail-Adressen der Co-Authoren.}{Alle}
            \fentry{email}{Link}{E-Mail-Adressen der Co-Authoren und Mailto-Link.}{Alle}
            \fentry{filterLabel}{outputLabel}{Label für Filter}{Alle}
            \fentry{dateLabel}{outputLabel}{Label für Filter von Datum.}{Alle}
            \fentry{dateUp}{commandButton}{Einreichungen und Gutachten werden aufsteigend angezeigt.}{Alle}
            \fentry{dateDown}{commandButton}{Einreichungen und Gutachten werden absteigend angezeigt.}{Alle}
            \fentry{reviewStateLabel}{outputLabel}{Label für Filtern der Gutachtenfreischaltung.}{E, A}
            \fentry{all}{checkbox}{Anzeigen aller Gutachten.}{E, A, G}
            \fentry{allLabel}{outputLabel}{Label für all.}{E, A, G}
            \fentry{locked}{checkbox}{Anzeigen nicht freigeschalteter Gutachten.}{E, A, G}
            \fentry{lockedLabel}{ouputLabel}{Laberl für locked.}{E, A, G}
            \fentry{unlocked}{checkbox}{Anzeigen freigeschalteter Gutachten.}{E, A, G}
            \fentry{unlockedLabel}{ouputLabel}{Laberl für unlocked.}{E, A, G}
            \fentry{reviewSubmitLabel}{outputLabel}{Label für Filter von eingereichten Gutachten.}{E, A}
            \fentry{submitted}{checkbox}{Anzeigen aller abgegebenen Gutachten.}{E,A}
            \fentry{submittedLabel}{outputLabel}{Label für submitted.}{E,A}
            \fentry{notSubmitted}{checkbox}{Anzeigen noch nicht abgegebener Gutachten.}{E, A}
            \fentry{notSubmittedLabel}{ouputLabel}{Label für notSubmitted.}{E, A}
            \fentry{versionLabel}{outputLabel}{Label für Versionsfilter.}{Alle}
            \fentry{version}{selectOneMenue}{Auswahl der Versionsnummern}{Alle}
            \fentry{reviewerLabel}{outputLabel}{Label für Reviewerfilter.}{Alle}
            \fentry{reviewer}{selectOneMenue}{Auswahl eines bestimmten Gutachters.}{E, A}
            \fentry{recommendationLabel}{outputLabel}{Label für Empfehlungen.}{E, A, G}
            \fentry{recommendation}{selectOneMenue}{Auswahl der Art von Empfehlungen.}{E, A, G}
            \fentry{searchLabel}{outputLabel}{Label für Suche.}{Alle}
            \fentry{search}{inputText}{Durchsuchen der Tabelle mit Matching-Words.}{Alle}
            \fentry{apply}{commandButton}{Wende Filtereinstellungen an.}{Alle}
            \fentry{submissionTable}{dataTable}{Liste aller Versionen}{Alle}
            \fentry{reviewTable}{dataTable}{Liste aller/der eigenen Gutachten.}{E, A, G}
            \fentry{pagination}{listPagination}{Jede Tabelle enthält eine Paginierung.}{Alle}
        }

    \emph{submissionTable} enthält die Spalten: Version, Datum, Deadline, Status und Download.
    Die Elemente der Liste, der unterschiedlichen Versionen der Einrichtung, sieht wie folgt aus:

    \ftable{
        \fentry{version}{outputText}{Versionsnummer der Einreichung}{Alle}
        \fentry{date}{outputText}{Datum der Einreichung}{Alle}
        \fentry{deadline}{outputText}{Deadline der Revision}{Alle}
        \fentry{state}{radioButton}{Status der Einreichung}{Alle}
        \fentry{pdf}{inputFile}{Download der Einreichung}{Alle}
    }

    \emph{reviewTable} enthält die Spalten: Version, Gutachter, Datum, Deadline, Status, Empfehlung, Kommentar und Download.
    Die Elemente der Liste von Gutachten ist wie folgt:

    \ftable{
        \fentry{version}{outputText}{Versionsnummer des Gutachtens.}{E, A, G}
        \fentry{reviewer}{outputText}{Namen des Gutachters.}{E, A, G}
        \fentry{date}{outputText}{Datum des Gutachtens.}{E, A, G}
        \fentry{deadline}{outputText}{Deadline des Gutachtens.}{E, A, G}
        \fentry{state}{commandButton}{Freigabe des Gutachtens}{E, A}
        \fentry{stateText}{outputText}{Freigabestatus des Gutachtens}{G}
        \fentry{recomendation}{checkbox}{Empfehlung eines Gutachters}{E, A, G}
        \fentry{comment}{outputText}{Kommentar des Gutachters.}{E, A, G}
        \fentry{pdf}{inputFile}{Download der Einreichung}{E, A, G}
    }
\end{samepage}

\begin{samepage}
    \textbf{newSubmission.xhtml} Hier können neue Paper eingereicht werden.
    \nopagebreak

    \ftable{

        \fentry{submissionName}{inputText}{Name des abzugebenden Papers.}{N}

        \fentry{forumName}{inputText}{Name des Forums, bei welchem abgegeben wird.}{N }

        \fentry{editorSearch}{inputText}{Angabe eines Editors}{N}

        \fentry{pfdUpload}{inputFile}{Angabe der Abgabedatei.}{N}

        \fentry{titel}{inputText}{Angabe eines Titels.}{N}

        coAuthorFirstName & inputText & Angabe des Vornamen eines Co-Autors.& N\\

        \fentry{coAuthorName}{inputText}{Angabe des Namen eines Co-Autors.}{N}

        \fentry{coAuthorEMail}{inputText}{Angabe der E-Mail eines Co-Autors.}{N}

        \fentry{submitCoAuthor}{commandButton}{Ausgabe des Co-Autors in einer Liste.}{N}

        \fentry{coAuthorList}{outputText}{Anzeige aller Co-Authoren in einer Liste.}{N}

        \fentry{deleteCoAuthor}{commandButton}{Löschen des zugehörigen Co-Authors.}{N}

        \fentry{submit}{commandButton}{Ausführen des Abgabe-Prozesses.}{N }
    }
\end{samepage}

\begin{samepage}
    \textbf{newReview.xhtml} Möchte ein Gutachter eines Papers seine Beurteilung abgeben, so ist dies auf dieser Seite möglich.
    \nopagebreak

    \ftable{

        \fentry{versionNumber}{outputText}{Versionsnummer des eingereichten Papers, zu welchem der Gutachter ein Gutachten abgeben kann.}{G,A}

        \fentry{recommendation}{checkbox}{Empfehlung des Gutachters ein Paper freizuschalten.}{G,A}

        \fentry{comment}{inputTextarea}{Kommentar eines Gutachters}{G,A}

        \fentry{reviewPdf}{inputFile}{Einzureichendes Gutachten.}{G,A }
    }
\end{samepage}


\begin{samepage}
    \textbf{resultList.xhtml} zeigt alle Ergebnisse einer Suche an.
    \nopagebreak

    Wird nach den eigenen Einreichungen, Gutachten oder Einreichungen in eingener editorialen Verantwortung gesucht, so erscheinen die Listen im gleiche Schema wie auch auf \hyperref[flt:homepage]{der Homepage}.
    Ergibt die Suche eine Liste von User, so wird diese wie in \hyperref[flt:userList]{der Userliste} für den Administrator und dem Editor angezeigt.
    \nopagebreak

    \ftable{

    }
\end{samepage}


\begin{samepage}
	\localauthor{Johann Schicho}\nopagebreak

	\textbf{scientificForumList.xhtml} HIer erhalten die Nutzer eine Übersicht über alle wissenschaftliche Foren.\nopagebreak

	\ftable{

		\fentry{filterLabel}{outputLabel}{Label für Filter}{Alle}

		\fentry{sc-fields-label}{outputLabel}{Label für Filter der Themengebiete}{Alle}

		\fentry{sc-field-select}{selectOneMenu}{Filtern der Themengebiete der Foren}{Alle}

		\fentry{deadline-label}{outputLabel}{Label für Filtern nach Deadlines (Abgelaufen/Zukunft)}{Alle}

		\fentry{deadline-select}{selectOneMenu}{Filtern der nach Deadlines}{Alle}

		\fentry{apply}{commandButton}{Wende Filtereinstellungen an.}{Alle}

		\fentry{nameUp}{commandButton}{Einreichungen werden aufsteigend nach Namen angezeigt.}{Alle}

		\fentry{nameDown}{commandButton}{Einreichungen werden absteigend nach Namen angezeigt.}{E,A}

		\fentry{deadlineUp}{commandButton}{Einreichungen werden aufsteigend nach Deadline angezeigt.}{Alle}

		\fentry{deadlineDown}{commandButton}{Einreichungen werden absteigend nach Deadline angezeigt.}{Alle}

		\fentry{forumsTable}{dataTable}{Liste aller Foren im System.}{Alle}

		\fentry{pagination}{listPagination}{Die Tabelle enthält eine Paginierung.}{Alle}
}
\end{samepage}


\begin{samepage}
	\localauthor{Stefanie Gürster}\nopagebreak

    \textbf{profile.xhtml} Die Profilseite eines angemeldeten Nutzers ist nur vom Nutzer selbst und vom Administrator editierbar.
    Ein Editor besitzt nur Leserechte. Alle zuvor gespeicherten Angaben werden in den jeweiligen Feldern angezeigt.
    \nopagebreak

    \ftable{

        \fentry{avatar}{graphicImage}{Aktueller Avatar.}{N,A,E}

        \fentry{newAvatar}{inputFile}{Hochladen eines neuen Avatars.}{N,A,E}

        \fentry{deleteAvatar}{commandButton}{Löschen des Avatars.}{N,A,E}

        \fentry{role}{outputText}{Angabe der Rolle eines Nutzers}{N,A,E}

        \fentry{isAdmin}{checkbox}{Zuteilung der Rolle des Administrators.}{A}

        \fentry{title}{inputText}{Titel des Nutzenden.}{N,A,E}

        \fentry{firstName}{inputText}{Vorname des Nutzenden.}{N,A,E}

        \fentry{name}{inputText}{Nachname des Nutzenden.}{N,A,E}

        \fentry{password}{inputSecret}{Leeres Eingabefeld für ein neues Passwort.}{N,A}

        \fentry{email}{inputText}{E-Mail des Nutzenden.}{N,A,E}

        \fentry{submissionNumber}{outputText}{Anzahl der eigenen Einreichungen}{N,A,E}

        \fentry{employer}{inputText}{Angabe des Arbeitgebers}{N,A,E}

        \fentry{specialty}{}{Spezialgebiete des Nutzers}{N,A,E}

        \fentry{dateOfBirth}{inputText}{Angabe des Geburtstages}{Nutzer}

        \fentry{save}{commandButton}{Daten werden persistiert.}{N,A,E}

        \fentry{delete}{commandButton}{Nutzer und alle damit verbundenen Daten werden gelöscht.}{N,A,E}
    }
\end{samepage}

Sollte ein Administrator einem anderen Nutzer die Rolle des Administrators zuweisen oder entziehen, so erscheint beim Auslösen des Save-Buttons ein Popup-Dialog.
In diesem Fenster wird der Administrator angewiesen, die Änderung mit seinem Passwort zu bestätigen.

\ftable{

    \fentry{password}{inputSecret}{Angabe eines Passworts.}{A}

    \fentry{reference}{outputText}{Hinweis zur Veränderung der Adminrolle.}{A}

    \fentry{abort}{commandButton}{Abbruch der Änderung.}{A}

    \fentry{save}{commandButton}{Speichern aller Änderungen.}{A}

}

\todo{bearbeiten kann es nur der Nutzer aber betrachten andere....+ was ist mit Passwort F200+ doch getrennte Profilseiten? + avatarbild}

\begin{samepage}
    \textbf{errorPage.xhtml} Auf diese Seite wird navigiert, wenn auf eine nicht existierende URL oder auf eine URL zugegriffen wird, auf die man keine Zugriffsrechte hat.
    \nopagebreak

    \ftable{

        \fentry{errorMessage}{outputText}{Anzeige einer Fehlermeldung.}{Alle}

    }
\end{samepage}


\begin{samepage}
	\localauthor{Johann Schicho} \nopagebreak

	\textbf{userList.xhtml} \label{flt:userList} Editoren und Administratoren erhalten hier einen Überblick über alle Nutzer. Die Seiten sind paginiert.
	\nopagebreak

	\ftable{

		\fentry{filterLabel}{outputLabel}{Label für Filter}{E,A}

		\fentry{roleLabel}{outputLabel}{Label für Filter Rolle des Nutzers}{E,A}

		\fentry{roleSelect}{selectOneMenu}{Filtern der Rolle des Nutzers}{E,A}

		\fentry{apply}{commandButton}{Wende Filtereinstellungen an.}{E,A}

		\fentry{nameUp}{commandButton}{Nutzer werden nach Namen aufsteigend angezeigt.}{E,A}

		\fentry{nameDown}{commandButton}{Nutzer werden absteigend nach Namen angezeigt.}{E,A}

		\fentry{roleUp}{commandButton}{Nutzer werden nach Rolle aufsteigend angezeigt.}{E,A}

		\fentry{roleDown}{commandButton}{Nutzer werden absteigend nach Rolle angezeigt.}{E,A}

		\fentry{userTable}{dataTable}{Liste aller Nutzer im System.}{E,A}

		\fentry{pagination}{listPagination}{Die Tabelle enthält eine Paginierung.}{E,A}
}
\end{samepage}

\begin{samepage}
    Für die Benutzerliste sind die folgenden Elemente vorgesehen, wobei diese in die Bereiche: Nutzerrolle, Name, E-Mail-Adresse, Arbeitgeber und Fachgebiete aufgegliedert ist.

    \ftable{
        \fentry{role}{outputText}{Rolle des Nutzers}{E, A}
    }
\end{samepage}

\todo{Dropdown für rollen.}

\begin{samepage}
	\localauthor{Stefanie Gürster}\nopagebreak

    \textbf{newUser.xhtml} Der Administrator kann hier einen neuen Nutzer anlegen.
    \nopagebreak

    \ftable{

        \fentry{password}{inputSecret}{Angabe des Passworts.}{A}

        \fentry{firstName}{inputText}{Angabe des Vornamen.}{A}

        \fentry{name}{inputText}{Angabe des Nachnamen.}{A}

        \fentry{titel}{inputText}{Angabe eines Titels.}{A}

        \fentry{email}{inputText}{Angabe einer E-Mail-Adresse.}{A}

        \fentry{isAdmin}{checkbox}{Zuordnung einer Administratorenrolle.}{A}

        \fentry{abort}{commandButton}{Abbruch des Vorgangs: Nutzererstellen.}{A}

        \fentry{save}{commandButton}{Speichern des neuen Nutzers.}{A }

    }
\end{samepage}

\begin{samepage}
    \textbf{administration.xhtml} Hier kann ein Administrator Konfigurationen vornehmen.
    \nopagebreak

    \ftable{

        \fentry{themeHeading}{outputText}{Überschrift für Farbthemes.}{A}

        \fentry{orangeTheme}{outputText}{Standart Theme der Applikation.}{A}

        \fentry{orange}{radioButton}{Auswahl des Standart Themes.}{A}

        \fentry{darkTheme}{outputText}{Dark Theme der Applikation.}{A}

        \fentry{dark}{radioButton}{Auswahl des Dark Themes.}{A}

        \fentry{blueTheme}{outputText}{Blue Theme der Applikation}{A}

        \fentry{blue}{radioButton}{Auswahl des Blue Themes.}{A}

        \fentry{greenTheme}{outputText}{Green Theme der Applikation.}{A}

        \fentry{green}{radioButton}{Auswahl des Green Themes.}{A}

        \fentry{logoHeading}{outputText}{Überschrift für Logo-Bereich.}{A}

        \fentry{logo}{graphicImage}{Logo des Systems.}{A}

        \fentry{changeLogo}{inputFile}{Hochladen eines neuen Logos.}{A}

        \fentry{institutionHeading}{outputText}{Überschrift für Einstellungen der Einrichtung.}{A}

        \fentry{institution}{inputText}{Angabe des Namens der Einrichtung.}{A}

        \fentry{imprint}{inputTextArea}{Angabe des Impressums der Einrichtung.}{A}

        \fentry{abort}{commandButton}{Abbruch des Änderungsvorgangs.}{A }

        \fentry{save}{commandButton}{Speichern der Änderungen.}{A}

        \fentry{newUser}{link}{Weiterleitung zu \emph{newUser.xhtml}.}{A}

        \fentry{newScientificForum}{link}{Weiterleitung zu \emph{newScientificForum.xhtml}.}{A}
    }
\end{samepage}

\begin{samepage}
    \textbf{newScientificForum.xhtml} Auf der Seite zum Erstellen eines wissenschaftlichen Forums kann ein Administrator dessen wesentlichen Daten festlegen.
    \nopagebreak

    \ftable{

        \fentry{forumName}{inputText}{Angabe des Name des Forums.}{A}

        \fentry{fName}{outputLabel}{Label Name des Forums.}{A}

        \fentry{emailEditor}{inputText}{Angabe der E-Mail-Adresse eines Editor.}{A}

        \fentry{emailE}{outputLabel}{Label für E-Mail-Adresse eines Editors.}{A}

        \fentry{addEditor}{commandButton}{Hinzufügen eines Editors in eine Liste.}{A}

        \fentry{deleteEditor}{commandButton}{Löschen des zugehörigen Editors von der Liste.}{A}

        \fentry{deadline}{inputText}{Hinzufügen einer Deadline des Forums.}{A}

        \fentry{deadlineLabel}{outputLabel}{Label für Deadline des Forums.}{A}

        \fentry{description}{inputTextArea}{Angabe einer Kurzbeschreibung.}{A}

        \fentry{descriptionLabel}{outputLabel}{Label für Kurzbeschreibung.}{A}

        \fentry{url}{inputText}{Link zur Konferenz oder zum Journal.}{A}

        \fentry{urlLabel}{outputLabel}{Label für URL des Forums.}{A}

        \fentry{reviewInstructions}{inputTextArea}{Angabe einer Anleitung für eine Begutachtung.}{A}

        \fentry{specialty}{selectOnMenu}{Auswahl von Fachgebieten.}{A}

        \fentry{specialtyLabel}{outputLabel}{Label für Fachgebiete.}{A}

        \fentry{newSpecialty}{inputText}{Hinzufügen neuer Fachgebiete.}{A}

        \fentry{addSpecialty}{commandButton}{Ausführen der Hinzufüge-Aktion.}{A}

        \fentry{save}{commandButton}{Speichern des neuen Forums.}{A}

        \fentry{abort}{commandButton}{Abbruch des Erstellungsprozesses.}{A}
    }
\end{samepage}

\begin{samepage}
    \textbf{editForum.xhtml} Hier kann der Administrator eine schon existierende Seite bearbeiten.
    \nopagebreak

    \ftable{
        \fentry{forumName}{inputText}{Angabe des Name des Forums.}{A}

        \fentry{fName}{outputLabel}{Label Name des Forums.}{A}

        \fentry{emailEditor}{inputText}{Angabe der E-Mail-Adresse eines Editor.}{A}

        \fentry{emailE}{outputLabel}{Label für E-Mail-Adresse eines Editors.}{A}

        \fentry{addEditor}{commandButton}{Hinzufügen eines Editors in eine Liste.}{A}

        \fentry{deleteEditor}{commandButton}{Löschen des zugehörigen Editors von der Liste.}{A}

        \fentry{deadline}{inputText}{Hinzufügen einer Deadline des Forums.}{A}

        \fentry{deadlineLabel}{outputLabel}{Label für Deadline des Forums.}{A}

        \fentry{description}{inputTextArea}{Angabe einer Kurzbeschreibung.}{A}

        \fentry{descriptionLabel}{outputLabel}{Label für Kurzbeschreibung.}{A}

        \fentry{url}{inputText}{Link zur Konferenz oder zum Journal.}{A}

        \fentry{urlLabel}{outputLabel}{Label für URL des Forums.}{A}

        \fentry{reviewInstructions}{inputTextArea}{Angabe einer Anleitung für eine Begutachtung.}{A}

        \fentry{scienceField}{selectOneMenus}{Auswahl von Fachgebieten.}{A}

        \fentry{specialtyLabel}{outputLabel}{Label für Fachgebiete.}{A}

        \fentry{newSpecialty}{inputText}{Hinzufügen neuer Fachgebiete.}{A}

        \fentry{addSpecialty}{commandButton}{Ausführen der Hinzufüge-Aktion.}{A}

        \fentry{save}{commandButton}{Speichern des Forums.}{A}

        \fentry{abort}{commandButton}{Abbruch des Bearbeitungsprozesses.}{A}
    }
\end{samepage}

\begin{samepage}
	\localauthor{Johann Schicho}\nopagebreak

    \textbf{scientificForum.xhtml} Die Ansicht eines Forums dient zur Ausgabe von Informationen über die jeweilige Konferenz oder das jeweilige Journal.
    \nopagebreak

    \ftable{

        \fentry{forumName}{outputText}{Name des Forums.}{Alle}

        \fentry{editor}{outputText}{Liste der verantwortliche Editoren.}{Alle}

        \fentry{deadline}{outputText}{Deadline des Forums.}{Alle}

        \fentry{deadlineLabel}{outputLabel}{Label für Deadline des Forums.}{Alle}

        \fentry{description}{outputText}{Kurzbeschreibung des Forums.}{Alle}

        \fentry{descriptionLabel}{outputLabel}{Label für Kurzbeschreibung.}{Alle}

        \fentry{url}{outputLink}{Link zur Konferenz oder zum Journal.}{Alle}

        \fentry{urlLabel}{outputLabel}{Label für URL des Forums.}{Alle}

        \fentry{reviewInstructions}{outputText}{Anleitung für eine Begutachtung.}{G, E, A}

        \fentry{specialty}{outputText}{Fachgebiet des Forums.}{Alle}

        \fentry{specialtyLabel}{outputLabel}{Label für Fachgebiete.}{Alle}

        \fentry{filterLabel}{outputLabel}{Label für Filter}{G,E,A}

        \fentry{stateLabel}{outputLabel}{Label für Filter vom Status der Einreichung}{G,E,A}

        \fentry{stateSelect}{selectOneMenu}{Filtern des Status von Einreichungen}{G,E,A}

        \fentry{dateLabel}{outputLabel}{Label für Filter von Datum.}{G,E,A}

        \fentry{apply}{commandButton}{Wende Filtereinstellungen an.}{G,E,A}

        \fentry{dateUp}{commandButton}{Einreichungen werden aufsteigend angezeigt.}{G,E,A}

        \fentry{dateDown}{commandButton}{Einreichungen werden absteigend angezeigt.}{G,E,A}

        \fentry{submissionTable}{dataTable}{Liste aller Einreichungen, welche der Nutzer bearbeitet.}{Alle}

        \fentry{pagination}{listPagination}{Die Tabelle enthält eine Paginierung.}{Alle}
    }
\end{samepage}

\begin{samepage}
    \textbf{imprint.xhtml} Das Impressum gibt die vom Administrator angegebenen Kontaktdaten des Betreibers wieder.\nopagebreak

    \ftable{

        \fentry{imprintHeading}{outputText}{Überschrift der Ansicht}{A}

        \fentry{imprint}{outputText}{Impressum des Betreibers.}{A}

    }
\end{samepage}
