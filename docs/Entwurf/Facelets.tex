%% Macros
\newcommand{\ftable}[1]{\begin{longtable}[H]{|m{2cm}|m{3cm}|m{6cm}|m{2.5cm}|}
                            \hline
                            \textbf{ID} & \textbf{Typ} & \textbf{Beschreibung} & \textbf{Rechte} \\
                            \hline
                            \hline
                            #1
\end{longtable}
}

\newcommand{\fentry}[4]{#1 & #2 & #3 & #4 \\
\hline}


\localauthor{Stefanie Gürster, Johann Schicho}

Im Folgendem Abschnitt werden Abkürzungen für die verschiedenen Rollen eingeführt:
\textbf{A} steht für Administrator, \textbf{AN} für einen anonymen Nutzer, \textbf{N} für einen angemeldeten Nutzer, \textbf{E} für einen Editor und \textbf{G} steht für einen Gutachter.
Ist eine Funktion für alle Benutzerrollen außer dem anonymen Nutzer vorgesehen, so werden diese unter dem Begriff \textbf{Alle} zusammengefasst.
Außerdem verfügt ein Administrator über alle Rechte der anderen Rollen und wird daher nur bei Schlüsselfunktionen genannt.

\subsection{Templates}

\localauthor{Johann Schicho}

% https://stackoverflow.com/questions/4003473/make-an-unbreakable-block-in-tex
% Mit samepage bleiben überschrift und tabelle auf der glechen Seite

\begin{samepage}
    \textbf{navigation.xhtml} \label{flt:navbar} ist die Kopfzeile der Webanwendung. Diese bietet die Suchfunktion an und Links zu verschiedenen Listen und dem Profil.
    \nopagebreak

    \ftable{
        \fentry{searchField}{inputText}{Suchleiste}{Alle}

        \fentry{search}{commandButton}{Suche ausführen.}{Alle}

        \fentry{userListLink}{link}{Link zur Übersichtsseite aller Nutzer}{E}

        \fentry{forumListLink}{link}{Link zur Übersichtsseite alle Journale und Konferenzen}{Alle}

        \fentry{logoutButton}{commandButton}{Loggt den Nutzer aus dem System aus und leitet zur Loginseite weiter}{Alle}

        \fentry{profileLink}{link}{Link zur Profilübersicht}{N}

    }
\end{samepage}

\begin{samepage}
    \textbf{main.xhtml} ist das Template, welches den Inhalt der Seiten zwischen Kopf- und Fußzeile einbettet.

    \ftable{
        \fentry{navigationBar}{include}{\hyperref[flt:navbar]{Kopfzeile}}{Alle}

        \fentry{mainContent}{insert}{\hyperref[flt:pages]{Seiteninhalt}}{Alle}

        \fentry{footerBar}{include}{\hyperref[flt:footer]{Fußzeile}}{Alle}

        \fentry{toolBar}{include}{\hyperref[flt:toolbar]{Seitenleiste}}{E}

    }
\end{samepage}

\begin{samepage}
    \textbf{footer.xhtml} \label{flt:footer} ist die Fußzeile der Webanwendung und für alle sichtbar.
    \nopagebreak

    \ftable{
        \fentry{imprintLink}{link}{Link zur Seite des Impressums}{Alle}

        \fentry{language}{outputText}{Anzeige der eingestellten Sprache.}{Alle}

    }
\end{samepage}

\begin{samepage}
    \textbf{toolbar.xhtml} \label{flt:toolbar} ist die Seitenleiste für Editoren und Administratoren auf der Einreichungsübersicht. Hier wird die Einreichung verwaltet.
    \nopagebreak

    \ftable{
        \fentry{addReviewerField}{inputText}{Eingabefeld zur Angabe einer E-Mailadresse}{E}

        \fentry{addDeadlineReviewer}{inputText}{Eingabefeld zur Angabe einer Deadline.}{E}

        \fentry{addReviewerBtn}{commandButton}{Knopf zum hinzufügen des Gutachters}{E}

        \fentry{reviewerTable}{dataTable}{Liste der Gutachter}{E}

        \fentry{removeReviewer}{commandButton}{Entferne einen bestimmten Gutachter}{E}

        \fentry{selectEditor}{selectOneMenu}{Auswahländerung des verwaltenden Editors}{E}

        \fentry{saveEditor}{commandButton}{Speichern der Änderung}{E}

        \fentry{revisionDeadline}{inputText}{Eingabe zur Angabe einer Deadline.}{E}

        \fentry{requireRevisionBtn}{commandButton}{Fordert den Einreicher auf, seine Einreichung zu überarbeiten}{E}

        \fentry{acceptSubmissionBtn}{commandButton}{Akzeptiere die Einreichung}{E}

        \fentry{rejectSubmissionBtn}{commandButton}{Lehne die Einreichung ab}{E}

    }
\end{samepage}

\begin{samepage}
    \textbf{pagination.xhtml} ist ein Composite Component zur Paginierung von Listen.
    \nopagebreak

    \ftable{
        \fentry{dataTable}{dataTable}{Hier werden die Spalten eingefügt. Durch Klicken auf die Spaltennamen werden die Einträge sortiert. Zusätzlich kann durch ein Textfeld die Liste gefiltert werden.}{Alle}

        \fentry{apply}{commandButton}{Mit diesem Knopf können gewählte Sortierung und Filterung angewandt werden.}{Alle}

        \fentry{navButtons}{panelGrid}{Hier befinden sich die Knöpfe zur Navigation zwischen den Seiten der Paginierungen.}{Alle}

		\fentry{firstPage}{commandButton}{Zurück zur ersten Seite.}{Alle}

		\fentry{prevPage}{commandButton}{Eine Seite zurück.}{Alle}

		\fentry{currentPage}{outputText}{Nummer der aktuellen Seite.}{Alle}

		\fentry{nextPage}{commandButton}{Eine Seite weiter.}{Alle}

		\fentry{lastPage}{commandButton}{Zur letzten Seite.}{Alle}

    }
\end{samepage}

\subsection{Seiten} \label{flt:pages}

\subsubsection{Anonymer Nutzer}
\localauthor{Stefanie Gürster}

Für jedes Facelet ist ein Bereich, für Anzeigen von Fehlern oder positiven Benachrichtigungen, vom Typ \emph{messages} angedacht.
Um Duplikation zu vermeiden, wird dies als Konvention für alle Facelets festgelegt.

\begin{samepage}
\textbf{welcome.xhtml} Auf der Login- bzw. Welcome-Seite wird \emph{LasEs} vorgestellt.
Zusätzlich gibt es ein Login-Formular zur Anmeldung im System.
Für nicht registrierte Nutzer wird man über einen gegebenen Link zu \emph{registration.xhtml} weitergeleitet.
\nopagebreak

\ftable{

    \fentry{welcomeHeading}{outputText}{Überschrift der Welcomepage.}{Alle}

    \fentry{welcomeText}{outputText}{Bewerben der Applikation mithilfe einer Kurzbeschreibung.}{Alle}

    \fentry{email}{inputText}{Textfeld für E-Mail Eingabe.}{Alle}

    \fentry{password}{inputSecret}{Textfeld für Passwort Eingabe.}{Alle}

    \fentry{login}{commandButton}{Ausführen des Login-Prozesses.}{Alle}

    \fentry{register}{link}{Weiterleitung zur Registrierung.}{Alle}

}
\end{samepage}

\begin{samepage}
    \textbf{registration.xhtml} Seite zur Registrierung anonymer Nutzer.
    \nopagebreak

    \ftable{
        \fentry{title}{inputText}{Angabe eines Titels.}{AN}

        \fentry{firstName}{inputText}{Angabe des Vornamens.}{AN}

        \fentry{name}{inputText}{Angabe des Nachnamen.}{AN}

        \fentry{password}{inputSecret}{Angabe eines Passwortes.}{AN}

        \fentry{email}{inputText}{Angabe einer validen E-Mail.}{AN}

        \fentry{register}{commandButton}{Ausführen des Registrations-Prozesses.}{AN}

        \fentry{welcomepage}{link}{Weiterleitung zur Login Seite.}{AN}
    }
\end{samepage}

\begin{samepage}
    \textbf{verification.xhtml} Wird bei erfolgreicher Verifikation der E-Mail angezeigt.
    \nopagebreak

    \ftable{

        \fentry{successText}{outputText}{Text zur erfolgreichen Verifizierung der E-Mail}{Alle}

        \fentry{GoToHome}{commandButton}{Button zur Weiterleitung auf die Homepage.}{Alle}

    }
\end{samepage}


\begin{samepage}
    \textbf{errorPage.xhtml} Auf diese Seite wird navigiert, wenn auf eine nicht existierende URL oder auf eine URL zugegriffen wird, auf die man keine Zugriffsrechte hat.
    \nopagebreak

    \ftable{

        \fentry{errorMessage}{outputText}{Anzeige einer Fehlermeldung.}{Alle,AN}
        \fentry{stackTrace}{outputText}{Stacktrace für den "Productive Mode".}{Development}

    }
\end{samepage}

\begin{samepage}
    \textbf{imprint.xhtml} Das Impressum gibt die vom Administrator angegebenen Kontaktdaten des Betreibers wieder.\nopagebreak

    \ftable{
        \fentry{imprintHeading}{outputText}{Überschrift der Ansicht}{A}

        \fentry{imprint}{outputText}{Impressum des Betreibers.}{A}
    }
\end{samepage}

\subsubsection{Angemeldeter Nutzer}
\localauthor{Stefanie Gürster}

\textbf{homepage.xhtml}\label{flt:homepage} Startseite die den Überblick über alle Einreichungen beinhaltet.

\begin{samepage}
    \ftable{
        \fentry{yourSubmission}{commandLink}{Reiter zum Anzeigen der eigenen Einreichungen.}{Alle}

        \fentry{yourReviews}{commandLink}{Reiter zum Anzeigen der zu begutachtenden Einreichungen}{G}

        \fentry{editorialOverview}{commandLink}{Reiter zum Anzeigen aller Einreichungen, welche man editiert.}{E}

        \fentry{stateSelect}{selectOneMenu}{Filtern des Status von Einreichungen}{Alle}

        \fentry{dateBefore}{inputText}{Eingabe eines \emph{spätestens} Datum}{Alle}

        \fentry{dateAfter}{inputText}{Eingabe eines \emph{frühestens} Datum}{Alle}

        \fentry{submissionPagination}{pagination}{Liste aller eignen Einreichungen.}{Alle}

        \fentry{reviewPagination}{pagination}{Liste aller Gutachten.}{G}

        \fentry{editorPagination}{pagination}{Liste aller Einreichungen für einen Editor.}{E}

    }
\end{samepage}

\begin{samepage}
    Die einzelnen Tabellen \emph{submissionPagination}, \emph{reviewPagination} und \emph{editorPagination} sind nach den gleichen Punkten gegliedert: Titel des Papers, Datum und Status der Einreichung und Name des zugehörigen Forums.

    \ftable{
        \fentry{title}{Link}{Titel des Papers und Weiterleitung zur Einreichung}{Alle}

        \fentry{date}{outputText}{Datum der Einreichung.}{Alle}

        \fentry{state}{outputText}{Status der Einreichung.}{Alle}

        \fentry{forum}{Link}{Name des zugehörigen Forums und Weiterleitung zum Forum.}{Alle}
    }
\end{samepage}

\begin{samepage}

    \textbf{submission.xhtml} ist die Übersichtsseite einer Abgabe.
    Hier werden alle mit der Einreichung verbundenen Aktivitäten abgebildet.
    Für den \textbf{Editor} und den \textbf{Administrator} wird zusätzlich eine \emph{toolbar.xhtml} eingebunden.
    \nopagebreak

    \ftable{
        \fentry{newReview}{commandLink}{Weiterleitung zur Einreichung eines neuen Gutachtens.}{G}

        \fentry{title}{outputText}{Titel des Papers.}{Alle}

        \fentry{forum}{Link}{Name des Forums und Weiterleitung zur Forumsübersicht.}{Alle}

        \fentry{author}{Link}{Name des Autors und Weiterleitung zum Profil.}{Alle}

        \fentry{email}{Link}{E-Mail-Adressen der Co-Autoren und Mailto-Link.}{Alle}

        \fentry{dateUp}{commandButton}{Einreichungen werden aufsteigend angezeigt.}{Alle}

        \fentry{dateDown}{commandButton}{Einreichungen werden absteigend angezeigt.}{Alle}

        \fentry{reviewState}{selectOneMenu}{Filtern der Gutachten nach ihrem Status.}{E,G}

        \fentry{submitted}{checkbox}{Anzeigen aller abgegebenen Gutachten.}{E}

        \fentry{notSubmitted}{checkbox}{Anzeigen noch nicht abgegebener Gutachten.}{E}

        \fentry{version}{selectOneMenu}{Auswahl der Versionsnummern}{Alle}

        \fentry{recommendation}{selectOneMenu}{Auswahl der Art von Empfehlungen.}{E,G}

        \fentry{submissionTable}{dataTable}{Liste aller Versionen}{Alle}

        \fentry{reviewTable}{dataTable}{Liste aller/der eigenen Gutachten.}{E,G}

        \fentry{pagination}{listPagination}{Jede Tabelle enthält eine Paginierung.}{Alle}
    }

    \emph{submissionTable} enthält die Spalten: Version, Datum, Deadline, Status und Download.
    Die Elemente der Liste, der unterschiedlichen Versionen der Einrichtung, sieht wie folgt aus:

    \ftable{
        \fentry{version}{outputText}{Versionsnummer der Einreichung}{Alle}

        \fentry{date}{outputText}{Datum der Einreichung}{Alle}

        \fentry{deadline}{outputText}{Deadline der Revision}{Alle}

        \fentry{state}{radioButton}{Status der Einreichung}{Alle}

        \fentry{pdf}{inputFile}{Download der Einreichung}{Alle}
    }

    \emph{reviewTable} enthält die Spalten: Version, Gutachter, Datum, Deadline, Status, Empfehlung, Kommentar und Download.
    Die Elemente der Liste von Gutachten ist wie folgt:

    \ftable{
        \fentry{version}{outputText}{Versionsnummer des Gutachtens.}{E,G}

        \fentry{reviewer}{outputText}{Namen des Gutachters.}{E,G}

        \fentry{date}{outputText}{Datum des Gutachtens.}{E,G}

        \fentry{deadline}{outputText}{Deadline des Gutachtens.}{E,G}

        \fentry{state}{commandButton}{Freigabe des Gutachtens}{E}

        \fentry{stateText}{outputText}{Freigabestatus des Gutachtens}{G}

        \fentry{recomendation}{checkbox}{Empfehlung eines Gutachters}{E,G}

        \fentry{comment}{outputText}{Kommentar des Gutachters.}{E,G}

        \fentry{pdf}{commandButton}{Download der Einreichung.}{E,G}
    }
\end{samepage}

\begin{samepage}
    \textbf{newSubmission.xhtml} Hier können neue Paper eingereicht werden.
    \nopagebreak

    \ftable{
        \fentry{submissionName}{inputText}{Name des abzugebenden Papers.}{N}

        \fentry{forumName}{inputText}{Name des Forums, bei welchem abgegeben wird.}{N}

        \fentry{editorSearch}{selectOneMenu}{Angabe eines Editors}{N}

        \fentry{pdfOutput}{outputText}{Angabe des ausgewählten Dateinamens.}{N}

        \fentry{pfdUpload}{inputFile}{Angabe der Abgabedatei.}{N}

        \fentry{titel}{inputText}{Angabe eines Titels.}{N}

        \fentry{coAuthorFirstName}{inputText}{Angabe des Vornamen eines Co-Autors.}{N}

        \fentry{coAuthorLastName}{inputText}{Angabe des Namen eines Co-Autors.}{N}

        \fentry{coAuthorEMail}{inputText}{Angabe der E-Mail eines Co-Autors.}{N}

        \fentry{submitCoAuthor}{commandButton}{Ausgabe des Co-Autors in einer Liste.}{N}

        \fentry{coAuthorList}{outputText}{Anzeige aller Co-Autoren in einer Liste.}{N}

        \fentry{deleteCoAuthor}{commandButton}{Löschen des zugehörigen Co-Authors.}{N}

        \fentry{submit}{commandButton}{Ausführen des Abgabe-Prozesses.}{N}
    }
\end{samepage}

\begin{samepage}
    \textbf{resultList.xhtml} zeigt alle Ergebnisse einer Suche an.
    \nopagebreak

    Wird nach den eigenen Einreichungen, nach denen die man begutachtet oder nach Einreichungen in eingener editorialen Verantwortung gesucht, so erscheinen die Listen im gleiche Schema wie auch auf \hyperref[flt:homepage]{der Homepage}.
    Ergibt die Suche eine Liste von User, so wird diese wie in \hyperref[flt:userList]{der Userliste} für den Administrator und dem Editor angezeigt.
    Ist das Ziel der Suche, wissenschaftliche Foren zu finden so erhält man bei passender Eingabe eine Liste im selben Format wie diejenige auf der \hyperref[flt:forumList]{Seite der Forumsliste.}
\end{samepage}

\begin{samepage}

    \textbf{scientificForumList.xhtml} \label{flt:forumList} Hier erhalten die Nutzer eine Übersicht über alle wissenschaftliche Foren.\nopagebreak

    \ftable{

        \fentry{scFieldSelect}{selectOneMenu}{Filtern der Themengebiete der Foren}{Alle}

        \fentry{deadlineSelect}{selectOneMenu}{Filtern der nach Deadlines}{Alle}

        \fentry{dateBefore}{inputText}{Eingabe eines \emph{zukünftigen} Deadlines}{Alle}

        \fentry{dateAfter}{inputText}{Eingabe eines \emph{vergangenen} Deadlines}{Alle}

        \fentry{dateBefore}{inputText}{Eingabe eines \emph{spätestens} Datum}{Alle}

        \fentry{dateAfter}{inputText}{Eingabe eines \emph{frühestens} Datum}{Alle}

        \fentry{forumsPagination}{pagination}{Liste aller Foren im System.}{Alle}
    }
\end{samepage}

\begin{samepage}

    Die Liste aller Foren im System wird in folgende Bereiche unterteilt: Name des Forums, Deadline und Kategorie.
    Dabei beinhaltet jede Zeile dieser Liste folgende Elemente:

    \ftable{
        \fentry{name}{link}{Name des Forums.}{Alle}

        \fentry{deadline}{outputText}{Abgabefrist des jeweiligen Forums.}{Alle}

        \fentry{scienceField}{outputText}{Kategorie des Forums.}{Alle}
    }
\end{samepage}

\begin{samepage}

    \textbf{profile.xhtml} Die Profilseite eines angemeldeten Nutzers ist nur vom Nutzer selbst und vom Administrator editierbar.
    Ein Editor besitzt nur Leserechte. Alle zuvor gespeicherten Angaben werden in den jeweiligen Feldern angezeigt.
    \nopagebreak

    \ftable{

        \fentry{avatar}{graphicImage}{Aktueller Avatar.}{N,E}

        \fentry{newAvatar}{inputFile}{Hochladen eines neuen Avatars.}{N,E}

        \fentry{deleteAvatar}{commandButton}{Löschen des Avatars.}{N,E}

        \fentry{role}{outputText}{Angabe der Rolle eines Nutzers}{N,E}

        \fentry{isAdmin}{checkbox}{Zuteilung der Rolle des Administrators.}{A}

        \fentry{title}{inputText}{Titel des Nutzenden.}{N,E}

        \fentry{firstName}{inputText}{Vorname des Nutzenden.}{N,E}

        \fentry{name}{inputText}{Nachname des Nutzenden.}{N,E}

        \fentry{password}{inputSecret}{Leeres Eingabefeld für ein neues Passwort.}{N}

        \fentry{email}{inputText}{E-Mail des Nutzenden.}{N,E}

        \fentry{submissionNumber}{outputText}{Anzahl der eigenen Einreichungen}{N,E}

        \fentry{employer}{inputText}{Angabe des Arbeitgebers}{N,E}

        \fentry{scieneField}{outputText}{Liste aller Fachgebiete}{N,E}

        \fentry{addScienceField}{inputText}{Spezialgebiete des Nutzers}{N,E}

        \fentry{addFieldButton}{commandButton}{Füge Fachgebiet zur Liste hinzu.}{N,E}

        \fentry{dateOfBirth}{inputText}{Angabe des Geburtstages}{N}

        \fentry{save}{commandButton}{Daten werden persistiert.}{N,E}

        \fentry{delete}{commandButton}{Nutzer und alle damit verbundenen Daten werden gelöscht.}{N,E}
    }
\end{samepage}

Sollte ein Administrator einem anderen Nutzer die Rolle des Administrators zuweisen oder entziehen, so erscheint beim Auslösen des Save-Buttons ein Popup-Dialog.
In diesem Fenster wird der Administrator angewiesen, die Änderung mit seinem Passwort zu bestätigen.

\ftable{

    \fentry{password}{inputSecret}{Angabe eines Passworts.}{A}

    \fentry{reference}{outputText}{Hinweis zur Veränderung der Adminrolle.}{A}

    \fentry{abort}{commandButton}{Abbruch der Änderung.}{A}

    \fentry{save}{commandButton}{Speichern aller Änderungen.}{A}

}

\begin{samepage}
    \textbf{scientificForum.xhtml} Die Ansicht eines Forums dient zur Ausgabe von Informationen über die jeweilige Konferenz oder das jeweilige Journal.
    \nopagebreak

    \ftable{

        \fentry{forumName}{outputText}{Name des Forums.}{Alle}

        \fentry{editor}{outputText}{Liste der verantwortliche Editoren.}{Alle}

        \fentry{deadline}{outputText}{Deadline des Forums.}{Alle}

        \fentry{description}{outputText}{Kurzbeschreibung des Forums.}{Alle}

        \fentry{url}{outputLink}{Link zur Konferenz oder zum Journal.}{Alle}

        \fentry{reviewInstructions}{outputText}{Anleitung für eine Begutachtung.}{G,E}

        \fentry{scienceField}{outputText}{Fachgebiet des Forums.}{Alle}

        \fentry{editorialPagination}{pagination}{Paginierte Liste aller Einreichungen, die der Editor verwaltet.}{E}

        \fentry{reviewesPagination}{pagination}{Paginierte Liste aller Einreichungen, die der Gutachter bearbeitet.}{E,G}

        \fentry{submissionPagination}{pagination}{Paginierte Liste aller Einreichungen, die der Nutzer eingereicht hat.}{Alle}
    }
\end{samepage}

\subsubsection{Gutachter}
\localauthor{Stefanie Gürster}

\begin{samepage}
    \textbf{newReview.xhtml} Möchte ein Gutachter eines Papers seine Beurteilung abgeben, so ist dies auf dieser Seite möglich.
    \nopagebreak

    \ftable{
        \fentry{versionNumber}{outputText}{Versionsnummer des eingereichten Papers, zu welchem der Gutachter ein Gutachten abgeben kann.}{G}

        \fentry{recommendation}{checkbox}{Empfehlung des Gutachters ein Paper zu akzeptieren.}{G}

        \fentry{comment}{inputTextarea}{Kommentar eines Gutachters}{G}

        \fentry{reviewPdf}{inputFile}{Einzureichendes Gutachten.}{G}

        \fentry{submitReview}{commandButton}{Gutachten einreichen.}{G}
    }
\end{samepage}

\subsubsection{Editor}
    \localauthor{Johann Schicho} \nopagebreak

\begin{samepage}
    \textbf{userList.xhtml} \label{flt:userList} Editoren und Administratoren erhalten hier einen Überblick über alle Nutzer. Die Seiten sind paginiert.\nopagebreak

    \ftable{
        \fentry{user-pagination}{pagination}{Die Tabelle enthält eine Paginierung.}{E}
    }
\end{samepage}

\begin{samepage}

    Für die Benutzerliste sind die folgenden Spalten vorgesehen, wobei diese in die Bereiche: Nutzerrolle, Name, E-Mail-Adresse, Arbeitgeber und Fachgebiete aufgegliedert ist.

    \ftable{
        \fentry{role}{outputText}{Rolle des Nutzers}{E}

        \fentry{name}{link}{Name des Nutzers. Link mit URL-Parameter zur Weiterleitung auf die Profilseite des Nutzers.}{E}

        \fentry{email}{link}{Mailto-Link mit E-Mail-Adresse des Nutzers.}{E}

        \fentry{employer}{outputText}{Arbeitgeber des Nutzers.}{E}

        \fentry{scienceField}{outputText}{Fachgebiete des Nutzers.}{E}
    }
\end{samepage}


\subsubsection{Admin}

\begin{samepage}\localauthor{Stefanie Gürster}

    \textbf{initialConfig.xhtml} Auf dieser Seite landet man beim ersten Systemstart. Sie ist nur für Administratoren zugänglich.
    Hier kann der Administrator die Datenbankschemata erstellen lassen.
    \nopagebreak

    \ftable{
        \fentry{dbConnectionState}{outputText}{Information über die Verbindung mit der Datenbank}{A}

        \fentry{createDbBtn}{commandButton}{Erstellt die Datenbankschemata}{A}
    }
\end{samepage}

\begin{samepage}

    \textbf{newUser.xhtml} Der Administrator kann hier einen neuen Nutzer anlegen.
    \nopagebreak

    \ftable{
        \fentry{password}{inputSecret}{Angabe des Passworts.}{A}

        \fentry{firstName}{inputText}{Angabe des Vornamen.}{A}

        \fentry{name}{inputText}{Angabe des Nachnamen.}{A}

        \fentry{titel}{inputText}{Angabe eines Titels.}{A}

        \fentry{email}{inputText}{Angabe einer E-Mail-Adresse.}{A}

        \fentry{isAdmin}{checkbox}{Zuordnung einer Administratorenrolle.}{A}

        \fentry{abort}{commandButton}{Abbruch des Vorgangs: Nutzererstellen.}{A}

        \fentry{save}{commandButton}{Speichern des neuen Nutzers.}{A}

    }
\end{samepage}

\begin{samepage}
    \textbf{administration.xhtml} Hier kann ein Administrator Konfigurationen vornehmen.
    \nopagebreak

    \ftable{

        \fentry{themeHeading}{outputText}{Überschrift für Farbthemes.}{A}

        \fentry{orange}{radioButton}{Auswahl des Standart Themes.}{A}

        \fentry{dark}{radioButton}{Auswahl des Dark Themes.}{A}

        \fentry{blue}{radioButton}{Auswahl des Blue Themes.}{A}

        \fentry{green}{radioButton}{Auswahl des Green Themes.}{A}

        \fentry{logoHeading}{outputText}{Überschrift für Logo-Bereich.}{A}

        \fentry{logo}{graphicImage}{Logo des Systems.}{A}

        \fentry{changeLogo}{inputFile}{Hochladen eines neuen Logos.}{A}

        \fentry{institutionHeading}{outputText}{Überschrift für Einstellungen der Einrichtung.}{A}

        \fentry{institution}{inputText}{Angabe des Namens der Einrichtung.}{A}

        \fentry{imprint}{inputTextArea}{Angabe des Impressums der Einrichtung.}{A}

        \fentry{abort}{commandButton}{Abbruch des Änderungsvorgangs.}{A }

        \fentry{save}{commandButton}{Speichern der Änderungen.}{A}

        \fentry{newUser}{link}{Weiterleitung zu \emph{newUser.xhtml}.}{A}

        \fentry{newScientificForum}{link}{Weiterleitung zu \emph{newScientificForum.xhtml}.}{A}
    }
\end{samepage}

\begin{samepage}
    \textbf{newScientificForum.xhtml} Auf der Seite zum Erstellen eines wissenschaftlichen Forums kann ein Administrator dessen wesentlichen Daten festlegen.
    \nopagebreak

    \ftable{

        \fentry{forumName}{inputText}{Angabe des Name des Forums.}{A}

        \fentry{emailEditor}{inputText}{Angabe der E-Mail-Adresse eines Editor.}{A}

        \fentry{addEditor}{commandButton}{Hinzufügen eines Editors in eine Liste.}{A}

        \fentry{deleteEditor}{commandButton}{Löschen des zugehörigen Editors von der Liste.}{A}

        \fentry{deadline}{inputText}{Hinzufügen einer Deadline des Forums.}{A}

        \fentry{description}{inputTextArea}{Angabe einer Kurzbeschreibung.}{A}

        \fentry{url}{inputText}{Link zur Konferenz oder zum Journal.}{A}

        \fentry{reviewInstructions}{inputTextArea}{Angabe einer Anleitung für eine Begutachtung.}{A}

        \fentry{scienceField}{selectOneMenu}{Auswahl von Fachgebieten.}{A}

        \fentry{newScienceField}{inputText}{Hinzufügen neuer Fachgebiete.}{A}

        \fentry{addScienceField}{commandButton}{Ausführen der Hinzufüge-Aktion.}{A}

        \fentry{save}{commandButton}{Speichern des neuen Forums.}{A}

        \fentry{abort}{commandButton}{Abbruch des Erstellungsprozesses.}{A}
    }
\end{samepage}

\begin{samepage}
    \textbf{editForum.xhtml} Hier kann der Administrator eine schon existierende Seite bearbeiten.
    \nopagebreak

    \ftable{
        \fentry{forumName}{inputText}{Angabe des Name des Forums.}{E}

        \fentry{emailEditor}{inputText}{Angabe der E-Mail-Adresse eines Editor.}{E}

        \fentry{addEditor}{commandButton}{Hinzufügen eines Editors in eine Liste.}{E}

        \fentry{deleteEditor}{commandButton}{Löschen des zugehörigen Editors von der Liste.}{E}

        \fentry{deadline}{inputText}{Hinzufügen einer Deadline des Forums.}{E}

        \fentry{description}{inputTextArea}{Angabe einer Kurzbeschreibung.}{E}

        \fentry{url}{inputText}{Link zur Konferenz oder zum Journal.}{E}

        \fentry{reviewInstructions}{inputTextArea}{Angabe einer Anleitung für eine Begutachtung.}{E}

        \fentry{scienceFieldOut}{outputText}{Liste der Fachgebiete.}{E}

        \fentry{scienceField}{selectOneMenu}{Auswahl von Fachgebieten.}{E}

        \fentry{newScienceField}{inputText}{Hinzufügen neuer Fachgebiete.}{A}

        \fentry{addScienceField}{commandButton}{Ausführen der Hinzufüge-Aktion.}{A}

        \fentry{save}{commandButton}{Speichern des Änderungen.}{E}

        \fentry{abort}{commandButton}{Abbruch des Bearbeitungsprozesses.}{E}

        \fentry{delete}{commandButton}{Löscht das Forum.}{A}
    }
\end{samepage}
