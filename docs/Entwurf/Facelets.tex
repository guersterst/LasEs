\localauthor{Stefanie Gürster}
Im Folgendem Abschnitt werden Abkürzungen für die verschiedenen Rollen eingeführt:
\textbf{A} steht für Administrator, \textbf{N} für einen angemeldeten Nutzer, \textbf{E} für einen Editor und \textbf{G} steht für einen Gutachter.
Ist eine Funktion für alle Benutzerrollen vorgesehen, so werden diese unter dem Begriff \textbf{Alle} zusammengefasst.

\todo{OutputText oder graphicimages für Bezeichnung hinzufügen}

\subsection{Templates}
\textbf{navigation.xhtml} ist die Kopfzeile der Webanwendung. Diese bietet die Suchfunktion an und Links zu verschiedenen Listen und dem Profil.

\begin{tabular}[H]{|m{2cm}|m{3cm}|m{6cm}|m{2.5cm}|}
	\hline
	\textbf{ID} & \textbf{Typ} & \textbf{Beschreibung} & \textbf{Sichtbarkeit} \\
	\hline
	\hline
	searchfield & inputText & Suchleiste & N \\
	\hline
	user-list-link & link & Link zur Übersichtsseite aller Nutzer & A, E \\
	\hline
	forum-list-link & link & Link zur Übersichtsseite alle Journale und Konferenzen & N \\
	\hline
	logout-button & commandButton & Loggt den Nutzer aus dem System aus und leitet zur Loginseite weiter & N \\
	\hline
	profile-link & link & Link zur Profilübersicht & N \\
	\hline
\end{tabular}

\textbf{main.xhtml} ist das Template, welches die Seiten zwischen Kopf- und Fußzeile einbettet.

\begin{tabular}[H]{|m{2cm}|m{3cm}|m{6cm}|m{2.5cm}|}
	\hline
	\textbf{ID} & \textbf{Typ} & \textbf{Beschreibung} & \textbf{Sichtbarkeit} \\
	\hline
	\hline
	navigation-bar & include & Kopfzeile & Alle\\
	\hline
	main-content & insert & Seiteninhalt & Alle \\
	\hline
	footer-bar & include & Fußzeile & Alle \\
	\hline
	tool-bar & include & Seitenleiste & A, E \\
	\hline
\end{tabular}

\textbf{footer.xhtml} ist die Fußzeile der Webanwendung und für alle sichtbar.

\begin{tabular}[H]{|m{2cm}|m{3cm}|m{6cm}|m{2.5cm}|}
	\hline
	\textbf{ID} & \textbf{Typ} & \textbf{Beschreibung} & \textbf{Sichtbarkeit} \\
	\hline
	\hline
	imprint-link & link & Link zur Seite des Impressums & Alle\\
	\hline
\end{tabular}

\begin{samepage} %https://stackoverflow.com/questions/4003473/make-an-unbreakable-block-in-tex

\textbf{toolbar.xhtml} ist die Seitenleiste für Editoren und Administratoren auf der Einreichungsübersichtsseite. Hier wird die Einreichung verwaltet.
\nopagebreak

\begin{tabular}[H]{|m{2cm}|m{3cm}|m{6cm}|m{2.5cm}|}
	\hline
	\textbf{ID} & \textbf{Typ} & \textbf{Beschreibung} & \textbf{Sichtbarkeit} \\
	\hline
	\hline
	add-reviewer-field & inputText & Eingabefeld zur Angabe einer E-Mailadresse & A, E \\
	\hline
	add-reviewer-btn & commandButton & Knopf zum hinzufügen des Gutachters & A, E \\
	\hline
	reviewer-table & dataTable & Liste der Gutachter & A, E \\
	\hline
	remove-reviewer & commandButton & Entferne einen bestimmten Gutachter & A, E \\
	\hline
	select-editor & selectOneMenu & Auswahländerung des verwaltenden Editors & A, E \\
	\hline
	save-editor & commandButton & Speichern des geänderten Editors & A, E \\
	\hline
	require-revision-btn & commandButton & Fordert den Einreicher auf, seine Einreichung zu überarbeiten & A, E \\
	\hline
	accept-submission-btn & commandButton & Akzeptiere die Einreichung & A, E \\
	\hline
	reject-submission-btn & commandButton & Lehne die Einreichung ab & A, E \\
	\hline
\end{tabular}

\end{samepage}

\subsection{Seiten}
initialConfig.xhtml

\textbf{welcome.xhtml} Auf der Login- bzw. Welcome-Seite wird \emph{LasEs} vorgestellt.
Zusätzlich gibt ein Login-Formular zur Anmeldung im System.
Für nicht registrierte Nutzer wird man über einem gegebenen Link zu \emph{registration.xhtml} weitergeleitet.


\begin{tabular}[H]{|m{2cm}|m{3cm}|m{6cm}|m{2.5cm}|}
    \hline
    \textbf{ID} & \textbf{Typ} & \textbf{Beschreibung} & \textbf{Sichtbarkeit} \\
    \hline
    \hline
    welcomeHeading & outputText & Überschrift der Welcomepage. & Alle\\
    \hline
    welcomeText & outputText & Bewerben der Applikation mithilfe einer Kurzbeschreibung. & Alle \\
    \hline
    username & inputText & Textfeld für Username Eingabe. & Alle \\
    \hline
    password & inputSecret & Textfeld für Passwort Eingabe. & Alle \\
    \hline
    login & commandButton & Ausführen des Login-Prozesses. & Alle \\
    \hline
    register & link & Weiterleitung zur Registrierung. & Alle \\
    \hline
\end{tabular}

\textbf{registration.xhtml} Seite zur Registrierung anonymer Nutzer.

\begin{tabular}[H]{|m{2cm}|m{3cm}|m{6cm}|m{2.5cm}|}
    \hline
    \textbf{ID} & \textbf{Typ} & \textbf{Beschreibung} & \textbf{Sichtbarkeit} \\
    \hline
    \hline
    title & inputText & Angabe eines Titels. & Alle \\
    \hline
    firstName & inputText & Angabe des Vornamens. & Alle \\
    \hline
    name & inputText & Angabe des Nachnamen. & Alle \\
    \hline
    password & inputSecret & Angabe eines Passwortes. & Alle \\
    \hline
    email & inputText & Angabe einer validen E-Mail. & Alle \\
    \hline
    register & commandButton & Ausführen des Registrations-Prozesses. & Alle \\
    \hline
    welcomepage & link & Weiterleitung zur Login Seite. & Alle\\
    \hline
\end{tabular}

\textbf{verification.xhtml} Wird bei erfolgreicher Verifikation der E-Mail angezeigt.

\begin{tabular}[H]{|m{2cm}|m{3cm}|m{6cm}|m{2.5cm}|}
    \hline
    \textbf{ID} & \textbf{Typ} & \textbf{Beschreibung} & \textbf{Sichtbarkeit} \\
    \hline
    \hline
    successText & outputText & Text zur erfolgreichen Verifizierung der E-Mail & Alle\\
    \hline
    countdown & timer & Timer bis zur Weiterleitung auf die Homepage. & Alle \\
    \hline
\end{tabular}

\textbf{homepage.xhtml} Startseite die den Überblick über alle Einreichungen beinhaltet.

submission.xhtml

\textbf{newSubmission.xhtml} Hier können neue Paper eingereicht werden.

\begin{tabular}[H]{|m{2cm}|m{3cm}|m{6cm}|m{2.5cm}|}
    \hline
    \textbf{ID} & \textbf{Typ} & \textbf{Beschreibung} & \textbf{Sichtbarkeit} \\
    \hline
    \hline
    submissionName & inputText & Name des abzugebenden Papers. & N\\
    \hline
    forumName & inputText & Name des Forums, bei welchem abgegeben wird. & N \\
    \hline
    editorSearch & inputText & Angabe eines Editors & N\\
    \hline
    pfdUpload & inputFile & Angabe der Abgabedatei. & N\\
    \hline
    titel & inputText & Angabe eines Titels. & N\\
    \hline
    coAuthorFirstName & inputText & Angabe des Vornamen eines Co-Autors.& N\\
    \hline
    coAuthorName & inputText & Angabe des Namen eines Co-Autors. & N\\
    \hline
    coAuthorEMail & inputText & Angabe der E-Mail eines Co-Autors. & N\\
    \hline
    submitCoAuthor & commandButton & Ausgabe des Co-Autors in einer Liste. & N\\
    \hline
    coAuthorList & outputText & Anzeige aller Co-Authoren in einer Liste. & N\\
    \hline
    deleteCoAuthor & commandButton & Löschen des zugehörigen Co-Authors. & N\\
    \hline
    submit & commandButton & Ausführen des Abgabe-Prozesses. & N \\
    \hline
\end{tabular}

\textbf{newReview.xhtml} Möchte ein Gutachter eines Papers seine Beurteilung abgeben, so ist dies auf dieser Seite möglich.

\begin{tabular}{|m{2cm}|m{3cm}|m{6cm}|m{2.5cm}|}
    \hline
    \textbf{ID} & \textbf{Typ} & \textbf{Beschreibung} & \textbf{Sichtbarkeit} \\
    \hline
    \hline
    versionNumber & outputText & Versionsnummer des eingereichten Papers, zu welchem der Gutachter ein Gutachten abgeben kann. & G,A\\
    \hline
    recommendation & checkbox & Empfehlung des Gutachters ein Paper freizuschalten. & G,A\\
    \hline
    comment & inputTextarea & Kommentar eines Gutachters & G,A\\
    \hline
    reviewPdf & inputFile & Einzureichendes Gutachten. & G,A \\
    \hline
\end{tabular}


resultList.xhtml
scientificForumList.xhtml

\textbf{profile.xhtml} Die Profilseite eines angemeldeten Nutzers ist nur vom Nutzer selbst und vom Administrator editierbar.
Ein Editor besitzt nur Leserechte. Alle zuvor gespeicherten Angaben werden in den jeweiligen Feldern angezeigt.

\begin{tabular}[H]{|m{2cm}|m{3cm}|m{6cm}|m{2.5cm}|}
    \hline
    \textbf{ID} & \textbf{Typ} & \textbf{Beschreibung} & \textbf{Sichtbarkeit} \\
    \hline
    \hline
    avatar & graphicImage & Aktueller Avatar. & N,A,E\\
    \hline
    newAvatar & inputFile & Hochladen eines neuen Avatars. & N,A,E\\
    \hline
    deleteAvatar & commandButton & Löschen des Avatars. & N,A,E\\
    \hline
    role & outputText & Angabe der Rolle eines Nutzers & N,A,E\\
    \hline
    isAdmin & checkbox & Zuteilung der Rolle des Administrators. & A\\
    \hline
    title & inputText & Titel des Nutzenden. & N,A,E\\
    \hline
    firstName & inputText & Vorname des Nutzenden. & N,A,E\\
    \hline
    name & inputText & Nachname des Nutzenden. & N,A,E\\
    \hline
    password & inputSecret & Leeres Eingabefeld für ein neues Passwort. & N,A\\
    \hline
    email & inputText & E-Mail des Nutzenden. & N,A,E\\
    \hline
    submissionNumber & outputText & Anzahl der eigenen Einreichungen & N,A,E\\
    \hline
    employer & inputText & Angabe des Arbeitgebers & N,A,E\\
    \hline
    specialty & & Spezialgebiete des Nutzers & N,A,E \\
    \hline
    dateOfBirth & inputText & Angabe des Geburtstages & Nutzer\\
    \hline
    save & commandButton & Daten werden persistiert. & N,A,E\\
    \hline
    delete & commandButton & Nutzer und alle damit verbundenen Daten werden gelöscht. & N,A,E\\
    \hline
\end{tabular}

Sollte ein Administrator einem Nutzer die Rolle des Admins zuweisen oder entziehen, so erscheint beim Auslösen des Save-Buttons ein Popup-Dialog.
In diesem Fenster  wird der Administrator angewiesen die Änderung mit seinem Passwort zu bestätigen.

\begin{tabular}[H]{|m{2cm}|m{3cm}|m{6cm}|m{2.5cm}|}
    \hline
    \textbf{ID} & \textbf{Typ} & \textbf{Beschreibung} & \textbf{Sichtbarkeit} \\
    \hline
    \hline
    password & inputSecret & Angabe eines Passworts. & A\\
    \hline
    reference & outputText & Hinweis zur Veränderung der Adminrolle. & A\\
    \hline
    abort & commandButton & Abbruch der Änderung. & A\\
    \hline
    save & commandButton & Speichern aller Änderungen. & A\\
    \hline
\end{tabular}

\todo{bearbeiten kann es nur der Nutzer aber betrachten andere....+ was ist mit Passwort F200+ doch getrennte Profilseiten? + avatarbild}

\textbf{errorPage.xhtml} Auf diese Seite wird navigiert, wenn auf eine nicht existierende URL oder auf eine URL zugegriffen wird, auf die man keine Zugriffsrechte hat.

\begin{tabular}[H]{|m{2cm}|m{3cm}|m{6cm}|m{2.5cm}|}
    \hline
    \textbf{ID} & \textbf{Typ} & \textbf{Beschreibung} & \textbf{Sichtbarkeit} \\
    \hline
    \hline
    errorMessage & outputText & Anzeige einer Fehlermeldung. & Alle\\
    \hline
\end{tabular}

userList.xhtml

\textbf{newUser.xhtml} Der Administrator kann hier einen neuen Nutzer anlegen.

\begin{tabular}[H]{|m{2cm}|m{3cm}|m{6cm}|m{2.5cm}|}
    \hline
    \textbf{ID} & \textbf{Typ} & \textbf{Beschreibung} & \textbf{Sichtbarkeit} \\
    \hline
    \hline
    password & inputSecret & Angabe des Passworts. & A\\
    \hline
    firstName & inputText & Angabe des Vornamen. & A\\
    \hline
    name & inputText & Angabe des Nachnamen. & A\\
    \hline
    titel & inputText & Angabe eines Titels. & A\\
    \hline
    email & inputText & Angabe einer E-Mail-Adresse. & A\\
    \hline
    isAdmin & checkbox & Zuordnung einer Administratorenrolle. & A\\
    \hline
    abrot & commandButton & Abbruch des Vorgangs: Nutzererstellen. & A\\
    \hline
    save & commandButton & Speichern des neuen Nutzers. & A \\
    \hline
\end{tabular}

\textbf{administration.xhtml} Hier kann ein Administrator Konfigurationen vornehmen.

\begin{tabular}[H]{|m{2cm}|m{3cm}|m{6cm}|m{2.5cm}|}
    \hline
    \textbf{ID} & \textbf{Typ} & \textbf{Beschreibung} & \textbf{Sichtbarkeit} \\
    \hline
    \hline
    themeHeading & outputText & Überschrift für Farbthemen. & A\\
    \hline
    orangeTheme & outputText & Standart Theme der Applikation. & A\\
    \hline
    orange & radioButton & Auswahl des Standart Themes. & A\\
    \hline
    darkTheme & outputText & Dark Theme der Applikation. & A\\
    \hline
    dark & radioButton & Auswahl des Dark Themes. & A\\
    \hline
    blueTheme & outputText & Blue Theme der Applikation & A\\
    \hline
    blue & radioButton & Auswahl des Blue Themes. & A\\
    \hline
    greenTheme & outputText & Green Theme der Applikation. & A\\
    \hline
    green & radioButton & Auswahl des Green Themes. & A\\
    \hline
    logoHeading & outputText & Überschrift für Logo-Bereich. & A\\
    \hline
    logo & graphicImage & Logo des Systems. & A\\
    \hline
    changeLogo & inputFile & Hochladen eines neuen Logos. & A\\
    \hline
    institutionHeading & outputText & Überschrift für Einstellungen der Einrichtung. & A\\
    \hline
    institution & inputText & Angabe des Namens der Einrichtung. & A\\
    \hline
    imprint & inputTextArea & Angabe des Impressums der Einrichtung. & A\\
    \hline
    abort & commandButton & Abbruch des Änderungsvorgangs. & A \\
    \hline
    save & commandButton & Speichern der Änderungen. & A\\
    \hline
    newUser & link & Weiterleitung zu \emph{newUser.xhtml}. & A\\
    \hline
    newScientificForum & link & Weiterleitung zu \emph{newScientificForum.xhtml}. & A\\
    \hline
\end{tabular}

\textbf{newScientificForum.xhtml} Auf der Seite zum Erstellen eines wissenschaftlichen Forums kann ein Administrator dessen wesentlichen Daten festlege.

\begin{tabular}[H]{|m{2cm}|m{3cm}|m{6cm}|m{2.5cm}|}
    \hline
    \textbf{ID} & \textbf{Typ} & \textbf{Beschreibung} & \textbf{Sichtbarkeit} \\
    \hline
    \hline
    forumName & inputText & Angabe des Name des Forums. & A\\
    \hline
    fName & outputLabel & Label Name des Forums. & A\\
    \hline
    emailEditor & inputText & Angabe der E-Mail-Adresse eines Editor. & A\\
    \hline
    emailE & outputLabel & Label für E-Mail-Adresse eines Editors. & A\\
    \hline
    addEditor & commandButton & Hinzufügen eines Editors in eine Liste. & A\\
    \hline
    deleteEditor & commandButton & Löschen des zugehörigen Editors von der Liste. & A\\
    \hline
    deadline & inputText & Hinzufügen einer Deadline des Forums. & A\\
    \hline
    deadlineLabel & outputLabel & Label für Deadline des Forums. & A\\
    \hline
    description & inputTextArea & Angabe einer Kurzbeschreibung. & A\\
    \hline
    descriptionLabel & outputLabel & Label für Kurzbeschreibung. & A\\
    \hline
    url & inputText & Link zur Konferenz oder zum Journal. & A\\
    \hline
    urlLabel & outputLabel & Label für URL des Forums. & A\\
    \hline
    reviewInstructions  & inputTextArea & Angabe einer Anleitung für eine Begutachtung. & A\\
    \hline
    specialty & selectOnMenus & Auswahl von Fachgebieten. & A\\
    \hline
    specialtyLabel & outputLabel & Label für Fachgebiete. & A\\
    \hline
    newSpecialty & inputText & Hinzufügen neuer Fachgebiete. & A\\
    \hline
    addSpecialty & commandButton & Ausführen der Hinzufüge-Aktion. & A\\
    \hline
    save & commandButton & Speichern des neuen Forums. & A\\
    \hline
    abort & commandButton & Abbruch des Erstellungsprozesses. & A\\
    \hline
\end{tabular}

\textbf{editForum.xhtml} Hier kann der Administrator eine schon existierende Seite bearbeiten.

\begin{tabular}[H]{|m{2cm}|m{3cm}|m{6cm}|m{2.5cm}|}
    \hline
    \textbf{ID} & \textbf{Typ} & \textbf{Beschreibung} & \textbf{Sichtbarkeit} \\
    \hline
    \hline
    forumName & inputText & Angabe des Name des Forums. & A\\
    \hline
    fName & outputLabel & Label Name des Forums. & A\\
    \hline
    emailEditor & inputText & Angabe der E-Mail-Adresse eines Editor. & A\\
    \hline
    emailE & outputLabel & Label für E-Mail-Adresse eines Editors. & A\\
    \hline
    addEditor & commandButton & Hinzufügen eines Editors in eine Liste. & A\\
    \hline
    deleteEditor & commandButton & Löschen des zugehörigen Editors von der Liste. & A\\
    \hline
    deadline & inputText & Hinzufügen einer Deadline des Forums. & A\\
    \hline
    deadlineLabel & outputLabel & Label für Deadline des Forums. & A\\
    \hline
    description & inputTextArea & Angabe einer Kurzbeschreibung. & A\\
    \hline
    descriptionLabel & outputLabel & Label für Kurzbeschreibung. & A\\
    \hline
    url & inputText & Link zur Konferenz oder zum Journal. & A\\
    \hline
    urlLabel & outputLabel & Label für URL des Forums. & A\\
    \hline
    reviewInstructions  & inputTextArea & Angabe einer Anleitung für eine Begutachtung. & A\\
    \hline
    specialty & selectOnMenus & Auswahl von Fachgebieten. & A\\
    \hline
    specialtyLabel & outputLabel & Label für Fachgebiete. & A\\
    \hline
    newSpecialty & inputText & Hinzufügen neuer Fachgebiete. & A\\
    \hline
    addSpecialty & commandButton & Ausführen der Hinzufüge-Aktion. & A\\
    \hline
    save & commandButton & Speichern des Forums. & A\\
    \hline
    abort & commandButton & Abbruch des Erstellungsprozesses. & A\\
    \hline
\end{tabular}

\textbf{scientificForum.xhtml} Die Ansicht eines Forums dient zur Ausgabe von Informationen über die jeweilige Konferenz oder das jeweilige Journal.

\begin{tabular}[H]{|m{2cm}|m{3cm}|m{6cm}|m{2.5cm}|}
    \hline
    \textbf{ID} & \textbf{Typ} & \textbf{Beschreibung} & \textbf{Sichtbarkeit} \\
    \hline
    \hline
    forumName & outputText & Name des Forums. & Alle\\
    \hline
    fName & outputLabel & Label Name des Forums. & Alle\\
    \hline
    editor & outputText & Liste der verantwortliche Editoren. & Alle\\
    \hline
    deadline & outputText & Deadline des Forums. & Alle\\
    \hline
    deadlineLabel & outputLabel & Label für Deadline des Forums. & Alle\\
    \hline
    description & outputText & Kurzbeschreibung des Forums. & Alle\\
    \hline
    descriptionLabel & outputLabel & Label für Kurzbeschreibung. & Alle\\
    \hline
    url & outputLink & Link zur Konferenz oder zum Journal. & Alle\\
    \hline
    urlLabel & outputLabel & Label für URL des Forums. & Alle\\
    \hline
    reviewInstructions  & outputText & Anleitung für eine Begutachtung. & Alle\\
    \hline
    specialty & outputText & Fachgebiet des Forums. & Alle\\
    \hline
    specialtyLabel & outputLabel & Label für Fachgebiete. & Alle\\
    \hline
    save & commandButton & Speichern des Forums. & Alle\\
    \hline
    abort & commandButton & Abbruch des Erstellungsprozesses. & Alle\\
    \hline
\end{tabular}

\textbf{imprint.xhtml} Das Impressum gibt die vom Administrator angegebenen Kontaktdaten des Betreibers wieder.

\begin{tabular}[H]{|m{2cm}|m{3cm}|m{6cm}|m{2.5cm}|}
    \hline
    \textbf{ID} & \textbf{Typ} & \textbf{Beschreibung} & \textbf{Sichtbarkeit} \\
    \hline
    \hline
    imprintHeading & outputText & Überschrift der Ansicht & A\\
    \hline
    imprint & outputText & Impressum des Betreibers. & A\\
    \hline
\end{tabular}

\todo{Forum bearbeiten können ? Löschen?}
