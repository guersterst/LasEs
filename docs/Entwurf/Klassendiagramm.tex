\localauthor{Sebastian Vogt, Johannes Garstenauer, Johann Schicho}

\newcommand{\classtable}[1]{\begin{longtable}[H]{|m{5cm}|m{9cm}|}
		\hline
		\textbf{Klassenname} & \textbf{Beschreibung} \\
		\hline
		\hline
		#1
	\end{longtable}
}

\newcommand{\classentry}[2]{#1 & #2 \\
	\hline}


\subsection{Klassendiagramm}

% Hier Klassendiagramm mal irgendwann einfügen.

\subsection{Klassenbeschreibungen}

\subsubsection{de.lases.global.transport}

\classtable{
	\classentry{Paper}{This DTO represents a paper.}
	\classentry{Privilege}{This represents a user-privilige.}
	\classentry{Review}{This DTO represents a review.}
	\classentry{ScienceField}{This DTO represents a field of Science.}
	\classentry{ScientificForum}{This DTO represents a forum}
	\classentry{Style}{This represents a user interface style.}
	\classentry{Submission}{This DTO represents a submission.}
	\classentry{SubmissionState}{This represents a submissions state.}
	\classentry{SystemSettings}{This DTO represents the system settings.}
	\classentry{User}{This DTO represents a user.}
}

\subsubsection{de.lases.global.logging}

\classtable{
	\classentry{Logging}{This Logger allows logging in diffrent log-levels.}
}