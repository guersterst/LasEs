\localauthor{Johann Schicho}

Der Implementierungszeitraum erstreckt sich vom 03.12.2021 zum 14.01.2022.
Dabei ist der Zeitraum in drei Milestones untergliedert, in welchen die Anwendung phasenweise entwickelt wird. Ein \emph{Arbeitspaket} umfasst dabei immer die vertikale Implementierung eines Features. Das heißt, ein Facelet mit zugehörigem Backing-Bean und den benötigten Services mit \emph{DTOs}. Diese Vorgehensweise ermöglicht eine weitgehend unabhängige Arbeit zwischen den Teammitgliedern.

Die Aufteilung wird im Folgenden genauer spezifiziert:

\begin{description}
	\item \textbf{Milestone 1} dauert bis zum 10.12.2021. Hier werden zunächst Infrastrukturelemente, wie Datenbankverbindung, Logging, Config Reader und Crypto, implementiert. Anschließend werden grundlegende Arbeitspakete der Anwendung darauf aufbauend umgesetzt. Dazu gehören unter anderem Login, Homepage und Submission.

	\item \textbf{Milestone 2} dauert bis zum 17.12.2021. Implementiert werden weitere aufwendigere Arbeitspakete. Dazu gehören zum Beispiel Profilseite, Anlegen eines neuen Forums, Forumsübersicht, Registrierungs- und Administrationsseite und das Einreichen von Gutachten.

	\item \textbf{Milestone 3} dauert bis zum 07.01.2022. In dieser Phase werden die letzten Arbeitspakete implementiert. Unter anderem das Impressum, Nutzerliste, automatischer E-Mail-Service und die Suchergebnisseite. Auch das Aussehen der Webanwendung wird durch CSS und Bootstrap vereinheitlicht.

	\item \textbf{Vorläufige Abgabe} ist am 14.01.2022. Diese umfasst nun das vorläufig vollständig implementierte System, das finale Handbuch und den Implementierungsbericht.
\end{description}
